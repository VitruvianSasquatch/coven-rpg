\chapter{Brewing}
\chaplabel{brewing}

\section{Creation and Application}

The \skillref{brewing} skill and \discref{brewing} discipline don't just cover potions brewed in a cauldron.
Potions, poultices, poisons, tinctures, salves, ointments, even beer, mead, wine and spirits.
Witches have many ways of turning \seclink{Herbs}{herbs}, and even other things, into more useful forms.

Each feat that allows a witch to prepare such a mixture lists the method of preparation and delivery.
The rules of such methods are presented below.

\subsection{Brewing and Chewing}

Different methods of preparation require different equipment, and take different periods of time.

\mixcreation{Cauldron}{cauldron}
Most potions are brewed in cauldrons, filled with water and brought to boil.
This requires, obviously, a cauldron, as well as a fire to heat it.
A smaller kettle might do in a pinch, but requires a Test.
A full cauldron will typically yield several doses.
Brewing in a cauldron requires around 15 minutes to bring the water to the boil, and another minute to mix the potion.

\mixcreation{Poultice}{poultice}
A poultice doesn't need to be brewed at all; the ingredients are simply chewed into a paste.
Some of the more dangerous poultices should definitely be ground with a mortar and pestle, however, rather than allowed anywhere near the mouth.
Creating a poultice requires less than a minute.

\mixcreation{Still}{still}
Some potions, or spirits, need to be distilled.
This requires quite a lot of dedicated equipment, a carefully maintained heat source, and several hours.

\subsection{Method of Delivery}

\mixdelivery{Drink}{drink}
A drink is about half a litre of liquid that must be drunk to take effect.
It can be quaffed as an Action, and takes effect immediately unless specified otherwise.

\mixdelivery{Spike}{spike}
A spike is a much smaller quantity of liquid than a potion, little enough that it could be slipped into a glass of wine without noticeably changing the volume.
It can be drunk willingly, but typically isn't.
It takes effect immediately, unless specified otherwise.

\mixdelivery{Topical}{topical}
A topical mixture is applied to the skin.
It typically requires more than an Action to smear it on, or bind a wad in place.
It generally only takes effect after a few minutes, but can kick in a little faster if applied to a wound or a mucous membrane.
Some need to be applied to the correct part of the body.
For example, if it is to treat a wound, it should be applied to the wound, and if it is to enhance the eyesight it should be applied to the eyes.

\mixdelivery{Injury}{injury}
These mixtures, typically harmful ones, must be delivered into the bloodstream via an injury.
The most expedient way to do this is to smear it on an arrow or an edged weapon, requiring an Action.
It's good for one cut, but otherwise remains on the weapon until rubbed off or washed away.
Beware rain.
It takes effect immediately, unless specified otherwise.

%TODO: Gaseous? Needs to be stored in an air-tight bottle?
%TODO: Incense? Needs to be burned, evaporated?

\section{Side-Effects}

More some of the more noxious mixtures a witch can make come with adverse effects by design, they are not the only way potions can hurt.
Many potions and other brews come with adverse side effects all by themselves, and these are compounded by the dangers of overdosing and combining brews.

\subsection{Overdosing}

Many potions carry harmful effects when taking too many doses.
These typically only occur if multiple doses would be in effect simulataneously; taking another dose after the first has worn off is safe unless specified.
Some of the effects of overdoses are given explicitly, but many are given as general guidelines.
The GM is left to adjudicate in the latter case.
Typically the worst effects of overdosing can be staved off with a \attref{might} Test, with the TN affected by how many excess doses have been taken, and how close in succession they were taken.

\subsection{Mixing Mixtures}

Mixing multiple potions can have adverse and unexpected effects.
These kick in when a character is under the effect of two substances that both affect the same \seclink{Attribute}{attributes} or other statistic.
The effects are unpredictable.
The GM is free to apply any appropriate penalty, possibly calling for a \attref{might} Test to avoid or alleviate the effects.
The following table is provided for inspiration.
The GM may roll 2 6-sided dice and compare their sum against the table to randomly determine an effect, if desired.

\begin{simpletable}{rX}
	\toprule
	2d6 & Effect\\
	\midrule
	2 & Apply severe overdose effects of the first mixture.\\
	3 & Exhaustion, unconsciousness and/or oxygen deprivation.\\
	4 & Apply moderate overdose effects of the first mixture.\\
	5 & Ignore any positive effects of the first mixture.\\
	6 & Double any detrimental effects of the first mixture.\\
	7 & Reroll twice on the table, taking both results.\\
	8 & Double any detrimental effects of the second mixture.\\
	9 & Ignore any positive effects of the second mixture.\\
	10 & Apply moderate overdose effects of the second mixture.\\
	11 & Twitching, seizure, overheating and/or organ failure.\\
	12 & Apply severe overdose effects of the second mixture.\\
	\bottomrule
\end{simpletable}

The effect of painkillers---to ignore {\damage}---does not count as a statistic for the purpose of mixing substances.
As such, a character may safely be under the effects of multiple painkillers as long as their other effects do not overlap.
Additionally, modifications to statistics due to withdrawal effects do not count for mixing.

\section{Feats}

\feat{Numbing Painkiller}{painkiller-grace}{15}{
	None
}{
	\mix{cauldron}{drink}{\herbref[willow bark]{2}}
	%Willow bark contains natural NSAIDs.
	
	The drinker may ignore 1 point of {\damage} for a few hours, but loses 1 \attref{grace} for the same duration.
	Two doses may be effective simultaneously.
	Further doses cause paralysis, and possibly organ failure.
}

\feat{Dimming Painkiller}{painkiller-mental}{10}{
	None
}{
	\mix{cauldron}{drink}{\herbref[poppy seed]{2}}
	%Poppies contain natural opioids.
	
	The drinker may ignore 1 point of {\damage} for a few hours, but loses 1 \attref{ken} and \attref{wit} for the same duration.
	Two doses may be effective simultaneously.
	Further doses cause unconsciousness, and possibly cessation of breathing.
}

\feat{Blurring Painkiller}{painkiller-heed}{15}{
	None
}{
	\mix{cauldron}{drink}{\herbref[barley]{2}}
	%Barley contains a lot of phenols, which are used in the synthesis of a lot of pharmaceuticals.
	%Malt barley is also used in producing alcohol, which can justify the blindness.
	
	The drinker may ignore 1 point of {\damage} for a few hours, but loses 1 \attref{heed} for the same duration.
	Two doses may be effective simultaneously.
	Further doses cause blindness, which can become permanent.
}

\feat{The Hard Stuff}{brewing-booze}{20}{
	None
}{
	You know how to make a drink that'll really put hairs on a man's chest.
	Or a woman's, at that.
	
	\mix{still}{drink}{Alcohol, \herbref[apple]{2}}
	%Scumble is 'mostly apples'.
	
	The drinker gains 1 \attref{might} for a few hours, and loses 2 \attref{wit} and \attref{heed} for the same duration.
	A second dose will render the drinker unconscious.
	Further doses are dangerously poisonous, causing vomiting, seizures and oxygen deprivation.
}

\feat{Empathogen}{charm-stimulant}{20}{
	None
}{
	\mix{cauldron}{drink}{\herbref[violet]{2}}
	%Violet contains piperonal, a common precursor to the empathogen MDMA (ecstasy).
	%Violets were also emblematic flowers of Aphrodite and Priapus.
	
	This potion affords the user a greater sense of empathy and connection with those around them.
	The drinker gains 1 \attref{charm} for a few hours, and loses 2 \attref{will} for the same duration.
	A second dose causes agitation and paranoia, instead reducing \attref{charm} by 1.
	Further doses cause the drinker to overheat, suffering heat stroke, and may lead to internal bleeding and organ failure.
}

\feat{Stimulant}{mental-stimulant}{25}{
	None
}{
	\mix{cauldron}{drink}{Ants, vinegar}
	%Ants have a gland which can produce phenylacetic acid.
	%This can be converted to phenylacetone using acetic anhydride, which can be derived from acetic acid.
	%Acetic acid is the main component of vinegar (beside water). Vinegar is simply fermented from ethanol.
	%Phenylacetone can be converted to amphetamine using the Leuckart reaction and formic acid (also from ants).
	%Amphetamine is a widely-known nootropic (cognitive enhancer).
	
	The drinker gains 1 \attref{wit}, \attref{will} and \attref{heed} for about an hour.
	The potion also staves off tiredness for the duration. %TODO: A mechanical effect for this?
	After the potion wears off, you pay the price of your temporarily enhanced performance.
	You suffer a \negative{1} penalty to all rolls for the next 24 hours.
	Additional doses within this period are ineffective.
	
	Being under the effect of two doses simultaneously causes headaches that counteract the increased attributes.
	Further doses can cause bleeding into the brain and death.
}

\feat{Stimulant Dragging}{mental-stimulant-extension}{15}{
	\skillref[1]{brewing},
	\featref{mental-stimulant}
}{
	A slight change to the formula of your \featref{mental-stimulant} allows its effect to be extended by additional doses.
	Drinking another dose as one begins wearing off extends the effect and staves off the withdrawal.
	However, the body can only sustain such enhanced performance for so long.
	Whenever you take a dose after the first, make a \attref{might} Test.
	The TN is 9 for the second dose, and increases by 3 for every subsequent dose.
	On a failure, you pass out, gain no benefit from the additional dose, and the withdrawal effects kick in.
	You cannot be roused for several minutes.
}

\feat{Hysterical Strength}{physical-stimulant}{15}{
	None
}{
	A person's muscles are stronger than they normally get to use, strong enough to break their own bones.
	There's a good reason you don't get to use the full strength, you see.
	But you've figured out how to unlock that extra potential, and damn the consequences!
	
	\mix{cauldron}{drink}{\herbref[joint pine]{3}}
	%Joint pine is Ephedra, which contains ephedrine.
	%Ephedrine is similar in structure and function to epinephrine (adrenaline).
	%Adrenaline is oft-blamed for hysterical strength.
	
	The drinker gains 1 \attref{might} and 1 \attref{grace} for a few minutes.
	For the duration, any strenuous activity causes the drinker to suffer a \dice{2} \seclink{Damage Test}{damage-tests}.
	Strenuous activity includes the \actionref{dash} and \actionref{attack} Actions, any Test using \attref{might} or \attref{grace} (excluding \seclink{Damage Tests}{damage-tests} as part of the \actionref{attack} Action), and other things at the GM's discretion.
	
	Being under the effect of two doses simultaneously does increase \attref{might} and \attref{grace} further, but causes \seclink{Damage Tests}{damage-tests} as a result of any movement at all; only lying still is safe.
	Further doses cause seizures, triggering the \seclink{Damage Tests}{damage-tests} themselves.
}

\feat{Healing Salves}{brewing-healing}{15}{
	\skillref[1]{brewing}
}{
	You know a wide range of minor poultices, salves and remedies for cuts, bruises and other physical injuries.
	As long as you have access to a reasonable supply of various \herbrefplural{2}, and time to chew up poultices, you may use your \skillref{brewing} skill in place of your \skillref{healing} skill on Tests to heal people and creatures of most physical injuries.
	Setting broken bones and performing surgery still requires \skillref{healing}.
	
	Similarly, you may use your \skillref{brewing} rank in place of your \skillref{healing} rank when determining the {\damage} healed by a patient during a day of rest.
}
