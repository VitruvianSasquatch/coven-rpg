\chapter{Brewing}
\chaplabel{brewing}

\section{Creation and Application}

\subsection{Brewing and Chewing}

\subsection{Method of Delivery}

\section{Feats}

\feat{Numbing Painkiller}{painkiller-grace}{15}{
	None
}{
	\materials{\herbref[willow bark]{2}}
	
	The drinker may ignore 1 point of \secrefraw{damage} for a few hours, but loses one point of \attref{grace} for the same duration.
	Two doses may be effective simultaneously.
	Further doses cause paralysis, and possibly organ failure.
}

\feat{Dimming Painkiller}{painkiller-wit}{15}{
	None
}{
	\materials{\herbref[poppy seed]{2}}
	
	The drinker may ignore 1 point of \secrefraw{damage} for a few hours, but loses one point of \attref{wit} for the same duration.
	Two doses may be effective simultaneously.
	Further doses cause unconsciousness, and possibly cessation of breathing.
}

\feat{The Hard Stuff}{brewing-booze}{20}{
	None
}{
	You know how to make a drink that'll really put hairs on a man's chest.
	Or a woman's, at that.
	This potion is made in a still, instead of a cauldron.
	
	\materials{Alcohol, \herbref[apple]{2}}
	
	The drinker gains 1 \attref{might} for a few hours, and loses 2 \attref{wit} for the same duration.
	A second dose will render the drinker unconscious, and further doses are dangerously poisonous.
}

\feat{Healing Salves}{brewing-healing}{15}{
	\skillref[1]{brewing}
}{
	You know a wide range of minor poultices, salves and remedies for cuts, bruises and other physical injuries.
	As long as you have access to a reasonable supply of various \herbrefplural{2}, and time to chew up poultices, you may use your \skillref{brewing} skill in place of your \skillref{healing} skill on Tests to heal people and creatures of most physical injuries.
	Setting broken bones and performing surgery still requires \skillref{healing}.
	
	Similarly, you may use your \skillref{brewing} rank in place of your \skillref{healing} rank when determining the \secrefraw{damage} healed by a patient during a day of rest.
}
