\chapter{Willing}
\chaplabel{willing}

\discref{willing} is the most raw and versatile application of a witch's magic.
Known to many layfolk as sorcery or spellcraft, it is the art of making something true simply by willing it hard enough.
Most \discref{willing} is performed without any of the accoutrements that accompany other forms of magic, and it doesn't follow the prescribed formulae of rites and brews.
This makes it the weakest form of magic in some ways, but its flexibility and ease of access more than make up for it.
So much so that every witch knows at least the basics.

Like any witchcraft, \discref{willing} is something anyone can do if they know how.
But there is a knack to it.
It requires that the witch not only \emph{want} something to be the case, but \emph{believe} that it already is.
That she outright refuses to accept any possibility that it might not, in fact, be the case.
It involves willfully deceiving not only oneself, but also the very universe.
Most people would never even think to try it, but it is among the first things any aspiring witch must learn.

The line between \discref{willing} and \discref{headology} can be a little blurred, at times.
Both have the ability to make things true by making people believe them.
Many Willers say that the difference is that \discref{willing} affects the real world, while \discref{headology} only affects other people's minds.
The Headologists point out that other people are just as much a part of the real world as any old rock is.
Some Headologists say that the difference is that \discref{headology} is about convincing other people, while \discref{willing} is about convincing yourself.
The Willers point out that it's about more than convincing yourself, it's about convincing the world.
And that includes other people.
A few say that there's no real difference at all, that it's just two ways of thinking about the same thing.
These tend to be the witches who are obnoxiously good at both, and everyone else pointedly ignores them.

One interesting property of \discref{willing} is that it cannot affect other people or animals.
It takes more than force of will to convince someone that they're a different shape; usually this entails talking to them.
This doesn't stop people getting knocked off their feet by a gust of wind, or crushed by a falling tree, however.
Witches interested in affecting people more directly are encouraged to pursue \discref{headology}.
Or swordplay.

Unlike many magical disciplines, which depend upon \attref{wit} for understanding or memorising their complexities, \discref{willing} depends upon raw \attref{will}, your own stubbornness and conviction against the fabric of reality.

\section{Feats}

\feat{Basic Willing}{willing-basic}{10}{
	None
}{
	You can perform very basic acts of \discref{willing} upon things you can touch, given a bit of time to focus your mind and an obvious physical cue.
	Examples include:
	\begin{itemize}
		\item Lighting kindling or a candle without a spark, by cupping your hands around it and blowing on it.
		%\item Colouring or mildly flavouring a small pot of water by stirring it.
		\item Scratching writing into stone using just a fingernail.
		\item Rubbing stains out of clothing using your bare hands.
		\item Combing your hair with just your fingers.
	\end{itemize}
	The amount of time required to produce an effect varies depending on the desired outcome, but should be more than an Action without a Test.
	This ability cannot produce a lasting effect by itself.
	You can light a fire, because that sustains itself once ignited, but you cannot create, destroy or melt a pebble.
}

\feat{Kindling}{willing-fire}{15}{
	\featref{willing-basic}
}{
	You've practiced \discref{willing} a fire to life, and it's getting a lot easier for you.
	You can now ignite a fire within a dozen metres as an Action, with nothing more than a quick glare.
	You no longer require kindling, but still need something a fire can catch on fairly easily, such as twigs, cloth or dry leaves.
	Lighting a log or floorboards is still beyond you.
	
	The fire begins small, so will be extinguished by rain or a moderate wind before it can catch.
	A person walking about or wriggling will automatically foil an attempt to ignite their clothes (perhaps without noticing), but a person sitting fairly still may not.
}

\feat{A Tool for the Job}{willing-tools-improvise}{20}{
	\featref{willing-basic}
}{
	Sometimes, the easiest way to convince someone of something is the hit them with a big stick until they agree with you.
	The world itself is no different.
	You've learned to make \discref{willing} easier using physical tools, even if they aren't the \emph{right} tools.
	
	Most simply, this means axes and knives cut just as well as ever in your hands, even if they've lost their edge.
	But you can take it even further, cutting carrots or trees with nothing more than an appropriately shaped stick.
	You can make any similarly-shaped object behave as the appropriate tool for a job.
	For a worse approximation, this may require a Test.
	A solid branch with a flat, sort of axe head shaped bit on the end will do a fine job of cutting down a tree.
	A solid branch without such an attachment would require a Test.
	A limp reed is going to be a real stretch.
	
	Such tools still obey the usual rules of \discref{willing}, and are of no additional use as weapons against people and animals.
	See \featref{headology-weapons-improvise} if you want weapons too.
}

\feat{A Hefty Tool}{willing-tools-effective}{15}{
	\skillref[1]{willing},
	\featref{willing-basic}
}{
	You can make an appropriate tool more effective when you use it.
	Or an inappropriate tool, with \featref{willing-tools-improvise}.
	
	Tools are several times more effective when you use them.
	For example, when you use a spade it always lifts clumps of dirt several times the size of the spade's head.
	You can bring down a tree that you can barely wrap your arms around with only 4 or 5 swings of an axe.
	You can bail out a rowboat with only a few scoops of a bucket.
	
	This only works as long as you are still using the tool.
	For instance, you cannot store any more water than normal in a bucket unless you are carrying it.
}

\feat{Bubbling Brook}{willing-water}{10}{
	\featref{willing-basic}
}{
	Water is considered by many to be an element of change.
	You've certainly figured out how to change it.
	While touching water, you can move it around with your mind.
	You can make it flow, swirl, form into fairly elaborate shapes, or even float into the air.
	
	You can only affect the water while it remains one continguous mass, which you must be touching.
	Afterwards, it flows normally again.
	You can only affect a couple of buckets-full at a time, and can't stretch it out over more than a couple of metres.
	You also can't move the water fast enough to hurt anybody.
	You can move other liquids if they are primarily water, such as wine, blood or most potions.
	As always with \discref{willing}, you cannot affect liquids inside a living person.
}

\feat{Water Walk}{water-walk}{20}{
	\skillref[1]{willing},
	\featref{willing-water}
}{
	You can walk on water, or any other liquid you could affect with \featref{willing-water}.
	This takes great concentration, and you cannot take an Action and move on the water's surface in the same Turn.
	You may take an Action if you stand still on the water, however.
	
	If the water is flowing, you will be carried with it.
	Staying upright on fast flowing or turbulent water may require a Test, and the effect requires you to stay on your feet; falling prone will cause you to fall into the water.
	You may take use an entire Turn to clamber onto the water, if you are swimming at the surface.
}

\feat{River Run}{water-walk-2}{15}{
	\skillref[2]{willing},
	\featref{water-walk}
}{
	Walking on water has become second nature to you.
	You may take Actions while moving.
	Additionally, flow and turbulence pose you no threat.
	You may treat water you are standing on as though it were not flowing, and you can remain on the water's surface even when prone.
	Lastly, climbing upright onto the water while swimming at the surface is treated as though you are merely standing from being prone.
}

