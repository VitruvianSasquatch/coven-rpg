\chapter{Familiars}
\chaplabel{familiars-expansion}

\section{Familiar Animals}

\familiar{Badger}{badger}{15}{
	\atttable{0}{1}{1}{1}{3}{1}{\negative 1}{1}
}{
	\speed{8}
}{
	\skillrefspeciality[2]{crafting}{Earthworks}, \skillref[1]{intimidation}
}{
	Badgers are striped, stocky, burrowing omnivores.
	They are normally docile, but have large claws, a strong bite, and a vicious streak when cornered.
	A badger's network of underground tunnels, called a sett, can stretch for a mile or more.
	Witches who take badgers for familiars are often reluctant to leave their territory, but will defend it to the death.
}{
	\ability{Bite \& Claws}{
		The badger rolls 4 dice for unarmed {\damagetests}.
	}
	
	\ability{Snuffling \& Rooting}{
		The badger rolls an extra die on \skillref{perception} Tests relying on smell.
		It rolls a second extra die if the Test is to detect something buried or underground.
	}
	
	\ability{Burrowing}{
		The badger can burrow through about a metre of loose earth in 2 minutes, or a metre of packed earth in 5 minutes.
		It leaves a tunnel behind it.
	}
	
	\familiaroption{Honey Badger}{10}{
		The honey badger rolls 5 dice for unarmed {\damagetests}.
		It increases its \statref{st} by 2.
	}
}

\familiar{Bees}{bees}{40}{
	\atttable{\negative 5}{1}{3}{\negative 1}{3}{1}{\negative 1}{2}
}{
	\flyspeed{6}
}{
	\skillref[1]{botany}, \skillref[1]{flying}, \skillref[1]{weaponry}
}{
	Bees are more than just social creatures.
	A bee, such as it is, barely has a mind at all.
	Only the \emph{hive}, as a whole, can be considered to have a real mind.
	As such, a witch does not take a bee as a familiar, but a hive.
	
	Taming a hive is no easy task, and the ritual to bind one as a familiar is further complicated by the distributed mind.
	Such a binding is an impressive feat, and a witch who has managed it can command a lot of respect from those who recognise this.
}{
	\ability{Hive}{
		While the hive stands and bees reside within, the swarm is not dead.
		Swarms of bees can leave the hive, though they cannot be away from the hive for more than a day.
		The loss of the swarm does not kill the hive, although a hive that loses many swarms in quick succession will not be able to provide more.
	}
	
	\ability{Swarm}{
		Being composed of many individuals, a swarm does not suffer injury in the same way most creatures.
		It is not subject to {\shock}, and is destroyed only when it has taken 15 {\damage}.
		However, the swarm grows depleted as is loses bees, and suffers a \negative{1} penalty to all Tests (including {\damagetests}) for every 3 {\damage} it has taken.
		A swarm can be healed only by being replenished from the hive.
	}
	
	\ability{Sting}{
		A swarm rolls 4 dice for unarmed {\damagetests}, which are not affected by the swarm's \attref{might}.
		This {\damage} is dealt by injected venom.
		A successful attack by the swarm also deals 3 {\damage} to the swarm itself, as bees are killed by stinging.
	}
}

\familiar{Dragonfly/Damselfly}{dragonfly}{10}{
	\atttable{\negative 5}{3}{0}{0}{1}{1}{0}{\negative 1}
}{
	\flyspeed{15}
}{
	\skillref[2]{flying}, \skillref[1]{stealth}, \skillref[2]{weaponry}
}{
	Dragonflies---and their cousins the damselflies---live for several years under the water, before they grow their wings and emerge.
	This adult stage only lasts a few months, and it is in this time that a witch must capture it and bind it as a familiar.
	It is a quick and nimble flier, and a voracious predator{\dots} of other insects.
}{
	\ability{Tiny Predator}{
		The dragonfly cannot effectively attack anything much larger than itself, but rolls 4 dice for unarmed {\damagetests} when picking on something its own size.
	}
	
	\ability{Dartwing}{
		The dragonfly's \statref{dr} is increased by 2.
	}
}

\familiar{Duck}{duck}{5}{
	\atttable{\negative 1}{1}{1}{2}{2}{1}{1}{0}
}{
	\speed{4}, \swimspeed{4}, \flyspeed{15}
}{
	\skillref[1]{flying}
}{
	Due to a duckling's tendency to imprint on humans, it is really quite easy for a witch to tame one and bind it as a familiar.
	Some ducklings find themselves as familiars within a day of hatching.
}{
	\ability{Waterfowl}{
		The duck can not only swim, but can take off from the water's surface.
	}
}

\familiar{Serpent}{serpent}{15}{
	\atttable{\negative 1}{1}{1}{2}{2}{2}{2}{1}
}{
	\speed{6}
}{
	\skillref[2]{deception}, \skillref[1]{intimidation}, \skillref[1]{perception}, \skillref[1]{weaponry}
}{
	Silver-tongued, slithering and and sly, a serpent is a favourite among some of the nastier witches.
}{
	\ability{Forked Tongue}{
		The serpent rolls an extra die on \skillref{perception} Tests relying on smell or taste.
	}
	
	\ability{Bite}{
		The serpent rolls 3 dice for unarmed {\damagetests}.
	}
	
	\familiaroption{Viper}{10}{
		The serpent can inject venom with its bite.
		If it does so, it rolls 5 dice for the {\damagetest}.
		The target must succeed on a {\tn} 15 \attref{might} Test or suffer paralysis over the next 5 minutes.
		Death often ensues, without medical attention.
	}
	
	\familiaroption{Constrictor}{10}{
		The serpent gains a \attref{might} score of 1 (instead of \negative{1}), \skillref[1]{athletics}, and rolls an extra die on Tests to entangle or restrain creatures.
		It rolls 4 dice for unarmed {\damagetests}, and may roll such a {\damagetest} without first making a Test to hit when it makes an \actionref{attack} against a target it has entangled.
		%TODO: Tie this into grappling rules, if they exist?
	}
}
