\chapter{Familiars}
\chaplabel{familiars-expansion}

\section{Familiar Animals}

\familiar{Bees}{bees}{40}{
	\atttable{\negative 5}{1}{3}{\negative 1}{3}{1}{\negative 1}{2}
}{
	\skillref[1]{botany}, \skillref[1]{flying}, \skillref[1]{weaponry}
}{
	Bees are more than just social creatures.
	A bee, such as it is, barely has a mind at all.
	Only the \emph{hive}, considered as a whole, can be considered to have a real mind.
	As such, a witch does not take a bee as a familiar, but a hive.
	
	Taming a hive is no easy task, and the ritual to bind one as a familiar is further complicated by the distributed mind.
	Such a binding is an impressive feat, and a hive familiar can command a witch some respect from those who recognise this.
}{
	\familiarability{Hive}{
		While the hive stands and bees reside within, the swarm is not dead.
		Swarms of bees can leave the hive, though they cannot be away from the hive for more than a day.
		The loss of the swarm does not kill the hive, although a hive that loses many swarms in quick succession will not be able to provide more.
	}
	
	\familiarability{Swarm}{
		Being composed of many individuals, a swarm does not take damage like most creatures.
		It is not subject to Shock, and is destroyed only when it has taken 15 damage.
		However, the swarm grows depleted as is loses bees, and suffers a \negative{1} penalty to all rolls for every 3 damage it has taken.
		A swarm can be healed only by being replenished from the hive.
	}
	
	\familiarability{Sting}{
		A swarm's unarmed attacks deal 4 dice of damage, and are not affected by the swarm's \attref{might}.
		This damage is dealt by injected venom.
		A successful attack by the swarm also deals 3 damage to the swarm itself, as bees are killed by stinging.
	}
}{}

\familiar{Duck}{duck}{5}{
	\atttable{\negative 1}{1}{1}{2}{2}{1}{1}{0}
}{
	\skillref[1]{flying}
}{
	Due to a duckling's tendency to imprint on humans, it is really quite easy for a witch to tame one and bind it as a familiar.
	Some ducklings find themselves as familiars within a day of hatching.
}{
	\familiarability{Waterfowl}{
		The duck can not only swim, but can take off from the water's surface.
	}
}{}
