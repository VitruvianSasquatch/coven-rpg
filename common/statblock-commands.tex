\newcommand\atttable[8]{%Arguments: might, grace, ken, wit, will, heed, charm, presence
	\begin{center}
		\begin{tabu}{X[c]X[c]X[c]X[c]}
			\toprule
			
			\attref{might} &
			\attref{ken} &
			\attref{will} &
			\attref{charm} \\
			
			#1 & #3 & #5 & #7 \\
			
			\midrule
			
			\attref{grace} &
			\attref{wit} &
			\attref{heed} &
			\attref{presence} \\
			
			#2 & #4 & #6 & #8 \\
			
			\bottomrule
		\end{tabu}
	\end{center}
}

\makeatletter
%Use the relay method to get more than 9 arguments.
%As described at https://texfaq.org/FAQ-moren9
%All statblock commands use the same latter half of the relay.
\newcommand\statblock[4][\@nil]{%Arguments: displaytitle (optional), title, labelprefix, labelsuffix
	\def\displaytitle{#1}
	\ifx\displaytitle\@nnil
		\def\displaytitle{#2}
	\fi
	\def\title{#2}
	\def\labelprefix{#3}
	\def\labelsuffix{#4}
	\statblockrelay
}
\newcommand\statblockrelay[5]{%Arguments: atttable, speed, skills, text, abilities
	\subsection[{\title}]{\displaytitle}
	\labelname{\title}
	\label{\labelprefix:\labelsuffix}
	\labelname{\title's}
	\label{\labelprefix possessive:\labelsuffix}
	{#1}
	
	\textbf{Speed:} {#2}
	
	\nopagebreak
	\textbf{Skills:} {#3}
	
	\nopagebreak
	{#4}
	
	{#5}
}

\newcommand\creature[3][\@nil]{%Arguments: displaytitle (optional), title, label
	\statblock[#1]{#2}{creature}{#3}
}

\newcommand\familiar[4][\@nil]{%Arguments: displaytitle, title, label, XP cost
	%TeX complains if we use \displaytitle for this, and then pass it to \statblock's \displaytitle.
	%So we use \familiardisplaytitle here.
	\def\familiardisplaytitle{#1}
	\ifx\familiardisplaytitle\@nnil%
		\def\familiardisplaytitle{#2}%
	\fi%
	\statblock[{\familiardisplaytitle} {[#4 XP]}]{#2}{familiar}{#3}
}
\makeatother

\newcommand\speed[1]{%
	#1%
}
\newcommand\flyspeed[1]{%
	#1 flying%
}
\newcommand\swimspeed[1]{%
	#1 swimming%
}
\newcommand\ability[2]{%Arguments: title, text
	\textbf{#1:} {#2}%
}
\newcommand\familiaroption[3]{%Arguments: title, XP cost, text
	\ability{{#1} [#2 XP]}{#3}
}

\newcommand\creatureref[1]{\nameref{creature:#1}}
\newcommand\creaturerefpossessive[1]{\nameref{creaturepossessive:#1}}
\newcommand\familiarref[1]{\nameref{familiar:#1}}
\newcommand\familiarrefpossessive[1]{\nameref{familiarpossessive:#1}}
