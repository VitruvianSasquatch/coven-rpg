\usepackage{hyperref}
\usepackage{nameref}

\usepackage{booktabs}
\usepackage{tabu}
\tabulinesep=1.2mm %Space tables more nicely.
\usepackage[table]{xcolor}
\newenvironment{simpletable}[1]{
	\begin{center}
	\taburowcolors 2{white .. gray!20}
	\begin{tabu}{#1}
}{
	\end{tabu}
	\end{center}
}

\makeatletter
\newcommand{\labelname}[1]{\def\@currentlabelname{#1}}%Changes the name used by nameref on the next label.
\makeatother

\newcommand\partlabel[1]{\label{part:#1}}
\newcommand\chaplabel[1]{\label{chap:#1}}
\newcommand\seclabel[1]{\label{sec:#1}}
\newcommand\partref[1]{Part~\ref{part:#1}}
\newcommand\chapref[1]{Chapter~\ref{chap:#1}}
\newcommand\chapnameref[1]{\nameref{chap:#1}}
\newcommand\secref[1]{the \nameref{sec:#1} section}
\newcommand\secrefraw[1]{\nameref{sec:#1}}
\newcommand\seclink[2]{\hyperref[sec:#2]{#1}}

\newcommand\discref[1]{\chapnameref{#1}}

\newcommand\feat[5]{%Arguments: title, label, XP cost, prerequisites, text.
	\subsection{#1 [#3 XP]}
	\labelname{#1}
	\label{feat:#2}
	\textbf{Prerequisites:} {#4}
	
	{#5}
} %TODO: Tags such as 'first circle'?
\newcommand\featref[1]{\nameref{feat:#1}}
%TODO: Make feat references add the chapter in brackets afterwards, if the feat is in a different chapter.

\makeatletter
\newcommand\skill[3][\@nil]{%Takes three arguments, the first of which is optional and defaults to nothing.
	\def\govdisc{#1}
	\subsubsection{#2}%
	\label{skill:#3}%
	\ifx\govdisc\@nnil%
		%
	\else%
		Governing discipline: \discref{\govdisc}%TODO: This does really nasty things to the paragraphing.
	\fi%
}
\makeatother


\newcommand\skillref[2][0]{%Takes two arguments, the first of which is optional and defaults to zero.
	\nameref{skill:#2}%
	\ifnum #1=0%
		%
	\else%
		~#1%
	\fi%
}
\newcommand\skillrefspeciality[3][0]{%Takes three arguments, the first of which is optional and defaults to zero.
	\nameref{skill:#2} (#3)%
	\ifnum #1=0%
		%
	\else%
		~#1%
	\fi%
}

\newcommand\attribute[2]{%
	\subsection{#1}
	\label{attribute:#2}
}

\newcommand\attref[1]{%
	\nameref{attribute:#1}%
}

\newcommand\testtype[2]{%
	$\text{\skillref{#2}} + \text{\attref{#1}}$%
}

\newcommand\statlabel[3]{%For derived statistics. Arguments: title, plural title, label
	\labelname{#1}%
	\label{stat:#3}%
	\labelname{#2}%
	\label{stats:#3}%
}
\newcommand\statref[1]{%
	\nameref{stat:#1}%
}
\newcommand\statrefplural[1]{%
	\nameref{stats:#1}%
}


\usepackage{amsfonts} %Gives the stuff we use to build \shortminus
\DeclareMathSymbol{\shortminus}{\mathbin}{AMSa}{"39}

\usepackage{xstring} %Gives \StrDel
\newcommand\dice[2][0]{%Takes two arguments, the first of which is optional and defaults to zero.
	#2d%
	\ifnum #1=0%
		%
	\else%
		\ifnum #1>0%
			$+$%
		\else%
			$\shortminus$%
		\fi%
		\StrDel{#1}{-}%Strips the minus from it, essentially giving absolute value.
	\fi%
}

\newcommand\negative[1]{$\shortminus$#1}
\newcommand\positive[1]{$+$#1}

\newcommand\titleemph[1]{\emph{#1}} %For the titles of books and such, including Coven itself.

\newcommand\storybreak{\bigskip}

\newcommand\atttable[8]{%Arguments: might, grace, ken, wit, will, heed, charm, presence
	\begin{center}
		\begin{tabu}{X[c]X[c]X[c]X[c]}
			\toprule
			
			\attref{might} &
			\attref{ken} &
			\attref{will} &
			\attref{charm} \\
			
			#1 & #3 & #5 & #7 \\
			
			\midrule
			
			\attref{grace} &
			\attref{wit} &
			\attref{heed} &
			\attref{presence} \\
			
			#2 & #4 & #6 & #8 \\
			
			\bottomrule
		\end{tabu}
	\end{center}
}

\newcommand\familiar[8]{%Arguments: title, label, XP cost, atttable, skills, text, abilities, options
	\subsection{#1 [#3 XP]}
	\labelname{#1}
	\label{familiar:#2}
	{#4}
	
	\textbf{Skills:} {#5}
	
	{#6}
	
	{#7}
	
	{#8}
}

\newcommand\familiarability[2]{%Arguments: title, text
	\textbf{#1:} {#2}
}

\newcommand\familiaroption[3]{%Arguments: title, XP cost, text
	\familiarability{{#1} [#2 XP]}{#3}
}

\newcommand\materials[1]{%
	\textbf{Materials:} {#1}.%
}

\newcommand\herblabel[2]{%Arguments: title, label
	\labelname{#1 Herb}%
	\label{herb:#2}%
	\labelname{#1 Herbs}%
	\label{herbs:#2}%
}

\newcommand\herbtype[1]{%A herb type (just rarity)
	\nameref{herb:#1}%
}
\newcommand\herbtypeplural[1]{%
	\nameref{herbs:#1}%
}

\makeatletter
\newcommand\herb[3][\@nil]{%A herb, with identity and type
	\def\akaname{#1}%Optional argument is an alternative name for the herb.
	#2 (%
	\ifx\akaname\@nnil%
		%
	\else%
		a.k.a.\ \akaname,\ 
	\fi%
	\herbtype{#3})%
}
\newcommand\herbs[2]{%
	#1 (\herbtypeplural{#2})%
}
\newcommand\Herb[3][\@nil]{%
	\def\akaname{#1}%
	\expandafter\MakeUppercase #2 (%
	\ifx\akaname\@nnil%
		%
	\else%
		a.k.a.\ \akaname,\ 
	\fi%
	\herbtype{#3})%
}
\newcommand\Herbs[2]{%
	\expandafter\MakeUppercase #1 (\herbtypeplural{#2})%
}
\makeatother


\newcommand\circlelabel[2]{%Arguments: title, label
	\labelname{#1 Ritual Circle}%
	\label{circle:#2}%
	\labelname{#1 Ritual Circles}%
	\label{circles:#2}%
}
\newcommand\circleref[1]{\nameref{circle:#1}}
\newcommand\circlerefplural[1]{\nameref{circles:#1}}

\newcommand\materiallabel[3]{%Arguments: title, plural title, label
	\labelname{#1}%
	\label{material:#3}%
	\labelname{#2}%
	\label{materials:#3}%
}
\newcommand\materialref[1]{\nameref{material:#1}}
\newcommand\materialrefplural[1]{\nameref{materials:#1}}

\newcommand\actionlabel[1]{\label{action:#1}}
\newcommand\actionref[1]{\nameref{action:#1}}

\newcommand\undeadlabel[3]{%Arguments: title, plural title, label
	\labelname{#1}%
	\label{undead:#3}%
	\labelname{#2}%
	\label{undeads:#3}%
}
\newcommand\undeadref[1]{\nameref{undead:#1}}
\newcommand\undeadrefplural[1]{\nameref{undeads:#1}}

\newcommand\mixcreation[2]{%Arguments: name, label
	\subsubsection{#1}%
	\label{mixcreation:#2}%
}
\newcommand\mixdelivery[2]{%Arguments: name, label
	\subsubsection{#1}%
	\label{mixdelivery:#2}%
}

\newcommand\mix[3]{%Arguments: creation, delivery, ingedients
	\textbf{\nameref{mixdelivery:#2}}, \textbf{\nameref{mixcreation:#1}}, \materials{#3}
}

%%%%%%%%%%%%%%%%%%%%%%%%%%%%%%
%Argument-less seclink macros.
%%%%%%%%%%%%%%%%%%%%%%%%%%%%%%
\newcommand\round{\seclink{Round}{structured-time}}
\newcommand\rounds{\seclink{Rounds}{structured-time}}
\newcommand\turn{\seclink{Turn}{structured-time}}
\newcommand\turns{\seclink{Turns}{structured-time}}
\newcommand\action{\seclink{Action}{structured-time}}
\newcommand\actions{\seclink{Actions}{structured-time}}

\newcommand\damage{\seclink{Damage}{damage}}
\newcommand\shock{\seclink{Shock}{shock}}
\newcommand\shocked{\seclink{Shocked}{shock}}
\newcommand\exhaustion{\seclink{Exhaustion}{exhaustion}}

\newcommand\diffterrain{\seclink{Difficult Terrain}{difficult-terrain}}

\renewcommand\symbol{\seclink{Symbol}{sympathetic-links}}
\newcommand\symbols{\seclink{Symbols}{sympathetic-links}}
\newcommand\symlink{\seclink{Sympathetic Link}{sympathetic-links}}
\newcommand\symlinks{\seclink{Sympathetic Links}{sympathetic-links}}
