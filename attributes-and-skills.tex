\chapter{Attributes and Skills}
\chaplabel{attributes-and-skills}

\section{Attributes}
\seclabel{attributes}

Attributes are a character's broad, innate capabilities.
They represent physical capacity and natural talent.
That is not to say they can't be improved; one can grow muscle through exercise and the brain is no different; but such improvement represents a more significant investment than picking up a new skill.
A character has six attributes: Might, Grace, Wit, Will, Charm and Presence.
For human characters, these range from 0 to 5, with 2 as the average for a human.
Non-human characters may have attributes outside this range.

A summary of these attributes is provided below, along with examples of using the attribute.
Note that many of the example Tests would be accompanied by an appropriate skill.

\subsection{Might}

Might represents physical strength, endurance and resilience.
It's used to lift things, smash things, resist diseases and endure hard labour, to put the hurt on people and to resist having the hurt put back on you.
Might is the attribute you use when rolling damage with melee weapons, and also determines the amount of damage required to put you down.
It can also prove useful when a brewer or botanist feeds you something you shouldn't have eaten.
Lastly, powerful legs let you run faster.

%This table seems unhelpful, on second thoughts.
%Would such grandiose descriptions transfer accurately to fairly small range of numbers in the game mechanics?
%
%\begin{simpletable}{lX}
%	\toprule
%	0 & Total weakling: Carrying your shopping back from the market is a struggle.\\
%	1 & Below average: Arm-wrestling ends embarassing for you.\\
%	2 & Average: A day's farm-work is tiring, but doable.\\
%	3 & Above average: You can carry bricks all day everyday.\\
%	4 & Exceptional: You could be a blacksmith.\\
%	5 & Incredible: You could run around in plate armour for hours.\\
%	\bottomrule
%\end{simpletable}

\begin{simpletable}{rX}
	\toprule
	TN & Example Task\\
	\midrule
	9 & Jumping across a \SI{3}{\metre} gap.\\
	12 & \\
	15 & \\
	18 & \\
	21 & \\
	\bottomrule
\end{simpletable}

\subsection{Grace}

Grace represents agility, dexterity and reflexes.
It's used to dodge swords, manoeuvre broomsticks, do backflips, dance waltzes, and hastily scratch runic circles into the floor without smudging them and letting the demons in.
Grace determines how hard you are to hit with a weapon or a fist and also contributes toward your speed.

%TODO: Table

\subsection{Wit}

Wit represents intelligence, memory and awareness.
It's used to recall knowledge, solve puzzles, spot lies, and ensure no detail escapes your notice.
Wit is the key attribute for many forms of magic, used to memorise and understand spells, recipes and rituals.

%TODO: Table.

\subsection{Will}

Will represents courage, dedication and conviction.
It's used to stand your ground, resist the influence of others, remain unfazed in embarassing situations, and push onwards in the face of adversity.
Will influences your pain threshold and is used to resist curses and mental influence, mundane or magical.

%TODO: Table.

\subsection{Charm}

Charm represents eloquence, wile and comeliness.
It's used to persuade, deceive or seduce people, to smarm your way into their good graces, and to imply things without outright saying them.
%TODO: Mechanical effects.

%TODO: Table.

\subsection{Presence}

Presence represents force of personality, air of authority and personal magnetism.
It's used to draw people's attention, boss them around, and make them wet themselves in terror.
%TODO: Mechanical effects.

\begin{simpletable}{rX}
	\toprule
	TN & Example Task\\
	\midrule
	9 & \\
	12 & \\
	15 & \\
	18 & \\
	21 & Silencing a raucous town hall with a polite cough.\\
	\bottomrule
\end{simpletable}

\section{Skills}

%TODO: Intro fluff paragraph about the skill of a witch.

%TODO: Recap how skills affect dice rolled for Tests, and how not every Test has an applicable skill.

A witch's skills can be divided into two categories.
The first consists of general skills, pertaining to things any witch might find herself doing.
The second consists of a witch's skills in her particular disciplines of magic.
These skills are normally of little use to a witch who does not practice such a discipline, although they can often be used to identify, and sometimes to counteract, the effects from it.

A list of the skills available to a witch, and examples of their use, is provided below.
If a skill corresponds to a particular discipline of magic, it can be increased by feats found in the appropriate chapter of \partref{disciplines}.
Otherwise, it can be increased by general feats. %TODO: Where to find these.

\newcommand\govdisc[1]{Governing discipline: #1} %TODO: Make these link to the relevant chapter, and insert the title of that chapter automatically.

\subsection{Botany}
\skilllabel{botany}

Used to raise crops and herbs in a witch's garden, find them out in the forest, or identify a fishy-looking leaf.

\subsection{Brewing}
\skilllabel{brewing}
\govdisc{Brewing}

Used to brew tinctures, tonics, elixirs and other potions.
This doesn't always require a cauldron: it also covers mixing poultices and the like.
Of course, you can also make booze.

\subsection{Crafting}
\skilllabel{crafting}

Used to make things, quite broadly.
Crafting covers the creation of any most kinds of objects, and is mundane skill much in use by non-witches.
Some kinds of crafting are still covered by other skills, such as \skillref{brewing}.

Crafting is unlike other skills in that a craftswoman must always specialise further.
Whenever you gain a rank of Crafting, select a specialty.
A list of specialties is provided below, and the GM may approve others.
Ranks for each specialty are tracked independently.
For example, a witch might have \skillrefspecialty[1]{crafting}{Carpentry}, \skillrefspecialty[2]{crafting}{Cooking} and \skillrefspecialty[1]{crafting}{Smithing}.

\begin{itemize}
	\item Carpentry
	\item Cooking
	\item Masonry
	\item Pottery
	\item Seamstressing
	\item Smithing
	\item Woodcarving
	%TODO: Evaluate and maybe expand.
\end{itemize}

\subsection{Deception}
\skilllabel{deception}

Many witches make it a rule not to lie.
That doesn't mean they always need to tell the whole truth, so this can still be a useful skill for them.

\subsection{Flying}
\skilllabel{flying}
\govdisc{Broomcraft}

Used by a witch on a broomstick, whether she's settling in for a cross-country flight, showing off with a barrel roll, or pulling a stalled stick out of a deep dive.

\subsection{Necromancy}
\skilllabel{necromancy}
\govdisc{Necromancy}

Used to pervert the natural order and bring the dead back to life.
Also used to send them on again, if hitting them over the head with a big stick won't suffice.

\subsection{Perception}
\skilllabel{perception}

Used by the uninitiated to see their present surroundings: to spot things out of place or to pick out details at a distance.
Also used to see the past and future, and places many miles away, for those who know how.

\begin{simpletable}{rX}
	\toprule
	TN & Example Task\\
	\midrule
	9 & Discerning the power of the necromancy animating a shambling corpse.\\
	12 & Identifying the purpose of a necromantic rite from the chalk circle left behind.\\
	15 & Filtering the true facts about vampires from the baseless rumours that surround them.\\
	18 & Discerning the power of the necromancy that previous animated a no-longer-shambling corpse.\\
	21 & Performing a complex necromantic ritual using nothing but two small sticks and a fresh egg.\\
	\bottomrule
\end{simpletable}

\subsection{Sympathetic Magic}
\skilllabel{sympathetic-magic}
\govdisc{Sympathetic Magic}

Used to manipulate people or things using effigies, poppets or other imitative talismans.
The idea of sympathetic magic is that one can't affect the imitation of a thing without affecting the thing itself.
%TODO: Used in crafting effigies?

\subsection{Weaponry}
\skilllabel{weaponry}

Used for everything from stabbing people with a concealed knife to clonking them over the head with a hefty staff, or even slugging them with a mean right hook.
Also used for chucking things and shooting them with a bow.

\subsection{Willing}
\skilllabel{willing}
\govdisc{Willing}

Used to force the universe to fall into line with what you know is true.
There is a real knack to convincing yourself of something well enough to make this work, and this skill governs how good you are at it.
