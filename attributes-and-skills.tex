\chapter{Attributes and Skills}
\chaplabel{attributes-and-skills}

\section{Attributes}
\seclabel{attributes}

Attributes are a character's broad, innate capabilities.
They represent physical capacity and natural talent.
That is not to say they can't be improved; one can grow muscle through exercise and the brain is no different; but such improvement represents a more significant investment than picking up a new skill.
A character has six attributes: \attref{might}, \attref{grace}, \attref{wit}, \attref{will}, \attref{charm} and \attref{presence}.
For human characters, these range from 0 to 5, with 2 as the average for a human.
Non-human characters may have attributes outside this range.

A summary of these attributes is provided below, along with examples of using the attribute.
Note that many of the example Tests would be accompanied by an appropriate skill.

\attribute{Might}{might}

\attref{might} represents physical strength, endurance and resilience.
It's used to lift things, smash things, resist diseases and endure hard labour, to put the hurt on people and to resist having the hurt put back on you.
\attref{might} is the attribute you use when rolling damage with melee weapons, and also determines the amount of damage required to put you down.
It can also prove useful when a brewer or botanist feeds you something you shouldn't have eaten.
Lastly, powerful legs let you run faster.

%This table seems unhelpful, on second thoughts.
%Would such grandiose descriptions transfer accurately to fairly small range of numbers in the game mechanics?
%
%\begin{simpletable}{lX}
%	\toprule
%	0 & Total weakling: Carrying your shopping back from the market is a struggle.\\
%	1 & Below average: Arm-wrestling ends embarassing for you.\\
%	2 & Average: A day's farm-work is tiring, but doable.\\
%	3 & Above average: You can carry bricks all day everyday.\\
%	4 & Exceptional: You could be a blacksmith.\\
%	5 & Incredible: You could run around in plate armour for hours.\\
%	\bottomrule
%\end{simpletable}

\begin{simpletable}{rX}
	\toprule
	TN & Example Task\\
	\midrule
	9 & Jumping across a \SI{3}{\metre} gap.\\
	12 & \\
	15 & \\
	18 & \\
	21 & \\
	\bottomrule
\end{simpletable}

\attribute{Grace}{grace}

\attref{grace} represents agility, dexterity and reflexes.
It's used to dodge swords, manoeuvre broomsticks, do backflips, dance waltzes, and hastily scratch runic circles into the floor without smudging them and letting the demons in.
\attref{grace} determines how hard you are to hit with a weapon and also contributes toward your speed.

%TODO: Table

\attribute{Wit}{wit}

\attref{wit} represents intelligence, memory and awareness.
It's used to recall knowledge, solve puzzles, and ensure no detail escapes your notice.
\attref{wit} is the key attribute for many forms of magic, used to memorise and understand spells, recipes and rituals.

%TODO: Table.

\attribute{Will}{will}

\attref{will} represents courage, dedication and conviction.
It's used to stand your ground, resist the influence of others, remain unfazed in embarassing situations, and push onwards in the face of adversity.
\attref{will} influences your pain threshold and is used to resist curses and mental influence, mundane or magical.
Force of \attref{will} can also be applied to directly influence the world in some forms of magic.

%TODO: Table.

\attribute{Charm}{charm}

\attref{charm} represents eloquence, wile and comeliness.
It's used to persuade, deceive or seduce people, to smarm your way into their good graces, and to imply things without outright saying them.
It's also important to reading a person, or a room.
%TODO: Mechanical effects.

%TODO: Table.

\attribute{Presence}{presence}

\attref{presence} represents force of personality, air of authority and personal magnetism.
It's used to draw people's attention, boss them around, and make them wet themselves in terror.
%TODO: Mechanical effects.

\begin{simpletable}{rX}
	\toprule
	TN & Example Task\\
	\midrule
	9 & \\
	12 & \\
	15 & \\
	18 & \\
	21 & Silencing a raucous town hall with a polite cough.\\
	\bottomrule
\end{simpletable}

\section{Skills}


%TODO: Intro fluff paragraph about the skill of a witch.

%TODO: Recap how skills affect dice rolled for Tests, and how not every Test has an applicable skill.

A witch's skills can be divided into two categories.
The first consists of general skills, pertaining to things any witch might find herself doing.
The second consists of a witch's skills in her particular disciplines of magic.
These skills are normally of little use to a witch who does not practice such a discipline, although they can often be used to identify, and sometimes to counteract, the effects from it.

A list of the skills available to a witch, and examples of their use, is provided below.
If a skill corresponds to a particular discipline of magic, it can be increased by feats found in the appropriate chapter of \partref{disciplines}.
Otherwise, it can be increased by general feats. %TODO: Where to find these.

\subsection{Specialities}

Some skills have specialities.
For these skills, a character does not gain ranks in the skill itself, but in one of its specialities.
Ranks for each speciality are tracked independently.
For example, a witch might have \skillrefspeciality[1]{crafting}{Carpenter}, \skillrefspeciality[2]{crafting}{Cook} and \skillrefspeciality[1]{crafting}{Smith}.
Each skill with specialities provides a list of recommended options, but the GM may approve others.

\subsection{General Skills}
\seclabel{general-skills}

\skill{Animal Ken}{animals}

Used to understand animals and interact with them: to calm them, tame them, ride them, commadn them, or predict how they might act.

\skill{Athletics}{athletics}

Used to run, jump, swim, climb, somersault and generally get about the place more easily and impressively.

\skill{Botany}{botany}

Used to raise crops and herbs in a witch's garden, find them out in the forest, or identify a fishy-looking leaf.

\skill{Crafting}{crafting}

Used to make things, quite broadly.
This covers the creation of most kinds of objects, although some kinds of crafting are still covered by other skills, such as \skillref{brewing}.
Available specialities include the following:

\begin{itemize}
	\item Carpenter
	\item Cook
	\item Jeweller
	\item Mason
	\item Potter
	\item Seamstress
	\item Smith
	\item Woodcarver
	%TODO: Evaluate and maybe expand.
\end{itemize}

\skill{Deception}{deception}

Used to mislead, lie, prevaricate or filibuster, without anyone catching on that you're doing it.
Many witches make it a rule not to lie.
That doesn't mean they always need to tell the whole truth, so this can still be a useful skill for them.

\skill{Healing}{healing}

Used to bind wounds, set bones, diagnose diseases and deliver children.
This covers first aid, extended care and even surgery.
It does not cover the use of herbs, poultices or potions; these fall under \skillref{botany} and \skillref{brewing}.
It can be used to diagnose a patient's sickness in the first place, however: an essential step in applying the correct potion.

\skill{Insight}{insight}

Used to read people, as individuals or crowds.
This can include judging people's attitude and confidence, telling when and why they're uncomfortable, picking up on tells that they're lying, or predicting whether an argument is likely to come to blows.
It can be particularly useful for guessing at people's levers and buttons when preparing to manipulate them.
As a skill that relies on social understanding, it is normally rolled with \attref{charm}.

\skill{Intimidation}{intimidation}

Used for making threats: anything from subtly suggesting that you know a secret somebody would rather wasn't public knowledge, to outright yelling that you'll break the bugger's knees if he doesn't sit down and shut up \emph{right now}!
This doesn't even have to involve speaking; turning half a mob into frogs can certainly discourage the rest from tangling with you.

\skill{Lore}{lore}

Used to know and recall assorted knowledge, such as history, geography and religious doctrine. %TODO: Double-check our stance on religion.
Many fields of knowledge, such as magic and \skillref{botany}, fall under their own skills; this covers those that don't.

\skill{Perception}{perception}

Used to see, hear or smell things.
This includes noticing things that are out of place, such as hearing someone sneaking up behind you or spotting that your hat is missing from its peg.
It also covers active attempts to discern things, such as picking out details on someone at the other end of a street, eavesdropping, or identifying a faint smell.
Lastly, this is the skill used when trying to follow the trail of an animal or person.

\skill{Performance}{performance}

Used to entertain, amuse or impress people, or perhaps just to distract them.
Not everything that entertains people must use this skill; people can easily be entertained by a show of a magic or a swordfight, which might use another skill.
\skillref{performance} covers things done primarily for entertainment.
Available specialities include the following:

\begin{itemize}
	\item Dancer
	\item Drummer
	\item Harper
	\item Joke-teller
	\item Piper
	\item Storyteller
	%TODO: Evaluate and maybe expand.
\end{itemize}

\skill{Persuasion}{persuasion}

Used to influence or convince a person or crowd: to make them believe a particular thing or act in a particular way.
This can be through subtle suggestion and manipulation, or through reasoned, logical argument.
If you're attempting to persuade someone to act based on a falsehood, this might require both a \skillref{deception} Test to avoid being caught in the lie, and \skillref{persuasion} Test to motivate them to act.

\skill{Socialising}{socialising}

Used to befriend people, mingle with them, build rapport, and get into their good graces.
A good socialiser is everybody's best friend within a few minutes of meeting them, and might be trusted with secrets people would never otherwise give up.

\skill{Stealth}{stealth}

Used to do things without being noticed, such as sneaking up behind someone, peeking out through a bush, or lifting a guard's knife from his belt.
You can even try to blend in with a crowd (take the Hat off first), or rifle a man's purse while he watches your other hand.

\skill{Weaponry}{weaponry}

Used for everything from stabbing people with a concealed knife to clonking them over the head with a hefty staff, or even slugging them with a mean right hook.
Also used for chucking things and shooting them with a bow.

\subsection{Discipline Skills}
\seclabel{discipline-skills}

\skill[brewing]{Brewing}{brewing}

Used to brew tinctures, tonics, elixirs and other potions.
This doesn't always require a cauldron: it also covers mixing poultices and the like.
Of course, you can also make booze.

\skill[divination]{Divination}{divination}

Used to see the past and future, and places many miles away.
It's not limited to seeing either; a diviner can eavesdrop on a conversation in the next village, or track a person better than any bloodhound.

\skill[broomcraft]{Flying}{flying}

Used by a witch on a broomstick, whether she's settling in for a cross-country flight, showing off with a barrel roll, or pulling a stalled stick out of a deep dive.
This is also the skill used for feats of flying by a winged familiar.

\skill[golemancy]{Golemancy}{golemancy}

Used to will life into inanimate creatures of clay, or other materials.
A skilled golemancer can make more golems, make them smarter, and, of course, force life into increasingly substandard bodies.

\skill[necromancy]{Necromancy}{necromancy}

Used to pervert the natural order and bring the dead back to life, or at least commune with them from beyond the Veil.
Also used to send them on again, if hitting them over the head with a big stick won't suffice.

\begin{simpletable}{rX}
	\toprule
	TN & Example Task\\
	\midrule
	9 & Discerning the power of the \discref{necromancy} animating a shambling corpse.\\
	12 & Identifying the purpose of a necromantic rite from the chalk circle left behind.\\
	15 & Filtering the true facts about vampires from the baseless rumours that surround them.\\
	18 & Discerning the power of the \discref{necromancy} that previously animated a no-longer-shambling corpse.\\
	21 & Performing a complex necromantic ritual using nothing but two small sticks and a fresh egg.\\
	\bottomrule
\end{simpletable}

\skill[sympathetic-magic]{Sympathetic Magic}{sympathetic-magic}

Used to manipulate people or things using effigies, poppets or other imitative talismans.
The idea of \discref{sympathetic-magic} is that one can't affect the imitation of a thing without affecting the thing itself.

\skill[willing]{Willing}{willing}

Used to force the universe to fall into line with what you know is true.
There is a real knack to convincing yourself of something well enough to make this work, and this skill governs how good you are at it.
