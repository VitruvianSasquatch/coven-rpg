\chapter{Necromancy}
\chaplabel{necromancy}

\section{Reanimation and Resurrection}

Necromancers draw a distinction between reanimation and resurrection.
Reanimation is a crude process, somewhat akin to \discref{golemancy}.
It's nothing more than the application of raw animating force to a corpse, to stand it up and get it shuffling around again.
The creature retains its instincts, its muscle memory and the like, but that's as much through what is left of its biology as it is through the will that animated it.
The results of reanimation are known as the undead.

Resurrection, by contrast, brings a creature back back to life, in full.
If the creature had a soul, it is restored to the body.
There may be a few ill effects of the process, not to mention whatever killed it in the first place, but these can typically be recovered from.
For all intents and purposes, the creature is just as much alive as it was in the first place.

\section{The Lurching Dead}

The reanimation of dead creatures, as practiced by so many necromancers, is a process filled with limitations and drawbacks.
Many of these can be overcome by an experienced necromancer, but, for the beginner, the rules of such creatures are presented here.

Reanimation comes in many forms: zombies, skeletons and more.
Each comes with its own associated changes to the statistics of a creature, listed in the following sections.
Many changes, however, are shared between all reanimated creatures and are listed here.

Most reanimations do nothing to heal \secrefraw{damage} to the corpse: both \secrefraw{damage} suffered before and during death, and any further \secrefraw{damage} done to the corpse since then.
Some reanimations also require an unrotted corpse.
It usually takes a little over a week before a corpse becomes too rotted for such a reanimation, although temperature and moisture can alter this.

A reanimated creature loses its memory and identity.
It remembers general skills such as how to hunt, but forgets such information as the location of its den, and loses any mannerisms that distinguished it in life.

Most reanimated creatures also lack many biological processes that they had in life.
They are immune to poisons, diseases and the like.
Additionally, they cannot heal themselves or be healed, and are unaffected by beneficial potions and such.
Lastly, they cannot produce any venom or other such substances, so any benefit of a venomous bite, sting or so on is lost.

For 24 hours after reanimation, the reanimated creature is under your control.
You can assert direct control over it, or give it general instructions which it will carry out to the best of its ability.
You control it mentally, without need for verbal instructions or gestures, however you can only provide instructions while you can see, hear or otherwise sense it.
You can only have control over one reanimated corpse at a time; reanimating another one will free the previous as though the 24 hours had expired.

When your control over the creature expires, it regains free will and begins to act as an animal of its kind normally would.
However, it is ravenously and insatiably hungry.
As such, it is generally considered good practice to put the creature down before this occurs.

\subsection{Zombies}
\undeadlabel{Zombie}{Zombies}{zombie}

A zombie is the simplest reanimation possible; the corpse, fully clothed in its own flesh, it simply stood up and walked around as it is.
Is it a clumsy creature, with most of the mind rotted away as well, and it only grows worse as the corpse rots further.

A corpse reanimated as a zombie loses 2 \attref{grace}, \attref{wit}, \attref{charm} and \attref{presence}.
It loses 4 Speed in addition to the loss from the reduces \attref{grace}.
Its Shock Threshold increases by 2, however.
If it could fly, it is now too clumsy to do so.

A corpse reanimated as a zombie is not healed of any \secrefraw{damage}.
If this reduces its Shock Threshold to 0 or below, the corpse is too mangled to successfully reanimate.

Reanimating a corpse as a zombie requires that it is unrotted.
Furthermore, it does nothing to slow the rot.
A zombie that rots too far loses animation.

\subsection{Skeletons}
\undeadlabel{Skeleton}{Skeletons}{skeleton}

A skeleton is the result of reanimating only the bones of a creature, the flesh rotted or carved away.
The bones arrange themselves in the air, supported by nothing but the will of the animating witch, and the creature's convication in its own shape.
The result is a creature far less clumsy than a zombie, but not so resilient.

A corpse reanimated as a skeleton loses 2 \attref{might}, \attref{wit}, \attref{charm} and \attref{presence}.
Its Shock Threshold is also reduced by 2.
The mere bones of wings are not sufficient to allow it to fly, if it previously could.

Requiring only the bones, a skeleton is not affected by most \secrefraw{damage} sustained by the corpse.
Only a critical success on a \seclink{Damage Test}{damage-tests}, or an intentional effort after death, will typically have broken any bones.
Likewise, it is not affected by \secrefraw{damage} in the course of its undeath; any blow insufficient to scatter it across the floor is insufficient to scratch its bones.

A skeleton lasts a long time without decomposing; at least a decade and even longer if kept dry.

\section{Feats}

\feat{Touching the Veil}{death-detection}{10}{
	None
}{
	When a soul departs our world for the next, its passage disrupts the Veil between worlds.
	A witch who knows what to look for can feel this disruption.
	
	You can feel where people have died, though this sense is damped by both distance and time.
	If you pass through the actual position of the death, you'll notice for up to about two weeks after it occurred.
	You automatically sense a death in the same room only for a few days after it's happened, and in the same house for only about a day.
	However, a Test can reveal slightly older or more distant deaths, if you are searching for them.
	You can't gain any information about the identity of the victim or the cause of death.
	
	Locations of mass or repeated death can leave their traces lingering for much longer.
	The site of a battlefield or sacrificial altar may be felt for many years after.
}

\feat{Raise Zombie}{animate-zombie}{20}{
	None
}{
	You can restore a terrible facsimile of life to the bodies of deceased animals, reanimating it as a \undeadref{zombie}.
	For now, you are limited to animals at least as large as a mouse, and no larger than a medium-size dog such as a bloodhound.
	You can't manage a human or any animal that has been a familiar, due to interference from the link with its soul.
	
	\materials{An animal corpse, a \circleref{small}, a lit candle which the ritual extinguishes}
	
	The reanimation ritual takes five minutes, and must be performed in the dark.
}

\feat{Maintain Control}{undead-maintain-control}{10}{
	\featref{zombies}
}{
	You can reassert control over a reanimated creature you already control, resetting the time before your control expires.
	
	\materials{A reanimated creature under your control, a \materialref{ritual-circle} of the same size required to initially animate the creature, a lit candle which the ritual extinguishes}
	
	The ritual takes five minutes, and must be performed in the dark.
}

\feat{Stitches}{undead-repair}{10}{
	\featref{zombies}
}{
	Normally, a corpse that has sustained enough \secrefraw{damage} to reduce its Shock Threshold to 0 or below is too heavily damaged to reanimate or resurrect.
	You've figured out a way around that, with the right repairs.
	This usually involves stitching the missing bits back on, or gluing some bones.
	
	The repair and reanimation requires a Test, with the TN determined by how badly damaged the corpse is, using your choice of \skillref{necromancy} or \skillref{healing}.
	Success repairs just enough \secrefraw{damage} to restore the creature's Shock Threshold to 1.
}

\feat{Raise Skeleton}{animate-skeleton}{15}{
	\featref{zombies}
}{
	After a few reanimations, most \undeadrefplural{zombies} are starting to come apart at the seams a bit.
	There comes a time when it's easier just to strip all the flesh off and make the bones stand up by themselves.
	You may reanimate the bones of an animal corpse as a \undeadref{skeleton}, subject to the same limitations as \featref{animate-zombie}.
	
	\materials{The bones of an animal corpse (with the flesh removed), a \circleref{small}, a lit candle which the ritual extinguishes, a short length of thread}
	
	The reanimation ritual takes five minutes, and must be performed in the dark.
}

\feat{Reanimate Familiar}{reanimate-familiar}{10}{
	\featref{zombies}
}{
	While a soul normally interferes with reanimating a creature, you've begun to figure out how to use it to your advantage, and achieve true resurrection.
	Unfortunately, you can't actually summon any souls back to their bodies yet.
	Not to worry, though, for you have quite ready access to one soul in particular: your familiar's, so inextricably bound to your own.
	
	If your familiar dies and you can recover the corpse, you can resurrect it, paying no XP cost beside that required to purchase this feat in the first place.
	This is a true resurrection and avoids most of the drawbacks of undeath.
	However, it still does not repair \secrefraw{damage} to the corpse, so your familiar may need medical attention following resurrection.
	In cases where your familiar's Shock Threshold has been reduced to zero or below, it will require \featref{undead-repair}.
	
	\materials{Your familiar's corpse, a \circleref{small}, a lit candle which the ritual extinguishes}
	
	The ritual takes five minutes, and must be performed in the dark.
}

\feat{Major Undead}{undead-larger}{20}{
	\skillref[1]{necromancy},
	\featref{zombies}
}{
	Larger bodies need more force to reanimate, but it's force you've learned to provide.
	This follows the same rules as \featref{zombies}, but requires a \circleref{medium} and ignores the upper size limit on the creature.
}
