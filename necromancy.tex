\chapter{Necromancy}
\chaplabel{necromancy}

\section{Reanimation and Resurrection}

Many necromancers draw a distinction between reanimation and resurrection.
Reanimation is a crude process, somewhat akin to \discref{golemancy}.
It's nothing more than the application of raw animating force to a corpse, to stand it up and get it shuffling around again.
The creature retains its instincts, its muscle memory and the like, but that's as much through what is left of its biology as it is through the will that animated it.
The results of reanimation are known as the undead.

Resurrection, by contrast, brings a creature back back to life, in full.
If the creature had a soul, it is restored to the body.
There may be a few ill effects of the process, not to mention whatever killed it in the first place, but these can typically be recovered from.
For all intents and purposes, the creature is just as much alive as it was in the first place.

In theory, at least.

The trouble is that nobody has ever achieved true resurrection.
Dozens of witches and hundreds of charlatans have all claimed to.
Many have even come incredibly close, but there has always been some slight snag.
The search, of course, continues, but many have given up all hope that it is possible.

\section{The Lurching Dead}

The reanimation of dead creatures, as practiced by so many necromancers, is a process filled with limitations and drawbacks.
Many of these can be overcome by an experienced necromancer, but, for the beginner, the rules of such creatures are presented here.

Reanimation comes in many forms: zombies, skeletons and more.
Each comes with its own associated changes to the statistics of a creature, listed in the following sections.
Many changes, however, are shared between all reanimated creatures and are listed here.

Most reanimations do nothing to heal \secrefraw{damage} to the corpse: both \secrefraw{damage} suffered before and during death, and any further \secrefraw{damage} done to the corpse since then.
Some reanimations also require an unrotted corpse.
It usually takes a little over a week before a corpse becomes too rotted for such a reanimation, although temperature and moisture can alter this.

A reanimated creature loses its memory and identity.
It remembers general skills such as how to hunt, but forgets such information as the location of its den, and loses any mannerisms that distinguished it in life.

Most reanimated creatures also lack many biological processes that they had in life.
They do not need to breathe, eat, drink or sleep.
They are immune to poisons, diseases and the like.
Additionally, they cannot heal themselves or be healed, and are unaffected by beneficial potions and such.
Lastly, they cannot produce any venom or other such substances, so any benefit of a venomous bite, sting or so on is lost.

For 24 hours after reanimation, the reanimated creature is under your control.
You can assert direct control over it, or give it general instructions which it will carry out to the best of its ability.
You control it mentally, without need for verbal instructions or gestures, however you can only provide instructions while you can see, hear or otherwise sense it.
You can only have control over one reanimated corpse at a time; reanimating another one will free the previous as though the 24 hours had expired.

When your control over the creature expires, it regains free will and begins to act as an animal of its kind normally would.
However, it is ravenously and insatiably hungry.
As such, it is generally considered good practice to put the creature down before this occurs.

\subsection{Zombies}
\undeadlabel{Zombie}{Zombies}{zombie}

A zombie is the simplest reanimation possible; the corpse, fully clothed in its own flesh, it simply stood up and walked around as it is.
Is it a clumsy creature, with most of the mind rotted away as well, and it only grows worse as the corpse rots further.

A corpse reanimated as a zombie loses 2 \attref{grace}, \attref{wit}, \attref{charm} and \attref{presence}.
It loses 4 Speed in addition to the loss from the reduced \attref{grace}.
Its Shock Threshold increases by 2, however.
If it could fly, it is now too clumsy to do so.

A corpse reanimated as a zombie is not healed of any \secrefraw{damage}.
If this reduces its Shock Threshold to 0 or below, the corpse is too mangled to successfully reanimate.

Reanimating a corpse as a zombie requires that it is unrotted.
Furthermore, it does nothing to slow the rot.
A zombie that rots too far loses animation.

\subsection{Skeletons}
\undeadlabel{Skeleton}{Skeletons}{skeleton}

A skeleton is the result of reanimating only the bones of a creature, the flesh rotted or carved away.
The bones arrange themselves in the air, supported by nothing but the will of the animating witch, and the creature's convication in its own shape.
The result is a creature far less clumsy than a zombie, but not so resilient.

A corpse reanimated as a skeleton loses 2 \attref{might}, \attref{wit}, \attref{charm} and \attref{presence}.
Its Shock Threshold is also reduced by 2.
The mere bones of wings are not sufficient to allow it to fly, if it previously could.
It also sinks in water, but may move along the bottom.

Requiring only the bones, a skeleton is not affected by most \secrefraw{damage} sustained by the corpse.
Only a critical success on a \seclink{Damage Test}{damage-tests}, or an intentional effort after death, will typically have broken any bones.
Likewise, it is not affected by \secrefraw{damage} in the course of its undeath; any blow insufficient to scatter it across the floor is insufficient to scratch its bones.

A skeleton lasts a long time without decomposing; at least a decade and even longer if kept dry.

\section{Feats}

\feat{Raise Zombie}{animate-zombie}{20}{
	None
}{
	You can restore a terrible facsimile of life to the bodies of deceased animals, reanimating it as a \undeadref{zombie}.
	For now, you are limited to animals at least as large as a mouse, and no larger than a medium-size dog such as a bloodhound.
	You can't manage a human or any animal that has been a familiar, due to interference from the link with its soul.
	
	\materials{An animal corpse, a \circleref{small}, a lit candle which the ritual extinguishes}
	
	The reanimation ritual takes five minutes, and must be performed in the dark.
}

\feat{Raise Skeleton}{animate-skeleton}{15}{
	\featref{animate-zombie}
}{
	After a few reanimations, most \undeadrefplural{zombie} are starting to come apart at the seams a bit.
	There comes a time when it's easier just to strip all the flesh off and make the bones stand up by themselves.
	You may reanimate the bones of an animal corpse as a \undeadref{skeleton}, subject to the same limitations as \featref{animate-zombie}.
	
	\materials{The bones of an animal corpse (with the flesh removed), a \circleref{small}, a lit candle which the ritual extinguishes, a short length of thread}
	
	The reanimation ritual takes five minutes, and must be performed in the dark.
}

\feat{Reanimate Familiar}{reanimate-familiar}{10}{
	\featref{animate-zombie}
}{
	While a soul normally interferes with reanimating a creature, you've begun to figure out how to use it to your advantage, beginning on the path towards resurrection.
	Unfortunately, you can't actually summon any souls back to their bodies yet.
	Not to worry, though, for you have quite ready access to one soul in particular: your familiar's, so inextricably bound to your own.
	
	If your familiar dies and you can recover the corpse, you can reanimate it, paying no XP cost beside that required to purchase this feat in the first place.
	You may use any of the reanimation rituals provided by %TODO: If this list remains two items long, change to "either of"
	\featref{animate-zombie} or \featref{animate-skeleton},
	provided you know them.
	
	Your familiar is subject to the usual modifications to its statistics, as appropriate to the kind of reanimation, except that its mental and social attributes (\attref{wit}, \attref{will}, \attref{charm}, \attref{presence}) are never changed.
	It also retains its memories, identity and free will, and is not subject to the usual hunger; as such it does not count towards the number of undead you may control.
	It is still subject to all the benefits and detriments of its loss of biological processes, such as immunity to suffocation, disease and potions.
	Lastly, it is still subject to usual rules for \secrefraw{damage} and rotting, so may require \featref{undead-repair}.
	
	Reanimating a familiar in this way does not prevent recovering it through the usual repetition of the binding ritual later (see \secref{familiar-injury-death}), although the normal XP cost must still be paid each time that it used.
}

\feat{Maintain Control}{undead-maintain-control}{10}{
	\featref{animate-zombie}
}{
	You can reassert control over a reanimated creature you already control, resetting the time before your control expires.
	
	\materials{A reanimated creature under your control, a \materialref{ritual-circle} of the same size required to initially animate the creature, a lit candle which the ritual extinguishes}
	
	The ritual takes five minutes, and must be performed in the dark.
}

\feat{Stitches}{undead-repair}{10}{
	\featref{animate-zombie}
}{
	Many reanimations and resurrections are ineffective on corpses which are too badly damaged.
	You've figured out a way around that, with the right repairs.
	This usually involves stitching the missing bits back on, or gluing some bones.
	
	The repair and reanimation requires a Test, with the TN determined by how badly damaged the corpse is, using your choice of \skillref{necromancy} or \skillref{healing}.
	A successful Test repairs at least enough \secrefraw{damage} to restore the creature's Shock Threshold to 1, and a high result may repair even more.
}

\feat{Scraps}{undead-repair-2}{10}{
	\skillref[1]{healing},
	\skillref[1]{necromancy},
	\featref{undead-repair}
}{
	You can do more than stitch a damaged corpse back together; you can stitch \emph{several} corpses together.
	You may assemble a corpse for reanimation out of parts from different corpses, stitched or otherwise attached together.
	A corpse assembled out of several different, but individually intact, parts can be much healthier than a corpse with several damaged and repaired parts.
	
	The pieces must all come from creatures of the same kind, and must be assembled to form a complete creature of that kind.
	The repair and reanimation still requires a Test, using your choice of \skillref{necromancy} or \skillref{healing}.
}

\feat{Major Undead}{undead-larger}{20}{
	\skillref[1]{necromancy},
	\featref{animate-zombie}
}{
	Larger bodies need more force to reanimate, but it's force you've learned to provide.
	When you perform a ritual to reanimate a creature, you may use a \circleref{medium} instead of a \circleref{small}, in order to ignore the upper size limit on the creature.
}

\feat{Touching the Veil}{death-detection}{10}{
	None
}{
	When a soul departs our world for the next, its passage disrupts the Veil between worlds.
	A witch who knows what to look for can feel this disruption.
	
	You can feel where people have died, though this sense is damped by both distance and time.
	If you pass through the actual position of the death, you'll notice for up to about two weeks after it occurred.
	You automatically sense a death in the same room only for a few days after it's happened, and in the same house for only about a day.
	However, a Test can reveal slightly older or more distant deaths, if you are searching for them.
	You can't gain any information about the identity of the victim or the cause of death.
	
	Locations of mass or repeated death can leave their traces lingering for much longer.
	The site of a battlefield or sacrificial altar may be felt for many years after.
}

\feat{Medium}{medium}{10}{
	None
}{
	It is possible for the souls of the dead to possess the bodies of the living, although most spirits are not strong enough to force their way in.
	A specially prepared mind may invite them in, however.
	
	You may enter a mediumship trance by consuming a \herbref[black henbane, a.k.a.\ stinking nightshade]{2} and meditating for a minute in a dark place.
	The trance naturally lasts a few minutes, during which the medium is unaware of her surroundings.
	
	While the medium is in a trance, she is vulnerable to possession by any nearby spirits of the dead.
	These can include the spirits of those who have died nearby, family members and friends of the medium or other nearby people, those who took some special interest in the medium (such as her mortal enemies), or sometimes even randomly passing spirits.
	A spirit is aware of the medium's identity, and must choose to possess her body.
	
	The possessing spirit gains full control of the medium's body.
	The medium's \attref{might} and \attref{grace} are retained, but the creature uses the spirit's \attref{wit}, \attref{will}, \attref{charm}, \attref{presence}, skills and feats.
	The spirit retains all the memories it had in life, and memories of any experiences it has had on the mortal plane since then, but has no recollection of the afterlife.
	It is aware that it has died, and that some time has passed since it died, but has no idea how much.
	
	The medium remains unaware of her body's surroundings throughout the possession, left alone with her thoughts.
	She is aware of the passage of time, although without external stimulus her internal clock is not very accurate.
	The possession lasts indefinitely, until the spirit reliquishes it, making possession by a non-benign spirit a very risky prospect.
	Whenever the spirit is shocked in some fashion, however, such as by \secrefraw{damage}, a slap to the face, a dunk in icy water or the revelation of a terrible secret, the medium may choose to try and reassert control over her own body.
	She makes a \attref{will} Test, opposed by the possessing spirit's \attref{will} Test.
	If the medium succeeds, she regains control over her own body and ends the possession.
	
	If the medium hopes to gain information from the possessing spirit, she is advised to have an assistant to ask the spirit questions, or at least to leave it a piece of paper with some questions and a quill to record its answers.
	%Mediums interested in limiting the harm a malignant spirit can wreak may be interested in \featref{containing-circle}. %TODO
}
