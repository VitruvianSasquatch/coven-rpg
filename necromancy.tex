\chapter{Necromancy}
\chaplabel{necromancy}

\section{The Lurching Dead}

Many \discref{necromancy} feats allow the reanimation of dead creatures.
This is a process filled with limitations and drawbacks.
Many of these can be overcome by an experienced necromancer, but, for the beginner, the rules of such creatures are presented here.

A reanimated animal is a shoddy excuse for a creature.
It retains its original statistics for the most part, except all its \secref{attributes} are reduced by 2, and its derived statistics are adjusted accordingly.
Furthermore, damage to the corpse is not healed; it has suffered as much damage as it had when it died, in addition to any damage done to the corpse since then.
If this reduces its Shock Threshold to zero or below, the corpse is too mangled to successfully reanimate.
If the animal could fly, the reanimated version cannot.

Furthermore, a reanimated animal has lost its memory and identity.
It remembers general skills such as how to hunt, but forgets such information as the location of its den, and loses any mannerisms that distinguished it in life.

For 24 hours after reanimation, the reanimated animal is under your control.
You can assert direct control over it, or give it general instructions which it will carry out to the best of its ability.
You control it mentally, without need for verbal instructions or gestures, however you can only provide instructions while you can see, hear or otherwise sense it.
You can only have control over one reanimated corpse at a time; reanimating another one will free the previous as though the 24 hours had expired.

When your control over the animal expires, it regains free will and begins to act as an animal of its kind normally would.
However, it is ravenously and insatiably hungry.
As such, it is generally considered good practice to put the animal down before this occurs.

Reanimation does nothing to preserve the corpse, and it will continue to rot as normal.
It generally takes a little over a week before the corpse is too rotted to become or remain animated, although temperature and moisture can alter this.

\section{Feats}

\feat{Touching the Veil}{death-detection}{10}{
	None
}{
	When a soul departs our world for the next, its passage disrupts the Veil between worlds.
	A witch who knows what to look for can feel this disruption.
	
	You can feel where people have died, though this sense is damped by both distance and time.
	If you pass through the actual position of the death, you'll notice for up to about two weeks after it occurred.
	You automatically sense a death in the same room only for a few days after it's happened, and in the same house for only about a day.
	However, a Test can reveal slightly older or more distant deaths, if you are searching for them.
	You can't gain any information about the identity of the victim or the cause of death.
	
	Locations of mass or repeated death can leave their traces lingering for much longer.
	The site of a battlefield or sacrificial altar may be felt for many years after.
}

\feat{Zombies}{zombies}{20}{
	None
}{
	You can restore a terrible facsimile of life to the bodies of deceased animals.
	For now, you are limited to animals at least as large as a mouse, and no larger than a medium-size dog such as a bloodhound.
	You can't manage a human or any animal that has been a familiar, due to interference from the link with its soul.
	
	\materials{An animal corpse, a \circleref{small}, a lit candle which the ritual extinguishes}
	
	The reanimation ritual takes five minutes, and must be performed in the dark.
}
