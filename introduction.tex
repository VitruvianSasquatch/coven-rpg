\chapter{Introduction}

\titleemph{Coven} is a role-playing game designed upon a simple premise: the player characters are witches and the party is a coven.
Every character shares the common tools of witchcraft: a familiar, a broomstick and, most importantly, a pointed hat.
However, that is often where the similarities end.
There are many different disciplines to witchcraft, and many different approaches even within a discipline.
From meticulous ritualists to soaring broom-riders, from shy girls to terrifying matriachs, hunched over a cauldron or chatting with squirrels in the forest, a coven can be a diverse lot.

In light of this, they don't always get along.
Witches can be somewhat solitary creatures by nature, tending to their own villages, dealing with their own problems.
But they do tend to keep tabs on one another, and a good witch recognises when things are bit much to handle by herself.
When the great spirits of the land are threatened, when \emph{things} push through from other realms, or when one of their own begins to cackle: these are the times witches come together.
And these are the adventures the players have with them.

\section{The Craft}

The Craft, the Art, the Way.
Witchery, Occultism, Thaumaturgy.
There are many names for witchcraft.
Few things define it, however.
In truth, it is nothing but knowledge of the diverse disciplines of magic, and the skill to apply it.

Witchcraft is not like the enchantments of the faeries or the sorcery of warlocks.
It's not a power one is born with, nor one absorbed in a moment.
It is learned through years of training, grasped through decades of practice, and never truly mastered.
Anyone can pick it up, given enough patience and determination.
But few even have the inclination.

For the power of witchcraft comes with more responsbility than most.
The responsibility to care for one's neighbours, one's charges, one's village.
To see them through sickness and through strife, to see them into the world and back out of it.
The responsibility to take up arms and defend them from the horrors of the night, of other realms, even the ones they bring upon themselves.
To lay down one's own life in defence of others.
And finally, the responsibility to train a successor, that the Craft may continue to serve one's village after one dies.
Everything that goes on in a witch's realm is her responsibility, and that is too great a burdern for many to bear.

Which brings us to the topic of the Black Craft.
Witchcraft is simply knowledge, to be used how it will.
Even possession, voodoo and necromancy are not evil acts in themselves, when turned to the purpose of good.
Evil begins when all the responsbility becomes too much for a witch.
When she wonders why she should be doing so much for these people who never do anything for themselves
When she believes that she is better than other people.
When she begins to cackle.
And so comes another responsibility of witches: to sit her down and give her a stern talking to.
Or, failing that, to show her the way out\dots
