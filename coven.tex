\documentclass[a4paper,10pt,twocolumn]{book}
\usepackage[all]{nowidow}
\usepackage{amsmath}
\usepackage{siunitx}
\usepackage{enumerate}

%\usepackage[margin=0.9in]{geometry}

\usepackage{titling}
%\setlength{\droptitle}{-0.5in} %Adjust margin above title

\usepackage{titlesec}
\titleformat{\chapter}[hang]{\normalfont\huge\bfseries}{\chaptertitlename\ \thechapter:}{1em}{} %Chapter number and name on same line
%TODO: This justifies split lines awfully, fix it.

\makeatletter
\renewcommand{\@seccntformat}[1]{} %Remove section numbers
\makeatother

\usepackage{tocloft}
\makeatletter
\renewcommand{\cftsecpresnum}{\begin{lrbox}{\@tempboxa}} %Removes section numbers from the table of contents.
\renewcommand{\cftsecaftersnum}{\end{lrbox}} %Latter half of the above.
\makeatother
\setlength{\cftsecnumwidth}{0pt} %Removes the space reserved for the section numbers in the table of contents.

\usepackage{hyperref}
\usepackage{nameref}

\usepackage{graphicx}
\usepackage{float} %Provides the float option [H] for a non-floating float.
\graphicspath{{./imgs/}}

\usepackage{hyperxmp} %Recommended for doclicense
\usepackage[
	type={CC},
	modifier={by-nc-sa},
	version={4.0},
]{doclicense}

\usepackage{booktabs}
\usepackage{tabu}
\tabulinesep=1.2mm %Space tables more nicely.
\usepackage[table]{xcolor}
\newenvironment{simpletable}[1]{
	\begin{center}
	\rowcolors{2}{}{gray!20}
	\begin{tabu}{#1}
}{
	\end{tabu}
	\end{center}
}


\newcommand\partlabel[1]{\label{part:#1}}
\newcommand\chaplabel[1]{\label{chap:#1}}
\newcommand\seclabel[1]{\label{sec:#1}}
\newcommand\partref[1]{Part~\ref{part:#1}}
\newcommand\chapref[1]{Chapter~\ref{chap:#1}}
\newcommand\chapnameref[1]{\nameref{chap:#1}}
\newcommand\secref[1]{the \nameref{sec:#1} section}
\newcommand\seclink[2]{\hyperref[sec:#2]{#1}}

\newcommand\discref[1]{\chapnameref{#1}}

\newcommand\feat[4]{ %Arguments: title, label, prerequisites, text.
	\subsection{#1}\label{feat:#2}
	\textbf{Prerequisites:} {#3}
	
	{#4}
} %TODO: XP cost, tags such as 'first circle'?
\newcommand\featref[1]{\nameref{feat:#1}}
%TODO: Make feat references add the chapter in brackets afterwards, if the feat is in a different chapter.

\makeatletter
\newcommand\skill[3][\@nil]{%Takes three arguments, the first of which is optional and defaults to nothing.
	\def\govdisc{#1}
	\subsection{#2}%
	\label{skill:#3}%
	\ifx\govdisc\@nnil%
		%
	\else%
		Governing discipline: \discref{\govdisc}%TODO: This does really nasty things to the paragraphing.
	\fi%
}
\makeatother


\newcommand\skillref[2][0]{%Takes two arguments, the first of which is optional and defaults to zero.
	\nameref{skill:#2}%
	\ifnum #1=0%
		%
	\else%
		~#1%
	\fi%
}
\newcommand\skillrefspecialty[3][0]{%Takes three arguments, the first of which is optional and defaults to zero.
	\nameref{skill:#2} (#3)%
	\ifnum #1=0%
		%
	\else%
		~#1%
	\fi%
}

\newcommand\attribute[2]{%
	\subsection{#1}
	\label{attribute:#2}
}

\newcommand\attref[1]{%
	\nameref{attribute:#1}%
}

\newcommand\testtype[2]{%
	$\text{\attref{#1}} + \text{\skillref{#2}}$%
}

\usepackage{amsfonts} %Gives the stuff we use to build \shortminus
\DeclareMathSymbol{\shortminus}{\mathbin}{AMSa}{"39}

\usepackage{xstring} %Gives \StrDel
\newcommand\dice[2][0]{%Takes two arguments, the first of which is optional and defaults to zero.
	#2d%
	\ifnum #1=0%
		%
	\else%
		\ifnum #1>0%
			$+$%
		\else%
			$\shortminus$%
		\fi%
		\StrDel{#1}{-}%Strips the minus from it, essentially giving absolute value.
	\fi%
}

\newcommand\negative[1]{$\shortminus$#1}

\newcommand\titleemph[1]{\emph{#1}} %For the titles of books and such, including Coven itself.

\newcommand\storybreak{\bigskip}

\title{Coven: An RPG of Witches}
\author{Christopher Brown}
\date{}

\begin{document}
\pagestyle{plain} %Fixes page numbers appearing in the top left on empty pages.

\maketitle

\onecolumn

\doclicenseThis

\twocolumn


\setcounter{tocdepth}{1} %No subsections or deeper.
\tableofcontents

\chapter{Introduction}

\titleemph{Coven} is a role-playing game designed upon a simple premise: the player characters are witches and the party is a coven.
Every character shares the common tools of witchcraft: a familiar, a broomstick and, most importantly, a pointed hat.
However, that is often where the similarities end.
There are many different disciplines to witchcraft, and many different approaches even within a discipline.
From meticulous ritualists to soaring broom-riders, from shy girls to terrifying matriachs, hunched over a cauldron or chatting with squirrels in the forest, a coven can be a diverse lot.

In light of this, they don't always get along.
Witches can be somewhat solitary creatures by nature, tending to their own villages, dealing with their own problems.
But they do tend to keep tabs on one another, and a good witch recognises when things are bit much to handle by herself.
When the great spirits of the land are threatened, when \emph{things} push through from other realms, or when one of their own begins to cackle: these are the times witches come together.
And these are the adventures the players have with them.

\section{The Craft}

The Craft, the Art, the Way.
Witchery, Occultism, Thaumaturgy.
There are many names for witchcraft.
Few things define it, however.
In truth, it is nothing but knowledge of the diverse disciplines of magic, and the skill to apply it.

Witchcraft is not like the enchantments of the faeries or the sorcery of warlocks.
It's not a power one is born with, nor one absorbed in a moment.
It is learned through years of training, grasped through decades of practice, and never truly mastered.
Anyone can pick it up, given enough patience and determination.
But few even have the inclination.

For the power of witchcraft comes with more responsbility than most.
The responsibility to care for one's neighbours, one's charges, one's village.
To see them through sickness and through strife, to see them into the world and back out of it.
The responsibility to take up arms and defend them from the horrors of the night, of other realms, even the ones they bring upon themselves.
To lay down one's own life in defence of others.
And finally, the responsibility to train a successor, that the Craft may continue to serve one's village after one dies.
Everything that goes on in a witch's realm is her responsibility, and that is too great a burdern for many to bear.

Which brings us to the topic of the Black Craft.
Witchcraft is simply knowledge, to be used how it will.
Even possession, voodoo and necromancy are not evil acts in themselves, when turned to the purpose of good.
Evil begins when all the responsbility becomes too much for a witch.
When she wonders why she should be doing so much for these people who never do anything for themselves
When she believes that she is better than other people.
When she begins to cackle.
And so comes another responsibility of witches: to sit her down and give her a stern talking to.
Or, failing that, to show her the way out\dots

\section{The World}

%TODO: Assumptions of the setting.

\section{Dice and Tests}

%Dice and tests system, critical success and failure.


\part{Character Creation}
\partlabel{character-creation}

\chapter{Character Creation Guide}
\chaplabel{character-creation-guide}

\section{Step Zero: Character Concept}

The most important part of a character is the concept.
Who is your character, what does she do?
You can try to flesh this all out now, or fill it in as you work through character creation.
Make sure your character is somebody you will enjoy roleplaying.

Also make sure your witch fits into the coven; discuss this with the other players and your GM.
It can be quite painful for everyone involved playing an unscrupulous necromancer in a coven of saccharine healers, or vice versa.
The GM should provide some idea of the tone intended for the game, to avoid this sort of trouble.
Diversity can also be good: make sure you know what you're letting yourselves in for if everybody in the group wants to play a potion-brewer.

\section{Step One: Starting Experience}

Your GM will assign you an amount of experience (XP) to use during character creation.
By default, this is 100 XP, though the GM is free to adjust this to suit a different style of characters or campaign.

100 XP is suitable for witches who have just completed their apprenticeship and are taking over their own steading.
40 XP might be more suitable for witches who are still in an apprenticeship, and may have only just bound their familiar.
200 XP might be appropriate for witches with a few years of caring for a steading under their belt.
Even more experienced witches might require even more starting XP.
For fairness, the GM should probably give all characters the same starting XP, unless there is a good reason otherwise.

Make a note of how much XP you have, and keep track as you spend it later.
It can be well worth keeping a record of everything you have spent XP on over the course of character creation and any later development.

\section{Step Two: Attributes}

Attributes are a witch's broad, innate capabilities.
Is she skinny and lithe, or broad and well-muscled; quick-witted, or bullheaded; domineering, or silver-tongued?

At character creation, you have 20 points to spend on your character's attributes.
Set each of the eight attributes to between 0 and 4, inclusive, such that they sum to 20.

Attributes represent much more innate ability than skills.
While they can be developed through much hard work, they are not often passively improved over the course of a lifetime.
As such, it is not recommended to change the number of points available for attributes as readily as one might change the starting XP.

\section{Step Three: Skills}

A witch does not begin totally unskilled.
Select 1 \seclink{general skill}{general-skills}; you begin with 2 ranks in this skill.
Select an additional 3 general skills; you begin with 1 rank in each of these.
Lastly, select 1 \seclink{discipline skill}{discipline-skills} and one \seclink{speciality skill}{speciality-skills}; you begin with 1 rank in each of these skills.

The GM is also free to adjust the number of skills and ranks granted to starting characters.
General skills represent general life experience, speciality skills tend to result from vocational experience, and discipline skills represent experience with magic and witchcraft.
However, acquiring more than 1 rank in a discipline skill without learning magic from the associated discipline is very rare; such ranks ought to be acquired through XP rather than granted at character creation.

\section{Step Four: Familiar}

A witch's familiar is her essential and constant companion, and is usually bound early in her apprenticeship.
The available familiars are detailed in \chapref{familiars}.
Select one, spending XP (including the XP for any options) from your starting XP if necessary.

A witch can never have more than one familiar.
However, with the GM's approval, you may decline to select your familiar yet.
In this case, you begin play without a familiar and must acquire it during play, spending the necessary XP then.
This can be useful if you want a familiar with a high XP cost, but want to spend your starting XP on other things.

\section{Derived Statistics}

\begin{simpletable}{ll}
	\toprule
	Statistic & Derivation\\
	\midrule
	Resilience & $3$\\
	Shock Threshold & $12 + \text{\attref{might}} + \text{\attref{will}}$\\
	Dodge Rating & $8 + \text{\attref{grace}} + \text{\attref{heed}}$\\
	Speed & $8 + \text{\attref{might}} + \text{\attref{grace}}$\\
	\bottomrule
\end{simpletable}

\section{Steading}

Most witches have a steading.
This is the area a witch watches over, a region she defends and protects the inhabitants of.
The duties a witch has to her steading are numerous and varied, but typically involve healing the inhabitants and protecting them from threats of a magical nature.
Some witches also perform midwifing, care for the land itself, or even take it upon themselves to deal with non-magical threats, such as invading armies.
A witch's responsibilities are not limited to her steading, and nothing stops her from responding to threats outside it.
But inside it, everything is certainly her responsibility.

Decide whether your witch has a steading.
How big is it?
One village, several, or an entire kingdom?
What duties does she perform within it?
Do the inhabitants appreciate what she does for them?

Also discuss this with your GM, and the other players.
Has the GM already described a village that could be your steading?
It is not unheard of for witches to share a steading, although this can obviously lead to disagreements.
Do you share a steading with your coven, or have you carved the local region into one steading each?


\chapter{Attributes and Skills}
\chaplabel{attributes-and-skills}

\section{Attributes}
\seclabel{attributes}

Attributes are a character's broad, innate capabilities.
They represent physical capacity and natural talent.
That is not to say they can't be improved---one can grow muscle through exercise and the brain is no different---but such improvement represents a more significant investment than picking up a new skill.
A character has six attributes: \attref{might}, \attref{grace}, \attref{wit}, \attref{will}, \attref{charm} and \attref{presence}.
For human characters, these range from 0 to 5, with 2 as the average for a human.
Non-human characters may have attributes outside this range.

A summary of these attributes is provided below, along with examples of using the attribute.
Note that many of the example Tests would be accompanied by an appropriate skill.

\attribute{Might}{might}

\attref{might} represents physical strength, endurance and resilience.
It's used to lift things, smash things, resist diseases and endure hard labour, to put the hurt on people and to resist having the hurt put back on you.
\attref{might} is the attribute you use when rolling damage with melee weapons, and also determines the amount of damage required to put you down.
It can also prove useful when a brewer or botanist feeds you something you shouldn't have eaten.
Lastly, powerful legs let you run faster.

%This table seems unhelpful, on second thoughts.
%Would such grandiose descriptions transfer accurately to fairly small range of numbers in the game mechanics?
%
%\begin{simpletable}{lX}
%	\toprule
%	0 & Total weakling: Carrying your shopping back from the market is a struggle.\\
%	1 & Below average: Arm-wrestling ends embarassing for you.\\
%	2 & Average: A day's farm-work is tiring, but doable.\\
%	3 & Above average: You can carry bricks all day everyday.\\
%	4 & Exceptional: You could be a blacksmith.\\
%	5 & Incredible: You could run around in plate armour for hours.\\
%	\bottomrule
%\end{simpletable}

\begin{simpletable}{rX}
	\toprule
	TN & Example Task\\
	\midrule
	9 & Jumping across a \SI{3}{\metre} gap.\\
	12 & \\
	15 & \\
	18 & \\
	21 & \\
	\bottomrule
\end{simpletable}

\attribute{Grace}{grace}

\attref{grace} represents agility, dexterity and reflexes.
It's used to dodge swords, manoeuvre broomsticks, do backflips, dance waltzes, and hastily scratch runic circles into the floor without smudging them and letting the demons in.
\attref{grace} determines how hard you are to hit with a weapon and also contributes toward your speed.

%TODO: Table

\attribute{Wit}{wit}

\attref{wit} represents intelligence, memory and awareness.
It's used to recall knowledge, solve puzzles, and ensure no detail escapes your notice.
\attref{wit} is the key attribute for many forms of magic, used to memorise and understand spells, recipes and rituals.

%TODO: Table.

\attribute{Will}{will}

\attref{will} represents courage, dedication and conviction.
It's used to stand your ground, resist the influence of others, remain unfazed in embarassing situations, and push onwards in the face of adversity.
\attref{will} influences your pain threshold and is used to resist curses and mental influence, mundane or magical.
Force of \attref{will} can also be applied to directly influence the world in some forms of magic.

%TODO: Table.

\attribute{Charm}{charm}

\attref{charm} represents eloquence, wile and comeliness.
It's used to persuade, deceive or seduce people, to smarm your way into their good graces, and to imply things without outright saying them.
It's also important to reading a person, or a room.
%TODO: Mechanical effects.

%TODO: Table.

\attribute{Presence}{presence}

\attref{presence} represents force of personality, air of authority and personal magnetism.
It's used to draw people's attention, boss them around, and make them wet themselves in terror.
%TODO: Mechanical effects.

\begin{simpletable}{rX}
	\toprule
	TN & Example Task\\
	\midrule
	9 & \\
	12 & \\
	15 & \\
	18 & \\
	21 & Silencing a raucous town hall with a polite cough.\\
	\bottomrule
\end{simpletable}

\section{Skills}


%TODO: Intro fluff paragraph about the skill of a witch.

%TODO: Recap how skills affect dice rolled for Tests, and how not every Test has an applicable skill.

A witch's skills can be divided into two categories.
The first consists of general skills, pertaining to things any witch might find herself doing.
The second consists of a witch's skills in her particular disciplines of magic.
These skills are normally of little use to a witch who does not practice such a discipline, although they can often be used to identify, and sometimes to counteract, the effects from it.

A list of the skills available to a witch, and examples of their use, is provided below.

\subsection{Improving Skills}

Ranks in skills may be purchased by spending XP.
The XP costs of increasing skills are provided in the following table.
A character must have the previous rank in a skill before purchasing the next rank.

\begin{simpletable}{lrr}
	\toprule
	Type & Rank & XP Cost\\
	\midrule
	General & 1 & 15\\
	General & 2 & 25\\
	General & 3 & 35\\
	Discipline & 1 & 40\\
	Discipline & 2 & 80\\
	Discipline & 3 & 120\\
	\bottomrule
\end{simpletable}

The XP cost for increasing a general skill is fixed, while progression in discipline skills is closely tied to progression in the discipline itself.
While it is possible to learn a lot about discipline of magic without ever practicing it, it is far easier to learn simply by setting out and using the magic.
As such, increasing a discipline skill costs 5 XP less for each feat a witch has purchased from its governing discipline, to a minimum cost of 0 XP.

\subsection{Specialities}

Some skills have specialities.
For these skills, a character does not gain ranks in the skill itself, but in one of its specialities.
Ranks for each speciality are tracked independently.
For example, a witch might have \skillrefspeciality[1]{crafting}{Carpenter}, \skillrefspeciality[2]{crafting}{Cook} and \skillrefspeciality[2]{crafting}{Smith}.
Each skill with specialities provides a list of recommended options, but the GM may approve others.

\subsection{General Skills}
\seclabel{general-skills}

\skill{Animal Ken}{animals}

Used to understand animals and interact with them: to calm them, tame them, ride them, command them, or predict how they might act.

\skill{Athletics}{athletics}

Used to run, jump, swim, climb, somersault and generally get about the place more easily and impressively.

\skill{Botany}{botany}

Used to raise crops and herbs in a witch's garden, find them out in the forest, or identify a fishy-looking leaf.

\skill{Crafting}{crafting}

Used to make things, quite broadly.
This covers the creation of most kinds of objects, although some kinds of crafting are still covered by other skills, such as \skillref{brewing}.
Available specialities include the following:

\begin{itemize}
	\item Carpenter
	\item Cook
	\item Jeweller
	\item Mason
	\item Potter
	\item Seamstress
	\item Smith
	\item Woodcarver
	%TODO: Evaluate and maybe expand.
\end{itemize}

\skill{Deception}{deception}

Used to mislead, lie, prevaricate or filibuster, without anyone catching on that you're doing it.
Many witches make it a rule not to lie.
That doesn't mean they always need to tell the whole truth, so this can still be a useful skill for them.

\skill{Healing}{healing}

Used to bind wounds, set bones, diagnose diseases and deliver children.
This covers first aid, extended care and even surgery.
It does not cover the use of herbs, poultices or potions; these fall under \skillref{botany} and \skillref{brewing}.
It can be used to diagnose a patient's sickness in the first place, however: an essential step in applying the correct potion.

\skill{Insight}{insight}

Used to read people, as individuals or crowds.
This can include judging people's attitude and confidence, telling when and why they're uncomfortable, picking up on tells that they're lying, or predicting whether an argument is likely to come to blows.
It can be particularly useful for guessing at people's levers and buttons when preparing to manipulate them.
As a skill that relies on social understanding, it is normally rolled with \attref{charm}.

\skill{Intimidation}{intimidation}

Used for making threats: anything from subtly suggesting that you know a secret somebody would rather wasn't public knowledge, to outright yelling that you'll break the bugger's knees if he doesn't sit down and shut up \emph{right now}!
This doesn't even have to involve speaking; turning half a mob into frogs can certainly discourage the rest from tangling with you.

\skill{Lore}{lore}

Used to know and recall assorted knowledge, such as history, geography and religious doctrine. %TODO: Double-check our stance on religion.
Many fields of knowledge, such as magic and \skillref{botany}, fall under their own skills; this covers those that don't.

\skill{Perception}{perception}

Used to see, hear or smell things.
This includes noticing things that are out of place, such as hearing someone sneaking up behind you or spotting that your hat is missing from its peg.
It also covers active attempts to discern things, such as picking out details on someone at the other end of a street, eavesdropping, or identifying a faint smell.
Lastly, this is the skill used when trying to follow the trail of an animal or person.

\skill{Performance}{performance}

Used to entertain, amuse or impress people, or perhaps just to distract them.
Not everything that entertains people must use this skill; people can easily be entertained by a show of a magic or a swordfight, which might use another skill.
\skillref{performance} covers things done primarily for entertainment.
Available specialities include the following:

\begin{itemize}
	\item Dancer
	\item Drummer
	\item Harper
	\item Joke-teller
	\item Piper
	\item Storyteller
	%TODO: Evaluate and maybe expand.
\end{itemize}

\skill{Persuasion}{persuasion}

Used to influence or convince a person or crowd: to make them believe a particular thing or act in a particular way.
This can be through subtle suggestion and manipulation, or through reasoned, logical argument.
If you're attempting to persuade someone to act based on a falsehood, this might require both a \skillref{deception} Test to avoid being caught in the lie, and \skillref{persuasion} Test to motivate them to act.

\skill{Socialising}{socialising}

Used to befriend people, mingle with them, build rapport, and get into their good graces.
A good socialiser is everybody's best friend within a few minutes of meeting them, and might be trusted with secrets people would never otherwise give up.
Additionally, the GM may call for a \skillref{socialising} Test to determine how well you know a member of your steading or a nearby one.

\skill{Stealth}{stealth}

Used to do things without being noticed, such as sneaking up behind someone, peeking out through a bush, or lifting a guard's knife from his belt.
You can even try to blend in with a crowd (take the Hat off first), or rifle a man's purse while he watches your other hand.

\skill{Weaponry}{weaponry}

Used for everything from stabbing people with a concealed knife to clonking them over the head with a hefty staff, or even slugging them with a mean right hook.
Also used for throwing things, or shooting them with a bow.

\subsection{Discipline Skills}
\seclabel{discipline-skills}

\skill[brewing]{Brewing}{brewing}

Used to brew tinctures, tonics, elixirs and other potions.
This doesn't always require a cauldron: it also covers mixing poultices and the like.
Of course, you can also make booze.

\skill[divination]{Divination}{divination}

Used to see the past and future, and places many miles away.
It's not limited to seeing either; a diviner can eavesdrop on a conversation in the next village, or track a person better than any bloodhound.

\skill[broomcraft]{Flying}{flying}

Used by a witch on a broomstick, whether she's settling in for a cross-country flight, showing off with a barrel roll, or pulling a stalled stick out of a deep dive.
This is also the skill used for feats of flying by a winged familiar.

\skill[golemancy]{Golemancy}{golemancy}

Used to will life into inanimate creatures of clay, or other materials.
A skilled golemancer can make more golems, make them smarter, and, of course, force life into increasingly substandard bodies.

\skill[necromancy]{Necromancy}{necromancy}

Used to pervert the natural order and bring the dead back to life, or at least commune with them from beyond the Veil.
Also used to send them on again, if hitting them over the head with a big stick won't suffice.

\begin{simpletable}{rX}
	\toprule
	TN & Example Task\\
	\midrule
	9 & Discerning the power of the \discref{necromancy} animating a shambling corpse.\\
	12 & Identifying the purpose of a necromantic rite from the chalk circle left behind.\\
	15 & Filtering the true facts about vampires from the baseless rumours that surround them.\\
	18 & Discerning the power of the \discref{necromancy} that previously animated a no-longer-shambling corpse.\\
	21 & Performing a complex necromantic ritual using nothing but two small sticks and a fresh egg.\\
	\bottomrule
\end{simpletable}

\skill[sympathetic-magic]{Sympathetic Magic}{sympathetic-magic}

Used to manipulate people or things using effigies, poppets or other imitative talismans.
The idea of \discref{sympathetic-magic} is that one can't affect the imitation of a thing without affecting the thing itself.

\skill[willing]{Willing}{willing}

Used to force the universe to fall into line with what you know is true.
There is a real knack to convincing yourself of something well enough to make this work, and this skill governs how good you are at it.


\chapter{Familiars}
\chaplabel{familiars}

A wizened old woman leans back in her rocking chair, eyes closed.
A white cat lies curled in her lap, its own eyes also shut, purring as she rubs its chin.

A handsome, tanned woman stands on the peak of a grassy hill, arm held aloft.
A falcon dives from above, alighting on her thick leather glove.
It casts its eyes north-west, then knowingly back at the witch.
With a sly grin, she casts the bird back into the air and strides downhill after it.

A bright-eyed girl, no more than thirteen, stands beside a bubbling cauldron, carefully teasing the seeds from a pine cone with a small knife.
``Sage leaf next, Harold?''
She looks up at the frog on the kitchen bench, as it croaks and nudgs one of the piles of herbs that surrounds it.
``Ohh, right. Rosemary. Of course{\dots}''
The girl shakes her head and tuts to herself as she counts out seven leaves into her hand and drops them in the cauldron.
Harold peers over from the bench, keeping a close eye on the brew as it slowly turns a deep blue.

\section{No Mere Beast}

A witch's familiar is no mere animal.
It is a fusion of summoned spirit, tamed beast, and a tiny sliver of soul from the witch herself.
Obtaining a familiar is one of the first steps for any witch-in-training, and the familiar often aids in the witch in her subsequent learning.

Familiars are intelligent creatures, in some cases even more intelligent than the witches they are bound to.
They understand language, though the limits of animal form mean that most are incapable of speech.
Despite this, the bond that a witch shares with her familiar allow them to communicate.
With simple looks and gestures a familiar can communicate great meaning to its witch, communicating as effectively as if through speech.
This ability does not extend to other witches, and especially not to layfolk, who may require a Test to interpret a familiar's communication.
Pointing and beckoning are typically fairly unambiguous, however.

A witch's communication with her familiar allows her to lean on its expertise when her own is lacking.
A witch may use her familiar's ranks in a skill in place of her own, as long as the Test takes at least a minute, and she can confer with her familiar through the duration.

\section{Binding a Familiar}

Binding a familiar takes place in a simple ritual, achievable by even the most novice witch, though often performed under direct tutelage.
The spirit to be bound to the familiar can be obtained in a number of ways: a lesser demon under contract, an amenable nature spirit or a spirit summoned from beyond.
It is not uncommon for a witch to use the spirit from the familiar of her teacher's teacher, or of an ancestor if witchcraft runs in her family.
The animal to become the familiar must be tamed by the witch, at least enough that it willing remains by her side throughout the ritual.
Many witches find this to be the hardest part of the ritual, and it means that some animals make for rather rare familiars.
Lastly, the witch must offer up a sliver of her own soul, to seal the bond.
She does so by feeding the familiar animal a drop of her own blood.

Upon completion of the ritual, the spirit and animal are fused to form a new entity, the familiar.
It takes on personality elements from both and the body of the animal.
Slight changes to its physical form often manifest, however, such as a coat that always remains strangely glossy, a slight chill to the touch, or sharper, whiter teeth.
Changes in eye colour are especially common.
Lastly, the sliver of the witch's soul included in the creation of a familiar also influences its personality.
It ensures that, although a witch and her familiar may not always get along, and may certainly disagree on the best way to go about something, they will always have one another's best interests at heart.

\section{Creating a Familiar}

From the perspective of character creation, there are many things to bear in mind when creating a familiar.
While the familiar is unlikely to take the foreground as much as the witch herself, they are still a character in their own right, and should be designed as such.

The most important decision is the form the familiar will take, the animal they were created from.
This determines the familiar's attributes, skills and abilities.
Note that familiars, as non-human characters, may have attributes below the human 0 to 5 range.

Beside its game statistics, it is also important to get an idea of your familiar as a character.
Try answering some of the following questions.

\begin{itemize}
	\item What is your familiar's name?
	\item Is your familiar male or female?
		Do you not know?
	\item At what stage in her life, and her training, did your witch bind her familiar?
	\item Which sort of spirit was used in the binding? %TODO: Remove this if it becomes mechanical.
	\item Do your witch and her familiar get along?
		Do they engage in playful banter?
		Philosophical debate?
	\item Does your familiar have any quirks, physical or mental?
\end{itemize}

Lastly, it is important to decide whether each familiar will be played by the player or the GM.
Both are valid, but if the GM is playing familiars they should typically act in their witch's best interests.

\section{Familiar Injury and Death}
\seclabel{familiar-injury-death}

Familiars suffer {\damage}, {\shock} and death just like other characters.
A witch is always aware when her familiar dies, feeling it as a searing pain in her very soul.

It is possible to recover a deceased familiar through a slight variant on the original binding ritual.
The familiar's spirit comes willingly, but another animal of the same kind must be provided.
The familiar, once reformed, may take either the new animal's appearance or its original one.

Repeating the ritual takes another sliver of the witch's soul, provided through another drop of blood.
As such, recovering a deceased familiar costs 10 XP every time.

\section{Familiar Animals}

A list of the types of animal available as familiars is presented below, along with the attributes, skills and abilities of the familiar.
Besides the abilities listed below, the players and GM are encouraged to apply common sense.
For instance, familiars lack thumbs and will struggle with door handles, and a weasel can squeeze through a smaller hole than a hound.

If you would like your familiar to be an animal not presented on the list below, discuss your option with your GM.
It might be possible to design a new familiar for you to use, or to use the statistics of a familiar presented here to represent something else.
Note that familiars are fairly small animals; the exclusion of anything larger than a medium-sized dog is intentional.

Many types of familiar, more powerful ones, come with an associated XP cost.
This is deducted from the witch's starting XP.
Some types of familiar also come with options which may be purchased for an additional XP cost.
These represent inherent differences in the animal used and must be purchased at the same time your familiar is created.
You may only select one option; they are mutually exclusive.

Lastly, bear in mind that some feats that can be purchased later depend upon particular types of familiar, and your familiar's later development is limited by its form.
As such, it can be worth taking a quick look at other feats you may be interested in taking when selecting your familiar.
%TODO: If those are all in a discipline chapter on familiars, direct people there.

\familiar{Cat}{cat}{15}{
	\atttable{\negative 1}{3}{2}{2}{2}{2}{3}{1}
}{
	\skillref[1]{athletics}, \skillref[1]{deception}, \skillref[1]{perception}, \skillref[2]{stealth}
}{
	Graceful and charming on the outside, cats can be incredibly sly and manipulative underneath.
	Just like many witches.
}{
	\familiarability{Natural Acrobat}{
		The cat rolls an extra die on Tests to retain its balance, land on its feet, or avoid damage from falling.
		%TODO: Reduce fall damage in some more definite fashion?
	}
	
	\familiarability{Claws}{
		The cat's unarmed attacks deal 4 dice of damage.
	}
}{}

\familiar{Dog}{dog}{15}{
	\atttable{1}{1}{1}{1}{3}{2}{0}{2}
}{
	\skillref[2]{intimidation}, \skillref[2]{perception}, \skillref[1]{weaponry}
}{
	A man's best friend, and often a witch's too.
	Dogs are a diverse lot, including hunting dogs, sheepdogs, sled dogs and more.
}{
	\familiarability{Bite}{
		The dog's unarmed attacks deal 5 dice of damage.
	}
}{
	\familiaroption{Scenthound}{5}{
		The dog rolls an extra die on \skillref{perception} skills relying on smell.
	}
}

\familiar{Ferret/Stoat/Weasel}{mustelid}{15}{
	\atttable{\negative 2}{3}{1}{1}{2}{2}{2}{0}
}{
	\skillref[1]{athletics}, \skillref[2]{stealth}, \skillref[1]{weaponry}
}{
	A ferret, stoat, weasel, polecat, ermine, mink or marten.
	Despite their small size, these creatures are ferocious predators.
	Their long, narrow bodies allow them to invade the burrows of much smaller animals, or the trousers of their witch's unfortunate foes.
}{
	\familiarability{Bite}{
		The ferret's unarmed attacks deal 4 dice of damage.
	}
	
	\familiarability{Slippery}{
		The ferret's \statref{dr} is increased by 2.
	}
}{}

\familiar{Frog/Toad}{frog}{5}{
	\atttable{\negative 2}{\negative 1}{1}{1}{2}{0}{\negative 1}{\negative 1}
}{
	\skillref[1]{brewing}
}{
	Frogs and toads make excellent companions to brewing witches, due to their natural affinity with water.
	Particularly with some of the stuff that gets into the murkier ponds around{\dots}
	
	It is important to try and keep their skin moist, but maybe refrain from dropping them in the cauldron.
}{
	\familiarability{Amphibian}{
		The frog can breathe underwater.
	}
	
	\familiarability{Leapfrog}{
		The frog can jump at least 3 metres from a standing start.
		It rolls an additional die on Tests made to jump.
	}
	
	%TODO: Something about keeping the skin moist? Maybe when there are exhaustion/fatigue rules.
}{}

\familiar{Raptor (Eagle/Falcon/Hawk)}{raptor}{25}{
	\atttable{\negative 1}{3}{1}{2}{2}{3}{\negative 1}{2}
}{
	\skillref[2]{flying}, \skillref[2]{perception}, \skillref[1]{weaponry}
}{
	A raptor is a buzzard, eagle, falcon, harrier, hawk, kite or osprey; a bird of prey.
	They are excellent fliers, have keen eyesight, and nobody would want to tangle with their wicked beak and talons.
}{
	\familiarability{Eagle Eyes}{
		The raptor rolls an extra die on \skillref{perception} Tests to see things at a long distance.
	}
	
	\familiarability{Beak \& Talons}{
		The raptor's unarmed attacks deal 4 dice of damage.
	}
}{}

\familiar{Rat/Mouse}{rat}{0}{
	\atttable{\negative 2}{1}{1}{1}{2}{1}{\negative 2}{\negative 1}
}{
	\skillref[1]{stealth}
}{
	The rat is a rather widely reviled animal, but it's certainly easy for a new witch looking for a familiar to find one.
	And it can get into smaller places than a cat or bird, which often proves helpful.
}{
	\familiarability{Filth-Liver}{
		The rat rolls an extra die on Tests to resist poison or disease.
	}
	
	\familiarability{Keen Smell}{
		The rat rolls an extra die on \skillref{perception} skills relying on smell.
	}
}{}

%\subsection{Beaver}

%Placeholder.

%\subsection{Bat}

%Placeholder.

%\subsection{Badger}

%Placeholder.

%\subsection{Chicken}

%Placeholder.

%\subsection{Crow/Raven}

%Placeholder.

%\subsection{Dove}

%Placeholder.

%\subsection{Fox}

%Placeholder.

%\subsection{Gecko}

%Placeholder.

%\subsection{Goose}

%Placeholder.

%\subsection{Hamster/Gerbil/Guinea Pig}

%Placeholder.

%\subsection{Hedgehog/Porcupine}

%Placeholder.

%\subsection{Lemming}

%Placeholder.

%\subsection{Lizard}

%Placeholder.

%\subsection{Mole}

%Placeholder.

%\subsection{Otter}

%Placeholder.

%\subsection{Owl}

%Placeholder.

%\subsection{Parrot}

%Placeholder.

%\subsection{Pigeon}

%Placeholder.

%\subsection{Rabbit/Hare}

%Placeholder.

%\subsection{Salamander/Newt}

%Placeholder.

%\subsection{Seabird}

%Placeholder.

%\subsection{Snake (Constrictor)}

%Placeholder.

%\subsection{Snake (Venomous)}

%Placeholder.

%\subsection{Songbird}

%Placeholder.

%\subsection{Spider}

%Placeholder.

%\subsection{Squirrel}

%Placeholder.

%\subsection{Swan}

%Placeholder.

%\subsection{Tortoise}

%Placeholder.

%\subsection{Vole/Shrew/Gopher}

%Placeholder.

%\subsection{Woodpecker}

%Placeholder.


\chapter{Tools of the Craft}
\chaplabel{equipment}

\section{The Hat}

A witch's pointed hat is the most important of her tools, in many regards.
There are no particular rules about the hat; its effects are left up to the GM.
But it always has an effect on people.
It may make them angry, reverent, reassured or afraid, but most importantly it makes sure they know that they are in the presence of a witch.

A witch's hat says a lot about her, particularly to other witches.
When you create your character, you can answer the following questions about your hat.

\begin{itemize}
	\item Did you make it yourself?
	\item How tall is it?
	\item Is it the traditional black, or some other colour?
	\item How long have you had it?
		Is it visibly worn?
		Well cared for?
	\item Is it plain, tastefully decorated, or covered in stars and sequins?
	\item Does it have any useful accessories?
		Pockets?
\end{itemize}

Many witches accompany their hats by a black cloak or other such attire.
Opinions on occult jewellery are mixed: some witches wear masses, others frown on it heavily.



\section{Broomsticks}

Sometimes, walking from one village to another just takes too long.
A lot of witches---to maintain their mystique or simply because the townsfolk wouldn't be happy otherwise---even choose to live quite a way from the nearest village.
Such circumstances make a broomstick an essential accessory for any witch.

Broomstick flight is no mean feat and while every witch picks up the rudiments, most can use it for nothing more than getting from A to B.
The broom needs a running start, has to be ridden sidesaddle, and has a turning circle several hundred metres across.
Detailed rules for flying a broomstick can be found in \chapref{broomcraft}.

Before it can be used, a broomstick needs to be trained to to fly.
This requires someone to fly it around on another broomstick so that it can learn its craft from one of its fellows.
It must be held parallel to the broom being ridden, to ensure it learns to fly in the correct direction.
The process takes about eight hours.
These hours need not be consecutive, but should all be done within a couple of weeks.
Once trained, a broom retains its flight skill for a long time.
Taking it out for a few hours each year is enough to keep its hand in.

At character creation, every witch is assumed to own a trained broomstick one way or another.
It was probably trained using the broom of whoever taught her witchcraft, at least if she's still using their first broom.
It might feel like an old friend at this point, the witch familiar with every knot and notch in its handle.
A more careless witch might have gone through a few brooms during her career.



\section{Common Magical Components}

The various rites and magics of the various disciplines of witchcraft require too many different materials to enumerate here.
However, there a few components that make a regular appearance.
Some details of their acquisition, construction and use are given here.

\subsection{Ritual Circles}
\materiallabel{Ritual Circle}{Ritual Circles}{ritual-circle}
\circlelabel{Small}{small}
\circlelabel{Medium}{medium}
\circlelabel{Large}{large}

A ritual circle describes any large arrangement of symbols or shapes required by a rite.
They are traditionally drawn on the floor in chalk, but other methods are far from uncommon; the visibility and accuracy are the only important aspects for most rites.
Some witches use paint for permanence, or even chisel their circles into stone.
Many a witch in a hurry has scratched their circles into the dirt with the toe of their boot.
Some witches even embroider their circles upon sheets of fabric that can be rolled up and laid down where needed.
However, a roll bearing even the smallest of circles is most of the height of a man.

Each rite requires a ritual circle of a particular design, different for every rite, but the same each time the rite is performed.
This means that scribing a circle just once and using it for many performances of the rite is a common practice.
Ritual circles are not even universally circular, although it is the most common shape and almost all have some sort of symmetry.
Squares, triangles and hexagons are not uncommon, and pentagrams are particularly common in certain disciplines.

Ritual circles are classified primarily by their size.
A rite can be performed with a larger circle than it requires, unless specified otherwise.
\begin{itemize}
	\item A \circleref{small} can be scribed entirely in arm's reach while standing in one spot.
		It can comfortably be drawn in a couple of minutes.
	\item A \circleref{medium} is a few paces across.
		Most houses should have a room large enough to draw one in, if the furniture is moved.
		It can comfortably be drawn in a quarter of an hour.
	\item A \circleref{large} is at least two dozen paces across.
		A ballroom or village hall is probably the only place one could be drawn indoors, so most are drawn outside.
		At least a couple of hours are required to draw such a circle without haste.
\end{itemize}

\subsection{Megalithic Circles}
\materiallabel{Megalithic Circle}{Megalithic Circles}{stone-circle}

Some rites require a circle of standing stones, called a \materialref{stone-circle}.
Such a circle must be at least the size of a \circleref{large}, with at least a dozen stones each taller than a man.
The arrangement and shape of the stones is unimportant, as long as it is recognisable as a ring of standing stones, and so the same circle can be used for all rites that require one.
Constructing a \materialref{stone-circle} is no easy task, typically requiring weeks of work by much of a village, even if the site is quite close to a stone quarry.

\subsection{Taglocks}
\materiallabel{Taglock}{Taglocks}{taglock}

A \materialref{taglock} is any part of a person's body, such as a piece of flesh, a strand of hair, a nail clipping, a drop of blood, or a gob of saliva.
It is often used to bind a spell to a particular target.
It can always be picked off a person---although taking a hair without being noticed might be difficult---but people often leave \materialrefplural{taglock} behind them, especially in places they frequent.
Finding a \materialref{taglock} in a place you suspect someone might have left on, such as their house or a bed they've slept in, typically uses \skillref{perception}.

\subsection{Poppets}
\materiallabel{Poppet}{Poppets}{poppet}

A \materialref{poppet} is an abstract representation of a person, although not a particular person.
Voodoo dolls are a typical example.
A \materialref{poppet} can be crafted from cloth, wood, clay, wax or other suitable material.
It should be recognisable as a human, bearing four appropriately-arranged limbs, a head, and two eyes.
However, if it is to be used in \discref{sympathetic-magic} affecting a non-human creature, it should resemble whichever creature the magic is intended to affect.
A \materialref{poppet} should be at least a handspan tall, though can be much larger.

\subsection{Effigies}
\materiallabel{Effigy}{Effigies}{effigy}

An \materialref{effigy} is much like a \materialref{poppet} and follows all the rules for one, except that it represents a particular person and must be crafted in their likeness.
It can be used only to affect the person it resembles.
Ideally, an \materialref{effigy} should be recognisable to even passing acquaintances of the person it is supposed to represent.
Less recognisable \materialrefplural{effigy} will require an appropriate Test to be used for magic.

\subsection{Blood}
\materiallabel{Blood}{}{blood}

Many spells call for \materialref{blood}, in varying quantities and from various creatures.
Extracting a mere drop of \materialref{blood} carries no ill effects.
Furthermore, a willing or restrained donor can provide \SI{100}{\milli\litre} of \materialref{blood} per point of \attref{might} with no ill effects, approximately once per week.
Above that, every \SI{100}{\milli\litre} of \materialref{blood} extracted deals 1 point of \secrefraw{damage}.
Creatures damaged in combat, by edged weapons, will also spill blood; approximately \SI{100}{\milli\litre} per point of \secrefraw{damage} dealt.
This blood will typically be lost in the dirt, however.

The above guidelines apply to humans, who typically contain about approximately 5 litres of blood in total.
Differently sized creatures will provide appropriately more or less blood.


\section{Herbs and Gardens}
\seclabel{herbs}
\herblabel{Ubiquitous}{1}
\herblabel{Common}{2}
\herblabel{Uncommon}{3}
\herblabel{Rare}{4}
\herblabel{Extraordinary}{5}

Herbs are an important component of most potions and poultices, as well as many spells.
Note that the term ``herb'' is used to encompass many things that are not technically herbs at all, such as fruit, fungi, tree bark and so on.

There are hundreds, perhaps thousands of different herbs, so instead of tracking every one, they are simply divided into categories based upon rarity.
Each potion or spell lists the highest rarity of herb that it requires.
Actual identities of the herbs may also be given, and can be used as guidelines to help improvise alternative spell components.

Herbs are classified as ubiquitous, common, uncommon, rare or extraordinary.
This covers both how common the herb is in the wild, and how difficult it is to cultivate in a garden.

\subsection{Finding Herbs}

Finding herbs growing in the wild uses \testtype{heed}{botany}.
A successful Test to find herbs provides enough for a few potions or poultices, or a few castings of a spell.
Under normal circumstances, this should be enough for the task at hand, or to restock a witch's supply.
But if the witch is trying to brew a potion for everyone in a castle, this supply might not cut it.

\begin{itemize}
	\item Ubiquitous herbs are incredibly easy to find and require no Test.
		They are primarily weeds that grow just about everywhere, whether people want them to or not.
		They should almost never take more than five minutes to find as long as you're outside, and even less if you're in a field or forest.
	\item Common herbs typically require searching a few hedgerows.
		They'll certainly turn up in an hour, and can be found much faster with a relatively easy Test.
	\item Uncommon herbs might require searching a large swathe of forest to even turn up one plant.
		This takes at least an hour, typically more, and requires a difficult Test.
	\item Rare herbs might not be found by searching an entire forest.
		Performing such a search exhaustively is infeasible, but a skilled botanist knows how to look in the right places.
		Still, this can take an entire day and requires a very difficult Test.
	\item Extraordinary herbs are not found in the wild under any but the most exceptional circumstances.
		They typically need to be traded from far-away places.
\end{itemize}

\subsection{Cultivating Herbs}

Many witches, especially brewers, keep a garden for growing herbs.
This is typically outside their cottage, but can be anywhere she likes.
However, tending a compelte garden requires about eight hours of work every week, which makes maintaining more than one a rather time-consuming task.
If the garden is left unattended for more than a week, it can require an even more considerable effort to get it back into shape.
The particularly needy or unruly herbs might die off, or even escape, during this time.

A garden is assumed to be accompanied by some storage of the herbs, so even herbs that need to be harvested at a particular time are available when they are needed.
A garden can provide an even greater supply of herbs than a search in the wild can, and may just about cut it to brew one potion for everyone in a castle.
But it can still be overtaxed, and this sort of thing shouldn't be tried too often.

Anyone can grow common or ubiquitous herbs in a garden.
As far as the weedy ubiquitous herbs go, most of the effort goes into keeping them under control.
Rarer herbs are harder to cultivate, however, requiring a skilled botanist (or a botanist with a skilled familiar).
%Uncommon herbs require Botany 1, rare herbs Botany 2 and extraordinary herbs Botany 3.
The following table gives the Botany skill required to cultivate a herb.

\begin{simpletable}{llX}
	\toprule
	Rarity & Skill & Examples\\
	\midrule
	Ubiquitous & - & Dock, Nettle, Clover\\
	Common & - & Lavender, Rosemary, Elderberry\\
	Uncommon & 1 & Tomato\\
	Rare & 2 & Truffle\\
	Extraordinary & 3 & Mandrake, Triffid\\
	%TODO: Expand
	\bottomrule
\end{simpletable}



\section{Improvised Tools}



\section{Weapons}
\seclabel{weapons}

Weapons are divided into several broad categories.
Players are free to describe their character's weapons how they wish, within the bounds of reason, placing them in one of the categories.
Anything a character might find at hand and hit people with can also be placed into a category.

A weapon's accuracy is added a flat bonus to rolls to hit, in place of an attribute.
A weapon's damage determines the number of dice rolled upon hitting.
The highest 3 dice are kept, as always, but the number of dice rolled are determined by the weapon instead of the wielder's skill.
The wielder's \attref{might} is added to the damage roll for melee or thrown weapons, but not for bows.

\begin{simpletable}{X[2.4]XXX[1.3]}
	\toprule
	Weapon & Accuracy & Damage & Range (metres)\\
	\midrule
	Fist & +2 & \dice{2} & Melee\\
	Club & +4 & \dice{4} & Melee\\
	Knife & +2 & \dice{5} & Melee\\
	Hand Weapon & +4 & \dice{5} & Melee\\
	Thrown Rock & +0 & \dice{2} & $5\times\text{\attref{might}}$\\
	Thrown Weapon & +0 & \dice{4} & $5\times\text{\attref{might}}$\\
	Bow & +2 & \dice{5} & 100\\
	\bottomrule
\end{simpletable}

\subsubsection{Fist}
A punch, a kick, or a headbutt.
Covers any attack you make without any weapon at all.

Some animals and familiars will have a different number of dice for their unarmed attacks, but use the same accuracy bonus unless this is also specified.

\subsubsection{Club}
A club, a walking stick, a chair, or a cauldron.
A club is just about anything you pick up and hit someone with.

\subsubsection{Knife}
A knife or dagger.
Easily concealed, and a staple of blood witches.
The short blade costs the wielder reach, but can do as much damage as a sword if you get the enemy in the tender parts.

\subsubsection{Hand Weapon}
A sword, an axe, a mace, a spear, a pike.
This category covers most things actually designed as a weapon and larger than a knife.

\subsubsection{Thrown Rock}
A genuine rock, but also a teapot, a boot or a frog.
Anything you might pick up and throw.
This includes weapons that aren't designed to be thrown.

\subsubsection{Thrown Weapon}
A spear, a knife, a hatchet.
Any weapon you can throw that was actually designed for the purpose.
Rocks from slingshots fall in this category too.

\subsection{Bow}
A bow and arrow.
Also covers crossbows, if the setting includes them.


\part{Playing the Game}
\partlabel{rules}

\chapter{The Broad of It}
\chaplabel{general-rules}

This chapter covers rules essential to day-to-day play.
Players and GMs alike should be familiar with at least the major points in here in order to play.
More specific rules, pertaining to various disciplines of magic, can be found in the appropriate chapters of \partref{disciplines}.

It is important to remember that this book cannot cover every situation that may arise during play.
The role of the GM includes adjudicating such scenarios, and the following section should contain guidelines to assist in that.
Furthermore, it is often helpful to do the same when the players simply cannot remember a rule, to avoid slowing down play while someone looks it up.
And lastly, remember that all the rules contained in this book are guidelines and suggestions.
Feel free to change them all that you want!
The most important thing is that everyone is having fun.

\section{Rounding Fractions}

In general, round down whenever you get a fraction, even if the fractional part is one half or greater.

\section{Tests}

Tests are the dice rolls used to determine the outcome of an action when there is element of chance and risk involved.
Several of the rules in this chapter and others will specify the appropriate Test to make with a particular action, but the GM should be calling for other kinds of Tests whenever appropriate as well.

A Test is typically made with a skill and an attribute, although having no applicable skill is not uncommon.
Often, the rule that required the Test specifies these.
Otherwise, the GM chooses as appropriate.
The character's skill determines how many dice she rolls for a Test.
If there is no skill applicable to the test, or if the character has no ranks in the applicable skill, she rolls 3 dice.
Each rank in the skill gives an additional rolled die, to a maximum of 6 with all three ranks.
Total together the highest 3 of the rolled dice and add the character's relevant attribute to this total.
The final total is compared against a Target Number (TN) set by the GM: if it meets or exceeds the TN the Test succeeds; otherwise it fails.

A Test where every die shows a 1 or 2 is a critical failure, and a Test where all 3 kept dice show a 6 is a critical success.
In addition to the Test automatically succeeding or failing, the GM is encouraged to apply an additional drawback or benefit to the result of the Test.
Critical failures on Tests involving dangerous magic can be especially catastrophic.

\subsection{Dice Notation}

A variant of standard RPG dice notation is used for Tests.
The size of the dice and the fact that only three are kept is omitted, as these are constants.
For example, \dice{4} indicates a 4 die Test with no bonus, and \dice[2]{3} indicates a 3 die Test with an attribute bonus of 2.

\subsection{An Example Test}

As an example, suppose Mistress Talbot is peering out of her window and attempting to identify which manner of undead dog has just shambled into her garden.
The GM declares this to be a \testtype{ken}{necromancy} Test, as she is attempting to recall information about the undead.
Mistress Talbot dabbled in \discref{necromancy} as a youth, and has one rank in the skill, so she rolls 4 dice.
However, her memory has begun to fade with age, so she has only 1 \attref{ken}.
The four dice show 4, 6, 2 and 3.
Her player totals the three highest dice, the 6, 4 and 3, for 13.
Then she adds Mistress Talbot's \attref{ken}, 1, for a grand total of 14.
Her player announces the total to the table.

The GM knows that the dog is a simple zombie, the most common variety of undead, but it was killed and raised only yesterday so the characteristic rot hasn't properly set in yet.
In light of this, she assigns a Target Number of 12: not too easy, but not particularly difficult either.
Hearing Mistress Talbot's total of 14, the GM knows that she has met the TN of 12: the Test has succeeded.
She announces that Mistress Talbot, by the creature's glassy eyes and stumbling gait, realises the midnight intruder is merely a zombie.
Reassured---she'd been fearing a ghoul or a hellhound---Mistress Talbot heads outside to see what the beast wants.
Though not without grabbing the poker from beside the fireplace, just in case.

\subsection{Target Numbers}

A Target Number (TN) represents the difficulty of the action that requires a Test.
The more difficult the action, the higher the target number, and the less likely the Test is to succeed.
In some situations, the same rule that requires a Test will specify its TN.
In other situations, the GM should select a TN she feels is appropriate.

Typical TNs range from approximately 9 to 21.
A Test with a TN lower than 9 is not normally worth it: a character with no skill and an average score in the relevant attribute will succeed more than \SI{95}{\percent} of the time.
Similarly, a Test with a TN higher than 21 is not normally worth it: a character needs a 5 in the relevant attribute to succeed without a critical success.
The following table shows a brief summary of the sorts of task particular TNs are suited to.

\begin{simpletable}{rX}
	\toprule
	TN & Task Difficulty\\
	\midrule
	9 & Easy: An average, unskilled person would normally manage this.\\
	12 & Moderate: An average, unskilled person would manage this about half the time.\\
	15 & Challenging: It takes skill to pull this off consistently.\\
	18 & Difficult: Even a skilled person is unlikely to achieve this consistently.\\
	21 & Legendary: This takes great skill, ability and good luck to perform.\\
	\bottomrule
\end{simpletable}

Instead of assigning a simple pass-or-fail TN, the GM may also employ graded success.
This is when a higher roll gives a higher level of success.
For instance, a higher roll on a Test to recall knowledge might mean that the character recalls more knowledge about the situation, while a higher roll on a check to influence a crowd might influence a greater proportion of the crowd.
This can also be used to apply success at a cost, where an intermediate roll, neither particularly high nor particualarly low, means that the character succeeds at their task but incurs some drawback in doing so.
For example, a coven might try to intimidate a guard to allow them into the castle.
Failure could indicate the guard calls for backup and resists, while a very high result on the Test would mean he is cowed and allows them to pass.
an intermediate result might mean that he allows the coven to pass, but sneaks off to find reinforcements and confront them later, while they are inside the castle.

\subsection{Using Tests}

Be careful not to call for a Test when it's not necessary.
If the action is a simple one that the character should be able to routinely perform, such as walking through a door or ransacking a room for something that isn't hidden, it doesn't require a Test.
(However, what is routine for one character might not be for another; a closed door can present a serious obstacle to many familiars.)
If the action is impossible, such as jumping over the moon or convincing the King to give up his crown without solid leverage, the player shouldn't make a Test.
If the character wouldn't succeed even with a critical success, a Test should never be rolled.
Lastly, if there is no penalty for failure, there is no need for a Test.
If the character will keep on trying until she succeeds, there's no need to make the player keep rolling Tests.

\subsection{Rolling Fewer Than Three Dice}

Some effects will modify the number of dice a character rolls for a Test, and this can bring the number of rolled dice below three.
In this case, all the rolled dice are added to the total as normal, but the maximum total that can be reached is obviously reduced.
Additionally, critical success is no longer possible, as this require three dice showing 6.
Critical failure, however, becomes far more likely, as it only requires that all dice show 1 or 2.

If the number of dice rolled for a Test would be reduced to zero, the Test cannot be performed.
If it is unavoidable, it is automatically treated as a critical failure.

\section{The Flow of Time}

\subsection{Narrative Time}

During normal play, the exact timing and duration of characters' action are unimportant, and not carefully tracked.
It is enough to know whether something took a matter of seconds or minutes, an hour or two, or a couple of days.
This is Narrative Time, and the GM is free to be as accurate or as loose as necessary with time periods.

The one element of Narrative Time with an impact upon the rules is that of Scenes, which are often used to measure the duration of effects.
It may be helpful to think of scenes like in a play.
The Scene typically changes when the action changes location (everyone walks from the church down to the village green), when there is a timeskip (everyone waits an hour for the sun to set) or there is a change in the cast of characters (the preparations for the party finish and the guests begin to arrive).
Changes in Scene, and the duration of effects that rely on them, are ultimately left up to the GM, but should often be obvious.

\subsection{Structured Time}
\seclabel{structured-time}

In tense situations with two opposing parties, exact timings and durations become important to track.
For this purpose, and to aid tactical thinking in such scenarios, the GM can move the game into Structured Time.
Direct combat is perhaps the most common application of this, but chase scenes may also use them.
With the correct magic, some of the participants might even be many miles apart.

Structured Time is divided into {\rounds} and {\turns}.
Every character participating in the Scene gets one {\turn} each {\round}.
Although the {\turns} are resolved in some order, all characters are assumed to be acting simultaneously and continuously.
If it becomes particularly relevant for some reason, assume each {\round} takes approximately 10 seconds.

On each {\turn}, a character may move a number of metres equal to their \statref{speed} and take one {\action}.
An {\action} is something that requires most of the character's effort during their {\turn}, such as attacking someone, performing a brief bit of magic, knocking a hole in a wall or quaffing a potion.
They may also take a reasonable number of minor actions that shouldn't require their full concentration, such as opening or slamming a door, drawing a sword, pointing at something or speaking a short sentence.
Not everything can be accomplished in one {\action}.
For example, winching a drawbridge closed may take several {\actions}, as might even one of the faster magical rites.
Some of the {\actions} available to a character are given in \secref{combat-actions}, but the GM is free adjudicate anything the characters try as one or more {\actions}.

\subsection{Initiative}

When the GM determines that the game should move into Structured Time, Initiative Tests are used to determine the order in which participants take their {\turns}.
Initiative determines how quickly characters notice the situation and are ready to act.

Initiative Tests can use any attribute and skill appropriate to the situation, as determined by the GM.
For example, an argument that boils over into a brawl might prompt Initiative Tests using \testtype{heed}{insight}, favouring characters who noticed tensions rising and fists clenching.
Combat that begins as characters race to the source of a scream might use \testtype{grace}{athletics}, favouring characters who arrive fastest.
If nothing in particular seems appropriate, default to a \attref{grace} or \attref{heed} Test with no applicable skill.

The GM may even assign different Tests to different characters, with appropriate bonuses or penalties.
For example, suppose a group of bandits ambush for a group of travellers.
The bandits roll \testtype{grace}{stealth} to spring from hiding, with a \positive{6} bonus as the ambushers.
The travellers roll \testtype{heed}{perception} to notice the bandits attacking.
The GM may assign a greater or lesser bonus to a better-laid ambush, or one staged in a suboptimal location.

Initiative Tests are not made against a particular TN.
Rather, all characters are ranked in order.
This is the Initiative Order, and remains the same on subsequent {\rounds}.
The character with the highest result takes their {\turn} first, and subsequent {\turns} proceed down the Initiative Order.
Once all characters have taken a {\turn}, return to the top of the Initiative Order for the next {\round}.

To save time, the GM may make a single Test for a group of similar NPCs, such that they all get the same result and take their {\turns} at the same time.
Similarly, a witch's familiar and all other creatures associated with her (such as a horse she is riding, or her golems and undead) use the witch's Initiative result and take their {\turns} at the same time as her.

\section{Movement}

Each {\turn} in Structured Time, a character can move a number of metres equal to her \statref{speed}, as well as taking an {\action}.
If she takes the \actionref{dash} {\action}, she may move a total number equal to twice her \statref{speed}.
This assumes that she is moving on foot over smooth ground.
This speed represents urgent movement over a short period.
A character trying to maintain this pace for more than a couple of minutes typically requires a \testtype{might}{athletics} Test to avoid tiring.
%TODO: Reference exhaustion rules here?

%TODO: Long-range travel times.

%TODO: Climbing and swimming.

\subsection{Difficult Terrain}
\seclabel{difficult-terrain}

{\diffterrain}, such as dense forest or a bog, slows characters trying to move through it.
As a simple default, movement through it is halved; it costs 2 metres from a character's \statref{speed} to move through 1 metre of {\diffterrain}.
The GM is free to impose a lesser or greater penalty for more or less severe terrain.

For some kinds of {\diffterrain}, the GM may offer players the option to ignore the movement penalty at an alternative cost.
For example, a character pushing through brambles may move at full speed, but be subjected to a \seclink{Damage Test}{damage-tests} for doing so.
A character moving on slick ice or along a narrow ledge may move at full speed, but must succeed on a \testtype{grace}{athletics} Test to avoid falling over, or off the edge{\dots}

\section{Injury}

Witchcraft is a dangerous business.
Between mad spirits, evil demons, foul undead, and disgruntled mobs of villagers, injury is inevitable.
And it's not only her own injuries that a witch has to deal with.
One of a witch's duties is to tend to the injuries of her neighbours, nursing them back to health after an accident or disease has laid them low.
Or, when they are beyond her help, easing their final moments.

A character's resistance to injury is determined by two statistics: \statref{res} and \statref{st}.
Most creatures of flesh and blood, including humans and familiars, have $3$ \statref{res}.
Other creatures, such as golems, may be more or less resilient.
A character's \statref{st} is equal to 12, plus their \attref{might}, plus their \attref{will}.

\subsection{Damage Tests}
\seclabel{damage-tests}

A \seclink{Damage Test}{damage-tests} is a special type of Test used to determine how much an effect hurts a character.
It is made like a normal Test, by rolling some number of dice and adding the highest 3 together, with a flat bonus.
In the case of an attack by one character upon another, the number of dice are determined by the weapon used and the flat bonus by the wielder's strength.
In other cases, the GM or the rules of the damaging effect assign the number of dice and the bonus.
For small effects, this can often be fewer than 3 dice.

The following table provides examples of the number of dice and the bonus for \seclink{Damage Tests}{damage-tests}.

\begin{simpletable}{Xl}
	\toprule
	Effect & Damage\\
	\midrule
	Touching a hot cauldron & \dice{1}\\
	Crawling through brambles & \dice{2}\\
	Wave-tossed against a boulder & \dice{3}\\
	Hit by a falling brick & \dice{4}\\
	Falling on a sword & \dice{5}\\ %TODO: Evaluate and expand.
	Hit by a falling tree & \dice[4]{5}\\
	\bottomrule
\end{simpletable}

Additionally, a \seclink{Damage Test}{damage-tests} is not made against a particular TN like most Tests.
Instead, it applies two effects to the target, {\shock} and {\damage}.
{\shock} is always tested for before {\damage} is applied.

Critical failure on a \seclink{Damage Test}{damage-tests} means no effect is applied at all; the blow was glancing and won't do more than bruise slightly.
Critical success on a \seclink{Damage Test}{damage-tests} may immediately kill the target or leave them with a lasting injury, at the GM's option, and always applies {\shock}.

\subsection{Shock}
\seclabel{shock}

If a \seclink{Damage Test}{damage-tests} meets or exceeds the target's \statref{st}, or critically succeeds, the target goes into {\shock}.
A character in {\shock} falls unconscious and cannot be roused while they remain in {\shock}.
If a character in {\shock} would go into {\shock} again due to another \seclink{Damage Test}{damage-tests}, they die.

Additionally, at the start of each of the {\shocked} character's {\turns}, roll a special Test against them.
This Test applies no flat bonus, and uses the same number of dice as the \seclink{Damage Test}{damage-tests} that sent the character into {\shock}: a character is more likely to bleed out from a sword wound than a punch.
If it meets or exceeds the {\shocked} character's \statref{st}, they die.
This Test is not considered to be a Test made by any character.

If the Test made every {\turn} ever totals 9 or less, unless it also meets or exceed their \statref{st}, the character is no longer in {\shock}.
However, the character remains unconscious and cannot be naturally roused before the end of the Scene.
A character can also be brought out of {\shock} by another character tending to them.
This requires an {\action} and a successful \testtype{ken}{healing} Test.
The TN for this Test is 3 times the number of dice that would be rolled against the {\shocked} character each round.

\subsection{Damage}
\seclabel{damage}

After {\shock} has been tested for, whether or not it occurs, the \seclink{Damage Test}{damage-tests} causes {\damage}.
To calculate {\damage}, divide the result of the \seclink{Damage Test}{damage-tests} by the target's \statref{res}.
For example, if the result of the \seclink{Damage Test}{damage-tests} is 13 and the damaged creature has 3 \statref{res}, they suffer 4 {\damage}.
{\damage} accumulates: a character who has previously suffered 3 {\damage} and suffers an additional 2 is now suffering from 5 {\damage}.

{\damage} has two effects.
Firstly, a character subtracts their current {\damage} from their \statref{st}.

Secondly, if a character's \statref{st} ever reaches zero, they die immediately.
This is very unlikely to happen through repeated {\damage}, as an earlier blow would send them into {\shock}, but can occur if a lot of painkillers wear off all at once.
%Secondly, if a character's current {\damage} ever equals or exceeds their original \statref{st} (unmodified by {\damage}), they die immediately.
%This applies even if they are ignoring the effect of some of their {\damage}, such as through painkillers.
%As effects that allow a character to ignore {\damage} are so common, it can be helpful to track a character's actual {\damage} and the {\damage} they are considered to be suffering from separately.
%The former represents injury: scrapes, bruises and cuts.
%The latter represents the pain suffered as a result of these.

\subsection{Healing \& Recovery}

{\damage} heals naturally over time, but it's a slow process.
Once per day, with a decent meal and at least about six hours of sleep, a character may recover from 1 point of {\damage}.
If the character spends the entire day resting, they may heal 1 additional point of {\damage}.
For a lightly wounded character, taking a stroll would be acceptable without disturbing a day of rest.
For a character with more serious wounds, they shouldn't move around too much, and may even require complete bed rest.
The GM should adjudicate this, taking into account how the {\damage} was sustained, but in general a character who is not suffering any penalty to rolls due to {\damage} doesn't need to be too careful.

Tending by a healer can hasten the natural recovery process, but only provides any benefit if the character is taking an entire day of rest.
For each rank their physician has in the \skillref{healing} skill, a character taking a day of rest may heal 1 additional point of {\damage}.
A single healer can tend many patients in a day, up to about a dozen.
They may tend themselves, but only if their activities tending others do not prevent them from taking a day of rest themselves.

\subsection{Exhaustion}
\seclabel{exhaustion}

Besides injury, an active witch runs the risk of {\exhaustion}.
From late night vigils to running after tricksy spirits, many things can leave a witch tired and longing for her bed.

When a character performs something exhausting, or goes a day without at least 6 hours of sleep, the GM may apply a level of {\exhaustion}, or call for a Test (typically \attref{might} or \attref{will}) to avoid one.
Each level of {\exhaustion} reduces two of a character's attributes by 1.
The GM selects appropriate attributes depending on the type of {\exhaustion}.
For example, {\exhaustion} as a result of a long foot chase might decrease \attref{might} and \attref{grace}.
Sleep deprivation might decrease \attref{wit} and \attref{heed}.
A long day of socialising, rushing from meeting to meeting, might even reduce \attref{charm} and \attref{presence}.

Multiple levels of {\exhaustion} may reduce the same attribute, leading to a total reduction of 2 or more.
Whenever a character suffers a second level of {\exhaustion} affecting the same attribute, the GM may call for a Test to avoid passing out until they can sleep it off.

A character may reduce their {\exhaustion} by 1 level when they get a good night's sleep: about eight hours.
An entire day of rest reduces {\exhaustion} by another level.
The player my choose which to attributes to recover when they reduce their {\exhaustion}.

\section{Combat}
\seclabel{combat}

\subsection{Actions in Combat}
\seclabel{combat-actions}

\subsubsection{Attack}
\actionlabel{attack}

You attack a creature or object, with a \seclink{weapon}{weapons} or unarmed.
You must be adjacent to the target to attack with a melee weapon, or within the listed range of a ranged weapon.
Make a Test using a number of dice determined, as normal, by your \skillref{weaponry} skill, and a flat bonus determined by your weapon's accuracy.
The Test is made against a TN equal to the target's \statref{dr}: 8, plus their \attref{grace}, plus their \attref{heed}.

If you succeed in your Test, you hit.
Make a \seclink{Damage Test}{damage-tests} against the target, rolling dice as determined by your weapon and adding your \attref{might}.

\subsubsection{Dash}
\actionlabel{dash}

You may move an additional number of metres equal to your \statref{speed} this {\turn}.

\subsubsection{Ready}
\actionlabel{ready}

You don't act immediately, but prepare to take an {\action} later.
Decide what {\action} you will take, and which circumstances trigger it.
When those circumstances come around, you may choose to take the readied {\action} or not.
If your next {\turn} comes around without you taking the readied {\action}, you lose the benefits of readying.
You must \actionref{ready} again if you want to continue to wait.

\section{Magic}

Magic consists of too many diverse disciplines and effects to be effectively summarised in this section; indeed this is the entire topic of \partref{disciplines}.
However, a few general guidelines apply.

It is generally assumed that any witch who knows a spell, rite or technique has the knowledge and practice to pull it off consistently; doing so does not require a Test unless specified otherwise.
However, this practice only applies under normal conditions, with adequate time and materials.
A witch may attempt to rush her magic, perform it using whatever she has to hand, or to perform it in difficult conditions, and each of these requires a Test.
Such Tests typically use \attref{wit} and the relevant skill for the discipline of magic, but not always.
More formulaic disciplines such as \discref{brewing} and \discref{ritual-magic} often use \attref{ken}, while other disciplines, such as \discref{willing} and \discref{golemancy}, rely primarily upon a witch's pure force of \attref{will}.
Furthermore, drawing a chalk circle hurriedly might use \attref{grace}, and grinding a poultice while riding a broomstick might use \skillref{flying}.

TNs for rushing or improvising magic are ultimately left up the GM, but some guidelines are provided below.

\subsection{Rushing Magic}

Generally, magic that would normally take at least an {\action} in combat cannot be performed in less than that time.
Exceptions may be made where the magic is used as part of the {\action} already being taken, to aid it or improve its effect, but the GM should still be careful allowing such things.
Otherwise, common sense may apply a limit to the minimum time magic can be performed in.
For example, if a potion requires boiling water, a witch needs some way to bring water to the boil in the time they want to brew their potion.

Where magic can be rushed, guideline TNs for doing so are given in the following table.

\begin{simpletable}{rX}
	\toprule
	TN & Example Task\\
	\midrule
	9 & Performing a simple rite in half the normal time.\\
	12 & Performing a complex rite in half the normal time.\\
	15 & Performing a simple rite in a tenth the normal time.\\
	18 & Performing a complex rite in a tenth the normal time.\\
	21 & Performing a simple 5 minute rite in one {\action}.\\
	\bottomrule
\end{simpletable}

\subsection{Improvising Materials}

This applies to both the tools used to conduct magic and the ingredients consumed by it, and works equally well in brewing and rites.
The most important part is that the witch can justify any substitution to herself.
From a gameplay perspective, this also means that the player should justify such improvisations to the GM.
This can be as simple as using a pool of water in place of a mirror, because both are reflective, or more extreme, such as using a fresh egg in place of blood, as both are the fluids of life.

\begin{simpletable}{rX}
	\toprule
	TN & Example Task\\
	\midrule
	9 & An unusual component that still meets the specifications, e.g.\ a ritual circle scratched in the dirt instead of drawn in chalk.\\
	12 & A component that retains the fundamental property, e.g.\ scrying through a pool of still water instead of a mirror.\\
	15 & A component that is close, but violates a specification, e.g.\ pig blood instead of human blood.\\
	18 & A component with a reasonable justification for relatedness, e.g.\ a fresh egg in place of blood.\\
	21 & A component with a weak justifcation for relatedness, e.g.\ apple juice in place of blood.\\
	\bottomrule
\end{simpletable}

\subsection{Consequences}

Magic is dangerous, especially when rushed or improvised.
The GM should feel free to reflect this in the consequences of failure on a magic Test, even when it is not a critical failure.
Failure on a magic Test need not indicate that nothing occurred, but might indicate that something unwanted or something rather tangential has occurred, or that the magic has succeeded, but with side effects.

For example, suppose a witch is attempting to brew a potion for hair regrowth, but has substituted several of the ingredients for similar ones they hoped would work.
A failure on the Test might mean that the potion successfully causes hair regrowth, but that the hair is the wrong colour or grows in more places that desired.

Other magics can have even more dangerous consequences.
A witch trying to scry through a puddle instead of a mirror might, on a narrow failure, only get an unclear image as the puddle is disturbed by wind.
But a more dire failure could mean that the target instead sees the witch herself through any nearby reflective surfaces, or that the imperfect scrying draws the attention of \emph{things} from other dimensions that look, reach or even climb out of the puddle.
Rituals to summon demons and the like can obviously have some of the most dangerous consequences of all, should they go wrong.


\part{Disciplines of Witchcraft}
\partlabel{disciplines}

\discipline{Brewing}{brewing}{Brewer}{Brewers}

\dropcapdiscref{brewing} may well be the least magical of witchcraft's disciplines.
In fact, it is not even restricted to witches.
Many wise folk can cook up a remedy for the most common ailments.
And almost every village has someone to brew their hops into beer.

As such, the feats in this discipline are not necessary for common brewing; they represent only those brews that require some trick, skill, or guarded piece of knowledge.
Anyone, even those without ranks in the \skillref{brewing} skill, may attempt to create more common brews.
Some, such as alcohol, are relatively simple.
A Test may determine the quality of the result, but almost anyone can create \emph{something} appropriate.

Remedies for diseases and ailments are a little more complicated.
That's not to say that it takes much skill to boil up a few strips of willow bark, but half of the common knowledge of folk medicine is ungrounded superstition.
It requires a \testtype{ken}{brewing} Test to create an appropriate remedy for most ailments, with the {\tn} of the Test determined by the rarity and severity of the ailment.
This normally requires access to a wide array of \materialrefplural{herb} and a boiling cauldron; the {\tn} might be increased as normal by trying to make do with inadequate materials.

For particularly simple remedies---such as helping a person to sleep, or easing an ache---it may suffice to use the raw \materialref{herb}, without any \discref{brewing}.
In this case, a \testtype{ken}{botany} Test might suffice.

\section{Brew-Toting}

A \practitioner{brewing} is not just a witch who can brew potions, but also a witch who tends to carry them.
It is assumed that a \practitioner{brewing} carries at least one dose of each potion she knows how to make; normally several doses of each.
Unless she begins providing potions to groups much larger than her coven, or keeping people dosed on a particular potion for hours on end, she should have plenty of doses.
If she does do this, however, the GM is welcome to declare that she runs out and has to brew some more.

This assumes, however, that she has ready access to all the ingredients required to brew her potions.
In most cases, this simply means that the witch has access to her {\garden}, and sufficient \skillref{botany} skill that her {\garden} contains the required \materialrefplural{herb}.
If she doesn't have ready access to the required ingredients, she will either have to gain access somehow, or improvise replacements.
In either case, the GM is free to have these endeavours roleplayed, and require Tests for them.

\section{Brews for Everyone}

Most \practitioners{brewing} sample their own fare.
But the main benefit of \discref{brewing} comes from sharing one's potions.
The other members of your coven; their familiars, and yours; friendly village folk; even an \featref{animal-companion}; everyone can benefit from your brews---except golems and the undead.

As such, many \practitioners{brewing} fall into something of a supportive role within their coven.
So don't be afraid to share your potions, to allow everyone to do what they do even better.

This doesn't have to come at the cost of doing your own thing, however.
You can easily brew a dose or two of every potion for every member of your coven, and can even make the other witches carry their own doses.
The brunt of your \discref{brewing} work will be done during downtime, leaving you plenty of time to get involved when you're out and about.

\section{Creation and Application}

The \skillref{brewing} skill and \discref{brewing} discipline don't just cover potions brewed in a cauldron.
Potions, poultices, poisons, tinctures, salves, ointments, even beer, mead, wine and spirits.
Witches have many ways of turning \materialrefplural{herb}, and even other ingredients, into more useful forms.

Each feat that allows a witch to prepare such a mixture lists the method of preparation and delivery.
The rules of such methods are presented below.

\subsection{Brewing and Chewing}

Different methods of preparation require different equipment, and take different periods of time.

\mixcreation{Cauldron}{cauldron}{
	Most potions are brewed in cauldrons, filled with water and brought to boil.
	This requires, obviously, a cauldron, as well as a fire to heat it.
	A smaller kettle might do in a pinch, but requires a Test.
	A full cauldron will typically yield several doses.
	Brewing in a cauldron requires around 15 minutes to bring the water to the boil, and another minute to mix the potion.
}

\mixcreation{Poultice}{poultice}{
	A poultice doesn't need to be brewed at all; the ingredients are simply chewed into a paste.
	Some of the more dangerous poultices should definitely be ground with a mortar and pestle, however, rather than allowed anywhere near the mouth.
	Creating a poultice requires less than a minute.
}

\mixcreation{Still}{still}{
	Some potions, or spirits, need to be distilled.
	This requires quite a lot of dedicated equipment, a carefully maintained heat source, and several hours.
}

\subsection{Method of Delivery}

Although most potions are drunk, there are many ways to get a mixture into a person's body.
Some faster, some slower, and some far easier to inflict on an unwilling victim.

\mixdelivery{Drink}{drink}{
	A \mixdeliveryref{drink} is about half a litre of liquid that must be drunk to take effect.
	It can be quaffed as an {\action}, and takes effect after 1 {\round}, unless specified otherwise.
}

\mixdelivery{Spike}{spike}{
	A \mixdeliveryref{spike} is a much smaller quantity of liquid than a \mixdeliveryref{drink}, little enough that it could be slipped into a glass of wine without being obvious.
	It can be drunk willingly, but typically isn't.
	It takes effect after 1 {\round}, unless specified otherwise.
	
	In a glass of wine or a cup of water, one dose will normally go unnoticed until it is too late.
	A second dose causes a noticeable change in taste, scent, or colour, which will typically be noticed unless the drinker isn't paying much attention.
	Larger and stronger drinks can conceal more doses, however.
}

\mixdelivery{Topical}{topical}{
	A \mixdeliveryref{topical} mixture is applied to the skin.
	It typically requires more than an {\action} to smear it on, or bind a wad in place.
	It generally only takes effect after a few minutes, but can kick in a little faster if applied to a wound or a mucous membrane.
	Some need to be applied to the correct part of the body.
	For example, if it is to treat a wound, it should be applied to the wound, and if it is to enhance the eyesight, it should be applied to the eyes.
}

\mixdelivery{Snuff}{snuff}{
	These mixtures are boiled or ground down to a powder, which must be inhaled into the nostrils.
	They can be taken as an {\action}, and take effect immediately.
	Giving them to someone unwilling requires forcing them to inhale in some fashion, but it can usually be achieved if you have the target at your mercy for a minute or more.
}

\mixdelivery{Injury}{injury}{
	These mixtures, typically harmful ones, must be delivered into the bloodstream via an injury.
	The most expedient way to do this is to smear it on an arrow, a \weaponref{blowgun} dart, or an edged weapon, requiring an {\action}.
	It's good for one cut, but otherwise remains on the weapon until rubbed off or washed away.
	Beware rain.
	It takes effect immediately, unless specified otherwise.
}

%TODO: Gaseous? Needs to be stored in an air-tight bottle?
%TODO: Incense? Needs to be burned, evaporated?

\subsection{Time to Effect}

Mixtures applied by different methods typically require different lengths of time to take effect, as specified above.
Some mixtures require more or less time to take effect, as specified in their descriptions.
If quaffed on a creature's {\turn}, a mixture that takes effect after 1 {\round} comes into play at the start of their {\turn} in the next {\round}.

The GM may allow a character to make \attref{might} Tests to stave off the effects of an unwanted mixture.
This might extend the time to effect by two or three times, but should not allow them to avoid the effect entirely, except in the most exceptional cases.

\subsection{Vomiting}

For ingested mixtures (a \mixdeliveryref{drink} or \mixdeliveryref{spike}), vomiting before it takes effect can massively ease this \attref{might} Test, and even allow avoiding the effect entirely.
Ingesting an emetic herb, such as \herb{veratrum or the toadstool ``emetic russula''}{2}, serves to induce vomiting.
However, these herbs do not act quickly enough to help with a mixture that takes effect in just 1 {\round}.
To prevent these, a witch might look into \featref{vomit-potion}.

\section{Side-Effects}

While some of the more noxious mixtures a witch can brew produce adverse effects by design, these are not the only ways a potion can hurt.
Many mixtures come with adverse side effects all by themselves, and these are compounded by the dangers of overdosing and combining brews.

\subsection{Overdosing}

Many potions carry harmful effects when taking too many doses.
These typically only occur if multiple doses would be in effect simultaneously; taking another dose after the first has worn off is safe unless specified.
Some of the effects of overdoses are given explicitly, but many are given as general guidelines.
The GM is left to adjudicate in the latter case.
Typically the worst effects of overdosing can be staved off with a \attref{might} Test, with the {\tn} affected by how many excess doses have been taken, and how close in succession they were taken.

If you wish to keep a brew's effect going continuously, and you have access to additional doses of the brew, it is easy enough to take another dose as you feel the first wearing off.
You can do this without any risk of overdosing.

\subsection{Mixing Mixtures}

Mixing multiple potions can have adverse and unexpected effects.
These kick in when a character is under the effect of two substances that both affect the same attribute, or other statistic.
The effects are unpredictable.
The GM is free to apply any appropriate penalty, possibly calling for a \attref{might} Test to avoid or alleviate the effects.
The following table is provided for inspiration.
The GM may roll 2 six-sided dice and compare their sum against the table to randomly determine an effect, if desired.

\begin{simpletable}{rX}
	\toprule
	\dice{2} & Effect\\
	\midrule
	2 & Apply severe overdose effects of the first mixture.\\
	3 & Exhaustion, unconsciousness and/or oxygen deprivation.\\
	4 & Apply moderate overdose effects of the first mixture.\\
	5 & Ignore any positive effects of the first mixture.\\
	6 & Double any detrimental effects of the first mixture.\\
	7 & Re-roll twice on the table, taking both results.\\
	8 & Double any detrimental effects of the second mixture.\\
	9 & Ignore any positive effects of the second mixture.\\
	10 & Apply moderate overdose effects of the second mixture.\\
	11 & Twitching, seizure, overheating and/or organ failure.\\
	12 & Apply severe overdose effects of the second mixture.\\
	\bottomrule
\end{simpletable}

The effect of painkillers---to ignore {\damage}---does not count as a statistic for the purpose of mixing substances.
As such, a character may safely be under the effects of multiple painkillers as long as their other effects do not overlap.
Additionally, withdrawal effects (such as those of \featref{mental-stimulant}) do not count for mixing.
%TODO: Use a better example when I have a withdrawal effect that modifies statistics.

\section{Antidotes}
\seclabel{antidotes}

While vomiting can help to avoid the effects of a potion \emph{before} they kick in, ending the effects once they are ongoing requires an {\antidote}.
\capital{\antidotes} exist not just for poisons, but for any potion or other mixture with an ongoing effect.
Each {\antidote} will only work on potions it is designed to counteract, however.
It will specify in its description what sorts of potion it is effective against.

\capital{\antidotes} can be delivered through different methods, just like potions themselves, and take the usual length of time to kick in.
Once it kicks in, the {\antidote} ends all ongoing effects of any potions it is designed to counteract.
One dose of {\antidote} can end the effect of several different potions, as long as it is designed to counteract all of them, or several doses of the same potion.

When an {\antidote} takes effect, treat it as though the duration of the potion has expired.
An {\antidote} ends both beneficial and detrimental effects, but can only end \emph{ongoing} effects.
Any {\damage} that has been dealt stays dealt.
Overdose effects are ended, but organs that have failed are not repaired, and so on.
Withdrawal effects (such as \featref{mental-stimulant}), are not ended by an {\antidote}.
In fact, taking an {\antidote} causes the withdrawal effects to kick in, as it ends the effect of the potion.

The effect of an {\antidote} is immediate, not ongoing.
This means that you cannot have an {\antidote} to an {\antidote}.

\section{Feats}

\feat{Numbing Painkiller}{painkiller-grace}{15}{
	\noprereq
}{
	\mix{cauldron}{drink}{\Herb{willow bark}{2}}
	%Willow bark contains natural NSAIDs.
	
	The drinker may ignore 1 point of {\damage} for a few hours, but loses 1 \attref{grace} for the same duration.
	Two doses may be effective simultaneously.
	Further doses cause paralysis, and possibly organ failure.
}

\feat{Dimming Painkiller}{painkiller-mental}{10}{
	\noprereq
}{
	\mix{cauldron}{drink}{\Herb{poppy seed}{2}}
	%Poppies contain natural opioids.
	
	The drinker may ignore 1 point of {\damage} for a few hours, but loses 1 \attref{ken} and \attref{wit} for the same duration.
	Two doses may be effective simultaneously.
	Further doses cause unconsciousness, and possibly cessation of breathing.
}

\feat{Blurring Painkiller}{painkiller-heed}{15}{
	\noprereq
}{
	\mix{cauldron}{drink}{\Herb{barley}{2}}
	%Barley contains a lot of phenols, which are used in the synthesis of a lot of pharmaceuticals.
	%Malt barley is also used in producing alcohol, which can justify the blindness.
	
	The drinker may ignore 1 point of {\damage} for a few hours, but loses 1 \attref{heed} for the same duration.
	Two doses may be effective simultaneously.
	Further doses cause blindness, which can become permanent.
}

\feat{The Hard Stuff}{might-potion}{20}{
	\noprereq
}{
	You know how to make a drink that'll really put hairs on a man's chest.
	Or a woman's, at that.
	
	\mix{still}{drink}{Alcohol, \herb{apple}{2}}
	%Scumble is 'mostly apples'.
	
	The drinker gains 1 \attref{might} for a few hours, and loses 2 \attref{wit} and \attref{heed} for the same duration.
	A second dose will render the drinker unconscious.
	Further doses are dangerously poisonous, causing vomiting, seizures and oxygen deprivation.
}

\feat{The Pure}{might-potion-2}{15}{
	\skillref[1]{brewing},
	\featref{might-potion}
}{
	\mix{still}{drink}{Alcohol, \herb{fennel}{2}, \herb[aniseed]{green anise}{3}, \herb{wormwood}{3}}
	%Wormwood, sweet fennel and green anise are the three herbs in absinthe.
	
	You brew your drink clean and pure.
	This functions as \featref{might-potion}, except it decreases \attref{wit} and \attref{heed} by only 1 point.
}

\feat{The Green Fairy}{might-potion-3}{20}{
	\skillref[2]{brewing},
	\featref{might-potion-2}
}{
	\mix{still}{drink}{Alcohol, \herb{fennel}{2}, \herb[aniseed]{green anise}{3}, \herbcreature{grand-wormwood}{5}}
	
	Your drink gives men the strength of horses.
	You don't want to see what it does to horses.
	This functions as \featref{might-potion-2}, except it increases \attref{might} by 2 points.
}

\feat{Empathogen}{charm-potion}{20}{
	\noprereq
}{
	\mix{cauldron}{drink}{\Herb{violet}{2}}
	%Violet contains piperonal, a common precursor to the empathogen MDMA (ecstasy).
	%Violets were also emblematic flowers of Aphrodite and Priapus.
	
	This potion affords the drinker a greater sense of empathy and connection with those around them.
	The drinker gains 1 \attref{charm} for a few hours, and loses 2 \attref{will} for the same duration.
	A second dose causes agitation and paranoia, instead reducing \attref{charm} by 1.
	Further doses cause the drinker to overheat, suffering heat stroke, and may lead to internal bleeding and organ failure.
}

\feat{Alpha's Potion}{presence-potion}{20}{
	\noprereq
}{
	\mix{cauldron}{drink}{\Herb{onion}{2}}
	%Onions have an obvious and pervasive smell that makes people cry.
	
	This potion grants the drinker increased confidence, a slightly louder voice, and a certain indefinable \emph{obviousness}.
	There's a certain smell goes with it, a sort of threat pheromone, but it sits below the conscious level for all but the most attentive people.
	The increased confidence that the potion provokes tends to go a bit too far, however, veering into arrogance.
	If the drinker isn't careful, they come across as, frankly, a right prick.
	
	The drinker gains 1 \attref{presence} for a few hours, and loses 2 \attref{charm} for the same duration.
	A second dose causes a total loss of social graces, decreasing \attref{charm} by a further 2 points, without further increasing \attref{presence}.
	It also causes the victim to sweat profusely and smell strongly of onions.
	Further doses cause degeneration into raving, incoherent lunacy, and turns the sweat into a glistening mucus that coats the skin.
}

\feat{Stimulant}{mental-stimulant}{25}{
	\noprereq
}{
	\mix{cauldron}{drink}{Ants, vinegar}
	%Ants have a gland which can produce phenylacetic acid.
	%This can be converted to phenylacetone using acetic anhydride, which can be derived from acetic acid.
	%Acetic acid is the main component of vinegar (beside water). Vinegar is simply fermented from ethanol.
	%Phenylacetone can be converted to amphetamine using the Leuckart reaction and formic acid (also from ants).
	%Amphetamine is a widely-known nootropic (cognitive enhancer).
	
	The drinker gains 1 \attref{wit}, \attref{will} and \attref{heed} for about an hour.
	The potion also staves off tiredness for the duration. %TODO: A mechanical effect for this?
	After the potion wears off, you pay the price of your temporarily enhanced performance.
	You suffer a \negative{1} penalty to all rolls for the next 24 hours.
	Additional doses within this period are ineffective.
	
	Being under the effect of two doses simultaneously causes a headache that counteracts the increased attributes.
	Further doses can cause bleeding into the brain and death.
}

\feat{Stimulant Dragging}{mental-stimulant-extension}{15}{
	\skillref[1]{brewing},
	\featref{mental-stimulant}
}{
	A slight change to the formula of your \featref{mental-stimulant} allows its effect to be extended by additional doses.
	Drinking another dose as one begins wearing off extends the effect and staves off the withdrawal.
	However, the body can only sustain such enhanced performance for so long.
	Whenever you take a dose after the first, make a \attref{might} Test.
	The {\tn} is 9 for the second dose, and increases by 3 for every subsequent dose.
	On a failure, you pass out, gain no benefit from the additional dose, and the withdrawal effects kick in.
	You cannot be roused for several minutes.
}

\feat{Hysterical Strength}{physical-stimulant}{15}{
	\noprereq
}{
	A person's muscles are stronger than they normally get to use, strong enough to break their own bones.
	There's a good reason you don't get to use the full strength, you see.
	But you've figured out how to unlock that extra potential, and damn the consequences!
	
	\mix{cauldron}{drink}{\Herb{joint pine}{3}}
	%Joint pine is Ephedra, which contains ephedrine.
	%Ephedrine is similar in structure and function to epinephrine (adrenaline).
	%Adrenaline is oft-blamed for hysterical strength.
	
	The drinker gains 1 \attref{might} and 1 \attref{grace} for a few minutes.
	For the duration, any strenuous activity causes the drinker to suffer a \dice{2} {\damagetest}.
	Strenuous activity includes the \actionref{dash} and \actionref{attack} {\actions}, any Test using \attref{might} or \attref{grace} (excluding {\damagetests} as part of the \actionref{attack} {\action}), and other things at the GM's discretion.
	
	Being under the effect of two doses simultaneously does increase \attref{might} and \attref{grace} further, but causes {\damagetests} as a result of any movement at all; only lying still is safe.
	Further doses cause seizures, triggering the {\damagetests} themselves.
}

\feat{Hysterical Restraint}{physical-stimulant-2}{20}{
	\skillref[1]{brewing},
	\featref{physical-stimulant}
}{
	A slight refinement to the formula for your \featref{physical-stimulant} affords the drinker more control over their newfound strength, limiting the harm they do to themselves.
	The {\damagetests} caused by your \featref{physical-stimulant} are only \dice{1}.
}

\feat{Hysterical Empowerment}{physical-stimulant-3}{20}{
	\skillref[1]{brewing},
	\featref{physical-stimulant-2}
}{
	By concentrating your \featref{physical-stimulant}, you can make it stronger, but more dangerous.
	Whenever you brew \featref{physical-stimulant}, select a number from 1 to 3.
	The potion provides this number as a bonus to both \attref{might} and \attref{grace}, but also determines the number of dice rolled on the {\damagetests} it causes.
	
	Doses beyond the first only ever provide 1 additional point to the attributes, regardless of their strength.
	However, they still carry all the same drawbacks.
}

\feat{Hysterical Overload}{physical-stimulant-4}{10}{
	\skillref[2]{brewing},
	\featref{physical-stimulant-3}
}{
	You can give your \featref{physical-stimulant} potion \emph{even more power}!
	When you use \featref{physical-stimulant-3}, you can brew potions that give an attribute bonus up to \positive{5}, although this raises the number of dice rolled for the {\damagetests} as normal.
}

\feat{Hysterical Moderation}{physical-stimulant-mitigate}{20}{
	\skillref[2]{brewing},
	\featref{physical-stimulant-3}
}{
	You can fine-tune the low end on your \featref{physical-stimulant} potions.
	When you use \featref{physical-stimulant-3} to brew a potion that provides exactly \positive{2} to \attref{might} and \attref{grace}, the {\damagetests} it causes are only \dice{1}.
}

\feat{Brew of Claws}{weapon-potion}{15}{
	\noprereq
}{
	\mix{poultice}{topical}{\Herb{bramble stems}{1}}
	
	Smeared onto hands or paws, this mixture causes the digits to calcify, and harden into wicked claws.
	They return to normal after about an hour.
	Until then the creature rolls additional dice for {\unarmed} {\damagetests} using the clawed appendages.
	
	A creature without an effective attack gains one, and rolls 2 dice, while a creature with an existing attack rolls at least 3 dice.
	A human, or other creature with proper hands, rolls 5 dice.
	
	However, the hardening makes it difficult to use the hand normally.
	You roll 1 die less on Test that use the clawed hand's manual dexterity, including attacks with weapons held in that hand.
	You may apply this \mixcreationref{poultice} to only one hand, granting you the improved {\damagetest} while leaving the other hand unaffected.
	If used on on a paw or other appendage of a non-human, the GM is free to apply penalties appropriate to the afflicted appendage.
	
	This \mixcreationref{poultice} uses only the stems of the brambles, not the thorns, so it is safe to chew.
	It leaves the mouth a little hardened, but causes no major problems.
}

\feat{Oakenhide Brew}{resilience-potion}{15}{
	\noprereq
}{
	\mix{cauldron}{drink}{\Herb{parsnip}{2}, \herb{oak bark}{2}}
	%Parsnip stems and leaves can cause skin rash.
	
	This potion causes the drinker's skin to harden in patches, sprouting bark-like growths.
	It lends a remarkable toughness, at the price of an awful stiffness.
	
	It takes effect over the course several minutes, eventually increasing the drinker's \statref{resilience} by 1, but reducing their \attref{grace} by 2.
	A second dose further increases \statref{resilience}, but makes the drinker so stiff that they cannot move at all.
	Further doses risk making this immobility permanent.
	
	The growths flake off after a few hours, returning the drinker to normal.
	The energy expended in growing and shedding these patches takes a couple of extra meals to recoup---someone using this potion more than once a day for an extended period simply can't keep up, and will starve.
}

\feat{Stonehide Brew}{resilience-potion-2}{15}{
	\skillref[1]{brewing},
	\featref{resilience-potion}
}{
	You've found a rare succulent plant which disguises itself as a pebble.
	Boiling its tough skin, you can create a potion that lends the hardness of stone to those who drink it.
	
	\mix{cauldron}{drink}{\Herb{parsnip}{2}, \herb{pebble plant}{4}}
	%Pebble plants are the genus Lithops.
	
	This functions as \featref{resilience-potion}, except it increases \statref{resilience} by 2.
}

\feat{Ironhide Brew}{resilience-potion-3}{15}{
	\skillref[2]{brewing},
	\featref{resilience-potion-2}
}{
	Using the legendary strength of the \creatureref{ironwood} tree, you can grow a skin that will nearly turn knives.
	
	\mix{cauldron}{drink}{\Herb{parsnip}{2}, \herbcreature{ironwood}{5}}
	
	This functions as \featref{resilience-potion}, except it increases \statref{resilience} by 3.
}

\feat{Supple Hide}{resilience-potion-mitigate}{15}{
	\skillref[2]{brewing},
	\featref{resilience-potion}
}{
	The growths produced by your \featref{resilience-potion}, \featref{resilience-potion-2}, and \featref{resilience-potion-3} are more supple, less restrictive.
	They reduce \attref{grace} by only 1.
}

\feat{Rapid Hide}{resilience-potion-fast}{15}{
	\skillref[1]{brewing},
	\featref{resilience-potion}
}{
	Your \featref{resilience-potion}, \featref{resilience-potion-2}, and \featref{resilience-potion-3} are more potent, faster-acting.
	They take effect in only 1 round.
}

\feat{Hedgehog Hide}{resilience-potion-spines}{10}{
	\featref{resilience-potion},
	\featref{weapon-potion}
}{
	When you brew a \featref{resilience-potion}, \featref{resilience-potion-2}, or \featref{resilience-potion-3}, you may cause the growths to form spikes.
	You must add a hedgehog's spine to the potion when you brew it.
	
	The spines grant the affected creature the same bonuses to {\unarmed} {\damagetests} as the \featref{weapon-potion}.
	The does not suffer the penalty to manual dexterity caused by the \featref{weapon-potion}, only the penalty to \attref{grace} caused by the \featref{resilience-potion}.
	
	Furthermore, when a creature with these spines is hit by an {\unarmed} attack, the attacker suffers a \dice{2} {\damagetest}.
}

\feat{Berserker Broth}{rage-potion}{15}{
	\noprereq
}{
	\mix{cauldron}{drink}{\Herb{rose}{2}}
	%Roses are blood-red and thorny.
	
	The potion fills the drinker with an all-consuming, empowering rage.
	For a few minutes, they gain 1 \attref{might} and \attref{will}.
	However, for the same duration, they want nothing more than to murder everything; people and other creatures.
	
	A person under the effect of this brew retains all their skills and knowledge; they will still use weapons and tools, and a witch retains her full ability with magic.
	However, the rage is urgent, and overwhelming.
	The drinker will not plan or delay their murders; they would rather draw a weapon and charge in than laying an ambush.
	They still have a sense of self-preservation, but it is a second priority, behind murder.
	Thankfully, however, the potion does not instill any desire to extend its own effect, even if doing so would give more time to murder people.
	
	If a player character drinks this, the GM \emph{may} allow them to keep control of their character, as long as they act in a appropriate fashion.
	However, the GM reserves the right to take control of the character if she is being insufficiently murderous, or insufficiently indiscriminate.
	
	A second does further increases \attref{might} and \attref{will}, but leaves the victim so mad that they cannot do anything but curl into the foetal position and scream.
	Further doses risk making this madness permanent.
}

\feat{Brighteye Drops}{sight-potion-light}{15}{
	\noprereq
}{
	\mix{cauldron}{topical}{\Herb{eyebright}{2}}
	
	Dripped into the eyes, this concoction enhances the eyesight.
	However, the user's night vision is utterly shot.
	
	For the next hour, the user rolls an extra die on \skillref{perception} Tests relying on sight in well-lit places.
	This replaces any existing bonus dice from an ability such as a \familiarrefpossessive{raptor}, but applies on top of the normal dice granted by the \skillref{perception} skill.
	
	However, the user is almost utterly blind in dimly lit locations.
	Most inside locations will be dimly lit, unless they have large or numerous windows, or an abundance of lamps.
	
	Further doses produce no additional effect.
	If taken alongside \featref{sight-potion-dark}, the user is blinded in both light and darkness.
}

\feat{Darkeye Drops}{sight-potion-dark}{10}{
	\noprereq
}{
	\mix{cauldron}{topical}{\Herb{carrot}{2}}
	
	The opposite of the \featref{sight-potion-light}, this potion grants excellent night vision, at the cost of an over-sensitivity to light.
	For an hour after application, the user suffers no penalties in low-light conditions, though they are as blind as anyone in complete darkness.
	
	For the same duration, bright lights---such as sunlight, or a lantern flame---are blinding.
	Looking into such light blinds the user for about a minute.
	
	Further doses produce no additional effect.
	If taken alongside \featref{sight-potion-light}, the user is blinded in both darkness and light.
}

\feat{Twitching Eyes}{sight-potion-switch}{15}{
	\skillref[1]{brewing},
	\featref{sight-potion-light},
	\featref{sight-potion-dark}
}{
	Normally, taking \featref{sight-potion-light} and \featref{sight-potion-dark} at the same time leaves the user completely blind.
	However, a slight change to the brewing process allows each to act as an {\antidote} to the other, in addition to its normal effects.
	This allows you switch back and forth between the two effects, through repeated applications of the mixtures.
}

\feat{Molenose Powder}{smell-potion}{10}{
	\noprereq
}{
	\mix{cauldron}{snuff}{\Herb{jasmine}{3}, an animal's nose}
	%Jasmine is noted for its fragrance.
	%The flowers look star-shaped, like a star-nosed mole's nose.
	
	Snorted into the nostrils, this concoction delivers an overwhelming fragrance of jasmine, which quickly fades to leave the user's sense of smell greatly enhanced.
	This enhanced sense lasts a few minutes, however, the concoction strikes the user completely blind for the same duration.
	
	Their sense of smell becomes fine enough to allow them detect people moving around a room, in real time.
	This should make walking around at a slow pace relatively easy, despite the blindness.
	With a difficult Test, they might even be able to run, without running into anything.
	Successfully attacking people, or evading attacks, is all but out of the question.
	
	For the duration, the user rolls 2 extra dice on \skillref{perception} Tests relying on smell.
	This replaces any existing bonus dice from an ability such as a \familiarrefpossessive{rat}, but applies on top of the normal dice granted by the \skillref{perception} skill.
	
	Additional doses burn out the nostrils, leaving the user with no sense of smell \emph{or} sight.
}

\feat{Mole's Eyes}{smell-potion-2}{10}{
	\skillref[1]{brewing},
	\featref{smell-potion}
}{
	By cutting \featref{smell-potion} with \herb{eyebright}{2}, you can ameliorate its effect on the eyesight.
	
	\mix{cauldron}{snuff}{\Herb{jasmine}{3}, \herb{eyebright}{2}, an animal's nose}
	
	This functions as \featref{smell-potion}, except it doesn't render the user completely blind.
	It still has a drastic effect on their vision, leaving even nearby objects as little more than blurry shapes.
	However, the user can safely walk around without trouble, and can even run without \emph{much} risk of running into something.
	Attacking or evading is even possible, albeit still at a large penalty.
	They also halve the number of dice they roll on \skillref{perception} Tests relying on sight.
}

\feat{Recovering Mole}{smell-potion-3}{15}{
	\skillref[2]{brewing},
	\featref{smell-potion-2}
}{
	A subtle change to the formula of your \featref{smell-potion} makes it last much longer, without hindering the eyes any more.
	
	The duration of your \featref{smell-potion} (and \featref{smell-potion-2} concoction) increases to an hour.
	However, this only increases the duration of its effect on smell; vision still recovers to normal after only a few minutes.
}

\feat{Hot Cuppa}{warm-potion}{10}{
	\noprereq
}{
	A cup of hot tea can certainly warm the body briefly, but this concoction can keep you warm for \emph{hours}.
	
	\mix{cauldron}{drink}{\Herb{horseradish}{2}}
	
	For a few hours, this concoction keeps the drinker warm and comfortable even in chilly weather.
	They can even go out in the snow without wrapping up, though freezing temperatures are enough to cause discomfort.
	They are highly resistant to hypothermia and frostbite---only an extended period in icy water is typically sufficient to afflict them.
	
	In warm temperatures, however---such as a midsummer's day, or near a large fire---the drinker is more susceptible to overheating, and heat stroke.
	Being under the effect of two doses at once similarly causes overheating, leading to heat stroke, and organ failure with further doses.
	If taken alongside \featref{cold-potion}, the drinker's temperature fluctuates wildly, switching between the two effects randomly.
}

\feat{Iced Cuppa}{cold-potion}{10}{
	\noprereq
}{
	Fed up of sweltering over a cauldron's fire, you've brewed up a solution.
	
	\mix{cauldron}{drink}{\Herb{mint}{2}}
	
	For a few hours, this concoction keeps the drinker comfortably cool even in sweltering heat.
	They won't suffer from heat exhaustion, or heat stroke.
	However, they are not protected from {\fire}, or anything else that would cause burning.
	
	However, the drinker becomes more susceptible to cold temperatures.
	Air temperatures below freezing are dangerous.
	Similarly, being under the effect of two doses at once causes hypothermia, and further doses risk frostbite in the extremities.
	If taken alongside \featref{warm-potion}---or any brew that causes overheating---the drinker's temperature fluctuates wildly, switching between the two effects randomly.
}

\feat{Vomiting Drops}{vomit-potion}{15}{
	\noprereq
}{
	\mix{cauldron}{spike}{\Herb{mistletoe}{2}}
	
	These drops cause immediate vomiting.
	They take effect immediately, and act quickly enough to be effective against ingested mixtures that take effect in just 1 {\round}.
	The drinker loses at least one {\action}, and usually several, while they heave and retch.
}

\feat{Sleeping Solution}{sleep-potion}{10}{
	\noprereq
}{
	\mix{cauldron}{spike}{\Herb{camomile}{2}}
	
	This soporific kicks in quite slowly, but the drinker should be asleep with 10 minutes, and remain that way for a few hours.
	A few doses are quite safe, and will accelerate the effect a little, but too many can put the victim into a coma.
}

\feat{Garlic Solution}{garlic-potion}{15}{
	\noprereq
}{
	\mix{still}{spike}{\Herb{garlic}{2}}
	
	A distillation of purest garlic essence, this solution goes straight to the sinuses and burns something awful.
	It takes effect immediately.
	The drinker must make a {\tn} 12 \attref{might} Test or pass out immediately, unable to be roused for several minutes.
	For the next hour or so, they have no useful sense of smell and a full breath to the face robs others of their own sense of smell.
	They can be easily tracked by scent as the stuff leaks from their pores.
	Remarkably, however, the solution itself has no obvious scent until ingested.
	
	Vampires suffer far worse, of course.
}

\feat{Projection Potion}{projection-potion}{15}{
	\skillref[1]{brewing},
	\featref{garlic-potion},
	\featref{projection-start}
}{
	\mix{still}{spike}{\Herb{garlic}{2}, \herb{morning glory}{3}}
	%Morning Glory is an Aztec entheogen.
	
	This solution has enough kick to knock the mind right out of the body.
	It takes effect immediately, ejecting the drinker's mind into the {\mentalrealm}.
}

\feat{Projection Poison}{projection-poison}{20}{
	\skillref[2]{brewing},
	\featref{projection-potion}
}{
	You can brew a variant of \featref{projection-potion} that is delivered through the bloodstream, rather than the mouth.
	This makes it a little easier to inflict on an unwilling target.
	
	\mix{still}{injury}{\Herb{garlic}{2}, \herbcreature{moly}{5}}
	
	This functions just as \featref{projection-potion}, except for its different method of delivery.
	Bear in mind that the victim can very quickly re-enter their own body.
	However, it will probably have fallen over before they do so, and a prepared mind could jump in and {\possess} the body first.
}

\feat{Medium's Potion}{medium-potion}{10}{
	\skillref[1]{brewing},
	\featref{projection-potion},
	\featref{medium}
}{
	By brewing the herb used to enter a \featref{medium} trance into the right concoction, you can skip the tedious meditating.
	Or feed it to someone else, if you like.
	
	You may include \herb[stinking nightshade]{black henbane}{2} in the recipe of your \featref{projection-potion}, or your \featref{projection-poison}, if you can make the latter.
	If you do so, then the user's body is placed into a \featref{medium} trance as their mind is ejected.
	
	As normal, the \featref{medium} may end the trance at any point by re-entering her body.
	However, a soul ready and waiting to {\possess} the body still gets the first chance.
}

\feat{Nettle's Bite}{poison-pain}{15}{
	\noprereq
}{
	You can grind up nettles while preserving, and enhancing, their sting.
	The resulting mixture won't do much lasting harm, but it hurts like nobody's business.
	
	\mix{poultice}{injury}{\Herb{nettle}{1}}
	
	If this poison is delivered by an attack, the target's \statref{shock-threshold} is treated as being 1 lower against the {\damagetest}.
	This does not increase the {\damage} dealt, but is more likely to send the target into {\shock}.
	
	Preparing this \mixcreationref{poultice} by chewing is incredibly painful, and leaves a swollen tongue that makes speaking difficult for hours---a mortar and pestle are recommended.
}

\feat{Festering Poison}{poison-infection}{10}{
	\noprereq
}{
	\mix{poultice}{injury}{\Herb{bloodwort}{2}}
	
	This wicked concoction invariably causes bleeding and infection when applied to a wound.
	Cleaning out the infection requires medical attention, and typically a Test.
	\capital{\damage} caused by a weapon coated in this poison, or any wounds it is applied to, will not heal until the infection has been cleared.
	If the infection goes untreated for several days, it can become lethal.
	Even if the infection is treated and the wound heals, it will often leave a wicked scar.
	
	Chewing up one or two doses of this \mixcreationref{poultice} is relatively safe, as long as you have no wounds around your mouth.
	Chewing a larger batch can cause bleeding and subsequent infection in the mouth.
}

\feat{Embalming Fluid}{embalming-fluid}{10}{
	\skillref[1]{brewing}
}{
	\mix{still}{drink}{Alcohol, ants, iron shavings}
	%Basing this on formaldehyde.
	%Formaldehyde is produced industrially by oxidation of methanol.
	%One catalyst used for this is a mixture of metals including iron.
	%Ants are for the connection between formic acid and formaldehyde.
	
	This fluid is not intended for consumption, but rather for soaking corpses.
	When used in {\embalming}, the corpse can preserved for years, or even decades if it is well cared for.
	One dose suffices to {\embalm} a rat or bird, but two doses are required for a cat, or more than a dozen for a human.
	
	If, for some reason, the potion is imbibed, it proves quite toxic.
	It causes pain, nausea and convulsions, progressing, over the course of a few minutes to permanent blindness, and probably death.
}

\feat{Healing Salves}{brewing-healing}{10}{
	\skillref[1]{brewing}
}{
	You know a wide range of minor poultices, salves and remedies for cuts, bruises and other physical injuries.
	As long as you have access to a reasonable supply of various \herbtypeplural{2}, and time to chew up poultices, you may use your \skillref{brewing} skill in place of your \skillref{healing} skill on Tests to heal people and creatures of most physical injuries.
	Setting broken bones and performing surgery still requires \skillref{healing}.
	
	Similarly, you may use your \skillref{brewing} rank in place of your \skillref{healing} rank when determining the {\damage} healed by a patient during a {\dayofrest}.
}

\feat{Healing Stimulant}{brewing-healing-2}{10}{
	\skillref[1]{brewing},
	\featref{brewing-healing}
}{
	\mix{cauldron}{drink}{\Herb{goldenrod}{2}}
	%Goldenrod is used as traditional medicine to heal wounds.
	
	This potion stimulates the body's natural growth and repair.
	However, it only works as long as the drinker rests.
	With one dose of the potion, the drinker heals twice as much damage during a {\dayofrest}.
	That is, they heal 4 points of {\damage}, plus any healed by their physician; the physician's healing is not doubled.
	The drinker needs to eat at least twice as much as normal, to fuel the regeneration.
	
	Taking more than one dose in a day causes overgrowth; scabs and scars develop even where the drinker is unwounded, and they can become deformed.
	Healing is no faster, but they need to eat even more.
}

\feat{Life Juice}{brewing-healing-3}{20}{
	\skillref[2]{brewing},
	\featref{brewing-healing-2}
}{
	\mix{cauldron}{drink}{\herbcreature{arbor-vitae}{5}}
	
	This potion acts like \featref{brewing-healing-2}, but much faster, and without the need to rest.
	The drinker heals up to 4 points of {\damage} over the course of a few minutes.
	However, they suffer one level of {\exhaustion}, affecting \attref{might} and \attref{grace}.
	
	Two simultaneously effective doses cause the same effect as an overdose of \featref{brewing-healing-2}.
	Additionally, consecutive doses continue to cause more {\exhaustion}, which may cause the drinker to pass out, as usual.
}

\feat{Regrowth Potion}{brewing-healing-regrow}{10}{
	\skillref[2]{brewing},
	\featref{brewing-healing-2}
}{
	Salamanders have the incredible ability to regrow lost limbs.
	By adding a salamander's leg to the recipe for your \featref{brewing-healing-2} or \featref{brewing-healing-3}, you can grant the same ability to the drinker.
	
	The drinker can regrow missing body parts.
	Fingers and ears will grow back with a single dose, while a limb or eye takes several doses.
	A dose of \featref{brewing-healing-2} must be accompanied by a {\dayofrest}, as usual, and multiple doses must be spread across multiple days.
}

\feat{Hair of the Dog}{brewing-healing-taglock}{20}{
	\skillref[2]{brewing},
	\featref{brewing-healing}
}{
	Folk medicine claims that, when bitten by a dog, you should apply a few hairs of the dog that bit you to the wound.
	It never worked with the raw hairs, but you've discovered a quick preparation that fixes that.
	
	\mix{poultice}{topical}{A \materialref{taglock} from the creature that attacked you, \herb{dock leaves}{1}, alcohol}
	
	Applied to a wound, this mixture will completely heal the wound over the course of a few minutes.
	This allows you to heal all the {\damage} dealt by the creature whose \materialref{taglock} is used in creating the poultice.
	However, it can only heal {\damage} dealt by {\unarmed} attacks---punches, bites, claws, and the like---not any {\damage} dealt with a weapon.
	Furthermore, it will only close wounds and heal {\damage}; it cannot regrow missing parts.
}

\feat{Exposure}{poison-resistance}{15}{
	\skillref[1]{brewing}
}{
	You've tasted a few too many of your own concoctions, but you're still alive.
	In fact, you're beginning to build up a bit of resistance.
	
	You roll an extra die on Tests to resist poisons or the like.
}

\feat{Bottled Sobriety}{antidote-alcohol}{10}{
	\skillref[1]{brewing}
}{
	\mix{cauldron}{spike}{\Herb{tea leaves}{4}}
	%Tea contains caffeine, commonly touted as sobriety-inducing.
	
	This concoction acts as an {\antidote} to alcohol of any kind, as well as any other mixture which lists alcohol as an ingredient, such as \featref{might-potion}.
	Alas, it won't cure a hangover.
}

\feat{Universal Antidote}{antidote-all}{25}{
	\skillref[3]{brewing},
	\featref{antidote-alcohol}
}{
	You have achieved one of the philosophers' stones of \discref{brewing}, the universal {\antidote}.
	
	\mix{cauldron}{snuff}{A toad's brain, \herbcreature{moly}{5}}
	%Moly is a mythical herb from the Odyssey that gave immunity to sorcery.
	%The toad brain is firstly a reference to toadstones, supposedly an antidote to any poison.
	%Secondly, it is established by familiars that toad's have a knowledge of brewing.
	%This knowledge might include the knowledge to counteract any brew.
	
	Taken by itself, this powder is inert; it has no effect.
	However, when this powder is combined with any other mixture, that mixture loses its effect and becomes precisely its own {\antidote}.
	This requires one dose of \featref{antidote-all} and one dose of the other mixture.
	The application method of the other mixture is not changed by this process: a \mixdeliveryref{spike} remains a \mixdeliveryref{spike}, \mixdeliveryref{snuff} remains \mixdeliveryref{snuff}, and so on.
}


Elle Weerstrom looked up from her parsley patch as the air swooshed overhead.
Black fabric flapped.

``Evenin' Linda.
Didn't expect to see you today.
What brings you up 'ere?''

A navy-lined cloak fluttered as the younger witch pulled her broomstick short and dropped to the ground.
``It's young Barnie, Elle.
He's got a mob together, marching on Buckle Hollow.
Says Musgrave's been sleeping with his wife.''

Elle brushed her gloves together, knocking dirt onto the lawn.
``Well, has he?''

``No!
I mean, they might've kissed a bit but{\dots}
They've got torches, Elle!
Pitchforks and torches!
C'mon, grab your broom.
We've got to stop them.''

Elle looked down at the ground, then up at the sky.
She sighed.
``Alright, we'll go.
But we're walkin'; there's a storm brewing.''

Linda looked up.
A single wisp of cloud drifted lazily across the azure sky.
``Looks alright to me.''

``It's on its way, mark my words.
Wouldn't want to be flyin' home in it.''
Elle strode towards her cottage.
``I'm goin' to get my coat.''

\storybreak

Sure enough, the sky was grey when the mob got to Buckle Hollow.
A fine drizzle filled the air.
The farm gate stood open, a figure between the posts in its place.
Her parka was pulled up against the rain, pointed hat tall above her crown.
The mob stopped in its tracks as a crack of lightning cast her silhouette upon them.

``Fine weather for arson, innit?''
Her voice seemed to carry further than it should in the damp air, reaching the ears of all present.
They shuffled their feet in the thickening mud.
``Yer a disappointment, the lot o' yer.''
More feet shuffled.
A voice rose in dissent, but Elle continued over it.

``Now, I know Musgrave ain't the finest man you've all met.
An' I ain't quite sure what he's been up to that's got you all riled up.
But I \emph{am} sure that it ain't nothin' worse than half o' you've done to yer own wives!
Honestly, torches lads?''
The rain intensified and the torches guttered.
One spluttered out.
``What were you goin' to burn?
The barn?
His house?
\emph{Him}?
Put 'em away, men.''

There was another shuffling of feet, and a few torches wobbled noncommittally.
A sudden gust of wind drove the rain sideways for a moment.
Every torch went out with a pathetic cough.
``Get home to yer own wives, an' stop worryin' about other people's.''

With a quiet mumble, a general grumble and a mutter of ``Soddin' linen's gonna be soaked{\dots}'' the mob turned around and began to trudge the other way.

``An' Barnie!''
The mob stopped in its tracks again.
One man turned around, a few others craned their necks to see.
``She mightn't be kissin' other blokes if you spent as much time in yer own bed as in the gutter out back o' the Head.''
A muffled chuckle ran through the mob before another peal of thunder cut it short.
Collectively, they slank off through the mud.

\chapter{Willing}
\chaplabel{willing}

\discref{willing} is the most raw and versatile application of a witch's magic.
Known to many layfolk as sorcery or spellcraft, it is the art of making something true simply by willing it hard enough.
Most \discref{willing} is performed without any of the accoutrements that accompany other forms of magic, and it doesn't follow the prescribed formulae of rites and brews.
This makes it the weakest form of magic in some ways, but its flexibility and ease of access more than make up for it.
So much so that every witch knows at least the basics.

Like any witchcraft, \discref{willing} is something anyone can do if they know how.
But there is a knack to it.
It requires that the witch not only \emph{want} something to be the case, but \emph{believe} that it already is.
That she outright refuses to accept any possibility that it might not, in fact, be the case.
It involves willfully deceiving not only oneself, but also the very universe.
Most people would never even think to try it, but it is among the first things any aspiring witch must learn.

The line between \discref{willing} and \discref{headology} can be a little blurred, at times.
Both have the ability to make things true by making people believe them.
Many Willers say that the difference is that \discref{willing} affects the real world, while \discref{headology} only affects other people's minds.
The Headologists point out that other people are just as much a part of the real world as any old rock is.
Some Headologists say that the difference is that \discref{headology} is about convincing other people, while \discref{willing} is about convincing yourself.
The Willers point out that it's about more than convincing yourself, it's about convincing the world.
And that includes other people.
A few say that there's no real difference at all, that it's just two ways of thinking about the same thing.
These tend to be the witches who are obnoxiously good at both, and everyone else pointedly ignores them.

One interesting property of \discref{willing} is that it cannot affect other people or animals.
It takes more than force of will to convince someone that they're a different shape; usually this entails talking to them.
This doesn't stop people getting knocked off their feet by a gust of wind, or crushed by a falling tree, however.
Witches interested in affecting people more directly are encouraged to pursue \discref{headology}.
Or swordplay.

Unlike many magical disciplines, which depend upon \attref{wit} for understanding or memorising their complexities, \discref{willing} depends upon raw \attref{will}, your own stubbornness and conviction against the fabric of reality.

\section{Feats}

\feat{Basic Willing}{basic-willing}{10}{
	None
}{
	You can perform very basic acts of \discref{willing} upon things you can touch, given a bit of time to focus your mind and an obvious physical cue.
	Examples include:
	\begin{itemize}
		\item Lighting kindling or a candle without a spark, by cupping your hands around it and blowing on it.
		%\item Colouring or mildly flavouring a small pot of water by stirring it.
		\item Scratching writing into stone using just a fingernail.
		\item Rubbing stains out of clothing using your bare hands.
		\item Combing your hair with just your fingers.
	\end{itemize}
	The amount of time required to produce an effect varies depending on the desired outcome, but should be more than an Action without a Test.
	This ability cannot produce a lasting effect by itself.
	You can light a fire, because that sustains itself once ignited, but you cannot create, destroy or melt a pebble.
}

\feat{Kindling}{fire-willing}{15}{
	\featref{basic-willing}
}{
	You've practiced \discref{willing} a fire to life, and it's getting a lot easier for you.
	You can now ignite a fire within a dozen metres as an Action, with nothing more than a quick glare.
	You no longer require kindling, but still need something a fire can catch on fairly easily, such as twigs, cloth or dry leaves.
	Lighting a log or floorboards is still beyond you.
	
	The fire begins small, so will be extinguished by rain or a moderate wind before it can catch.
	A person walking about or wriggling will automatically foil an attempt to ignite their clothes (perhaps without noticing), but a person sitting fairly still may not.
}

\feat{A Tool for the Job}{willing-tools-improvise}{20}{
	\featref{basic-willing}
}{
	Sometimes, the easiest way to convince someone of something is the hit them with a big stick until they agree with you.
	The world itself is no different.
	You've learned to make \discref{willing} easier using physical tools, even if they aren't the \emph{right} tools.
	
	Most simply, this means axes and knives cut just as well as ever in your hands, even if they've lost their edge.
	But you can take it even further, cutting carrots or trees with nothing more than an appropriately shaped stick.
	You can make any similarly-shaped object behave as the appropriate tool for a job.
	For a worse approximation, this may require a Test.
	A solid branch with a flat, sort of axe head shaped bit on the end will do a fine job of cutting down a tree.
	A solid branch without such an attachment would require a Test.
	A limp reed is going to be a real stretch.
	
	Such tools still obey the usual rules of \discref{willing}, and are of no additional use as weapons against people and animals.
	See \featref{headology-weapons-improvise} if you want weapons too.
}

\feat{Bubbling Brook}{water-willing}{10}{
	\featref{basic-willing}
}{
	Water is considered by many to be an element of change.
	You've certainly figured out how to change it.
	While touching water, you can move it around with your mind.
	You can make it flow, swirl, form into fairly elaborate shapes, or even float into the air.
	
	You can only affect the water while it remains one continguous mass, which you must be touching.
	Afterwards, it flows normally again.
	You can only affect a couple of buckets-full at a time, and can't stretch it out over more than a couple of metres.
	You also can't move the water fast enough to hurt anybody.
	You can move other liquids if they are primarily water, such as wine, blood or most potions.
	As always with \discref{willing}, you cannot affect liquids inside a living person.
}

\feat{Water Walk}{water-walk}{20}{
	\skillref[1]{willing},
	\featref{water-willing}
}{
	You can walk on water, or any other liquid you could affect with \featref{water-willing}.
	This takes great concentration, and you cannot take an Action and move on the water's surface in the same Turn.
	You may take an Action if you stand still on the water, however.
	
	If the water is flowing, you will be carried with it.
	Staying upright on fast flowing or turbulent water may require a Test, and the effect requires you to stay on your feet; falling prone will cause you to fall into the water.
	You may take use an entire Turn to clamber onto the water, if you are swimming at the surface.
}

\feat{River Run}{water-walk-2}{15}{
	\skillref[2]{willing},
	\featref{water-walk}
}{
	Walking on water has become second nature to you.
	You may take Actions while moving.
	Additionally, flow and turbulence pose you no threat.
	You may treat water you are standing on as though it were not flowing, and you can remain on the water's surface even when prone.
	Lastly, climbing upright onto the water while swimming at the surface is treated as though you are merely standing from being prone.
}



\discipline{Headology}{headology}{Headologist}{Headologists}

\dropcapdiscref{headology} is really no magic at all.
Rather, it is the art of making other people use their own magic.

One common misconception among apprentice witches is that \discref{headology} is the ability to affect people's minds.
Their mentors must quickly disabuse them of this notion.
Everyone has the ability to affect people's minds, and uses it every day.
It's called talking.
It can make someone like you or hate you, make them smarter or more stupid, even make them believe that the sky is purple, if you're really good.
It's an incredibly ability---the most important one a witch can have, in the opinion of most.
Enough people, sufficiently motivated, can move mountains.
But talking, by itself, is not \discref{headology}.

\discref{headology} is the step that comes after.
\discref{headology} is making people's minds affect the world---letting them move mountains without all the shovels and wheelbarrows it normally requires.
Making them into \practitioners{willing}, without them even knowing it.

Every person, and even every animal, has the ability to affect the world through \discref{willing}.
Most never realise this, and would struggle to control the power even if they knew.
But a witch who knows the trick of it can unlock another person's ability.
And if she's convinced them of the correct things first, she can direct it with her words.
This is the basis of most \discref{headology}.

\section{Convincing People}

Almost every feat in \discref{headology} requires a witch to convince somebody of something, before it has any affect.
Talking to people is a complicated subject, and there are no strict rules for this.
As such \discref{headology} is more subject to the whims of the GM than many other disciplines.
The following paragraphs provide many guidelines for adjudicating this, but the GM should also apply common sense, and remember to ensure that everyone is having fun.
If the the \practitioner{headology} is consistently overshadowing the rest of the coven, it's probably proving too easy to convince people of things, and vice versa.
And if they've put on a particularly awesome show to convince someone, just let it work.

Firstly, if there is doubt as to whether a \practitioner{headology} has convinced someone, call for an {\opposed} Test.
On the part of the \practitioner{headology}, this will normally use \attref{charm} or \attref{presence}, and \skillref{persuasion} or \skillref{deception}.
\skillref{intimidation} and \skillref{socialising} might come into it fairly often, as well.
On the part of the intended victim, this might use \attref{will}, to hold onto a conviction, or \testtype{heed}{insight}, to see through a trick.

\subsection{Modifiers to Convincing}

For anything but the simplest effects---such as \featref{curse}---simply stating something is not enough to convince someone, no matter how persuasive your tone.
In these circumstances, the GM should simply not let the victim be convinced, or at least apply a major penalty to the Test to convince them.
Often, some variety of evidence or trick is required.

For example, a prince who's simply told he's a frog is unlikely to fall for it.
But a prince who's told he's a frog, then gets knocked out, and wakes up in a pond with his skin covered in slime---he's more likely to buy it.
With the right evidence, a witch might not need to speak to the victim at all.
A prince who wakes up in a pond, surrounded by other frogs, all dressed in the armour of his personal guard---he's going to leap to his own conclusions.

As such, the GM should use the circumstances to put modifiers on rolls to convince people.
This should often be a negative modifier without a good argument or some evidence, while presenting a solid piece of evidence can give a positive modifier.
An elaborate---but solid and successfully executed---plan for convincing someone will often bypass the need for a direct Test altogether.
The type of effect being applied should also influence the modifier.
It is much easier to convince someone that they're under a simple bad-luck curse than that they're a frog.

Lastly, the \practitionerpossessive{headology} reputation can be important.
If the victim knows that she is a powerful witch, this can go a long way by itself.
If she specifically has a reputation for turning people into frogs, people are likely to believe her pretty easily when she says she's turning them into a frog too.
Even more so if they've just seen her do it to one of their friends.
Practically, this means that a \practitioner{headology} often needs to make it clear that she's a witch, by wearing the {\hat}.

\subsection{The Trick of Headology}

One unfortunate catch of \discref{headology} is that it only works as long as the victim is unaware of quite what's being pulled on them.
As soon as someone realises that they'll only turn into a frog if they believe they're a frog, they'll never believe it.
Even if they try.
This means that it is impossible to \emph{willingly} be the subject of \discref{headology}.

It also means that a \practitioner{headology} needs to be careful not to let their victims catch on to what's happening.
This is not commonly a problem with normal folk, unless someone explicitly explains it them.
Superstition runs rife, and a witch who uses a lot of \discref{headology} is likely to provoke more fear and respect than understanding.
However, a \practitioner{headology} ought to maintain a certain mystique about her craft, to ensure no clever clogs goes digging too deeply.

With other witches, however, tend to catch on quite quickly.
A witch who has seen a particular trick of \discref{headology} used a couple of times tends to figure it out, and thereby become immune to it, whether she wants to or not.
Other \practitioners{headology} tend to be even quicker on the uptake, and are likely to catch on the very first time they see a trick, if the witch using it on them isn't careful.
A witch who knows and uses a trick herself can never be affected by it, except, perhaps, in the most exceptional caper of all time.

Perhaps mostly importantly, this means that you can never use your \discref{headology} on your own coven, unless you are careful to keep them in the dark about a new trick you've picked up.
Even then, it won't last long.

\section{Feats}

\feat{Curse}{curse}{15}{
	\noprereq
}{
	It's a well known fact that someone who believes they will fail is more likely to do so.
	It doesn't take a drop of magic to make that true, but not everyone knows how to leverage it.
	You do.
	
	If someone believes that you have cursed them, or even if you can convince them that they have been cursed by something else, they suffer bad luck.
	Whenever they make a Test, dice that roll a 3 count towards a critical failure.
	The GM is also encouraged to make their critical failures a little more dire.
	This bad luck persists as long as the supposed curse is present in their minds; it might help to remind them now and again.
	
	This only applies if they believe they are under a rather broad curse, or specifically a bad luck curse.
	An overly specific curse---for example, ``May your crops wither in your fields,'' or ``May your nose fall from your face''---does nothing to focus their mind on their own failure and will have no effect.
}

\feat{Mentally Scarred}{headology-wound}{10}{
	\featref{curse}
}{
	You have mastered a more specific form of curse---a curse of physical wounding.
	
	If you can convince someone that they are wounded, they develop the wounds they believe they have.
	This directly causes {\damage}---not a {\damagetest}---appropriate to the kind of wound they develop.
}

\feat{Mind over Magic}{foil-magic}{15}{
	\noprereq
}{
	For all the magic circles and burning incense, magic ultimately comes from the mind.
	Not only do you know this, but you know \emph{how to exploit it}.
	
	If you can convince a practitioner of magic that their magic won't work, then it won't.
}

\feat{Doubt \& Despair}{foil-magic-2}{25}{
	\featref{foil-magic}
}{
	Under your tender care, even the smallest seed of doubt can flourish into a blossoming tree of failure.
	
	If you can make a practitioner so much as doubt the efficacy of their magic, or their own ability to work it, then the magic will either fail to work or, at the GM's option, backfire.
}

\feat{Mind Like a Razor}{headology-weapons-improvise}{10}{
	\featref{willing-tools-improvise}
}{
	If you can convince your foes that what you wield is a weapon, their flesh will believe you.
	You may treat an item you wield or throw as a \weaponref{knife}, \weaponref{hand-weapon} or \weaponref{thrown-weapon} (depending on its size and whether you're throwing it) if you can convince the target that it can cut (or otherwise deal damage) like one.
	A demonstration against an inanimate object, or another foe, will often suffice.
	Even your bare hands can cut like \weaponrefplural{knife} if you convince your foes that they can.
}

\feat{Change Blindness}{headology-stealth}{10}{
	\noprereq
}{
	You may hide in plain sight by leveraging the fact that people don't \emph{expect} to see you there.
	This uses a \testtype{charm}{stealth} Test.
	You must remain silent and quite still, though you may creep around slowly.
	
	In order to make use of this feat, anyone you are hiding from must have no reason to expect to see you, or anyone.
	If they see much out of place---a drawer opened or a vase knocked over---they might look for whoever did it and will immediately spot you.
	Furthermore, you can only use it if the people you are hiding from have some degree of familiarity with the location; they must have seen it before, at least.
	Somebody entering a room for the very first time doesn't know what to expect and will see it as it is, you included.
	
	Lastly, somebody seeing a group or crowd of people has no reason not to expect other people with them.
	This feat does not allow you to hide in such a situation, unless everyone in the group has the feat.
	%TODO: Is there a feat that helps blending in with a crowd?
}

\feat{Elsewhere}{headology-stealth-2}{15}{
	\featref{headology-stealth}
}{
	While \featref{headology-stealth} lets you hide from people who aren't expecting \emph{anyone}, you've now figured out how to hide from people who aren't expecting \emph{you}.
	As long as someone is convinced \emph{you} won't be somewhere---for instance, you've told them you'll be somewhere else---they won't see you there.
	Note that it is not enough for them not to expect you there---except as falls under the purview of \featref{headology-stealth}---they must expect you not to be there.
	
	This only holds up as long as you aren't too too intrusive.
	For example, you shouldn't pass in front of something they are paying attention to, make any loud noises, or open any doors they are looking at.
	However, you might be able to get away with moving things around.
	Even if somebody notices that something has been moved, they ought not to suspect \emph{you} to have done it, as long as they still believe you are somewhere else.
	Tests to avoid being noticed, if it is in doubt, use \testtype{charm}{stealth}.
	
	Furthermore, this feat does allow you to go unnoticed in a crowd, as long as the person watching has good reason to believe you won't there.
}

\feat{You Shall Not Pass!}{headology-barrier}{20}{
	\noprereq
}{
	You may erect barriers inside people's heads, allowing them to project them into reality.
	If you convince someone that they cannot pass some barrier, they become unable to.
	Even if they are thrown bodily against the barrier, they will bounce off it.
	This does not prevent them throwing stones, poking a stick, using magic across the barrier, or the like.
	
	The barrier can be of any shape or nature that you can convince the target of.
	For example, you might draw a line in the sand, convince them that they cannot enter a house, or tell them that they cannot touch you.
}

\feat{Fake Sympathy}{headology-sympathetic-magic}{25}{
	\skillref[1]{sympathetic-magic},
	any feat giving a use for {\symlinks}
}{
	Although you know how to perform \discref{sympathetic-magic}, you've also figured out how to skimp on the magic and just use \discref{headology}.
	You may establish a fake {\symlink} just by convincing the target that you have established one.
	They need not understand the actual mechanisms of \discref{sympathetic-magic}---in fact, it's probably better if they don't---they just need to know that by affecting the {\symbol}, you can affect them.
	Establishing this fake link does not require the usual Test, only any Tests to convince the target.
	It does not count towards your maximum number of {\symlinks}
	It lasts as long as the target continues to believe it does---as such, it is not subject to {\stress}.
	
	You may transmit any effects along this fake link that you could along a normal {\symlink}---anything you possess the appropriate \discref{sympathetic-magic} feat for.
	However, you may only do so by showing the target what you are doing, and even explaining it if necessary.
	For example, \featref{sympathetic-speak} is useless: if the target cannot hear the sounds anyway, they don't know what to expect, and receive nothing.
	
	This {\symlink} doesn't actually exist in any sense, so you cannot modify it in any way you could modify a normal {\symlink}.
	However, nor can anybody else, and it is not impeded by anything that would impede a normal {\symlink}, unless the target is aware of and believes in such impedance.
}

\feat{Placebo}{headology-brewing}{15}{
	\noprereq
}{
	Often, the promise of a cure is more important than the cure itself.
	You can save a lot of time brewing this way, if you just talk to people.
	
	If you know how to make a brew, and have a mixture of approximately the right size, consistency, and colour, you might be able to use that instead.
	If you can convince someone that what they're taking will have the effect of that brew, then it acts as that brew for them.
	This works not only with brews that you have a feat to make, but also the same minor remedies that you might otherwise make with a \skillref{brewing} Test.
	However, if a brew requires a feat to make, and you don't have that feat, this won't work.
	
	This only works if they are convinced at the time they take the brew; it can't work retroactively.
	As such, it's not all that much use for poisoning people.
}

\feat{Retroactive Placebo}{headology-brewing-2}{15}{
	\featref{headology-brewing}
}{
	If you try to convince someone that you've poisoned their wine, they're hardly likely to drink it.
	But if you convince someone that you'd poisoned the wine they've just drunk, they might well drop dead.
	
	You may use \featref{headology-brewing} even if you convince someone \emph{after} they take the brew.
	Bear in mind that, obviously, they're unlikely to believe you if there's no way you could have touched the mixture they drank.
	
	The time taken for the brew to kick in is counted from when they took it, not when you convince them.
	As such, the effect will often kick in immediately after you convince them.
	It doesn't matter if this means it kicks in late, as long as they can believe they've been resisting it, or it's rather slow-acting.
	Convincing someone that yesterday's poison is only now affecting them might be difficult, though.
	
	This also functions for \featref{headology-brewing-antidote} and \featref{headology-brewing-antidote-2}; you can convince people that a mixture was an {\antidote} after they take it.
}

\feat{Poison is in the Mind}{headology-brewing-antidote}{10}{
	\featref{headology-brewing}
}{
	Sometimes it's useful to end the effect of a potion without giving away that it was fake all along.
	In this case, you can give someone an {\antidote}.
	Just as fake as the original, of course.
	
	If you can convince someone that a mixture they take is an {\antidote}, it functions as one.
	It counteracts whichever mixtures you convince them that it will.
	However, it can \emph{only} counteract mixtures that were applied using \featref{headology-brewing} in the first place; it is not effective against any real brew.
}

\feat{Placebo Panacea}{headology-brewing-antidote-2}{20}{
	\featref{headology-brewing-antidote}
}{
	You can counteract even the deadliest poisons with plain water, if your powers of persuasion are up to scratch.
	When you use \featref{headology-brewing-antidote}, the fake {\antidote} may counteract \emph{any} brew, even a real one.
}


Today was not going well for Linda Greene.
It had started out alright.
A brisk walk in the frosty air at sunrise, a quick trip up to the castle to drop off a couple of poultices for the servants there.
The cook had even given her a big side of braised ham for her help.
But things had gone downhill pretty quickly when the warty old crone had strolled in and started turning people into frogs.

Now here she was, speeding over the mountaintops, hair and cloak whipped back by the frozen wind, and a crazy old hag hot on her tail.
The crone had a wicked-looking knife clutched between her teeth.
Been screaming that she was going to gut the king with it, or somthing unpleasant like that.
Well, the king was safe for now, even if he was croaking rather indignantly.
Linda had stuffed him down her blouse so his now-cold-blooded majesty wouldn't freeze in the mountain air.
It did explain the indignancy, perhaps, but Linda had bigger problems on her mind.
The hag was gaining on her.

Linda leant right forwards and threw the stick into a dive.
She picked up speed as she shed altitude, but the hag quickly followed suit.
Her feet brushed the snow as she skimmed down the far side of the mountain

\dots %TODO: Write the middle of the chase.

She wasn't the best flyer in the world, she knew that.
She wasn't even the best in the kingdom; young Wren up Salwich way could fly circles around her.
And this hag, too, was clearly better than her.
But the problem with being the best was that there were some things you didn't actually get to practice that much.
Some things that Linda, who was the first to admit that her reach often exceeded her grasp, got to practice all too often.

%TODO: Stall the sticks, and have Linda recover. The hag falls into a snowdrift.

\chapter{Broomcraft}
\chaplabel{broomcraft}

A broom is primarily a witch's method of getting from A to B: from village to village, out into a distant forest, or all the way up the city.
It's not the easiest mode of transport, and it can be quite terrifying at first, but a witch can pick up the rudiments in a week or two's practice.
This is as far as most witches go.
But some, with enough practice, skill and flamboyance, can turn it into a real art.

\section{Laws of Aviation}

For an unpracticed witch, there are a lot limitations to broomstick flying.
After all, she is sitting on a thin stick floating hundreds of metres in the air.
First and foremost, it is easiest to balance on a broomstick if one sits side-saddle, and this is all an unpracticed witch is capable of.
This does, however, make it a lot harder to turn, and to fly at high speeds.
Barrel rolls are right out.

\subsection{Taking Off}

Getting the broomstick off the ground in the first place is no easy task.
A broom needs a running start before the magic will catch, and even then it isn't consistent.
The witch must hold the broomstick level as she runs along the ground, then jump on it quickly when it starts.

Attempting to start a broom requires an Action and a 15 metre run-up.
A character must move this distance in a straight line on one Turn, and may Dash as part of the broom-starting action if necessary. %TODO: Ensure Dashing is in the rules.
They must also succeed on a TN 12 Grace + Flying Test or the broom fails to start.
As normal, the Test is not required if there is no time pressure, as the witch may run up and down as many times as necessary until the broom starts.

The Test to start a broom may be more difficult in adverse conditions; the following table provides suggestions for the TN of such Tests.
It is possible to achieve the necessary run-up through falling, although such a thing is \emph{very} difficult and the consequences for failure are obviously drastic.

\begin{simpletable}{rX}
	\toprule
	TN & Conditions\\
	\midrule
	12 & Nominal.\\
	15 & Blowing a gale.\\
	18 & In a bog.\\
	21 & While falling.\\
	\bottomrule
\end{simpletable}

\subsection{Lift}

A broomstick can carry one witch, and about as much equipment as she could easily walk around with on the ground.
It can also carry a familiar, as long as it's of reasonable size.
A cat is fine, a beagle is borderline, a wolfhound is right out.

A little bit of extra weight, or something inconveniently large, makes the broomstick unwieldy.
Tests to take off or perform manoeuvres are more difficult, and the broom's maximum speed may be reduced.
A lot of extra weight, such as a passenger, makes proper flight impossible.
The broomstick cannot take off, and cannot remain in level flight.
It might still be possible to bring it down and land safely, with an appropriate Test.



\section{Feats}

\feat{Ride Astride}{astride}{
	None
}{
	By sitting astride your broom, instead of side-saddle, you can go faster and turn more sharply without falling off.
	It's harder to balance, but you've got the hang of it now.
	
	%TODO
}

\feat{Chocks Away}{astride}{
	None
}{
	There's a simple knack to starting a broom, and you've got it down pat now.
	You don't need a Test to start a broom under normal conditions (although you still need the run-up), and the TN of any Test to start the broom under difficult conditions is reduced by 3.
}

\feat{Broom Whisperer}{untrained-broom}{
	\skillref[1]{flying}
}{
	You've got the knack of flying for yourself now, and don't need a broom to be trained to fly it.
	You can even train a broom this way, although without one of its own to learn from the process takes about 24 hours of flight time.
}


\chapter{Sympathetic Magic}
\disclabel{sympathetic-magic}{Sympathist}{Sympathists}

\section{Sympathetic Links \& Symbols}
\seclabel{sympathetic-links}

Central to the practice of \discref{sympathetic-magic} is the creation and manipulation of {\symbols}.
A {\symbol} is a representation of a creature or object, and by affecting the {\symbol} a witch may cause a mirroring effect upon the target.
Not every \materialref{poppet} or \materialref{effigy} is automatically a {\symbol}.
It must by magically bound to the target by a {\symlink}.

A novice witch can only maintain one {\symlink} at a time.
It's not that maintaining one is particularly arduous; once established, a {\symlink} remains in place indefinitely, as long as the target is not resisting it.
Rather, two {\symlinks} tend to tangle themselves up, like pieces of string left together in a drawer.
Soon enough, both are totally useless and they have to be cut to separate them.

A {\symlink} by itself does nothing, but a \practitioner{sympathetic-magic} soon learns to use it to transmit numerous things: sensations, physical effects and more.
A {\symlink} doesn't always transmit everything it is capable of transmitting: only what the witch who established it wants it to.
The witch can change what the link transmits at any point she chooses, regardless of proximity to the {\symbol} or the target.
However, she has no particular sense of what is being transmitted by the link, and must watch the {\symbol} or the target if she wants to know.
As such, leaving {\symbols} lying around is a slightly dangerous proposition.

\subsection{Establishing a Sympathetic Link}

The simplest method for establishing a {\symlink} actually relies upon a trick of \discref{headology}.
The target must be \emph{expecting} the link, allowing the witch the opportunity to fasten it in place.
As such, at first, the witch can only establish {\symlinks} with people as the target, using a \materialref{poppet} or \materialref{effigy} as the {\symbol}.

Establishing the link requires an {\action}.
The target must see the {\symbol}, and the witch must declare to the target that she is binding them together.
Many witches adopt a standard incantation for this, often some piece of mumbo jumbo that suits the mystique they wish to cultivate.
The important thing is that the target understands the intent---that they are \emph{convinced} by it is not so important as in ``true'' \discref{headology}.

The target's expectation provides a hook that the witch may fasten the {\symlink} to.
If the target welcomes the {\symlink}, this is easy---it is established automatically and remains in place indefinitely.
Otherwise, establishing the link requires a \testtype{wit}{sympathetic-magic} Test {\opposed} by the target's \testtype{will}{sympathetic-magic}.

\subsection{Severing a Sympathetic Link}
\seclabel{break-sympathetic-link}

A witch can sever any {\symlink} she has established as an {\action}, or as part of establishing any new {\symlink}.
Additionally, a {\symlink} is severed if the {\symbol} or target are destroyed, or die.

Otherwise, a {\symlink} to an object, a willing creature, or a creature who is unaware they are the target of a {\symlink} at all, will persist indefinitely.
However, a {\symlink} to a creature that knows it is the target of a link, and does not wish to be, will be dislodged over time.
It automatically breaks after a minute, but can be broken sooner if it is {\stressed}.
This applies even if the creature previously accepted the link, but now wants rid of it.

Some uses of a {\symlink} will cause it considerable {\stress}, giving an unwilling creature another change to dislodge the link.
In this case, repeat the Test used to establish a link---your \testtype{wit}{sympathetic-magic} {\opposed} by the target's \testtype{will}{sympathetic-magic}.
If the target wins the Test, the {\symlink} is broken.
Actions that {\stress} a link will say so in their relevant feats.

\section{Feats}

\feat{Stable Sympathy}{symlink-stable}{20}{
	\skillref[1]{sympathetic-magic}
}{
	An unwilling target will soon throw off a {\symlink}, but you've learned to stabilise your links against this, leaving them fastened strong in the face of adversity.
	However, this requires some preparation.
	
	By using an \materialref{effigy} in the likeness of the target as the {\symbol}, the {\symlink} does not expire over time, even when resisted.
	However, this does not allow it to resist {\stress}.
	This still requires the usual Test to establish the link in the first place.
}

\feat{Taglock Binding}{symlink-taglock}{20}{
	\skillref[1]{sympathetic-magic}
}{
	Normally, the hook to fasten a {\symlink} in place is provided by the target's \emph{expectation} of a link.
	This is the simplest and strongest way, but not the only one.
	
	You can establish {\symlinks} to creatures, using a \materialref{taglock}, and a \materialref{poppet} or \materialref{effigy} as the {\symbol}.
	Establishing the {\symlink} uses an {\action}, while touching the \materialref{taglock} and the {\symbol}.
	However, {\symlinks} fastened in this way are weaker.
	Anything that would {\stress} the link---or destroy it, as with \featref{sympathetic-damage}---simply snaps the {\symlink} without taking effect.
	Remember, however, that you can always choose not to try and transmit anything that would {\stress}, and hence break, the link.
}

\feat{Twin Links}{symlink-extra}{20}{
	\skillref[1]{sympathetic-magic}
}{
	You may maintain two {\symlinks} simultaneously.
}

\feat{Sympathetic Jerk}{sympathetic-puppet}{15}{
	None
}{
	An expert \practitioner{sympathetic-magic} can make their target dance on the puppet strings of their {\symlink}.
	You aren't there yet, but you've taken the first step.
	
	You cannot control your target's movements, but you---or someone else holding the {\symbol}---can \emph{disrupt} them by jerking the {\symbolpossessive} limb the wrong way at the opportune time.
	If the target is just walking and talking normally, this doesn't do more than faintly disturb them.
	But if they are performing something highly physical or precise---running, jumping, aiming a weapon, or sewing, for example---it can severely disrupt them.
	Jerking the correct limb at the correct time requires knowing what the target is doing, or at least being able to take a very good guess.
	Normally, this means being able to see them.
	
	Typically, you can use this by taking the \actionref{ready} {\action} in order to disrupt the target's next {\action}, while holding their {\symbol}.
	Common disruptions include making them miss an \actionref{attack}, or making them trip and fall prone when jumping or taking the \actionref{dash} {\action}.
	The GM ultimately decides the result of any disruption.
	Disruptions like those listed above do not require a Test, but if the outcome is in doubt, the GM may call for an {\opposedtest}.
	This typically uses \testtype{wit}{sympathetic-magic} for the witch, and might use something like \testtype{grace}{athletics} or \testtype{grace}{weaponry} for the target.
}

\feat{Sympathetic Puppet}{sympathetic-puppet-2}{25}{
	\skillref[1]{sympathetic-magic},
	\featref{sympathetic-puppet}
}{
	You can control someone's actions through a {\symlink}.
	Only intermittently, and not precisely, but that doesn't make it much less terrifying.
	
	As an {\action}, someone can puppet a target by manipulating its linked {\symbolpossessive} limbs.
	The manipulator takes a physical {\action} on behalf of the target, which may be moving up to its \statref{speed} using the \actionref{dash} {\action}.
	This also deprives the target of their {\action} on their next {\turn}---unless that {\action} would be purely non-physical---although they may still make their usual movement.
	
	Using this {\stresses} the {\symlink}.
	
	Puppetry is quite difficult to do precisely.
	You can control limbs, and you can even open and close the hands and jaw, if the {\symbol} has the appropriate anatomy to manipulate.
	But speaking is impossible, and any work with the fingers requires you to manipulate the {\symbolpossessive} fingers with the same precision---a difficult proposition using your own bulky fingers.
	
	The manipulator suffers a \negative{6} penalty to any \emph{physical} Tests they must make on the target's behalf.
	These Tests typically use \attref{grace}, to finely manipulate the {\symbol}, and whichever skill would be used for performing the {\action} normally.
	However, \skillrefspeciality{performance}{Puppeteer} can be used in place of the normal skill.
}

\feat{Sympathetic Destruction}{sympathetic-damage}{20}{
	None
}{
	When a {\symbol} is destroyed, you can send its death throes lashing along the {\symlink}, tearing at its target.
	Roll a {\damagetest} against the target, using \testtype{wit}{sympathetic-magic}.
	This works against objects, as well as creatures.
	
	Tearing a {\symbol} apart typically requires an {\action}, though you might find a faster way to destroy it.
	The destruction of the {\symbol} obviously terminates the {\symlink}.
}

\feat{Sympathetic Stabbing}{sympathetic-damage-2}{15}{
	\skillref[1]{sympathetic-magic},
	\featref{sympathetic-damage}
}{
	You no longer need to destroy a {\symbol} outright to wound the target.
	When a {\symbol} is significantly damaged in some way---sticking a pin in it is traditional---you may roll a {\damagetest} against the target, using \testtype{wit}{sympathetic-magic}.
	This works against objects, as well as creatures.
	Using this effect {\stresses} the {\symlink}.
	
	Attacking a {\symbol} to activate this should typically require an {\action}, though you might find a faster way to damage it.
}

\feat{Sympathetic Buoyancy}{sympathetic-weight}{10}{
	None
}{
	The mass of a {\symbol} affects the mass of its target: a stone or iron \materialref{poppet} will make a person heavier while a wood or cloth one will make them lighter.
	Not hugely so---no more than about \SI{25}{\percent}---but enough to make a person easily float or sink, and to aid or hinder jumping and climbing.
	%TODO: Mechanical effects on jumping, etc.
	
	This effect can be used on objects as well as creatures, making them easier or harder to lift and carry.
}

\feat{Sympathetic Sleep}{sympathetic-sleep}{10}{
	None
}{
	A {\symbol} can rest in place of its target, allowing the target to work through most of the night.
	The rest, the {\symbol} needs to be tucked into a small bed, with soft bedding, a pillow, and sheets.
	It needs to be in a quiet, dim location, and generally to be in conditions where a person could easily sleep.
	The {\symbol} cannot be used for any other \discref{sympathetic-magic} while it is resting.
	
	As long as the {\symbol} rests for at least 8 hours each day, the target can get by on only 1 hour of sleep each day without any ill effects.
	However, the target does not recover from {\damage} and {\exhaustion} as a result of this rest.
}

\feat{Sympathetic Insomnia}{sympathetic-sleep-deprive}{15}{
	\skillref[1]{sympathetic-magic},
	\featref{sympathetic-sleep},
	\featref{symlink-stable}
}{
	By keeping a {\symbol} awake, you can deprive its target of restful sleep.
	If the {\symbol} is subjected to loud noises, bright lights, stony bedding, or other significant discomforts while the target sleeps, the sleep will be fitful and restless.
	The sleep does not help them recover from {\damage} or {\exhaustion} (although they may still benefit from a day of rest).
	If this goes on for several nights, they may begin suffering {\exhaustion} due to sleep deprivation.
}

\feat{Sympathetic Narcolepsy}{sympathetic-sleep-cause}{15}{
	\skillref[1]{sympathetic-magic},
	\featref{sympathetic-sleep},
	\featref{symlink-stable}
}{
	\featref{sympathetic-sleep} lets a {\symbol} sleep instead of the target.
	You've reversed this, and may instead let the {\symbol} send the target to sleep.
	
	If you tuck a {\symbol} in, as you would for \featref{sympathetic-sleep}, then you may cause it to bring on tiredness in the target.
	This does not kick in for a minute, while the {\symbol} falls asleep.
	After this minute, make a \testtype{wit}{sympathetic-magic} {\opposed} by the target's \attref{will} Test.
	If you succeed, the target falls into a deep sleep.
	They cannot be roused for 8 hours (so long as the {\symbol} continues to sleep), but benefit as though they were sleeping naturally.
	
	Succeed or fail, this will not work on the same target again for another 24 hours.
	They've either slept of the tiredness, or fought through it.
}

\feat{Sympathetic Warmth}{sympathetic-heat}{10}{
	None
}{
	The temperature of a {\symbol} affects the temperature of its target.
	Uncomfortable temperatures remain comfortable as long as the {\symbol} is at a comfortable temperature, and comfortable temperatures become uncomfortable if the {\symbol} is warmed or chilled.
	This effect cannot create dangerous temperatures---hot enough to cause heat stroke or cold enough to cause hypothermia---but can counteract them if the {\symbol} is inversely heated or cooled.
	Temperatures sufficiently extreme to cause {\damage}, such as fire or anything that would directly freeze the flesh, are outside the reach of this effect.
}

\feat{Sympathetic Combustion}{sympathetic-fire}{15}{
	\skillref[1]{sympathetic-magic},
	\featref{sympathetic-damage},
	\featref{sympathetic-heat}
}{
	When you burn someone in effigy, they really burn.
	If a {\symbol} is destroyed by fire, and you use \featref{sympathetic-damage}, the target also catches fire.
	A person ignited this way begins at \dice{3} {\fire}.
}

\feat{Sympathetic Malady}{sympathetic-attribute-reduce}{10}{
	None
}{
	You may afflict a target with various maladies by though a {\symlink}.
	You may reduce one of their attributes by 1 point by causing some appropriate affliction to the {\symbol}.
	For instance, you could reduce the target's \attref{grace} by binding their {\symbolpossessive} arms and legs, their \attref{heed} by blindfolding their {\symbol}, or their \attref{charm} by giving their {\symbol} some obvious disfigurement.
	A target may only be subject to one of these effects at a time, per witch who is affecting them.
}

\feat{Sympathetic Communication}{sympathetic-speak}{20}{
	\skillref[1]{sympathetic-magic}
}{
	You can send sounds along a {\symlink}, like a string telephone.
	A creature can hear sounds that originate near its {\symbol}, as long as it is conscious and not deafened.
	It can avoid this by plugging its ears, although this obviously leaves it deaf to its own surroundings as well.
	The {\symbol} has a very short range of hearing; speaking through it essentially requires picking it up and holding it near the mouth.
}

\feat{Sympathetic Pestering}{sympathetic-speak-2}{15}{
	\skillref[1]{sympathetic-magic},
	\featref{sympathetic-speak},
	\featref{sympathetic-sleep-deprive}
}{
	When sending sounds along a {\symlink} using \featref{sympathetic-speak}, you may send them directly into the target's mind, bypassing its ears.
	The target hears them even if it is deaf, or has its ears plugged.
	You may even be able to wake the target up with loud enough sounds, if it is asleep.
}

\feat{Sympathetic Ventriloquism}{sympathetic-puppet-speak}{10}{
	\skillref[2]{sympathetic-magic},
	\featref{sympathetic-puppet-2},
	\featref{sympathetic-speak-2}
}{
	Puppeteering the vocal cords requires a lot more precision than swinging the limbs around.
	However, it doesn't take as much force---using this effect does not {\stress} the {\symlink}.
	
	While a {\symbolpossessive} jaw is flapped around, the target will speak anything said into the {\symbolpossessive} ear.
	This obviously requires that the {\symbol} possesses an appropriate jaw.
	The target speaks in its own voice, so an animal cannot be made to speak particularly well.
	
	This does not prevent the target from talking whenever this is not being actively used, so you have to force the target to talk constantly if you want to prevent it getting a word in edgeways.
}

\feat{Sympathetic Knot}{sympathetic-knot}{15}{
	\skillref[1]{sympathetic-magic},
	\featref{symlink-extra}
}{
	Normally when {\symlinks} get tangled, it renders both useless.
	However, if you knot them together intentionally, carefully, you can take advantage of it.
	
	You can knot together two or more {\symlinks} of the same kind---to creatures or to objects---as an {\action}.
	This requires that you are touching at least one end of each {\symlink} to be involved in the knot.
	For example, knotting together two {\symlinks} from \materialrefplural{poppet} to people requires you to be touching both \materialrefplural{poppet}, both people, or the \materialref{poppet} from one link and the person from the other.
	
	You can also undo a knot as an {\action}, but again you must be touching at least one end of every {\symlink} in the knot---you can only undo knots in their entirety, and not remove just one {\symlink}.
	Similarly, severing any {\symlink} in the knot severs all of them.
	You can only knot or unknot your own {\symlinks}.
	
	While two {\symlinks} are knotted, anything transmitted by any {\symbol} in the knot affects every target in the knot.
	You may still control what each {\symbol} transmits, but it always transmits to all targets.
}

\feat{Unbarred Sympathy}{sympathetic-ignore-barrier}{15}{
	\skillref[2]{sympathetic-magic}
}{
	Most barriers that interfere with magical effects don't break a {\symlink}, they just prevent it transmitting.
	But a finger on a string doesn't stop it from vibrating; it just restricts it.
	You can circumvent it if you know how.
	
	Barriers created by a \featref{circle-contain}, \featref{circle-exclude}, \featref{circle-contain-exclude}, or the like no longer impede transmission by your {\symlinks}.
	You still can't establish a {\symlink} that would be blocked by such a barrier, however.
}

\feat{Threading the Barrier}{sympathetic-ignore-barrier-2}{10}{
	\skillref[3]{sympathetic-magic},
	\featref{sympathetic-ignore-barrier}
}{
	If air can pass a magical barrier, why not a {\symlink}.
	It's just like threading a needle: it takes a bit of dexterity and your eyesight better be good, but it's hardly \emph{impossible}.
	
	You may establish a {\symlink} even through the barrier created by a \featref{circle-contain}, \featref{circle-exclude}, \featref{circle-contain-exclude}, or the like.
	%You can't always do it first time, however, and the GM may require a Test if you are in a hurry.
}


\chapter{Golemancy}
\chaplabel{golemancy}

\section{Animating a Golem}

A golem must be animated as part of creation, and the witch doing the animation must be involved in its creation, even if she is not the primary craftswoman.
To animate a golem, a witch simply touches it and wills it life; many consider Golemancy to be a particularly specialised application of \discref{willing}.
A novice golemancer has only enough animating force to maintain one golem at a time.
If she animates a new golem, the previous golem immediately becomes inanimate.
A witch may also withdraw the animating force from a golem she has animated at any time, though if this is to be done urgently (perhaps the golem has gone rogue), the GM may require a Test.

The crafting and animation, although strongly interlinked, are separate processes.
Tests related to the craftsmanship use an appropriate Craft skill and an appropriate attribute.
Tests related to the golem's animation, such as giving it instructions, use \testtype{wit}{golemancy}.
A witch can only animate a particular material into a golem if she has taken the appropriate feat.

\section{A Golem's Instructions}

A witch just beginning to dabble in \discref{golemancy} only has the skill to make very simple-minded, single-purpose creatures, although she will learn more nuance later.
These golems are imbued with a single instruction at the moment of their creation.
They will follow this until its completion, whereupon they will simply stand still and await destruction.
The instruction must be very simple, and the golem has minimal ability to improvise around it.
It should not have any conditional aspects, and the golem is unable to respond to any form of communication.
Example instruction are given below.

\begin{itemize}
	\item Deliver this note to the castle.
	\item Fetch my broom.
	\item Kill that man.
	\item Sweep the floor every evening.
	\item Extinguish any fires you see.
\end{itemize}

Additional information necessary to the completion of the task, such as the location of the castle or the identity of an intended victim, may be imparted with the instruction.
The golem will trust this information and cannot adapt if it is wrong, for example if the victim has been disguised.

Giving instructions with nuance, or instructions with multiple linked parts (such as ``go to the castle and kill the King'') requires a Test, with a TN set by the GM based on the complexity of the instruction.
A failure either prevents the golem from animating or, at the GM's option, corrupts the instructions.

\section{A Golem's Statistics}

A golem's physical statistics are determined by the material and method of its construction, and are specified in the appropriate feat.
These include its \attref{might}, \attref{grace}, and response to damage.
The GM is also advised to apply common sense to other consequences of a golem's construction: for example, a clay golem will sink in water, a wooden golem will float, and a gingerbread golem will go soggy and fall apart.
A golem lacks the other four attributes (\attref{wit}, \attref{will}, \attref{charm} and \attref{presence}) entirely; it automatically fails any Test that would require them.
It has no ranks in any skills.

A golem knows no languages; it cannot read, write, or comprehend speech.
It cannot speak, and furthermore cannot vocalise in any fashion.
The sounds it can make are limited to such things as clapping its hands and stamping its feet.

A golem has senses as good as a human, although only if its craftsmanship gives it the appropriate anatomy.
For example, a gingerbread golem with two currants for eyes can see, but if baked without the currants it will be blind.
A clay golem can only smell if a nose is sculpted upon its face.

\section{Feats}

\feat{Gingerbread Man}{gingerbread-golem}{4}{
	None
}{
	The simplest golems are not baked of clay, but of dough.
	When you bake a humanoid figure from gingerbread, approximately the size of a human hand, you may animate it as a gingerbread golem.
	
	A gingerbread golem has \negative{2} \attref{might} and 2 \attref{grace}.
	It is destroyed if it suffers any damage.
	
	Additionally, a gingerbread golem has a limited lifespan.
	After about a week, it grows stale and can no longer move.
	Moisture or water, even a couple of minutes in rain, will destroy it sooner.
}


\discipline{Necromancy}{necromancy}{Necromancer}{Necromancers}

\section{Reanimation and Resurrection}

Many \practitioners{necromancy} draw a distinction between reanimation and resurrection.
Reanimation is a crude process, somewhat akin to \discref{golemancy}.
It's nothing more than the application of raw animating force to a corpse, to stand it up and get it shuffling around again.
The creature retains its instincts, its muscle memory and the like, but that's as much through what is left of its biology as it is through the will that animated it.
The results of reanimation are known as the undead.

Resurrection, by contrast, brings a creature back back to life, in full.
If the creature had a soul, it is restored to the body.
There may be a few ill effects of the process, not to mention whatever killed it in the first place, but these can typically be recovered from.
For all intents and purposes, the creature is just as much alive as it was in the first place.

In theory, at least.

The trouble is that nobody has ever achieved true resurrection.
Dozens of witches and hundreds of charlatans have all claimed to.
Many have even come incredibly close, but there has always been some slight snag.
The search, of course, continues, but many have given up all hope that it is possible.

\section{Reanimation Rituals \& the Undead}
\seclabel{reanimation-rituals}

Undead, the products of reanimation, come in many different forms: \undeadrefplural{zombie}, \undeadrefplural{skeleton}, \undeadrefplural{ghoul}, and more.
The rituals to reanimate these creatures are just as numerous, but they all share mark{\'e}d similarities.
And many \practitioners{necromancy}, who consider these rituals to form the heart and soul of the discipline, learn a lot of techniques for improving them.

For convenience, the feats which modify these rituals refer to them collectively as the {\reanimationrituals}, and the feats which provide them are listed here.
\begin{itemize}
	\item \capital{\featref{animate-zombie}}
	\item \capital{\featref{animate-skeleton}}
	\item \capital{\featref{animate-ghoul}}
	\item \capital{\featref{animate-draugr}}
	\item \capital{\featref{animate-sea-draugr}}
	\item \capital{\featref{animate-fossil}}
	\item \capital{\featref{animate-fire-skeleton}}
	\item \capital{\featref{animate-shade}}
	\item \capital{\featref{animate-wraith}}
\end{itemize}
Obviously, you cannot perform any variant of a ritual unless you have the feat to perform that ritual in the first place.

\subsection{The Lurching Dead}

The statistics and capabilities of an undead creature are based upon the statistics and capabilities of the creature whose corpse it is raised from.
Each variety of undead comes with its own associated changes to these statistics, listed in the following sections.
Many changes, however, are shared between all reanimated creatures, and are listed here.

Most reanimations do nothing to heal {\damage} to the corpse: both {\damage} suffered before and during death, and any further {\damage} done to the corpse since then.
If this reduces its \statref{shock-threshold} to 0 or below, the corpse is too mangled to successfully reanimate.
Some reanimations also require an unrotted corpse.
It usually takes a little over a week before a corpse becomes too rotted for such a reanimation, although temperature and moisture can alter this.
Corpses can be preserved by {\embalming}.

A reanimated creature loses its memory and identity.
It retains general knowledge such as how to hunt, but forgets such information as the location of its den, and loses any mannerisms that distinguished it in life.
It retains {\generalskills}, but loses {\specialityskills}, {\disciplineskills}, and feats.

Most reanimated creatures also lack many biological processes that they had in life.
They do not need to breathe, eat, drink, or sleep.
They are immune to poisons, diseases and the like.
Additionally, they cannot heal themselves or be healed, and are unaffected by potions and such.
They do not suffer from {\shock}, and are simply deanimated if they suffer a {\damagetest} equalling or exceeding their \statref{shock-threshold}.
Lastly, they cannot produce any venom or other such substances, so any benefit of a venomous bite, sting, or the like is lost.

When you reanimate a creature, the resulting undead is under your control.
You control it mentally, without need for verbal instructions or gestures.
However, you can only maintain control while you remain conscious, and nearby---at least near enough to see or otherwise sense it.
If you fall asleep, fall unconscious, die, get too far away from the creature, or leave your body for the {\mentalrealm}, you lose control over it.
A novice \practitioner{necromancy} has no way to regain control over an undead creature once she loses it, other than to kill the creature and reanimate the corpse again.

When you lose control over an undead, it regains free will.
It begins to act as an animal of its kind normally would, seeking out whichever food it would normally eat.
However, it is eternally, ravenously, and insatiably hungry.
It's generally considered good practice to put an undead down, rather than to lose control of it.

A witch who can raise an undead creature learns to control one, and only one, at a time.
When you reanimate a second creature, you must either relinquish control of the previous one, or immediately lose control of the new one.

\undead{Zombie}{Zombies}{zombie}{
	A \undeadref{zombie} is the simplest reanimation possible; the corpse, fully clothed in its own flesh, is simply stood up and walked around as it is.
	Is it a clumsy creature, with most of the mind rotted away as well, and it only grows worse as the corpse rots further.
	
	A corpse reanimated as a \undeadref{zombie} loses 2 points from all attributes except \attref{might} and \attref{will}.
	Its \statref{speed} is halved.
	Its \statref{shock-threshold} increases by 2, however.
	If it could fly, it is now too clumsy to do so.
	
	A corpse reanimated as a \undeadref{zombie} is not healed of any {\damage}.
	The reanimation also requires that the corpse is unrotted.
	Furthermore, it does nothing to slow the rot.
	A \undeadref{zombie} that rots too far loses animation.
}

\undead{Ghoul}{Ghouls}{ghoul}{
	A \undeadref{ghoul} retains greater mental and physical faculties than a \undeadref{zombie}, but this comes at a dangerous price.
	A \undeadref{ghoul} is sustained only by consuming the flesh of its own kind.
	For example, a rabbit \undeadref{ghoul} must consume rabbit flesh, and a human \undeadref{ghoul} must consume human flesh.
	
	A corpse reanimated as a \undeadref{ghoul} loses 2 points from its \attref{ken}, \attref{wit}, and \attref{charm} scores.
	It retains the ability to fly, if it could in life.
	
	A corpse reanimated as a \undeadref{ghoul} is not healed of any {\damage}.
	However, an animated \undeadref{ghoul} may heal {\damage} by consuming the flesh of its own kind.
	An entire corpse is sufficient to restore {\damage} equal to its maximum \statref{shock-threshold}, with smaller portions restoring proportionally smaller amounts.
	A ghoul can consume an entire corpse in less than a minute, and there is no end to its hunger: it could consume corpses for hours on end without being sated.
	\undeadrefplural{ghoul} created from partial corpses, such as with \featref{undead-head}, need only eat their own mass to count it as a full corpse.
	
	Reanimating a corpse as a \undeadref{ghoul} also requires that it is unrotted.
	While animated and fed at least one full corpse each week, however, a \undeadref{ghoul} does not continue to rot.
	
	Lastly, a \undeadrefpossessive{ghoul} unnatural hunger makes it harder to control than most undead.
	It must be fed a full corpse at least once a week, or it always breaks free of the \practitionerpossessive{necromancy} control.
	Even a \undeadref{ghoul} reanimated as a \undeadref{souled} is not immune to this: it goes insane with hunger after a week, and does not return to sanity until it has fed again.
}

\undead{Draugr}{Draugar}{draugr}{
	With all the moisture drawn out of a corpse, it is not only prevented from rotting, but can also be freed from the bloated clumsiness that afflicts \undeadrefplural{zombie}.
	The result is a \undeadref{draugr} (plural \undeadrefplural{draugr}).
	The better preservation also grants it a better memory and senses.
	
	A corpse reanimated as a \undeadref{draugr} loses 2 points from its \attref{wit}, \attref{charm}, and \attref{presence} scores.
	It retains the ability to fly, if it could in life.
	
	A corpse reanimated as a \undeadref{draugr} is not healed of any {\damage}.
	The reanimation requires that the corpse is unrotted, and it also must be {\embalmed} by desiccation (drying out).
	This usually is usually done using salt, but can happen naturally to creatures that die in deserts.
	
	To remain animated, the \undeadref{draugr} must be kept dry; water bloats the corpse, starts it rotting again, and immediately ends its animation.
	It might manage a 30 second sprint through light rain, but heavier rain is too much.
	Given a heavy leather coat, it might just about be able to travel through rain, but the moisture will still get to it in a couple of hours.
	The \undeadref{draugr} is aware of this limitation, and will avoid moisture when not under a \practitionerpossessive{necromancy} direct control.
	
	\undeadrefplural{draugr} are often used to guard ancient tombs, sealed inside where water cannot intrude.
}

\undead{Sea-Draugr}{Sea-Draugar}{sea-draugr}{
	The \undeadrefplural{sea-draugr} are, in many ways, the complete opposite of the regular \undeadrefplural{draugr}.
	Creatures of seas and lakes, they can only be animated from the corpses of those who died by drowning.
	They revel in their water-bloated flesh, lurking beneath the surface and dragging their prey to join them in their watery grave.
	
	A \undeadref{sea-draugr} is subject to the same rules as a regular \undeadref{draugr}.
	It also gains a swimming speed equal to its land speed, or half its flying speed, whichever is greater.
	
	However, instead of remaining dry, an \undeadref{sea-draugr} must remain soaked.
	It functions best when immersed in water, and begins to weaken about five minutes after it emerges.
	After about ten minutes, it dries out too much and completely loses animation.
	
	Regular wetting can extend this time; it might get half an hour in rain, or even an indefinite time if the rain is sufficiently torrential.
	\emph{Continuous} attention using \featref{willing-water-vapour}, or a couple of uses of \featref{willing-water-vapour-2} every 5 minutes, also suffices.
	However, it must be completely immersed for at least 8 hours each day.
	Just like a \undeadref{draugr}, a \undeadref{sea-draugr} is aware of this limitation, and will seek out water.
	
	As long as the corpse remains animated as a \undeadref{sea-draugr}, and sufficiently wetted, it will not continue to rot.
}

\undead{Skeleton}{Skeletons}{skeleton}{
	A \undeadref{skeleton} is the result of reanimating only the bones of a creature, the flesh rotted or carved away.
	The bones arrange themselves in the air, supported by nothing but the will of the animating witch, and the creature's conviction in its own shape.
	The result is a creature far less clumsy than a zombie, but not so resilient.
	
	A corpse reanimated as a \undeadref{skeleton} loses 2 points from all attributes except \attref{grace} and \attref{will}.
	Its \statref{shock-threshold} is also reduced by 2, in addition to the loss from the reduced \attref{might}.
	The mere bones of wings are not sufficient to allow it to fly, if it previously could.
	It also sinks in water, but may move along the bottom.
	
	Requiring only the bones, a \undeadref{skeleton} is not affected by most {\damage} sustained by the corpse.
	Only a critical success on a {\damagetest}, or an intentional effort after death, will typically have broken any bones.
	Likewise, it is not affected by {\damage} in the course of its undeath; any blow insufficient to scatter it across the floor is insufficient to scratch its bones.
	
	A \undeadref{skeleton} lasts a long time without decomposing; at least a decade, and even longer if kept dry.
}

\undead{Living Fossil}{Living Fossils}{fossil}{
	A \undeadref{fossil} is much like a \undeadref{skeleton}, except that the bones have been mineralised, impregnated with stone.
	The essence of earth permeates the creature, strengthening it.
	
	A \undeadref{fossil} uses the same rules as a \undeadref{skeleton}, except that its \attref{might} is not reduced.
	Additionally, fossilised bones do not decay, lasting millennia and more.
}

\undead{Blazing Skeleton}{Blazing Skeletons}{fire-skeleton}{
	A \undeadref{fire-skeleton} can only be made from the charred bones of a creature that died burning.
	Flames race across its bones as it walks, and its eye sockets blaze like the sun.
	It spreads destruction wherever it steps, leaving fire and ash in its wake.
	
	A \undeadref{fire-skeleton} mostly uses the same rules as a \undeadref{skeleton}, with a handful of differences.
	Firstly, the bones used to animate must be charred by fire.
	This should not be enough to totally destroy the bones, but even badly fire-damaged bones have no ill effects on the resulting \undeadref{fire-skeleton}.
	
	Secondly, the \undeadref{fire-skeleton} always burns, as long as it is animated.
	It burns without fuel, and without damaging itself---in fact, it is immune to all harm from fire and heat.
	The fire goes out if the \undeadref{fire-skeleton} loses animation.
	Conversely, the \undeadref{fire-skeleton} loses animation if the fire is extinguished, such as by being immersed in water.
	The fire is somewhat robust, however; it can survive moderate rain, simply causing the droplets to boil away.
	
	The \undeadrefpossessive{fire-skeleton} flames produce heat, and can combust things, just like normal flame.
	They will ignite most combustible materials they touch.
	A \undeadrefpossessive{fire-skeleton} {\unarmed} attacks also {\ignite} the target, at \dice{2}, or add 1 die of {\fire} to a target who is already burning.
}

\undead{Shade}{Shades}{shade}{
	A \undeadrefplural{shade} bridges the gap between ghosts and the corporeal undead.
	It is formed from a creature's body, but always appear to be wreathed in shadow.
	Every part of them is dark: black or grey.
	Its facial features are indistinct, or even absent.
	
	Although it is formed from a creature's body, a \undeadref{shade} is insubstantial.
	It often finds its fingers passing straight through objects, like shadows flitting over them.
	Although this makes it harder to affect the world, it also affords the \undeadref{shade} a degree of protection.
	Swords can pass right through it, without even disturbing it.
	
	Light, however, brings the \undeadref{shade} form, clarity.
	This makes it vulnerable.
	Worse still, sunlight can burn it away entirely, destroying it.
	
	A \undeadref{shade} loses 2 points from its \attref{ken}, \attref{charm}, and \attref{presence} scores.
	Furthermore, its insubstantial nature causes it to lose 5 points from its \attref{might} score.
	However, it suffers no penalties to vision in low-light conditions, or even complete darkness.
	And, in complete darkness, it is immune to all physical harm.
	
	Light, even dim light, makes the \undeadref{shade} vulnerable again.
	If the light falls only on part of its body, only that part is vulnerable.
	Sunlight, however, is worse.
	The \undeadref{shade} suffers a \dice{5} {\damagetest} every {\round} that it is exposed to direct sunlight.
	Reducing exposure can reduce the number of dice rolled for the {\damagetest}, but even if it wrapped entirely in thick black cloth, leaving just its eyes exposed so that it might see causes it to suffer a \dice{1} {\damagetest} every round.
	
	Reanimating a corpse as a \undeadref{shade} requires that it is unrotted, but it does not continue to rot while it is animated.
	When the \undeadref{shade} is deanimated, it leaves the corpse fully corporeal again, albeit with a slightly dark pallor, and still affected by any {\damage} the \undeadref{shade} suffered.
	If the \undeadref{shade} is deanimated in sunlight, however, it burns away entirely, leaving no corpse---not even ash.
}

\undead{Wraith}{Wraiths}{wraith}{
	A \undeadref{wraith} is the invention of a foul \practitioner{necromancy} from a bygone era.
	She sacrificed dozens of people to the darkness of a \creatureref{stygian-nightshade}, then recovered their flayed corpses for reanimation.
	The result was a variety of \undeadref{shade} that carried the \creaturerefpossessive{stygian-nightshade} wicked claws, able to rend flesh despite their intangibility.
	The process has been refined since, and needs nothing more than a sprig of \creatureref{stygian-nightshade}.
	However, it still only works on the corpses of creatures that died violent deaths.
	
	A \undeadref{wraith} appears just like a \undeadref{shade}, and uses all the same rules, except for two differences.
	Firstly, the \undeadref{wraith} can affect the physical world with full force; it loses no \attref{might}.
	
	Secondly, it sprouts wicked claws of stygian darkness, increasing the number of dice it rolls for {\unarmed} {\damagetests}.
	A creature without an effective attack gains one, and rolls 2 dice, while a creature with an existing attack rolls at least 3 dice.
	A human, or other creature with proper hands, rolls 5 dice.
}

\undead{Haunt}{Haunts}{souled}{
	%TODO: Souled, Haunts or something else?
	A \undeadref{souled} is the result of necromancy that is beginning to lift itself from mere reanimation towards the ideal of resurrection.
	It is the result of imbuing a soul into a more conventionally reanimated undead such as a \undeadref{zombie} or \undeadref{skeleton}.
	It is subject to the usual modifications to its statistics, as appropriate to the kind of reanimation.
	
	However, a \undeadref{souled} retains its memories, identity, and free will, and is not subject to the usual hunger.
	It is not controlled by the \practitioner{necromancy} who reanimated it.
	Furthermore, its \attref{might} and \attref{grace} are the only attributes subject to change; the other six are always unchanged.
	It retains all its skills and feats.
	It is still subject to all the benefits and detriments of its loss of biological processes, such as immunity to suffocation, disease, and potions.
	Lastly, it is still subject to usual rules for {\damage} and rotting, so may require \featref{undead-repair}.
	
	A creature can only be reanimated as a single \undeadref{souled} at a time, even when using feats such as \featref{undead-part}.
	
	Names for \undeadrefplural{souled} vary considerably, with many \undeadrefplural{souled} themselves finding the term unpleasant.
	They may refer to themselves as the Souled, or using some other name.
}

\section{Embalming}
\seclabel{embalming}

Decomposition can be such a pain for a \practitioner{necromancy}, putting valuable corpses to waste.
Most corpses barely last more than a week before they are too rotted to make some kinds of undead, such as \undeadrefplural{zombie} and \undeadrefplural{ghoul}.
A \practitioner{necromancy} can always strip away the flesh and raise \undeadrefplural{skeleton}, but these might not suit her needs.
Instead, she might turn to {\embalming}.

Anyone can attempt to {\embalm} a corpse.
The process is a mixture of surgery, and treatment with substances that slow decomposition.
Various substances can be used, with varying effectiveness.
Soaking a corpse in strong alcohol can preserve it for a month or more.
Drying it with salt can preserve it indefinitely, as long as it is not wetted again.
Very long periods of preservation can be achieved with dedicated \featref{embalming-fluid}.

\capital{\embalming} a corpse typically takes a few hours, and requires a \testtypespeciality{ken}{crafting}{Embalmer} Test.
Failure means that the corpse, or some parts of it, won't be preserved, or at least won't last as long as they could.
Particularly bad results can cause {\damage} to corpse.

\capital{\embalming} does nothing to repair {\damage} to the corpse, or to reduce rotting that has already occurred.
It only slows or prevents further rotting.

\section{Phylacteries}
\seclabel{phylacteries}

Resurrection, even to the limited extent that it is possible, requires the return of the creature's soul.
While an experienced \practitioner{necromancy} might reach through the Veil between worlds to pluck the soul from whatever afterlife it may be enjoying, it can be a lot easier to keep the soul shackled to the mortal realm.
Such is the purpose of a {\phylactery}.

A {\phylactery} is an object into which a shard of a person's soul has been bound, enchanted so that the rest of the soul will join it when it would otherwise pass on.
\capital{\phylacteries} must be created from a clay jar, at least the size of a fist but possibly larger.
They are no more robust than the jars they are created from, and their destruction frees the shard of soul within.
The destruction of a {\phylactery} is always felt by the person whose soul it contains, just as the death of a familiar, but otherwise carries no ill effects.
If the {\phylactery} is destroyed after the person has died and their entire soul has passed into it, their soul is released to pass on to the afterlife.

A person can have no more than one {\phylactery} at a time, and the previous one must always be destroyed before a new one can be created.
Likewise, a single object cannot be the {\phylactery} for more than one person at a time.

It should be noted that a {\phylactery} does nothing to \emph{prevent} a person's death; it only makes it easier to restore them afterwards.
However, as long as a witch's soul remains in the {\phylactery} and does not depart this realm, the witch's death does not kill her familiar.

\section{Feats}

\feat{Raise Zombie}{animate-zombie}{20}{
	\noprereq
}{
	You can restore a terrible facsimile of life to the bodies of deceased animals, reanimating it as a \undeadref{zombie}.
	For now, you are limited to animals at least as large as a mouse, and no larger than a medium-size dog such as a bloodhound.
	You can't manage a human, or any animal that has been a familiar, due to interference from the link with its soul.
	
	\materials{An animal corpse, a \circleref{small}, a lit candle which the ritual extinguishes}
	
	The reanimation ritual takes five minutes, and must be performed in the dark.
}

\feat{Raise Skeleton}{animate-skeleton}{15}{
	\featref{animate-zombie}
}{
	After a few reanimations, most \undeadrefplural{zombie} are starting to come apart at the seams a bit.
	There comes a time when it's easier just to strip all the flesh off and make the bones stand up by themselves.
	You may reanimate the bones of an animal corpse as a \undeadref{skeleton}, subject to the same limitations as \featref{animate-zombie}.
	
	\materials{The bones of an animal corpse (with the flesh removed), a \circleref{small}, a lit candle which the ritual extinguishes}
	
	The reanimation ritual takes five minutes, and must be performed in the dark.
}

\feat{Raise Ghoul}{animate-ghoul}{20}{
	\skillref[1]{necromancy},
	\featref{animate-zombie}
}{
	\undeadrefplural{ghoul} are faster and scarier than \undeadrefplural{zombie}, but also \emph{hungrier}.
	You may reanimate an animal corpse as a \undeadref{ghoul}, subject to the same limitations as \featref{animate-zombie}.
	
	\materials{An animal corpse, an additional corpse to be consumed by the \undeadref{ghoul}, a \circleref{small}, a lit candle which the ritual extinguishes}
	
	The reanimation ritual takes five minutes, and must be performed in the dark.
	At the conclusion of the ritual, the newly-arisen \undeadref{ghoul} must immediately be fed a complete corpse---of the same kind of animal as the ghoul---or it does not fall under the \practitionerpossessive{necromancy} control.
}

\feat{Raise Draugr}{animate-draugr}{20}{
	\skillref[1]{necromancy},
	\featref{animate-zombie}
}{
	By animating corpses as \undeadrefplural{draugr}, you can keep them around longer than mere \undeadrefplural{zombie}.
	You may reanimate a desiccated animal corpse as a \undeadref{draugr}, subject to the same limitations as \featref{animate-zombie}.
	
	\materials{An animal corpse {\embalmed} by desiccation (drying out), a \circleref{small}, a lit candle which the ritual extinguishes}
	
	The reanimation ritual takes five minutes, and must be performed in the dark.
}

\feat{Raise Sea-Draugr}{animate-sea-draugr}{10}{
	\skillref[1]{necromancy},
	\featref{animate-draugr}
}{
	A small variation on the ritual to animate \undeadrefplural{draugr} lets you create the opposite.
	You may reanimate a drowned, soaked animal corpse as a \undeadref{sea-draugr}, subject to the same limitations as \featref{animate-zombie}.
	
	\materials{A water-soaked animal corpse that died by drowning, a \circleref{small}, a lit candle which the ritual extinguishes}
	
	The reanimation ritual takes five minutes, and must be performed in the dark.
	The corpse must be kept soaked during the ritual, so keep a few buckets of water handy.
	And make sure the \materialref{ritual-circle} won't be washed away.
}

\feat{Raise Fossil}{animate-fossil}{20}{
	\skillref[1]{necromancy},
	\featref{animate-skeleton}
}{
	Fossilisation is naturally a slow process, but a dedicated \practitioner{necromancy} can accelerate the process.
	You may reanimate the bones of an animal corpse as a \undeadref{fossil}, subject to the same limitations as \featref{animate-zombie}.
	
	\materials{The bones of an animal corpse (with or without flesh) buried in a bog, a \circleref{small}, a small heap of finely crushed rock, a lit candle which the ritual extinguishes}
	
	Beginning the reanimation ritual requires five minutes, but the \undeadref{fossil} does not rise for 24 hours.
	For the entire 24 hours, the candle must remain lit, the \materialref{ritual-circle} must remain intact, and the area must remain dark.
	The witch need not be present for the whole duration, however.
	
	Over the course of the 24 hour period, the rock dust is drawn into the bog and incorporated into the bones, and any remaining flesh rots away.
	At the conclusion, the \undeadref{fossil} is animated and claws its way to the surface.
	The witch must be present at this point to assert immediate control over the \undeadref{fossil}, or it becomes uncontrolled.
}

\feat{Raise Blazing Skeleton}{animate-fire-skeleton}{20}{
	\skillref[2]{necromancy},
	\featref{animate-skeleton}
}{
	Playing with fire is dangerous, and playing with \undeadrefplural{fire-skeleton} is even worse.
	But you've decided it's worth the risk.
	You may reanimate the charred bones of an animal that died burning as a \undeadref{fire-skeleton}, subject to the same limitations as \featref{animate-zombie}.
	
	\materials{The charred bones of an animal that died burning (with the flesh removed), a \circleref{small}, a lit candle which the ritual extinguishes}
	
	The reanimation ritual takes five minutes, and must be performed in the dark.
	The ritual ignites the bones, but it requires fire to do it.
	Normally the candle suffices, but if the candle is substituted for a {\phylactery}, a flame must still be provided.
}

\feat{Raise Shade}{animate-shade}{20}{
	\skillref[1]{necromancy},
	\featref{animate-zombie}
}{
	A \undeadref{shade} is a valuable tool in a \practitionerpossessive{necromancy} arsenal, silent and deadly in the dark.
	You may reanimate an animal corpse as a \undeadref{shade}, subject to the same limitations as \featref{animate-zombie}.
	
	\materials{An animal corpse, a \circleref{small}, a lit candle which the ritual extinguishes}
	
	The reanimation ritual takes five minutes, and must be performed in the dark, \emph{at night}.
}

\feat{Raise Wraith}{animate-wraith}{20}{
	\skillref[2]{necromancy},
	\featref{animate-shade}
}{
	Using the victim of a violent death, and a sprig of \creatureref{stygian-nightshade} placed in its mouth, you can raise a powerful and violent variety of \undeadref{shade}: a \undeadref{wraith}.
	You may reanimate an animal corpse as a \undeadref{wraith}, subject to the same limitations as \featref{animate-zombie}.
	
	\materials{The corpse of an animal which died violently, \herbcreature{stygian-nightshade}{5}, a \circleref{small}, a lit candle which the ritual extinguishes}
	
	The reanimation ritual takes five minutes, and must be performed in the dark, \emph{at night}.
}

\feat{Deanimate}{deanimate}{10}{
	\featref{animate-zombie}
}{
	You can withdraw the animating force from a creature you have reanimated, returning it to death.
	This is far neater than beating it to death.
	
	\materials{An undead under your control, a \materialref{ritual-circle} of the same size required to initially animate the undead, an unlit candle which the ritual lights}
	
	The ritual takes five minutes, and must be performed in a brightly lit location.
	The undead must remain within the \materialref{ritual-circle} for the duration.
	You can affect multiple undead with one casting of the ritual, as long as they are in within the \materialref{ritual-circle} for the duration.
	The size of the \materialref{ritual-circle} required is determined by the size of the largest undead in the group.
}

\feat{Uncontrolled Deanimation}{deanimate-2}{10}{
	\featref{deanimate}
}{
	You can use the \featref{deanimate} ritual against even an uncontrolled undead.
}

\feat{Offensive Deanimation}{deanimate-3}{10}{
	\skillref[1]{necromancy},
	\featref{deanimate}
}{
	You can use the \featref{deanimate} ritual against any undead, be it under your control, under another \practitionerpossessive{necromancy} control, uncontrolled, or even a \undeadref{souled}.
}

\feat{Maintain Control}{undead-control}{10}{
	\featref{animate-zombie}
}{
	Keeping control of your undead can pose a real challenge, especially when you need to sleep.
	You've learned a ritual to place them in a sort of suspended animation, keeping them ready to reassert control over when you awake.
	
	\materials{An undead under your control, a \materialref{ritual-circle} of the same size required to initially animate the undead, a lit candle}
	
	The ritual takes five minutes, and the undead must remain within the \materialref{ritual-circle} for the duration of casting the ritual.
	When you have finished casting the ritual, the undead enters suspended animation; it simply remains in the \materialref{ritual-circle}, unmoving.
	It still counts against the number of undead you are controlling, but you need not remain nearby, or conscious.
	At any point when you are within range, you may end the ritual's effect and reassert control over the undead.
	
	The effect of the ritual lasts as long as the candle remains lit.
	With a large enough candle, you can get 2 or 3 days out it.
	If the candle is extinguished, or removed, you must be in a position to immediately resume active control of the undead---otherwise it becomes uncontrolled.
	
	If you have the ability to control multiple undead, you may affect multiple undead with one \materialref{ritual-circle} and one invocation of the ritual.
	The size of the \materialref{ritual-circle} required is determined by the size of the largest undead in the group.
	You may end the effect of the ritual on each undead individually.
}

\feat{Assert Control}{undead-control-2}{10}{
	\skillref[1]{necromancy},
	\featref{undead-control},
	\featref{deanimate-2}
}{
	Rather than drawing the animation out of undead, you can simply substitute your own animating force, taking control of the undead.
	This uses the \featref{deanimate} ritual, but requires an additional material: a lit candle which the ritual extinguishes---the same as animating an undead in the first place.
	
	On the completion of the ritual, instead of losing animation, the undead comes under your control.
	Undead you take control of this way still count against the maximum number of undead you can control, and exceeding this limit will free an earlier undead from your control, just as raising a new one would.
	
	Obviously, this is useless against an undead you already control, but it is useful in combination with \featref{deanimate-2} or \featref{deanimate-3}.
	You can even take control of a type of undead that you could not normally raise in this way.
	However, you cannot use this against undead that cannot normally be subject to a \practitionerpossessive{necromancy} control, such as \undeadrefplural{souled}---you must simply deanimate these.
}

\feat{Precision Control}{control-deanimate-small}{10}{
	\skillref[1]{necromancy},
	\featref{undead-large},
	\featref{undead-control} or \featref{deanimate}
}{
	You can use a \circleref{small} for the \featref{undead-control} or \featref{deanimate} rituals---assuming you have the feat to perform the ritual at all---regardless of the size of circle you would require to animate the creature in the first place.
	Note, however, that the undead must fit inside the circle, so you will need slightly bigger than a \circleref{small} for an elephant, or the like.
}

\feat{Rapid Control}{control-deanimate-fast}{20}{
	\skillref[2]{necromancy},
	\featref{undead-control} or \featref{deanimate}
}{
	You have become far faster at manipulating an undead's animating force---it can hardly be called a ritual anymore, though it still requires a \materialref{ritual-circle}.
	You can perform the \featref{undead-control} or \featref{deanimate} rituals in just one {\action}.
}

\feat{Contact Control}{control-deanimate-fast-2}{20}{
	\skillref[3]{necromancy},
	\featref{control-deanimate-small},
	\featref{control-deanimate-fast}
}{
	You can perform the \featref{undead-control} or \featref{deanimate} rituals without a \materialref{ritual-circle}, simply by touching the undead you want to affect.
	You still need all the requisite candles, however.
	Furthermore, you can only affect one undead per action if you do not use a \materialref{ritual-circle}.
}

\feat{Stitches}{undead-repair}{10}{
	\featref{animate-zombie}
}{
	Many reanimations and resurrections are ineffective on corpses which are too badly damaged.
	By sealing wounds, stitching severed parts back on, and gluing bones together, you can solve this.
	Any parts you reattach must come from the original creature.
	
	The repair and reanimation requires a Test, with the {\tn} determined by how badly damaged the corpse is, using your choice of \skillref{necromancy} or \skillref{healing}.
	A successful Test repairs at least enough {\damage} to restore the creature's \statref{shock-threshold} to 1, and a high result may repair even more.
	You must perform the repairs while the corpse is dead; you cannot repair it while it is animated.
}

\feat{Scraps}{undead-repair-2}{10}{
	\skillref[1]{necromancy},
	\skillref[1]{healing},
	\featref{undead-repair}
}{
	You can do more than stitch a damaged corpse back together; you can stitch \emph{several} corpses together.
	When using \featref{undead-repair}, you may assemble the corpse to be animated out of parts from different corpses.
	A corpse assembled out of several individually intact parts can be healthier than a single, damaged corpse.
	
	The pieces must all come from creatures of the same kind; all from humans, all from dogs, and so on.
	They must be assembled to form a creature of that kind; you cannot make a six-legged dog.
}

\feat{Chimera}{undead-repair-3}{25}{
	\skillref[3]{necromancy},
	\skillref[2]{healing},
	\featref{undead-repair-2}
}{
	You have mastered the art of assembling corpses, creating horrifying, chimeric monstrosities.
	When using \featref{undead-repair-2}, the parts needn't all come from the same kind of animal.
	They needn't form a normal creature, either; you could stitch extra legs on to a dog.
	
	The creature can typically use replacement anatomy easily; for example, if you replace a human's arms with a bear's.
	New anatomy, however---an extra pair of limbs, for example---may take several hours, or even days to learn.
	Neither a human with an animal mouth, nor an animal with a human mouth, can speak properly.
	
	The GM may invent a set of statistics for the resulting creature, based upon the component creatures and modified, as usual, by its kind of undeath.
}

\feat{Darning}{undead-repair-active}{15}{
	\skillref[1]{necromancy},
	\skillref[2]{healing},
	\featref{undead-repair}
}{
	You may make repairs to a corpse even while it is currently animated.
	Any Tests made to do so use \skillref{healing}.
	You may even reattach severed parts, though these must be the original parts unless you also have \featref{undead-repair-2}.
	\capital{\featref{undead-repair-3}} even allows you to attach parts from different kinds of creature.
}

\feat{Knitted Resurrection}{undead-repair-active-2}{15}{
	\skillref[2]{necromancy},
	\skillref[3]{healing},
	\featref{undead-repair-active}
}{
	You have discovered a means to resurrect dead tissue by attaching it to living tissue.
	This allows you to reattach severed parts to a person or animal.
	Any Tests made to do so use \skillref{healing}.
	
	This doesn't do much, if anything, to heal {\damage}; no more than normal surgery.
	The reattached parts, however, become living parts of the creature, for all intents and purposes.
	Some rot---up to about a week---is tolerable, though disgusting, and will be healed naturally after reattachment.
	
	The reattached parts must be the original parts, unless you also have \featref{undead-repair-2}.
	If you do have \featref{undead-repair-2}, however, you may replace failed organs, or broken limbs, with healthy ones from another creature.
	The target must remain alive throughout the entire process, so replacing a heart is incredibly difficult, and replacing a brain is impossible.
	\capital{\featref{undead-repair-3}} even allows you to attach parts from different kinds of creature.
}

\feat{Major Undead}{undead-large}{20}{
	\skillref[1]{necromancy},
	\featref{animate-zombie}
}{
	Larger bodies need more force to reanimate, but it's force you've learned to provide.
	When you perform a {\reanimationritual}, you may use a \circleref{medium} instead of a \circleref{small}, in order to ignore the upper size limit on the creature.
	You still cannot reanimate a creature with a soul, such as a human; you need \featref{undead-human} to do so.
}

\feat{Undead Head}{undead-head}{10}{
	\featref{animate-zombie}
}{
	Rather than bothering to reanimate larger creatures, you can just reanimate parts of them.
	The head, specifically, the seat of consciousness.
	
	You may reanimate a creature's severed head using a {\reanimationritual}.
	The usual restrictions apply; for example, you cannot reanimate a human unless you also have \featref{undead-human}.
	However, when evaluating whether you need \featref{undead-large} and a \circleref{medium}, consider only the size of the creature's head, not the whole creature.
	As such, any head short of an elephant's only needs a \circleref{small}.
	
	The resulting creature has all the limitations you would expect from a severed head.
	It can't move, and can only bite people who put their hands in its mouth.
	It has no \attref{might} or \attref{grace} scores for most purposes, though it retains its \attref{might} score for calculating its \statref{shock-threshold}, and for biting.
	It can still see, hear, and so on, and vocalise or speak as it could in life.
}

\feat{Partial Undead}{undead-part}{15}{
	\skillref[1]{necromancy},
	\featref{undead-head}
}{
	Sometimes, a \practitioner{necromancy} has to make do with just the parts of corpses that they can find.
	Sometimes, of course, they just want a hand.
	
	You may reanimate just part of a creature using a {\reanimationritual}.
	As with \featref{undead-head}, you cannot reanimate a human without \featref{undead-human}, and whether you need to use \featref{undead-large} is determined by the size of the part you reanimate.
	However small the part you reanimate, the result still counts as one undead under your control.
	
	The abilities of a reanimated part are determined by which part it is, and the GM should adjudicate this according to common sense.
	For example, a part without the head has no sense of sight, hearing, smell, or taste; only touch.
	Similarly, a single limb is likely to move more slowly than a full creature, and will suffer a considerable penalty to attacks---if it can attack at all.
	\capital{\statref{dodge-rating}} is also likely to be reduced.
	\capital{\statref{shock-threshold}} and \statref{resilience} are unlikely to be affected, however.
	Furthermore, the {\damage} the resulting undead part has sustained is determined only by the {\damage} to the relevant part, and it is not considered to have taken any {\damage} merely by being severed from the rest of the body.
	
	If you reanimate a part without a mouth, this causes problems when the creature breaks free of your control and becomes hungry.
	It cannot eat, so is behaviour will become increasingly erratic.
	It may begin to simply hunt and stockpile food, or something else, at the GM's discretion.
	Partial \undeadrefplural{ghoul} are even worse; if they cannot be fed, they cannot be controlled at all.
}

\feat{Sever Soul}{undead-human}{20}{
	\skillref[1]{necromancy},
	\featref{undead-large} or \featref{undead-head}
}{
	Reanimating a creature that once possessed a soul has previously proven impossible, due to interference from the residual link.
	You've learned to sever these links, and hence reanimate these creatures.
	
	You may reanimate a human, or an animal that was once a familiar, using a {\reanimationritual}.
	You must use an iron blade as part of the ritual, to sever the link.
	
	Reanimating an entire human requires a \circleref{medium}, and \featref{undead-large}.
	Reanimating smaller parts of a human, using \featref{undead-head} or \featref{undead-part}, may not.
	
	A reanimated familiar has lost the link to its witch, and is now just a normal animal of its kind.
	See \featref{reanimate-familiar} to reanimate your familiar without losing this link.
}

\feat{Undead Familiar}{reanimate-familiar}{10}{
	\featref{animate-zombie}
}{
	While a soul normally interferes with reanimating a creature, you've begun to figure out how to use it to your advantage, beginning on the path towards resurrection.
	Unfortunately, you can't actually summon any souls back to their bodies yet.
	Not to worry, though, for you have quite ready access to one soul in particular: your familiar's, so inextricably bound to your own.
	
	If your familiar dies and you can recover the corpse, you can reanimate it, paying no XP cost beside that required to purchase this feat in the first place.
	You may use any {\reanimationritual}, and it becomes the appropriate kind of \undeadref{souled}.
	
	Reanimating a familiar in this way does not prevent recovering it through the usual repetition of the binding ritual later (see the section \secref{familiar-injury-death}), although the normal XP cost must still be paid each time that method used.
}

\feat{Pull Yourself Together!}{skeleton-reassemble}{20}{
	\skillref[2]{necromancy},
	\skillref[1]{healing},
	\featref{animate-skeleton},
	\featref{undead-repair-active},
	\featref{undead-part}
}{
	By animating each bone of a \undeadref{skeleton} separately, you give it the ability to reassemble itself when scattered.
	When you animate a \undeadref{skeleton}, \undeadref{fossil}, or \undeadref{fire-skeleton}, you may give it this ability.
	This can make it quite hard to deanimate, however, so bear in mind that you don't have to.
	
	If an undead with this ability is destroyed by a {\damagetest}, or some other trauma that scatters its bones without destroying them, it does not lose animation.
	Over the next few minutes, the bones will draw themselves back together and reform the undead creature, which will begin normal operation again.
	The bones can be restrained to prevent this, but even so, will keep trying to draw themselves together, indefinitely.
	Even a \practitioner{necromancy} in control of the undead cannot give an order to prevent this; it is unresponsive to orders until reassembled.
	
	Thankfully, this does not entirely prevent deanimating the skeleton.
	It will lose animation if enough of the bones are broken, or, in the case of a \undeadref{fire-skeleton}, extinguished.
	Alternatively, the \featref{deanimate} ritual still works.
}

\feat{Undead Phoenix}{skeleton-reassemble-phoenix}{20}{
	\skillref[3]{necromancy},
	\skillref[1]{healing},
	\featref{skeleton-reassemble},
	\featref{animate-fire-skeleton}
}{
	You have learned an obscure ritual to emulate the legendary phoenix, creating an avian undead that is reborn from fire in an instant.
	When you use \featref{animate-fire-skeleton} on the bones of a flying bird, you may and use this feat.
	If you do so, the resulting undead gains all the effects of \featref{skeleton-reassemble}, as well as several extras.
	
	Firstly, despite its skeletal nature, the bird can fly just as well as it could in life.
	Secondly, its flame burns unstoppably; it cannot be extinguished in any way, except by deanimating it.
	Lastly, it can reassemble itself faster, taking only moments, not minutes.
	After it is destroyed, it reforms on its next {\turn}, though it cannot do anything else on that {\turn}.
	
	However, it is said that the phoenix is a unique bird---that only one ever exists at a time.
	While not quite true here, \emph{you} can certainly only create one.
	The last one you created must lose animation before you can animate another one---it is not enough to simply lose control of it.
}

\feat{Phylactery}{phylactery}{10}{
	\skillref[1]{necromancy},
	\featref{reanimate-familiar},
	\featref{undead-large} or \featref{undead-head}
}{
	You've learned to restore your familiar's soul to its body in the event of its death.
	The next step is simply to perform the same procedure upon \emph{yourself}.
	This is complicated by the fact that you are dead, of course, so you ought to have a very good plan in place for pulling this off.
	Examples include a resurrection pact with a trustworthy friend who knows this same procedure, having your familiar do it (\featref{phylactery-familiar}), or ensuring you can stick around to do it yourself (\featref{projection-lifeline-phylactery}).
	
	Firstly, this feat allows you to extract a sliver of your own soul and place it in a {\phylactery}.
	The ritual to do so requires an hour, and must be performed in a dark place.
	It costs 10 XP each time you perform the ritual, as you extract another sliver of your soul.
	
	\materials{The clay jar to become the {\phylactery}, a drop of your own blood, \herb[deadly nightshade]{belladonna}{2}, a \circleref{medium}}
	
	Secondly, you have learned to use a {\phylactery} in a reanimation ritual.
	This must be the {\phylactery} containing the soul of the person whose body is being reanimated, but you may do this with anybody's {\phylactery}, not just your own.
	
	You may use any {\reanimationritual}, reanimating the person as the appropriate kind of \undeadref{souled}.
	The {\phylactery} takes the place of the candle in the ritual.
	The ritual typically requires a \circleref{medium} and \featref{undead-large}---unless you take advantage of \featref{undead-head}, or \featref{undead-part}.
	You do not need \featref{undead-human}---you are incorporating the soul, not severing it.
}

\feat{Familiar Resurrection}{phylactery-familiar}{10}{
	\skillref[1]{necromancy},
	\featref{phylactery}
}{
	Reanimating yourself is hard, what with being dead and all.
	So you've taught your familiar to do it for you.
	
	Your familiar can perform the {\reanimationritual} variant granted by \featref{phylactery}.
	It is only the link between the soul in your {\phylactery} and the sliver of the same soul in your familiar that affords it the magical intuition to do this, so it can only perform the ritual in order to reanimate \emph{you}.
}

\feat{Self-Sacrifice}{phylactery-lethal}{10}{
	\skillref[1]{necromancy},
	\featref{phylactery}
}{
	Dividing your soul is always costly, making the typical method of creating a {\phylactery} rather draining.
	Fortunately, it is possible to create a {\phylactery} without dividing your soul.
	It's simple, really---you just shift your \emph{entire} soul into the {\phylactery} at once.
	This is, of course, lethal.
	
	\materials{The clay jar to become the {\phylactery}, a deadly dose of \herb[deadly nightshade]{belladonna}{2}, a \circleref{medium}}
	
	The ritual requires only a minute, and must be performed in a dark place.
	It has no XP cost.
	It kills your body and traps your entire soul in the {\phylactery}.
	
	If you have \featref{projection-start-2} or a \featref{projection-potion}, and an alternative {\lifeline}, you can leave your body in time that your mind survives the ritual.
	If you have \featref{projection-lifeline-phylactery}, you may even transition seamlessly to using the {\lifeline} provided by the {\phylactery} you are creating.
}

\feat{Touching the Veil}{death-detection}{10}{
	\noprereq
}{
	When a soul departs our world for the next, its passage disrupts the Veil between worlds.
	A witch who knows what to look for can feel this disruption.
	
	You can feel where people have died, though this sense is damped by both distance and time.
	If you pass through the actual position of the death, you'll notice for up to about two weeks after it occurred.
	You automatically sense a death in the same room only for a few days after it's happened, and in the same house for only about a day.
	However, a Test can reveal slightly older or more distant deaths, if you are searching for them.
	You can't gain any information about the identity of the victim or the cause of death.
	
	Locations of mass or repeated death can leave their traces lingering for much longer.
	The site of a battlefield or sacrificial altar may be felt for many years after.
}

\feat{Medium}{medium}{10}{
	\featref{projection-start}
}{
	It is possible for the souls of the dead to {\possess} the bodies of the living, although most souls are not strong enough to force their way in.
	A specially prepared mind and body may invite them in, however.
	
	You may enter a mediumship trance by consuming \herb[stinking nightshade]{black henbane}{2} and meditating for a minute in a dark place.
	Upon doing so, you enter the {\mentalrealm} and may act as normal there.
	The trance lasts until your mind returns to your body, which it may do as normal.
	
	While your body remains in trance, it is vulnerable to {\possession} by any nearby souls of the dead.
	These can include the souls of those who have died nearby, family and friends of you or other nearby people, those who took some special interest in you (such as your mortal enemies), or sometimes even randomly passing souls.
	A soul is aware of your identity, and must choose to {\possess} your body.
	
	The soul gains indefinite control of your body, using the normal rules for {\possession}.
	This makes {\possession} by a malevolent soul a very risky prospect.
	The soul retains all the memories it had in life, and memories of any experiences it has had on the mortal plane since then, but has no recollection of the afterlife.
	It is aware that it has died, and that some time has passed since it died, but has no idea how much.
	
	If you hope to gain information from the {\possessing} soul, you are advised to have an assistant to ask it questions, or at least to leave it a piece of paper with some questions and a quill.
	Mediums interested in limiting the harm a malevolent soul can wreak may be interested in a \featref{circle-contain}.
}

\feat{Piercing the Veil}{medium-death-location}{10}{
	\featref{medium},
	\featref{death-detection}
}{
	The point where a soul departed this world is the easiest point for it to return, and with a prepared mind you may reach out and offer it an invitation.
	If you enter a \featref{medium} trance at the location of a creature's death, as detected by \featref{death-detection}, you may offer that creature's soul an invitation to possess you.
	If it elects not to {\possess} you, you may end the trance early, before another soul has a chance to {\possess} you.
}


\end{document}
