\documentclass[a4paper,10pt,twocolumn]{book}
\usepackage[all]{nowidow}
\usepackage{amsmath}
\usepackage{siunitx}
\usepackage{enumerate}

%\usepackage[margin=0.9in]{geometry}

\usepackage{titling}
%\setlength{\droptitle}{-0.5in} %Adjust margin above title

\usepackage{titlesec}
\titleformat{\chapter}[hang]{\normalfont\huge\bfseries}{\chaptertitlename\ \thechapter:}{1em}{} %Chapter number and name on same line
%TODO: This justifies split lines awfully, fix it.

\makeatletter
\renewcommand{\@seccntformat}[1]{} %Remove section numbers
\makeatother

\usepackage{tocloft}
\makeatletter
\renewcommand{\cftsecpresnum}{\begin{lrbox}{\@tempboxa}} %Removes section numbers from the table of contents.
\renewcommand{\cftsecaftersnum}{\end{lrbox}} %Latter half of the above.
\makeatother
\setlength{\cftsecnumwidth}{0pt} %Removes the space reserved for the section numbers in the table of contents.

\usepackage{hyperref}
\usepackage{nameref}

\usepackage{graphicx}
\usepackage{float} %Provides the float option [H] for a non-floating float.
\graphicspath{{./imgs/}}

\usepackage{hyperxmp} %Recommended for doclicense
\usepackage[
	type={CC},
	modifier={by-nc-sa},
	version={4.0},
]{doclicense}

\usepackage{booktabs}
\usepackage{tabu}
\tabulinesep=1.2mm %Space tables more nicely.
\usepackage[table]{xcolor}
\newenvironment{simpletable}[1]{
	\begin{center}
	\rowcolors{2}{}{gray!20}
	\begin{tabu}{#1}
}{
	\end{tabu}
	\end{center}
}


\newcommand\partlabel[1]{\label{part:#1}}
\newcommand\chaplabel[1]{\label{chap:#1}}
\newcommand\seclabel[1]{\label{sec:#1}}
\newcommand\partref[1]{Part~\ref{part:#1}}
\newcommand\chapref[1]{Chapter~\ref{chap:#1}}
\newcommand\chapnameref[1]{\nameref{chap:#1}}
\newcommand\secref[1]{the \nameref{sec:#1} section}

\newcommand\feat[4]{ %Arguments: title, label, prerequisites, text.
	\subsection{#1}\label{feat:#2}
	\textbf{Prerequisites:} {#3}
	
	{#4}
} %TODO: XP cost, tags such as 'first circle'?
\newcommand\featref[1]{\nameref{feat:#1}}
%TODO: Make feat references add the chapter in brackets afterwards, if the feat is in a different chapter.

\usepackage{amsfonts} %Gives the stuff we use to build \shortminus
\DeclareMathSymbol{\shortminus}{\mathbin}{AMSa}{"39}

\usepackage{xstring} %Gives \StrDel
\newcommand\dice[2][0]{%Takes two arguments, the first of which is optional and defaults to zero.
	#2d%
	\ifnum #1=0%
		%
	\else%
		\ifnum #1>0%
			$+$%
		\else%
			$\shortminus$%
		\fi%
		\StrDel{#1}{-}%Strips the minus from it, essentially giving absolute value.
	\fi%
}

\newcommand\negative[1]{$\shortminus$#1}

\newcommand\titleemph[1]{\emph{#1}} %For the titles of books and such, including Coven itself.

\newcommand\storybreak{\bigskip}

\title{Coven: An RPG of Witches}
\author{Christopher Brown}
\date{}

\begin{document}
\pagestyle{plain} %Fixes page numbers appearing in the top left on empty pages.

\maketitle

\onecolumn

\doclicenseThis

\twocolumn


\setcounter{tocdepth}{1} %No subsections or deeper.
\tableofcontents

\chapter{Introduction}

\titleemph{Coven} is a role-playing game designed upon a simple premise: the player characters are witches and the party is a coven.
Every character shares the common tools of witchcraft: a familiar, a broomstick and, most importantly, a pointed hat.
However, that is often where the similarities end.
There are many different disciplines to witchcraft, and many different approaches even within a discipline.
From meticulous ritualists to soaring broom-riders, from shy girls to terrifying matriachs, hunched over a cauldron or chatting with squirrels in the forest, a coven can be a diverse lot.

In light of this, they don't always get along.
Witches can be somewhat solitary creatures by nature, tending to their own villages, dealing with their own problems.
But they do tend to keep tabs on one another, and a good witch recognises when things are bit much to handle by herself.
When the great spirits of the land are threatened, when \emph{things} push through from other realms, or when one of their own begins to cackle: these are the times witches come together.
And these are the adventures the players have with them.

\section{The Craft}

The Craft, the Art, the Way.
Witchery, Occultism, Thaumaturgy.
There are many names for witchcraft.
Few things define it, however.
In truth, it is nothing but knowledge of the diverse disciplines of magic, and the skill to apply it.

Witchcraft is not like the enchantments of the faeries or the sorcery of warlocks.
It's not a power one is born with, nor one absorbed in a moment.
It is learned through years of training, grasped through decades of practice, and never truly mastered.
Anyone can pick it up, given enough patience and determination.
But few even have the inclination.

For the power of witchcraft comes with more responsbility than most.
The responsibility to care for one's neighbours, one's charges, one's village.
To see them through sickness and through strife, to see them into the world and back out of it.
The responsibility to take up arms and defend them from the horrors of the night, of other realms, even the ones they bring upon themselves.
To lay down one's own life in defence of others.
And finally, the responsibility to train a successor, that the Craft may continue to serve one's village after one dies.
Everything that goes on in a witch's realm is her responsibility, and that is too great a burdern for many to bear.

Which brings us to the topic of the Black Craft.
Witchcraft is simply knowledge, to be used how it will.
Even possession, voodoo and necromancy are not evil acts in themselves, when turned to the purpose of good.
Evil begins when all the responsbility becomes too much for a witch.
When she wonders why she should be doing so much for these people who never do anything for themselves
When she believes that she is better than other people.
When she begins to cackle.
And so comes another responsibility of witches: to sit her down and give her a stern talking to.
Or, failing that, to show her the way out\dots

\section{The World}

%TODO: Assumptions of the setting.

\section{Dice and Tests}

%Dice and tests system, critical success and failure.


\part{Character Creation}
\partlabel{character-creation}

\chapter{Character Creation Guide}
\chaplabel{character-creation-guide}

\section{Step Zero: Character Concept}

The most important part of a character is the concept.
Who is your character, what does she do?
You can try to flesh this all out now, or fill it in as you work through character creation.
Make sure your character is somebody you will enjoy roleplaying.

Also make sure your witch fits into the coven; discuss this with the other players and your GM.
It can be quite painful for everyone involved playing an unscrupulous necromancer in a coven of saccharine healers, or vice versa.
The GM should provide some idea of the tone intended for the game, to avoid this sort of trouble.
Diversity can also be good: make sure you know what you're letting yourselves in for if everybody in the group wants to play a potion-brewer.

\section{Step One: Starting Experience}

Your GM will assign you an amount of experience (XP) to use during character creation.
By default, this is 100 XP, though the GM is free to adjust this to suit a different style of characters or campaign.

100 XP is suitable for witches who have just completed their apprenticeship and are taking over their own steading.
40 XP might be more suitable for witches who are still in an apprenticeship, and may have only just bound their familiar.
200 XP might be appropriate for witches with a few years of caring for a steading under their belt.
Even more experienced witches might require even more starting XP.
For fairness, the GM should probably give all characters the same starting XP, unless there is a good reason otherwise.

Make a note of how much XP you have, and keep track as you spend it later.
It can be well worth keeping a record of everything you have spent XP on over the course of character creation and any later development.

\section{Step Two: Attributes}

Attributes are a witch's broad, innate capabilities.
Is she skinny and lithe, or broad and well-muscled; quick-witted, or bullheaded; domineering, or silver-tongued?

At character creation, you have 20 points to spend on your character's attributes.
Set each of the eight attributes to between 0 and 4, inclusive, such that they sum to 20.

Attributes represent much more innate ability than skills.
While they can be developed through much hard work, they are not often passively improved over the course of a lifetime.
As such, it is not recommended to change the number of points available for attributes as readily as one might change the starting XP.

\section{Step Three: Skills}

A witch does not begin totally unskilled.
Select 1 \seclink{general skill}{general-skills}; you begin with 2 ranks in this skill.
Select an additional 3 general skills; you begin with 1 rank in each of these.
Lastly, select 1 \seclink{discipline skill}{discipline-skills} and one \seclink{speciality skill}{speciality-skills}; you begin with 1 rank in each of these skills.

The GM is also free to adjust the number of skills and ranks granted to starting characters.
General skills represent general life experience, speciality skills tend to result from vocational experience, and discipline skills represent experience with magic and witchcraft.
However, acquiring more than 1 rank in a discipline skill without learning magic from the associated discipline is very rare; such ranks ought to be acquired through XP rather than granted at character creation.

\section{Step Four: Familiar}

A witch's familiar is her essential and constant companion, and is usually bound early in her apprenticeship.
The available familiars are detailed in \chapref{familiars}.
Select one, spending XP (including the XP for any options) from your starting XP if necessary.

A witch can never have more than one familiar.
However, with the GM's approval, you may decline to select your familiar yet.
In this case, you begin play without a familiar and must acquire it during play, spending the necessary XP then.
This can be useful if you want a familiar with a high XP cost, but want to spend your starting XP on other things.

\section{Derived Statistics}

\begin{simpletable}{ll}
	\toprule
	Statistic & Derivation\\
	\midrule
	Resilience & $3$\\
	Shock Threshold & $12 + \text{\attref{might}} + \text{\attref{will}}$\\
	Dodge Rating & $8 + \text{\attref{grace}} + \text{\attref{heed}}$\\
	Speed & $8 + \text{\attref{might}} + \text{\attref{grace}}$\\
	\bottomrule
\end{simpletable}

\section{Steading}

Most witches have a steading.
This is the area a witch watches over, a region she defends and protects the inhabitants of.
The duties a witch has to her steading are numerous and varied, but typically involve healing the inhabitants and protecting them from threats of a magical nature.
Some witches also perform midwifing, care for the land itself, or even take it upon themselves to deal with non-magical threats, such as invading armies.
A witch's responsibilities are not limited to her steading, and nothing stops her from responding to threats outside it.
But inside it, everything is certainly her responsibility.

Decide whether your witch has a steading.
How big is it?
One village, several, or an entire kingdom?
What duties does she perform within it?
Do the inhabitants appreciate what she does for them?

Also discuss this with your GM, and the other players.
Has the GM already described a village that could be your steading?
It is not unheard of for witches to share a steading, although this can obviously lead to disagreements.
Do you share a steading with your coven, or have you carved the local region into one steading each?


\chapter{Attributes and Skills}
\chaplabel{attributes-and-skills}

\section{Attributes}
\seclabel{attributes}

Attributes are a character's broad, innate capabilities.
They represent physical capacity and natural talent.
That is not to say they can't be improved---one can grow muscle through exercise and the brain is no different---but such improvement represents a more significant investment than picking up a new skill.
A character has six attributes: \attref{might}, \attref{grace}, \attref{wit}, \attref{will}, \attref{charm} and \attref{presence}.
For human characters, these range from 0 to 5, with 2 as the average for a human.
Non-human characters may have attributes outside this range.

A summary of these attributes is provided below, along with examples of using the attribute.
Note that many of the example Tests would be accompanied by an appropriate skill.

\attribute{Might}{might}

\attref{might} represents physical strength, endurance and resilience.
It's used to lift things, smash things, resist diseases and endure hard labour, to put the hurt on people and to resist having the hurt put back on you.
\attref{might} is the attribute you use when rolling damage with melee weapons, and also determines the amount of damage required to put you down.
It can also prove useful when a brewer or botanist feeds you something you shouldn't have eaten.
Lastly, powerful legs let you run faster.

%This table seems unhelpful, on second thoughts.
%Would such grandiose descriptions transfer accurately to fairly small range of numbers in the game mechanics?
%
%\begin{simpletable}{lX}
%	\toprule
%	0 & Total weakling: Carrying your shopping back from the market is a struggle.\\
%	1 & Below average: Arm-wrestling ends embarassing for you.\\
%	2 & Average: A day's farm-work is tiring, but doable.\\
%	3 & Above average: You can carry bricks all day everyday.\\
%	4 & Exceptional: You could be a blacksmith.\\
%	5 & Incredible: You could run around in plate armour for hours.\\
%	\bottomrule
%\end{simpletable}

\begin{simpletable}{rX}
	\toprule
	TN & Example Task\\
	\midrule
	9 & Jumping across a \SI{3}{\metre} gap.\\
	12 & \\
	15 & \\
	18 & \\
	21 & \\
	\bottomrule
\end{simpletable}

\attribute{Grace}{grace}

\attref{grace} represents agility, dexterity and reflexes.
It's used to dodge swords, manoeuvre broomsticks, do backflips, dance waltzes, and hastily scratch runic circles into the floor without smudging them and letting the demons in.
\attref{grace} determines how hard you are to hit with a weapon and also contributes toward your speed.

%TODO: Table

\attribute{Wit}{wit}

\attref{wit} represents intelligence, memory and awareness.
It's used to recall knowledge, solve puzzles, and ensure no detail escapes your notice.
\attref{wit} is the key attribute for many forms of magic, used to memorise and understand spells, recipes and rituals.

%TODO: Table.

\attribute{Will}{will}

\attref{will} represents courage, dedication and conviction.
It's used to stand your ground, resist the influence of others, remain unfazed in embarassing situations, and push onwards in the face of adversity.
\attref{will} influences your pain threshold and is used to resist curses and mental influence, mundane or magical.
Force of \attref{will} can also be applied to directly influence the world in some forms of magic.

%TODO: Table.

\attribute{Charm}{charm}

\attref{charm} represents eloquence, wile and comeliness.
It's used to persuade, deceive or seduce people, to smarm your way into their good graces, and to imply things without outright saying them.
It's also important to reading a person, or a room.
%TODO: Mechanical effects.

%TODO: Table.

\attribute{Presence}{presence}

\attref{presence} represents force of personality, air of authority and personal magnetism.
It's used to draw people's attention, boss them around, and make them wet themselves in terror.
%TODO: Mechanical effects.

\begin{simpletable}{rX}
	\toprule
	TN & Example Task\\
	\midrule
	9 & \\
	12 & \\
	15 & \\
	18 & \\
	21 & Silencing a raucous town hall with a polite cough.\\
	\bottomrule
\end{simpletable}

\section{Skills}


%TODO: Intro fluff paragraph about the skill of a witch.

%TODO: Recap how skills affect dice rolled for Tests, and how not every Test has an applicable skill.

A witch's skills can be divided into two categories.
The first consists of general skills, pertaining to things any witch might find herself doing.
The second consists of a witch's skills in her particular disciplines of magic.
These skills are normally of little use to a witch who does not practice such a discipline, although they can often be used to identify, and sometimes to counteract, the effects from it.

A list of the skills available to a witch, and examples of their use, is provided below.

\subsection{Improving Skills}

Ranks in skills may be purchased by spending XP.
The XP costs of increasing skills are provided in the following table.
A character must have the previous rank in a skill before purchasing the next rank.

\begin{simpletable}{lrr}
	\toprule
	Type & Rank & XP Cost\\
	\midrule
	General & 1 & 15\\
	General & 2 & 25\\
	General & 3 & 35\\
	Discipline & 1 & 40\\
	Discipline & 2 & 80\\
	Discipline & 3 & 120\\
	\bottomrule
\end{simpletable}

The XP cost for increasing a general skill is fixed, while progression in discipline skills is closely tied to progression in the discipline itself.
While it is possible to learn a lot about discipline of magic without ever practicing it, it is far easier to learn simply by setting out and using the magic.
As such, increasing a discipline skill costs 5 XP less for each feat a witch has purchased from its governing discipline, to a minimum cost of 0 XP.

\subsection{Specialities}

Some skills have specialities.
For these skills, a character does not gain ranks in the skill itself, but in one of its specialities.
Ranks for each speciality are tracked independently.
For example, a witch might have \skillrefspeciality[1]{crafting}{Carpenter}, \skillrefspeciality[2]{crafting}{Cook} and \skillrefspeciality[2]{crafting}{Smith}.
Each skill with specialities provides a list of recommended options, but the GM may approve others.

\subsection{General Skills}
\seclabel{general-skills}

\skill{Animal Ken}{animals}

Used to understand animals and interact with them: to calm them, tame them, ride them, command them, or predict how they might act.

\skill{Athletics}{athletics}

Used to run, jump, swim, climb, somersault and generally get about the place more easily and impressively.

\skill{Botany}{botany}

Used to raise crops and herbs in a witch's garden, find them out in the forest, or identify a fishy-looking leaf.

\skill{Crafting}{crafting}

Used to make things, quite broadly.
This covers the creation of most kinds of objects, although some kinds of crafting are still covered by other skills, such as \skillref{brewing}.
Available specialities include the following:

\begin{itemize}
	\item Carpenter
	\item Cook
	\item Jeweller
	\item Mason
	\item Potter
	\item Seamstress
	\item Smith
	\item Woodcarver
	%TODO: Evaluate and maybe expand.
\end{itemize}

\skill{Deception}{deception}

Used to mislead, lie, prevaricate or filibuster, without anyone catching on that you're doing it.
Many witches make it a rule not to lie.
That doesn't mean they always need to tell the whole truth, so this can still be a useful skill for them.

\skill{Healing}{healing}

Used to bind wounds, set bones, diagnose diseases and deliver children.
This covers first aid, extended care and even surgery.
It does not cover the use of herbs, poultices or potions; these fall under \skillref{botany} and \skillref{brewing}.
It can be used to diagnose a patient's sickness in the first place, however: an essential step in applying the correct potion.

\skill{Insight}{insight}

Used to read people, as individuals or crowds.
This can include judging people's attitude and confidence, telling when and why they're uncomfortable, picking up on tells that they're lying, or predicting whether an argument is likely to come to blows.
It can be particularly useful for guessing at people's levers and buttons when preparing to manipulate them.
As a skill that relies on social understanding, it is normally rolled with \attref{charm}.

\skill{Intimidation}{intimidation}

Used for making threats: anything from subtly suggesting that you know a secret somebody would rather wasn't public knowledge, to outright yelling that you'll break the bugger's knees if he doesn't sit down and shut up \emph{right now}!
This doesn't even have to involve speaking; turning half a mob into frogs can certainly discourage the rest from tangling with you.

\skill{Lore}{lore}

Used to know and recall assorted knowledge, such as history, geography and religious doctrine. %TODO: Double-check our stance on religion.
Many fields of knowledge, such as magic and \skillref{botany}, fall under their own skills; this covers those that don't.

\skill{Perception}{perception}

Used to see, hear or smell things.
This includes noticing things that are out of place, such as hearing someone sneaking up behind you or spotting that your hat is missing from its peg.
It also covers active attempts to discern things, such as picking out details on someone at the other end of a street, eavesdropping, or identifying a faint smell.
Lastly, this is the skill used when trying to follow the trail of an animal or person.

\skill{Performance}{performance}

Used to entertain, amuse or impress people, or perhaps just to distract them.
Not everything that entertains people must use this skill; people can easily be entertained by a show of a magic or a swordfight, which might use another skill.
\skillref{performance} covers things done primarily for entertainment.
Available specialities include the following:

\begin{itemize}
	\item Dancer
	\item Drummer
	\item Harper
	\item Joke-teller
	\item Piper
	\item Storyteller
	%TODO: Evaluate and maybe expand.
\end{itemize}

\skill{Persuasion}{persuasion}

Used to influence or convince a person or crowd: to make them believe a particular thing or act in a particular way.
This can be through subtle suggestion and manipulation, or through reasoned, logical argument.
If you're attempting to persuade someone to act based on a falsehood, this might require both a \skillref{deception} Test to avoid being caught in the lie, and \skillref{persuasion} Test to motivate them to act.

\skill{Socialising}{socialising}

Used to befriend people, mingle with them, build rapport, and get into their good graces.
A good socialiser is everybody's best friend within a few minutes of meeting them, and might be trusted with secrets people would never otherwise give up.
Additionally, the GM may call for a \skillref{socialising} Test to determine how well you know a member of your steading or a nearby one.

\skill{Stealth}{stealth}

Used to do things without being noticed, such as sneaking up behind someone, peeking out through a bush, or lifting a guard's knife from his belt.
You can even try to blend in with a crowd (take the Hat off first), or rifle a man's purse while he watches your other hand.

\skill{Weaponry}{weaponry}

Used for everything from stabbing people with a concealed knife to clonking them over the head with a hefty staff, or even slugging them with a mean right hook.
Also used for throwing things, or shooting them with a bow.

\subsection{Discipline Skills}
\seclabel{discipline-skills}

\skill[brewing]{Brewing}{brewing}

Used to brew tinctures, tonics, elixirs and other potions.
This doesn't always require a cauldron: it also covers mixing poultices and the like.
Of course, you can also make booze.

\skill[divination]{Divination}{divination}

Used to see the past and future, and places many miles away.
It's not limited to seeing either; a diviner can eavesdrop on a conversation in the next village, or track a person better than any bloodhound.

\skill[broomcraft]{Flying}{flying}

Used by a witch on a broomstick, whether she's settling in for a cross-country flight, showing off with a barrel roll, or pulling a stalled stick out of a deep dive.
This is also the skill used for feats of flying by a winged familiar.

\skill[golemancy]{Golemancy}{golemancy}

Used to will life into inanimate creatures of clay, or other materials.
A skilled golemancer can make more golems, make them smarter, and, of course, force life into increasingly substandard bodies.

\skill[necromancy]{Necromancy}{necromancy}

Used to pervert the natural order and bring the dead back to life, or at least commune with them from beyond the Veil.
Also used to send them on again, if hitting them over the head with a big stick won't suffice.

\begin{simpletable}{rX}
	\toprule
	TN & Example Task\\
	\midrule
	9 & Discerning the power of the \discref{necromancy} animating a shambling corpse.\\
	12 & Identifying the purpose of a necromantic rite from the chalk circle left behind.\\
	15 & Filtering the true facts about vampires from the baseless rumours that surround them.\\
	18 & Discerning the power of the \discref{necromancy} that previously animated a no-longer-shambling corpse.\\
	21 & Performing a complex necromantic ritual using nothing but two small sticks and a fresh egg.\\
	\bottomrule
\end{simpletable}

\skill[sympathetic-magic]{Sympathetic Magic}{sympathetic-magic}

Used to manipulate people or things using effigies, poppets or other imitative talismans.
The idea of \discref{sympathetic-magic} is that one can't affect the imitation of a thing without affecting the thing itself.

\skill[willing]{Willing}{willing}

Used to force the universe to fall into line with what you know is true.
There is a real knack to convincing yourself of something well enough to make this work, and this skill governs how good you are at it.


\chapter{Familiars}
\chaplabel{familiars}

A wizened old woman leans back in her rocking chair, eyes closed.
A white cat lies curled in her lap, its own eyes also shut, purring as she rubs its chin.

A handsome, tanned woman stands on the peak of a grassy hill, arm held aloft.
A falcon dives from above, alighting on her thick leather glove.
It casts its eyes north-west, then knowingly back at the witch.
With a sly grin, she casts the bird back into the air and strides downhill after it.

A bright-eyed girl, no more than thirteen, stands beside a bubbling cauldron, carefully teasing the seeds from a pine cone with a small knife.
``Sage leaf next, Harold?''
She looks up at the frog on the kitchen bench, as it croaks and nudgs one of the piles of herbs that surrounds it.
``Ohh, right. Rosemary. Of course{\dots}''
The girl shakes her head and tuts to herself as she counts out seven leaves into her hand and drops them in the cauldron.
Harold peers over from the bench, keeping a close eye on the brew as it slowly turns a deep blue.

\section{No Mere Beast}

A witch's familiar is no mere animal.
It is a fusion of summoned spirit, tamed beast, and a tiny sliver of soul from the witch herself.
Obtaining a familiar is one of the first steps for any witch-in-training, and the familiar often aids in the witch in her subsequent learning.

Familiars are intelligent creatures, in some cases even more intelligent than the witches they are bound to.
They understand language, though the limits of animal form mean that most are incapable of speech.
Despite this, the bond that a witch shares with her familiar allow them to communicate.
With simple looks and gestures a familiar can communicate great meaning to its witch, communicating as effectively as if through speech.
This ability does not extend to other witches, and especially not to layfolk, who may require a Test to interpret a familiar's communication.
Pointing and beckoning are typically fairly unambiguous, however.

A witch's communication with her familiar allows her to lean on its expertise when her own is lacking.
A witch may use her familiar's ranks in a skill in place of her own, as long as the Test takes at least a minute, and she can confer with her familiar through the duration.

\section{Binding a Familiar}

Binding a familiar takes place in a simple ritual, achievable by even the most novice witch, though often performed under direct tutelage.
The spirit to be bound to the familiar can be obtained in a number of ways: a lesser demon under contract, an amenable nature spirit or a spirit summoned from beyond.
It is not uncommon for a witch to use the spirit from the familiar of her teacher's teacher, or of an ancestor if witchcraft runs in her family.
The animal to become the familiar must be tamed by the witch, at least enough that it willing remains by her side throughout the ritual.
Many witches find this to be the hardest part of the ritual, and it means that some animals make for rather rare familiars.
Lastly, the witch must offer up a sliver of her own soul, to seal the bond.
She does so by feeding the familiar animal a drop of her own blood.

Upon completion of the ritual, the spirit and animal are fused to form a new entity, the familiar.
It takes on personality elements from both and the body of the animal.
Slight changes to its physical form often manifest, however, such as a coat that always remains strangely glossy, a slight chill to the touch, or sharper, whiter teeth.
Changes in eye colour are especially common.
Lastly, the sliver of the witch's soul included in the creation of a familiar also influences its personality.
It ensures that, although a witch and her familiar may not always get along, and may certainly disagree on the best way to go about something, they will always have one another's best interests at heart.

\section{Creating a Familiar}

From the perspective of character creation, there are many things to bear in mind when creating a familiar.
While the familiar is unlikely to take the foreground as much as the witch herself, they are still a character in their own right, and should be designed as such.

The most important decision is the form the familiar will take, the animal they were created from.
This determines the familiar's attributes, skills and abilities.
Note that familiars, as non-human characters, may have attributes below the human 0 to 5 range.

Beside its game statistics, it is also important to get an idea of your familiar as a character.
Try answering some of the following questions.

\begin{itemize}
	\item What is your familiar's name?
	\item Is your familiar male or female?
		Do you not know?
	\item At what stage in her life, and her training, did your witch bind her familiar?
	\item Which sort of spirit was used in the binding? %TODO: Remove this if it becomes mechanical.
	\item Do your witch and her familiar get along?
		Do they engage in playful banter?
		Philosophical debate?
	\item Does your familiar have any quirks, physical or mental?
\end{itemize}

Lastly, it is important to decide whether each familiar will be played by the player or the GM.
Both are valid, but if the GM is playing familiars they should typically act in their witch's best interests.

\section{Familiar Injury and Death}
\seclabel{familiar-injury-death}

Familiars suffer {\damage}, {\shock} and death just like other characters.
A witch is always aware when her familiar dies, feeling it as a searing pain in her very soul.

It is possible to recover a deceased familiar through a slight variant on the original binding ritual.
The familiar's spirit comes willingly, but another animal of the same kind must be provided.
The familiar, once reformed, may take either the new animal's appearance or its original one.

Repeating the ritual takes another sliver of the witch's soul, provided through another drop of blood.
As such, recovering a deceased familiar costs 10 XP every time.

\section{Familiar Animals}

A list of the types of animal available as familiars is presented below, along with the attributes, skills and abilities of the familiar.
Besides the abilities listed below, the players and GM are encouraged to apply common sense.
For instance, familiars lack thumbs and will struggle with door handles, and a weasel can squeeze through a smaller hole than a hound.

If you would like your familiar to be an animal not presented on the list below, discuss your option with your GM.
It might be possible to design a new familiar for you to use, or to use the statistics of a familiar presented here to represent something else.
Note that familiars are fairly small animals; the exclusion of anything larger than a medium-sized dog is intentional.

Many types of familiar, more powerful ones, come with an associated XP cost.
This is deducted from the witch's starting XP.
Some types of familiar also come with options which may be purchased for an additional XP cost.
These represent inherent differences in the animal used and must be purchased at the same time your familiar is created.
You may only select one option; they are mutually exclusive.

Lastly, bear in mind that some feats that can be purchased later depend upon particular types of familiar, and your familiar's later development is limited by its form.
As such, it can be worth taking a quick look at other feats you may be interested in taking when selecting your familiar.
%TODO: If those are all in a discipline chapter on familiars, direct people there.

\familiar{Cat}{cat}{15}{
	\atttable{\negative 1}{3}{2}{2}{2}{2}{3}{1}
}{
	\skillref[1]{athletics}, \skillref[1]{deception}, \skillref[1]{perception}, \skillref[2]{stealth}
}{
	Graceful and charming on the outside, cats can be incredibly sly and manipulative underneath.
	Just like many witches.
}{
	\familiarability{Natural Acrobat}{
		The cat rolls an extra die on Tests to retain its balance, land on its feet, or avoid damage from falling.
		%TODO: Reduce fall damage in some more definite fashion?
	}
	
	\familiarability{Claws}{
		The cat's unarmed attacks deal 4 dice of damage.
	}
}{}

\familiar{Dog}{dog}{15}{
	\atttable{1}{1}{1}{1}{3}{2}{0}{2}
}{
	\skillref[2]{intimidation}, \skillref[2]{perception}, \skillref[1]{weaponry}
}{
	A man's best friend, and often a witch's too.
	Dogs are a diverse lot, including hunting dogs, sheepdogs, sled dogs and more.
}{
	\familiarability{Bite}{
		The dog's unarmed attacks deal 5 dice of damage.
	}
}{
	\familiaroption{Scenthound}{5}{
		The dog rolls an extra die on \skillref{perception} skills relying on smell.
	}
}

\familiar{Ferret/Stoat/Weasel}{mustelid}{15}{
	\atttable{\negative 2}{3}{1}{1}{2}{2}{2}{0}
}{
	\skillref[1]{athletics}, \skillref[2]{stealth}, \skillref[1]{weaponry}
}{
	A ferret, stoat, weasel, polecat, ermine, mink or marten.
	Despite their small size, these creatures are ferocious predators.
	Their long, narrow bodies allow them to invade the burrows of much smaller animals, or the trousers of their witch's unfortunate foes.
}{
	\familiarability{Bite}{
		The ferret's unarmed attacks deal 4 dice of damage.
	}
	
	\familiarability{Slippery}{
		The ferret's \statref{dr} is increased by 2.
	}
}{}

\familiar{Frog/Toad}{frog}{5}{
	\atttable{\negative 2}{\negative 1}{1}{1}{2}{0}{\negative 1}{\negative 1}
}{
	\skillref[1]{brewing}
}{
	Frogs and toads make excellent companions to brewing witches, due to their natural affinity with water.
	Particularly with some of the stuff that gets into the murkier ponds around{\dots}
	
	It is important to try and keep their skin moist, but maybe refrain from dropping them in the cauldron.
}{
	\familiarability{Amphibian}{
		The frog can breathe underwater.
	}
	
	\familiarability{Leapfrog}{
		The frog can jump at least 3 metres from a standing start.
		It rolls an additional die on Tests made to jump.
	}
	
	%TODO: Something about keeping the skin moist? Maybe when there are exhaustion/fatigue rules.
}{}

\familiar{Raptor (Eagle/Falcon/Hawk)}{raptor}{25}{
	\atttable{\negative 1}{3}{1}{2}{2}{3}{\negative 1}{2}
}{
	\skillref[2]{flying}, \skillref[2]{perception}, \skillref[1]{weaponry}
}{
	A raptor is a buzzard, eagle, falcon, harrier, hawk, kite or osprey; a bird of prey.
	They are excellent fliers, have keen eyesight, and nobody would want to tangle with their wicked beak and talons.
}{
	\familiarability{Eagle Eyes}{
		The raptor rolls an extra die on \skillref{perception} Tests to see things at a long distance.
	}
	
	\familiarability{Beak \& Talons}{
		The raptor's unarmed attacks deal 4 dice of damage.
	}
}{}

\familiar{Rat/Mouse}{rat}{0}{
	\atttable{\negative 2}{1}{1}{1}{2}{1}{\negative 2}{\negative 1}
}{
	\skillref[1]{stealth}
}{
	The rat is a rather widely reviled animal, but it's certainly easy for a new witch looking for a familiar to find one.
	And it can get into smaller places than a cat or bird, which often proves helpful.
}{
	\familiarability{Filth-Liver}{
		The rat rolls an extra die on Tests to resist poison or disease.
	}
	
	\familiarability{Keen Smell}{
		The rat rolls an extra die on \skillref{perception} skills relying on smell.
	}
}{}

%\subsection{Beaver}

%Placeholder.

%\subsection{Bat}

%Placeholder.

%\subsection{Badger}

%Placeholder.

%\subsection{Chicken}

%Placeholder.

%\subsection{Crow/Raven}

%Placeholder.

%\subsection{Dove}

%Placeholder.

%\subsection{Fox}

%Placeholder.

%\subsection{Gecko}

%Placeholder.

%\subsection{Goose}

%Placeholder.

%\subsection{Hamster/Gerbil/Guinea Pig}

%Placeholder.

%\subsection{Hedgehog/Porcupine}

%Placeholder.

%\subsection{Lemming}

%Placeholder.

%\subsection{Lizard}

%Placeholder.

%\subsection{Mole}

%Placeholder.

%\subsection{Otter}

%Placeholder.

%\subsection{Owl}

%Placeholder.

%\subsection{Parrot}

%Placeholder.

%\subsection{Pigeon}

%Placeholder.

%\subsection{Rabbit/Hare}

%Placeholder.

%\subsection{Salamander/Newt}

%Placeholder.

%\subsection{Seabird}

%Placeholder.

%\subsection{Snake (Constrictor)}

%Placeholder.

%\subsection{Snake (Venomous)}

%Placeholder.

%\subsection{Songbird}

%Placeholder.

%\subsection{Spider}

%Placeholder.

%\subsection{Squirrel}

%Placeholder.

%\subsection{Swan}

%Placeholder.

%\subsection{Tortoise}

%Placeholder.

%\subsection{Vole/Shrew/Gopher}

%Placeholder.

%\subsection{Woodpecker}

%Placeholder.


\chapter{Tools of the Craft}
\chaplabel{equipment}

\section{The Hat}

A witch's pointed hat is the most important of her tools, in many regards.
There are no particular rules about the hat; its effects are left up to the GM.
But it always has an effect on people.
It may make them angry, reverent, reassured or afraid, but most importantly it makes sure they know that they are in the presence of a witch.

A witch's hat says a lot about her, particularly to other witches.
When you create your character, you can answer the following questions about your hat.

\begin{itemize}
	\item Did you make it yourself?
	\item How tall is it?
	\item Is it the traditional black, or some other colour?
	\item How long have you had it?
		Is it visibly worn?
		Well cared for?
	\item Is it plain, tastefully decorated, or covered in stars and sequins?
	\item Does it have any useful accessories?
		Pockets?
\end{itemize}

Many witches accompany their hats by a black cloak or other such attire.
Opinions on occult jewellery are mixed: some witches wear masses, others frown on it heavily.



\section{Broomsticks}

Sometimes, walking from one village to another just takes too long.
A lot of witches---to maintain their mystique or simply because the townsfolk wouldn't be happy otherwise---even choose to live quite a way from the nearest village.
Such circumstances make a broomstick an essential accessory for any witch.

Broomstick flight is no mean feat and while every witch picks up the rudiments, most can use it for nothing more than getting from A to B.
The broom needs a running start, has to be ridden sidesaddle, and has a turning circle several hundred metres across.
Detailed rules for flying a broomstick can be found in \chapref{broomcraft}.

Before it can be used, a broomstick needs to be trained to to fly.
This requires someone to fly it around on another broomstick so that it can learn its craft from one of its fellows.
It must be held parallel to the broom being ridden, to ensure it learns to fly in the correct direction.
The process takes about eight hours.
These hours need not be consecutive, but should all be done within a couple of weeks.
Once trained, a broom retains its flight skill for a long time.
Taking it out for a few hours each year is enough to keep its hand in.

At character creation, every witch is assumed to own a trained broomstick one way or another.
It was probably trained using the broom of whoever taught her witchcraft, at least if she's still using their first broom.
It might feel like an old friend at this point, the witch familiar with every knot and notch in its handle.
A more careless witch might have gone through a few brooms during her career.



\section{Common Magical Components}

The various rites and magics of the various disciplines of witchcraft require too many different materials to enumerate here.
However, there a few components that make a regular appearance.
Some details of their acquisition, construction and use are given here.

\subsection{Ritual Circles}
\materiallabel{Ritual Circle}{Ritual Circles}{ritual-circle}
\circlelabel{Small}{small}
\circlelabel{Medium}{medium}
\circlelabel{Large}{large}

A ritual circle describes any large arrangement of symbols or shapes required by a rite.
They are traditionally drawn on the floor in chalk, but other methods are far from uncommon; the visibility and accuracy are the only important aspects for most rites.
Some witches use paint for permanence, or even chisel their circles into stone.
Many a witch in a hurry has scratched their circles into the dirt with the toe of their boot.
Some witches even embroider their circles upon sheets of fabric that can be rolled up and laid down where needed.
However, a roll bearing even the smallest of circles is most of the height of a man.

Each rite requires a ritual circle of a particular design, different for every rite, but the same each time the rite is performed.
This means that scribing a circle just once and using it for many performances of the rite is a common practice.
Ritual circles are not even universally circular, although it is the most common shape and almost all have some sort of symmetry.
Squares, triangles and hexagons are not uncommon, and pentagrams are particularly common in certain disciplines.

Ritual circles are classified primarily by their size.
A rite can be performed with a larger circle than it requires, unless specified otherwise.
\begin{itemize}
	\item A \circleref{small} can be scribed entirely in arm's reach while standing in one spot.
		It can comfortably be drawn in a couple of minutes.
	\item A \circleref{medium} is a few paces across.
		Most houses should have a room large enough to draw one in, if the furniture is moved.
		It can comfortably be drawn in a quarter of an hour.
	\item A \circleref{large} is at least two dozen paces across.
		A ballroom or village hall is probably the only place one could be drawn indoors, so most are drawn outside.
		At least a couple of hours are required to draw such a circle without haste.
\end{itemize}

\subsection{Megalithic Circles}
\materiallabel{Megalithic Circle}{Megalithic Circles}{stone-circle}

Some rites require a circle of standing stones, called a \materialref{stone-circle}.
Such a circle must be at least the size of a \circleref{large}, with at least a dozen stones each taller than a man.
The arrangement and shape of the stones is unimportant, as long as it is recognisable as a ring of standing stones, and so the same circle can be used for all rites that require one.
Constructing a \materialref{stone-circle} is no easy task, typically requiring weeks of work by much of a village, even if the site is quite close to a stone quarry.

\subsection{Taglocks}
\materiallabel{Taglock}{Taglocks}{taglock}

A \materialref{taglock} is any part of a person's body, such as a piece of flesh, a strand of hair, a nail clipping, a drop of blood, or a gob of saliva.
It is often used to bind a spell to a particular target.
It can always be picked off a person---although taking a hair without being noticed might be difficult---but people often leave \materialrefplural{taglock} behind them, especially in places they frequent.
Finding a \materialref{taglock} in a place you suspect someone might have left on, such as their house or a bed they've slept in, typically uses \skillref{perception}.

\subsection{Poppets}
\materiallabel{Poppet}{Poppets}{poppet}

A \materialref{poppet} is an abstract representation of a person, although not a particular person.
Voodoo dolls are a typical example.
A \materialref{poppet} can be crafted from cloth, wood, clay, wax or other suitable material.
It should be recognisable as a human, bearing four appropriately-arranged limbs, a head, and two eyes.
However, if it is to be used in \discref{sympathetic-magic} affecting a non-human creature, it should resemble whichever creature the magic is intended to affect.
A \materialref{poppet} should be at least a handspan tall, though can be much larger.

\subsection{Effigies}
\materiallabel{Effigy}{Effigies}{effigy}

An \materialref{effigy} is much like a \materialref{poppet} and follows all the rules for one, except that it represents a particular person and must be crafted in their likeness.
It can be used only to affect the person it resembles.
Ideally, an \materialref{effigy} should be recognisable to even passing acquaintances of the person it is supposed to represent.
Less recognisable \materialrefplural{effigy} will require an appropriate Test to be used for magic.

\subsection{Blood}
\materiallabel{Blood}{}{blood}

Many spells call for \materialref{blood}, in varying quantities and from various creatures.
Extracting a mere drop of \materialref{blood} carries no ill effects.
Furthermore, a willing or restrained donor can provide \SI{100}{\milli\litre} of \materialref{blood} per point of \attref{might} with no ill effects, approximately once per week.
Above that, every \SI{100}{\milli\litre} of \materialref{blood} extracted deals 1 point of \secrefraw{damage}.
Creatures damaged in combat, by edged weapons, will also spill blood; approximately \SI{100}{\milli\litre} per point of \secrefraw{damage} dealt.
This blood will typically be lost in the dirt, however.

The above guidelines apply to humans, who typically contain about approximately 5 litres of blood in total.
Differently sized creatures will provide appropriately more or less blood.


\section{Herbs and Gardens}
\seclabel{herbs}
\herblabel{Ubiquitous}{1}
\herblabel{Common}{2}
\herblabel{Uncommon}{3}
\herblabel{Rare}{4}
\herblabel{Extraordinary}{5}

Herbs are an important component of most potions and poultices, as well as many spells.
Note that the term ``herb'' is used to encompass many things that are not technically herbs at all, such as fruit, fungi, tree bark and so on.

There are hundreds, perhaps thousands of different herbs, so instead of tracking every one, they are simply divided into categories based upon rarity.
Each potion or spell lists the highest rarity of herb that it requires.
Actual identities of the herbs may also be given, and can be used as guidelines to help improvise alternative spell components.

Herbs are classified as ubiquitous, common, uncommon, rare or extraordinary.
This covers both how common the herb is in the wild, and how difficult it is to cultivate in a garden.

\subsection{Finding Herbs}

Finding herbs growing in the wild uses \testtype{heed}{botany}.
A successful Test to find herbs provides enough for a few potions or poultices, or a few castings of a spell.
Under normal circumstances, this should be enough for the task at hand, or to restock a witch's supply.
But if the witch is trying to brew a potion for everyone in a castle, this supply might not cut it.

\begin{itemize}
	\item Ubiquitous herbs are incredibly easy to find and require no Test.
		They are primarily weeds that grow just about everywhere, whether people want them to or not.
		They should almost never take more than five minutes to find as long as you're outside, and even less if you're in a field or forest.
	\item Common herbs typically require searching a few hedgerows.
		They'll certainly turn up in an hour, and can be found much faster with a relatively easy Test.
	\item Uncommon herbs might require searching a large swathe of forest to even turn up one plant.
		This takes at least an hour, typically more, and requires a difficult Test.
	\item Rare herbs might not be found by searching an entire forest.
		Performing such a search exhaustively is infeasible, but a skilled botanist knows how to look in the right places.
		Still, this can take an entire day and requires a very difficult Test.
	\item Extraordinary herbs are not found in the wild under any but the most exceptional circumstances.
		They typically need to be traded from far-away places.
\end{itemize}

\subsection{Cultivating Herbs}

Many witches, especially brewers, keep a garden for growing herbs.
This is typically outside their cottage, but can be anywhere she likes.
However, tending a compelte garden requires about eight hours of work every week, which makes maintaining more than one a rather time-consuming task.
If the garden is left unattended for more than a week, it can require an even more considerable effort to get it back into shape.
The particularly needy or unruly herbs might die off, or even escape, during this time.

A garden is assumed to be accompanied by some storage of the herbs, so even herbs that need to be harvested at a particular time are available when they are needed.
A garden can provide an even greater supply of herbs than a search in the wild can, and may just about cut it to brew one potion for everyone in a castle.
But it can still be overtaxed, and this sort of thing shouldn't be tried too often.

Anyone can grow common or ubiquitous herbs in a garden.
As far as the weedy ubiquitous herbs go, most of the effort goes into keeping them under control.
Rarer herbs are harder to cultivate, however, requiring a skilled botanist (or a botanist with a skilled familiar).
%Uncommon herbs require Botany 1, rare herbs Botany 2 and extraordinary herbs Botany 3.
The following table gives the Botany skill required to cultivate a herb.

\begin{simpletable}{llX}
	\toprule
	Rarity & Skill & Examples\\
	\midrule
	Ubiquitous & - & Dock, Nettle, Clover\\
	Common & - & Lavender, Rosemary, Elderberry\\
	Uncommon & 1 & Tomato\\
	Rare & 2 & Truffle\\
	Extraordinary & 3 & Mandrake, Triffid\\
	%TODO: Expand
	\bottomrule
\end{simpletable}



\section{Improvised Tools}



\section{Weapons}
\seclabel{weapons}

Weapons are divided into several broad categories.
Players are free to describe their character's weapons how they wish, within the bounds of reason, placing them in one of the categories.
Anything a character might find at hand and hit people with can also be placed into a category.

A weapon's accuracy is added a flat bonus to rolls to hit, in place of an attribute.
A weapon's damage determines the number of dice rolled upon hitting.
The highest 3 dice are kept, as always, but the number of dice rolled are determined by the weapon instead of the wielder's skill.
The wielder's \attref{might} is added to the damage roll for melee or thrown weapons, but not for bows.

\begin{simpletable}{X[2.4]XXX[1.3]}
	\toprule
	Weapon & Accuracy & Damage & Range (metres)\\
	\midrule
	Fist & +2 & \dice{2} & Melee\\
	Club & +4 & \dice{4} & Melee\\
	Knife & +2 & \dice{5} & Melee\\
	Hand Weapon & +4 & \dice{5} & Melee\\
	Thrown Rock & +0 & \dice{2} & $5\times\text{\attref{might}}$\\
	Thrown Weapon & +0 & \dice{4} & $5\times\text{\attref{might}}$\\
	Bow & +2 & \dice{5} & 100\\
	\bottomrule
\end{simpletable}

\subsubsection{Fist}
A punch, a kick, or a headbutt.
Covers any attack you make without any weapon at all.

Some animals and familiars will have a different number of dice for their unarmed attacks, but use the same accuracy bonus unless this is also specified.

\subsubsection{Club}
A club, a walking stick, a chair, or a cauldron.
A club is just about anything you pick up and hit someone with.

\subsubsection{Knife}
A knife or dagger.
Easily concealed, and a staple of blood witches.
The short blade costs the wielder reach, but can do as much damage as a sword if you get the enemy in the tender parts.

\subsubsection{Hand Weapon}
A sword, an axe, a mace, a spear, a pike.
This category covers most things actually designed as a weapon and larger than a knife.

\subsubsection{Thrown Rock}
A genuine rock, but also a teapot, a boot or a frog.
Anything you might pick up and throw.
This includes weapons that aren't designed to be thrown.

\subsubsection{Thrown Weapon}
A spear, a knife, a hatchet.
Any weapon you can throw that was actually designed for the purpose.
Rocks from slingshots fall in this category too.

\subsection{Bow}
A bow and arrow.
Also covers crossbows, if the setting includes them.


\part{Playing the Game}
\partlabel{rules}

\chapter{The Broad of It}
\chaplabel{general-rules}

This chapter covers rules essential to day-to-day play.
Players and GMs alike should be familiar with at least the major points in here in order to play.
More specific rules, pertaining to various disciplines of magic, can be found in the appropriate chapters of \partref{disciplines}.

It is important to remember that this book cannot cover every situation that may arise during play.
The role of the GM includes adjudicating such scenarios, and the following section should contain guidelines to assist in that.
Furthermore, it is often helpful to do the same when the players simply cannot remember a rule, to avoid slowing down play while someone looks it up.
And lastly, remember that all the rules contained in this book are guidelines and suggestions.
Feel free to change them all that you want!
The most important thing is that everyone is having fun.

\section{Rounding Fractions}

In general, round down whenever you get a fraction, even if the fractional part is one half or greater.

\section{Tests}

Tests are the dice rolls used to determine the outcome of an action when there is element of chance and risk involved.
Several of the rules in this chapter and others will specify the appropriate Test to make with a particular action, but the GM should be calling for other kinds of Tests whenever appropriate as well.

A Test is typically made with a skill and an attribute, although having no applicable skill is not uncommon.
Often, the rule that required the Test specifies these.
Otherwise, the GM chooses as appropriate.
The character's skill determines how many dice she rolls for a Test.
If there is no skill applicable to the test, or if the character has no ranks in the applicable skill, she rolls 3 dice.
Each rank in the skill gives an additional rolled die, to a maximum of 6 with all three ranks.
Total together the highest 3 of the rolled dice and add the character's relevant attribute to this total.
The final total is compared against a Target Number (TN) set by the GM: if it meets or exceeds the TN the Test succeeds; otherwise it fails.

A Test where every die shows a 1 or 2 is a critical failure, and a Test where all 3 kept dice show a 6 is a critical success.
In addition to the Test automatically succeeding or failing, the GM is encouraged to apply an additional drawback or benefit to the result of the Test.
Critical failures on Tests involving dangerous magic can be especially catastrophic.

\subsection{Dice Notation}

A variant of standard RPG dice notation is used for Tests.
The size of the dice and the fact that only three are kept is omitted, as these are constants.
For example, \dice{4} indicates a 4 die Test with no bonus, and \dice[2]{3} indicates a 3 die Test with an attribute bonus of 2.

\subsection{An Example Test}

As an example, suppose Mistress Talbot is peering out of her window and attempting to identify which manner of undead dog has just shambled into her garden.
The GM declares this to be a \testtype{ken}{necromancy} Test, as she is attempting to recall information about the undead.
Mistress Talbot dabbled in \discref{necromancy} as a youth, and has one rank in the skill, so she rolls 4 dice.
However, her memory has begun to fade with age, so she has only 1 \attref{ken}.
The four dice show 4, 6, 2 and 3.
Her player totals the three highest dice, the 6, 4 and 3, for 13.
Then she adds Mistress Talbot's \attref{ken}, 1, for a grand total of 14.
Her player announces the total to the table.

The GM knows that the dog is a simple zombie, the most common variety of undead, but it was killed and raised only yesterday so the characteristic rot hasn't properly set in yet.
In light of this, she assigns a Target Number of 12: not too easy, but not particularly difficult either.
Hearing Mistress Talbot's total of 14, the GM knows that she has met the TN of 12: the Test has succeeded.
She announces that Mistress Talbot, by the creature's glassy eyes and stumbling gait, realises the midnight intruder is merely a zombie.
Reassured---she'd been fearing a ghoul or a hellhound---Mistress Talbot heads outside to see what the beast wants.
Though not without grabbing the poker from beside the fireplace, just in case.

\subsection{Target Numbers}

A Target Number (TN) represents the difficulty of the action that requires a Test.
The more difficult the action, the higher the target number, and the less likely the Test is to succeed.
In some situations, the same rule that requires a Test will specify its TN.
In other situations, the GM should select a TN she feels is appropriate.

Typical TNs range from approximately 9 to 21.
A Test with a TN lower than 9 is not normally worth it: a character with no skill and an average score in the relevant attribute will succeed more than \SI{95}{\percent} of the time.
Similarly, a Test with a TN higher than 21 is not normally worth it: a character needs a 5 in the relevant attribute to succeed without a critical success.
The following table shows a brief summary of the sorts of task particular TNs are suited to.

\begin{simpletable}{rX}
	\toprule
	TN & Task Difficulty\\
	\midrule
	9 & Easy: An average, unskilled person would normally manage this.\\
	12 & Moderate: An average, unskilled person would manage this about half the time.\\
	15 & Challenging: It takes skill to pull this off consistently.\\
	18 & Difficult: Even a skilled person is unlikely to achieve this consistently.\\
	21 & Legendary: This takes great skill, ability and good luck to perform.\\
	\bottomrule
\end{simpletable}

Instead of assigning a simple pass-or-fail TN, the GM may also employ graded success.
This is when a higher roll gives a higher level of success.
For instance, a higher roll on a Test to recall knowledge might mean that the character recalls more knowledge about the situation, while a higher roll on a check to influence a crowd might influence a greater proportion of the crowd.
This can also be used to apply success at a cost, where an intermediate roll, neither particularly high nor particualarly low, means that the character succeeds at their task but incurs some drawback in doing so.
For example, a coven might try to intimidate a guard to allow them into the castle.
Failure could indicate the guard calls for backup and resists, while a very high result on the Test would mean he is cowed and allows them to pass.
an intermediate result might mean that he allows the coven to pass, but sneaks off to find reinforcements and confront them later, while they are inside the castle.

\subsection{Using Tests}

Be careful not to call for a Test when it's not necessary.
If the action is a simple one that the character should be able to routinely perform, such as walking through a door or ransacking a room for something that isn't hidden, it doesn't require a Test.
(However, what is routine for one character might not be for another; a closed door can present a serious obstacle to many familiars.)
If the action is impossible, such as jumping over the moon or convincing the King to give up his crown without solid leverage, the player shouldn't make a Test.
If the character wouldn't succeed even with a critical success, a Test should never be rolled.
Lastly, if there is no penalty for failure, there is no need for a Test.
If the character will keep on trying until she succeeds, there's no need to make the player keep rolling Tests.

\subsection{Rolling Fewer Than Three Dice}

Some effects will modify the number of dice a character rolls for a Test, and this can bring the number of rolled dice below three.
In this case, all the rolled dice are added to the total as normal, but the maximum total that can be reached is obviously reduced.
Additionally, critical success is no longer possible, as this require three dice showing 6.
Critical failure, however, becomes far more likely, as it only requires that all dice show 1 or 2.

If the number of dice rolled for a Test would be reduced to zero, the Test cannot be performed.
If it is unavoidable, it is automatically treated as a critical failure.

\section{The Flow of Time}

\subsection{Narrative Time}

During normal play, the exact timing and duration of characters' action are unimportant, and not carefully tracked.
It is enough to know whether something took a matter of seconds or minutes, an hour or two, or a couple of days.
This is Narrative Time, and the GM is free to be as accurate or as loose as necessary with time periods.

The one element of Narrative Time with an impact upon the rules is that of Scenes, which are often used to measure the duration of effects.
It may be helpful to think of scenes like in a play.
The Scene typically changes when the action changes location (everyone walks from the church down to the village green), when there is a timeskip (everyone waits an hour for the sun to set) or there is a change in the cast of characters (the preparations for the party finish and the guests begin to arrive).
Changes in Scene, and the duration of effects that rely on them, are ultimately left up to the GM, but should often be obvious.

\subsection{Structured Time}
\seclabel{structured-time}

In tense situations with two opposing parties, exact timings and durations become important to track.
For this purpose, and to aid tactical thinking in such scenarios, the GM can move the game into Structured Time.
Direct combat is perhaps the most common application of this, but chase scenes may also use them.
With the correct magic, some of the participants might even be many miles apart.

Structured Time is divided into {\rounds} and {\turns}.
Every character participating in the Scene gets one {\turn} each {\round}.
Although the {\turns} are resolved in some order, all characters are assumed to be acting simultaneously and continuously.
If it becomes particularly relevant for some reason, assume each {\round} takes approximately 10 seconds.

On each {\turn}, a character may move a number of metres equal to their \statref{speed} and take one {\action}.
An {\action} is something that requires most of the character's effort during their {\turn}, such as attacking someone, performing a brief bit of magic, knocking a hole in a wall or quaffing a potion.
They may also take a reasonable number of minor actions that shouldn't require their full concentration, such as opening or slamming a door, drawing a sword, pointing at something or speaking a short sentence.
Not everything can be accomplished in one {\action}.
For example, winching a drawbridge closed may take several {\actions}, as might even one of the faster magical rites.
Some of the {\actions} available to a character are given in \secref{combat-actions}, but the GM is free adjudicate anything the characters try as one or more {\actions}.

\subsection{Initiative}

When the GM determines that the game should move into Structured Time, Initiative Tests are used to determine the order in which participants take their {\turns}.
Initiative determines how quickly characters notice the situation and are ready to act.

Initiative Tests can use any attribute and skill appropriate to the situation, as determined by the GM.
For example, an argument that boils over into a brawl might prompt Initiative Tests using \testtype{heed}{insight}, favouring characters who noticed tensions rising and fists clenching.
Combat that begins as characters race to the source of a scream might use \testtype{grace}{athletics}, favouring characters who arrive fastest.
If nothing in particular seems appropriate, default to a \attref{grace} or \attref{heed} Test with no applicable skill.

The GM may even assign different Tests to different characters, with appropriate bonuses or penalties.
For example, suppose a group of bandits ambush for a group of travellers.
The bandits roll \testtype{grace}{stealth} to spring from hiding, with a \positive{6} bonus as the ambushers.
The travellers roll \testtype{heed}{perception} to notice the bandits attacking.
The GM may assign a greater or lesser bonus to a better-laid ambush, or one staged in a suboptimal location.

Initiative Tests are not made against a particular TN.
Rather, all characters are ranked in order.
This is the Initiative Order, and remains the same on subsequent {\rounds}.
The character with the highest result takes their {\turn} first, and subsequent {\turns} proceed down the Initiative Order.
Once all characters have taken a {\turn}, return to the top of the Initiative Order for the next {\round}.

To save time, the GM may make a single Test for a group of similar NPCs, such that they all get the same result and take their {\turns} at the same time.
Similarly, a witch's familiar and all other creatures associated with her (such as a horse she is riding, or her golems and undead) use the witch's Initiative result and take their {\turns} at the same time as her.

\section{Movement}

Each {\turn} in Structured Time, a character can move a number of metres equal to her \statref{speed}, as well as taking an {\action}.
If she takes the \actionref{dash} {\action}, she may move a total number equal to twice her \statref{speed}.
This assumes that she is moving on foot over smooth ground.
This speed represents urgent movement over a short period.
A character trying to maintain this pace for more than a couple of minutes typically requires a \testtype{might}{athletics} Test to avoid tiring.
%TODO: Reference exhaustion rules here?

%TODO: Long-range travel times.

%TODO: Climbing and swimming.

\subsection{Difficult Terrain}
\seclabel{difficult-terrain}

{\diffterrain}, such as dense forest or a bog, slows characters trying to move through it.
As a simple default, movement through it is halved; it costs 2 metres from a character's \statref{speed} to move through 1 metre of {\diffterrain}.
The GM is free to impose a lesser or greater penalty for more or less severe terrain.

For some kinds of {\diffterrain}, the GM may offer players the option to ignore the movement penalty at an alternative cost.
For example, a character pushing through brambles may move at full speed, but be subjected to a \seclink{Damage Test}{damage-tests} for doing so.
A character moving on slick ice or along a narrow ledge may move at full speed, but must succeed on a \testtype{grace}{athletics} Test to avoid falling over, or off the edge{\dots}

\section{Injury}

Witchcraft is a dangerous business.
Between mad spirits, evil demons, foul undead, and disgruntled mobs of villagers, injury is inevitable.
And it's not only her own injuries that a witch has to deal with.
One of a witch's duties is to tend to the injuries of her neighbours, nursing them back to health after an accident or disease has laid them low.
Or, when they are beyond her help, easing their final moments.

A character's resistance to injury is determined by two statistics: \statref{res} and \statref{st}.
Most creatures of flesh and blood, including humans and familiars, have $3$ \statref{res}.
Other creatures, such as golems, may be more or less resilient.
A character's \statref{st} is equal to 12, plus their \attref{might}, plus their \attref{will}.

\subsection{Damage Tests}
\seclabel{damage-tests}

A \seclink{Damage Test}{damage-tests} is a special type of Test used to determine how much an effect hurts a character.
It is made like a normal Test, by rolling some number of dice and adding the highest 3 together, with a flat bonus.
In the case of an attack by one character upon another, the number of dice are determined by the weapon used and the flat bonus by the wielder's strength.
In other cases, the GM or the rules of the damaging effect assign the number of dice and the bonus.
For small effects, this can often be fewer than 3 dice.

The following table provides examples of the number of dice and the bonus for \seclink{Damage Tests}{damage-tests}.

\begin{simpletable}{Xl}
	\toprule
	Effect & Damage\\
	\midrule
	Touching a hot cauldron & \dice{1}\\
	Crawling through brambles & \dice{2}\\
	Wave-tossed against a boulder & \dice{3}\\
	Hit by a falling brick & \dice{4}\\
	Falling on a sword & \dice{5}\\ %TODO: Evaluate and expand.
	Hit by a falling tree & \dice[4]{5}\\
	\bottomrule
\end{simpletable}

Additionally, a \seclink{Damage Test}{damage-tests} is not made against a particular TN like most Tests.
Instead, it applies two effects to the target, {\shock} and {\damage}.
{\shock} is always tested for before {\damage} is applied.

Critical failure on a \seclink{Damage Test}{damage-tests} means no effect is applied at all; the blow was glancing and won't do more than bruise slightly.
Critical success on a \seclink{Damage Test}{damage-tests} may immediately kill the target or leave them with a lasting injury, at the GM's option, and always applies {\shock}.

\subsection{Shock}
\seclabel{shock}

If a \seclink{Damage Test}{damage-tests} meets or exceeds the target's \statref{st}, or critically succeeds, the target goes into {\shock}.
A character in {\shock} falls unconscious and cannot be roused while they remain in {\shock}.
If a character in {\shock} would go into {\shock} again due to another \seclink{Damage Test}{damage-tests}, they die.

Additionally, at the start of each of the {\shocked} character's {\turns}, roll a special Test against them.
This Test applies no flat bonus, and uses the same number of dice as the \seclink{Damage Test}{damage-tests} that sent the character into {\shock}: a character is more likely to bleed out from a sword wound than a punch.
If it meets or exceeds the {\shocked} character's \statref{st}, they die.
This Test is not considered to be a Test made by any character.

If the Test made every {\turn} ever totals 9 or less, unless it also meets or exceed their \statref{st}, the character is no longer in {\shock}.
However, the character remains unconscious and cannot be naturally roused before the end of the Scene.
A character can also be brought out of {\shock} by another character tending to them.
This requires an {\action} and a successful \testtype{ken}{healing} Test.
The TN for this Test is 3 times the number of dice that would be rolled against the {\shocked} character each round.

\subsection{Damage}
\seclabel{damage}

After {\shock} has been tested for, whether or not it occurs, the \seclink{Damage Test}{damage-tests} causes {\damage}.
To calculate {\damage}, divide the result of the \seclink{Damage Test}{damage-tests} by the target's \statref{res}.
For example, if the result of the \seclink{Damage Test}{damage-tests} is 13 and the damaged creature has 3 \statref{res}, they suffer 4 {\damage}.
{\damage} accumulates: a character who has previously suffered 3 {\damage} and suffers an additional 2 is now suffering from 5 {\damage}.

{\damage} has two effects.
Firstly, a character subtracts their current {\damage} from their \statref{st}.

Secondly, if a character's \statref{st} ever reaches zero, they die immediately.
This is very unlikely to happen through repeated {\damage}, as an earlier blow would send them into {\shock}, but can occur if a lot of painkillers wear off all at once.
%Secondly, if a character's current {\damage} ever equals or exceeds their original \statref{st} (unmodified by {\damage}), they die immediately.
%This applies even if they are ignoring the effect of some of their {\damage}, such as through painkillers.
%As effects that allow a character to ignore {\damage} are so common, it can be helpful to track a character's actual {\damage} and the {\damage} they are considered to be suffering from separately.
%The former represents injury: scrapes, bruises and cuts.
%The latter represents the pain suffered as a result of these.

\subsection{Healing \& Recovery}

{\damage} heals naturally over time, but it's a slow process.
Once per day, with a decent meal and at least about six hours of sleep, a character may recover from 1 point of {\damage}.
If the character spends the entire day resting, they may heal 1 additional point of {\damage}.
For a lightly wounded character, taking a stroll would be acceptable without disturbing a day of rest.
For a character with more serious wounds, they shouldn't move around too much, and may even require complete bed rest.
The GM should adjudicate this, taking into account how the {\damage} was sustained, but in general a character who is not suffering any penalty to rolls due to {\damage} doesn't need to be too careful.

Tending by a healer can hasten the natural recovery process, but only provides any benefit if the character is taking an entire day of rest.
For each rank their physician has in the \skillref{healing} skill, a character taking a day of rest may heal 1 additional point of {\damage}.
A single healer can tend many patients in a day, up to about a dozen.
They may tend themselves, but only if their activities tending others do not prevent them from taking a day of rest themselves.

\subsection{Exhaustion}
\seclabel{exhaustion}

Besides injury, an active witch runs the risk of {\exhaustion}.
From late night vigils to running after tricksy spirits, many things can leave a witch tired and longing for her bed.

When a character performs something exhausting, or goes a day without at least 6 hours of sleep, the GM may apply a level of {\exhaustion}, or call for a Test (typically \attref{might} or \attref{will}) to avoid one.
Each level of {\exhaustion} reduces two of a character's attributes by 1.
The GM selects appropriate attributes depending on the type of {\exhaustion}.
For example, {\exhaustion} as a result of a long foot chase might decrease \attref{might} and \attref{grace}.
Sleep deprivation might decrease \attref{wit} and \attref{heed}.
A long day of socialising, rushing from meeting to meeting, might even reduce \attref{charm} and \attref{presence}.

Multiple levels of {\exhaustion} may reduce the same attribute, leading to a total reduction of 2 or more.
Whenever a character suffers a second level of {\exhaustion} affecting the same attribute, the GM may call for a Test to avoid passing out until they can sleep it off.

A character may reduce their {\exhaustion} by 1 level when they get a good night's sleep: about eight hours.
An entire day of rest reduces {\exhaustion} by another level.
The player my choose which to attributes to recover when they reduce their {\exhaustion}.

\section{Combat}
\seclabel{combat}

\subsection{Actions in Combat}
\seclabel{combat-actions}

\subsubsection{Attack}
\actionlabel{attack}

You attack a creature or object, with a \seclink{weapon}{weapons} or unarmed.
You must be adjacent to the target to attack with a melee weapon, or within the listed range of a ranged weapon.
Make a Test using a number of dice determined, as normal, by your \skillref{weaponry} skill, and a flat bonus determined by your weapon's accuracy.
The Test is made against a TN equal to the target's \statref{dr}: 8, plus their \attref{grace}, plus their \attref{heed}.

If you succeed in your Test, you hit.
Make a \seclink{Damage Test}{damage-tests} against the target, rolling dice as determined by your weapon and adding your \attref{might}.

\subsubsection{Dash}
\actionlabel{dash}

You may move an additional number of metres equal to your \statref{speed} this {\turn}.

\subsubsection{Ready}
\actionlabel{ready}

You don't act immediately, but prepare to take an {\action} later.
Decide what {\action} you will take, and which circumstances trigger it.
When those circumstances come around, you may choose to take the readied {\action} or not.
If your next {\turn} comes around without you taking the readied {\action}, you lose the benefits of readying.
You must \actionref{ready} again if you want to continue to wait.

\section{Magic}

Magic consists of too many diverse disciplines and effects to be effectively summarised in this section; indeed this is the entire topic of \partref{disciplines}.
However, a few general guidelines apply.

It is generally assumed that any witch who knows a spell, rite or technique has the knowledge and practice to pull it off consistently; doing so does not require a Test unless specified otherwise.
However, this practice only applies under normal conditions, with adequate time and materials.
A witch may attempt to rush her magic, perform it using whatever she has to hand, or to perform it in difficult conditions, and each of these requires a Test.
Such Tests typically use \attref{wit} and the relevant skill for the discipline of magic, but not always.
More formulaic disciplines such as \discref{brewing} and \discref{ritual-magic} often use \attref{ken}, while other disciplines, such as \discref{willing} and \discref{golemancy}, rely primarily upon a witch's pure force of \attref{will}.
Furthermore, drawing a chalk circle hurriedly might use \attref{grace}, and grinding a poultice while riding a broomstick might use \skillref{flying}.

TNs for rushing or improvising magic are ultimately left up the GM, but some guidelines are provided below.

\subsection{Rushing Magic}

Generally, magic that would normally take at least an {\action} in combat cannot be performed in less than that time.
Exceptions may be made where the magic is used as part of the {\action} already being taken, to aid it or improve its effect, but the GM should still be careful allowing such things.
Otherwise, common sense may apply a limit to the minimum time magic can be performed in.
For example, if a potion requires boiling water, a witch needs some way to bring water to the boil in the time they want to brew their potion.

Where magic can be rushed, guideline TNs for doing so are given in the following table.

\begin{simpletable}{rX}
	\toprule
	TN & Example Task\\
	\midrule
	9 & Performing a simple rite in half the normal time.\\
	12 & Performing a complex rite in half the normal time.\\
	15 & Performing a simple rite in a tenth the normal time.\\
	18 & Performing a complex rite in a tenth the normal time.\\
	21 & Performing a simple 5 minute rite in one {\action}.\\
	\bottomrule
\end{simpletable}

\subsection{Improvising Materials}

This applies to both the tools used to conduct magic and the ingredients consumed by it, and works equally well in brewing and rites.
The most important part is that the witch can justify any substitution to herself.
From a gameplay perspective, this also means that the player should justify such improvisations to the GM.
This can be as simple as using a pool of water in place of a mirror, because both are reflective, or more extreme, such as using a fresh egg in place of blood, as both are the fluids of life.

\begin{simpletable}{rX}
	\toprule
	TN & Example Task\\
	\midrule
	9 & An unusual component that still meets the specifications, e.g.\ a ritual circle scratched in the dirt instead of drawn in chalk.\\
	12 & A component that retains the fundamental property, e.g.\ scrying through a pool of still water instead of a mirror.\\
	15 & A component that is close, but violates a specification, e.g.\ pig blood instead of human blood.\\
	18 & A component with a reasonable justification for relatedness, e.g.\ a fresh egg in place of blood.\\
	21 & A component with a weak justifcation for relatedness, e.g.\ apple juice in place of blood.\\
	\bottomrule
\end{simpletable}

\subsection{Consequences}

Magic is dangerous, especially when rushed or improvised.
The GM should feel free to reflect this in the consequences of failure on a magic Test, even when it is not a critical failure.
Failure on a magic Test need not indicate that nothing occurred, but might indicate that something unwanted or something rather tangential has occurred, or that the magic has succeeded, but with side effects.

For example, suppose a witch is attempting to brew a potion for hair regrowth, but has substituted several of the ingredients for similar ones they hoped would work.
A failure on the Test might mean that the potion successfully causes hair regrowth, but that the hair is the wrong colour or grows in more places that desired.

Other magics can have even more dangerous consequences.
A witch trying to scry through a puddle instead of a mirror might, on a narrow failure, only get an unclear image as the puddle is disturbed by wind.
But a more dire failure could mean that the target instead sees the witch herself through any nearby reflective surfaces, or that the imperfect scrying draws the attention of \emph{things} from other dimensions that look, reach or even climb out of the puddle.
Rituals to summon demons and the like can obviously have some of the most dangerous consequences of all, should they go wrong.


\part{Disciplines of Witchcraft}
\partlabel{disciplines}

Elle Weerstrom looked up from her parsley patch as the air swooshed overhead.
Black fabric flapped.

``Evenin' Linda.
Didn't expect to see you today.
What brings you up 'ere?''

A navy-lined cloak fluttered as the younger witch pulled her broomstick short and dropped to the ground.
``It's young Barnie, Elle.
He's got a mob together, marching on Buckle Hollow.
Says Musgrave's been sleeping with his wife.''

Elle brushed her gloves together, knocking dirt onto the lawn.
``Well, has he?''

``No!
I mean, they might've kissed a bit but{\dots}
They've got torches, Elle!
Pitchforks and torches!
C'mon, grab your broom.
We've got to stop them.''

Elle looked down at the ground, then up at the sky.
She sighed.
``Alright, we'll go.
But we're walkin'; there's a storm brewing.''

Linda looked up.
A single wisp of cloud drifted lazily across the azure sky.
``Looks alright to me.''

``It's on its way, mark my words.
Wouldn't want to be flyin' home in it.''
Elle strode towards her cottage.
``I'm goin' to get my coat.''

\storybreak

Sure enough, the sky was grey when the mob got to Buckle Hollow.
A fine drizzle filled the air.
The farm gate stood open, a figure between the posts in its place.
Her parka was pulled up against the rain, pointed hat tall above her crown.
The mob stopped in its tracks as a crack of lightning cast her silhouette upon them.

``Fine weather for arson, innit?''
Her voice seemed to carry further than it should in the damp air, reaching the ears of all present.
They shuffled their feet in the thickening mud.
``Yer a disappointment, the lot o' yer.''
More feet shuffled.
A voice rose in dissent, but Elle continued over it.

``Now, I know Musgrave ain't the finest man you've all met.
An' I ain't quite sure what he's been up to that's got you all riled up.
But I \emph{am} sure that it ain't nothin' worse than half o' you've done to yer own wives!
Honestly, torches lads?''
The rain intensified and the torches guttered.
One spluttered out.
``What were you goin' to burn?
The barn?
His house?
\emph{Him}?
Put 'em away, men.''

There was another shuffling of feet, and a few torches wobbled noncommittally.
A sudden gust of wind drove the rain sideways for a moment.
Every torch went out with a pathetic cough.
``Get home to yer own wives, an' stop worryin' about other people's.''

With a quiet mumble, a general grumble and a mutter of ``Soddin' linen's gonna be soaked{\dots}'' the mob turned around and began to trudge the other way.

``An' Barnie!''
The mob stopped in its tracks again.
One man turned around, a few others craned their necks to see.
``She mightn't be kissin' other blokes if you spent as much time in yer own bed as in the gutter out back o' the Head.''
A muffled chuckle ran through the mob before another peal of thunder cut it short.
Collectively, they slank off through the mud.


\chapter{Willing}
\chaplabel{willing}

\discref{willing} is the most raw and versatile application of a witch's magic.
Known to many layfolk as sorcery or spellcraft, it is the art of making something true simply by willing it hard enough.
Most \discref{willing} is performed without any of the accoutrements that accompany other forms of magic, and it doesn't follow the prescribed formulae of rites and brews.
This makes it the weakest form of magic in some ways, but its flexibility and ease of access more than make up for it.
So much so that every witch knows at least the basics.

Like any witchcraft, \discref{willing} is something anyone can do if they know how.
But there is a knack to it.
It requires that the witch not only \emph{want} something to be the case, but \emph{believe} that it already is.
That she outright refuses to accept any possibility that it might not, in fact, be the case.
It involves willfully deceiving not only oneself, but also the very universe.
Most people would never even think to try it, but it is among the first things any aspiring witch must learn.

The line between \discref{willing} and \discref{headology} can be a little blurred, at times.
Both have the ability to make things true by making people believe them.
Many Willers say that the difference is that \discref{willing} affects the real world, while \discref{headology} only affects other people's minds.
The Headologists point out that other people are just as much a part of the real world as any old rock is.
Some Headologists say that the difference is that \discref{headology} is about convincing other people, while \discref{willing} is about convincing yourself.
The Willers point out that it's about more than convincing yourself, it's about convincing the world.
And that includes other people.
A few say that there's no real difference at all, that it's just two ways of thinking about the same thing.
These tend to be the witches who are obnoxiously good at both, and everyone else pointedly ignores them.

One interesting property of \discref{willing} is that it cannot affect other people or animals.
It takes more than force of will to convince someone that they're a different shape; usually this entails talking to them.
This doesn't stop people getting knocked off their feet by a gust of wind, or crushed by a falling tree, however.
Witches interested in affecting people more directly are encouraged to pursue \discref{headology}.
Or swordplay.

Unlike many magical disciplines, which depend upon \attref{wit} for understanding or memorising their complexities, \discref{willing} depends upon raw \attref{will}, your own stubbornness and conviction against the fabric of reality.

\section{Feats}

\feat{Basic Willing}{basic-willing}{10}{
	None
}{
	You can perform very basic acts of \discref{willing} upon things you can touch, given a bit of time to focus your mind and an obvious physical cue.
	Examples include:
	\begin{itemize}
		\item Lighting kindling or a candle without a spark, by cupping your hands around it and blowing on it.
		%\item Colouring or mildly flavouring a small pot of water by stirring it.
		\item Scratching writing into stone using just a fingernail.
		\item Rubbing stains out of clothing using your bare hands.
		\item Combing your hair with just your fingers.
	\end{itemize}
	The amount of time required to produce an effect varies depending on the desired outcome, but should be more than an Action without a Test.
	This ability cannot produce a lasting effect by itself.
	You can light a fire, because that sustains itself once ignited, but you cannot create, destroy or melt a pebble.
}

\feat{Kindling}{fire-willing}{15}{
	\featref{basic-willing}
}{
	You've practiced \discref{willing} a fire to life, and it's getting a lot easier for you.
	You can now ignite a fire within a dozen metres as an Action, with nothing more than a quick glare.
	You no longer require kindling, but still need something a fire can catch on fairly easily, such as twigs, cloth or dry leaves.
	Lighting a log or floorboards is still beyond you.
	
	The fire begins small, so will be extinguished by rain or a moderate wind before it can catch.
	A person walking about or wriggling will automatically foil an attempt to ignite their clothes (perhaps without noticing), but a person sitting fairly still may not.
}

\feat{A Tool for the Job}{willing-tools-improvise}{20}{
	\featref{basic-willing}
}{
	Sometimes, the easiest way to convince someone of something is the hit them with a big stick until they agree with you.
	The world itself is no different.
	You've learned to make \discref{willing} easier using physical tools, even if they aren't the \emph{right} tools.
	
	Most simply, this means axes and knives cut just as well as ever in your hands, even if they've lost their edge.
	But you can take it even further, cutting carrots or trees with nothing more than an appropriately shaped stick.
	You can make any similarly-shaped object behave as the appropriate tool for a job.
	For a worse approximation, this may require a Test.
	A solid branch with a flat, sort of axe head shaped bit on the end will do a fine job of cutting down a tree.
	A solid branch without such an attachment would require a Test.
	A limp reed is going to be a real stretch.
	
	Such tools still obey the usual rules of \discref{willing}, and are of no additional use as weapons against people and animals.
	See \featref{headology-weapons-improvise} if you want weapons too.
}

\feat{Bubbling Brook}{water-willing}{10}{
	\featref{basic-willing}
}{
	Water is considered by many to be an element of change.
	You've certainly figured out how to change it.
	While touching water, you can move it around with your mind.
	You can make it flow, swirl, form into fairly elaborate shapes, or even float into the air.
	
	You can only affect the water while it remains one continguous mass, which you must be touching.
	Afterwards, it flows normally again.
	You can only affect a couple of buckets-full at a time, and can't stretch it out over more than a couple of metres.
	You also can't move the water fast enough to hurt anybody.
	You can move other liquids if they are primarily water, such as wine, blood or most potions.
	As always with \discref{willing}, you cannot affect liquids inside a living person.
}

\feat{Water Walk}{water-walk}{20}{
	\skillref[1]{willing},
	\featref{water-willing}
}{
	You can walk on water, or any other liquid you could affect with \featref{water-willing}.
	This takes great concentration, and you cannot take an Action and move on the water's surface in the same Turn.
	You may take an Action if you stand still on the water, however.
	
	If the water is flowing, you will be carried with it.
	Staying upright on fast flowing or turbulent water may require a Test, and the effect requires you to stay on your feet; falling prone will cause you to fall into the water.
	You may take use an entire Turn to clamber onto the water, if you are swimming at the surface.
}

\feat{River Run}{water-walk-2}{15}{
	\skillref[2]{willing},
	\featref{water-walk}
}{
	Walking on water has become second nature to you.
	You may take Actions while moving.
	Additionally, flow and turbulence pose you no threat.
	You may treat water you are standing on as though it were not flowing, and you can remain on the water's surface even when prone.
	Lastly, climbing upright onto the water while swimming at the surface is treated as though you are merely standing from being prone.
}



\discipline{Headology}{headology}{Headologist}{Headologists}

\dropcapdiscref{headology} is really no magic at all.
Rather, it is the art of making other people use their own magic.

One common misconception among apprentice witches is that \discref{headology} is the ability to affect people's minds.
Their mentors must quickly disabuse them of this notion.
Everyone has the ability to affect people's minds, and uses it every day.
It's called talking.
It can make someone like you or hate you, make them smarter or more stupid, even make them believe that the sky is purple, if you're really good.
It's an incredibly ability---the most important one a witch can have, in the opinion of most.
Enough people, sufficiently motivated, can move mountains.
But talking, by itself, is not \discref{headology}.

\discref{headology} is the step that comes after.
\discref{headology} is making people's minds affect the world---letting them move mountains without all the shovels and wheelbarrows it normally requires.
Making them into \practitioners{willing}, without them even knowing it.

Every person, and even every animal, has the ability to affect the world through \discref{willing}.
Most never realise this, and would struggle to control the power even if they knew.
But a witch who knows the trick of it can unlock another person's ability.
And if she's convinced them of the correct things first, she can direct it with her words.
This is the basis of most \discref{headology}.

\section{Convincing People}

Almost every feat in \discref{headology} requires a witch to convince somebody of something, before it has any affect.
Talking to people is a complicated subject, and there are no strict rules for this.
As such \discref{headology} is more subject to the whims of the GM than many other disciplines.
The following paragraphs provide many guidelines for adjudicating this, but the GM should also apply common sense, and remember to ensure that everyone is having fun.
If the the \practitioner{headology} is consistently overshadowing the rest of the coven, it's probably proving too easy to convince people of things, and vice versa.
And if they've put on a particularly awesome show to convince someone, just let it work.

Firstly, if there is doubt as to whether a \practitioner{headology} has convinced someone, call for an {\opposed} Test.
On the part of the \practitioner{headology}, this will normally use \attref{charm} or \attref{presence}, and \skillref{persuasion} or \skillref{deception}.
\skillref{intimidation} and \skillref{socialising} might come into it fairly often, as well.
On the part of the intended victim, this might use \attref{will}, to hold onto a conviction, or \testtype{heed}{insight}, to see through a trick.

\subsection{Modifiers to Convincing}

For anything but the simplest effects---such as \featref{curse}---simply stating something is not enough to convince someone, no matter how persuasive your tone.
In these circumstances, the GM should simply not let the victim be convinced, or at least apply a major penalty to the Test to convince them.
Often, some variety of evidence or trick is required.

For example, a prince who's simply told he's a frog is unlikely to fall for it.
But a prince who's told he's a frog, then gets knocked out, and wakes up in a pond with his skin covered in slime---he's more likely to buy it.
With the right evidence, a witch might not need to speak to the victim at all.
A prince who wakes up in a pond, surrounded by other frogs, all dressed in the armour of his personal guard---he's going to leap to his own conclusions.

As such, the GM should use the circumstances to put modifiers on rolls to convince people.
This should often be a negative modifier without a good argument or some evidence, while presenting a solid piece of evidence can give a positive modifier.
An elaborate---but solid and successfully executed---plan for convincing someone will often bypass the need for a direct Test altogether.
The type of effect being applied should also influence the modifier.
It is much easier to convince someone that they're under a simple bad-luck curse than that they're a frog.

Lastly, the \practitionerpossessive{headology} reputation can be important.
If the victim knows that she is a powerful witch, this can go a long way by itself.
If she specifically has a reputation for turning people into frogs, people are likely to believe her pretty easily when she says she's turning them into a frog too.
Even more so if they've just seen her do it to one of their friends.
Practically, this means that a \practitioner{headology} often needs to make it clear that she's a witch, by wearing the {\hat}.

\subsection{The Trick of Headology}

One unfortunate catch of \discref{headology} is that it only works as long as the victim is unaware of quite what's being pulled on them.
As soon as someone realises that they'll only turn into a frog if they believe they're a frog, they'll never believe it.
Even if they try.
This means that it is impossible to \emph{willingly} be the subject of \discref{headology}.

It also means that a \practitioner{headology} needs to be careful not to let their victims catch on to what's happening.
This is not commonly a problem with normal folk, unless someone explicitly explains it them.
Superstition runs rife, and a witch who uses a lot of \discref{headology} is likely to provoke more fear and respect than understanding.
However, a \practitioner{headology} ought to maintain a certain mystique about her craft, to ensure no clever clogs goes digging too deeply.

With other witches, however, tend to catch on quite quickly.
A witch who has seen a particular trick of \discref{headology} used a couple of times tends to figure it out, and thereby become immune to it, whether she wants to or not.
Other \practitioners{headology} tend to be even quicker on the uptake, and are likely to catch on the very first time they see a trick, if the witch using it on them isn't careful.
A witch who knows and uses a trick herself can never be affected by it, except, perhaps, in the most exceptional caper of all time.

Perhaps mostly importantly, this means that you can never use your \discref{headology} on your own coven, unless you are careful to keep them in the dark about a new trick you've picked up.
Even then, it won't last long.

\section{Feats}

\feat{Curse}{curse}{15}{
	\noprereq
}{
	It's a well known fact that someone who believes they will fail is more likely to do so.
	It doesn't take a drop of magic to make that true, but not everyone knows how to leverage it.
	You do.
	
	If someone believes that you have cursed them, or even if you can convince them that they have been cursed by something else, they suffer bad luck.
	Whenever they make a Test, dice that roll a 3 count towards a critical failure.
	The GM is also encouraged to make their critical failures a little more dire.
	This bad luck persists as long as the supposed curse is present in their minds; it might help to remind them now and again.
	
	This only applies if they believe they are under a rather broad curse, or specifically a bad luck curse.
	An overly specific curse---for example, ``May your crops wither in your fields,'' or ``May your nose fall from your face''---does nothing to focus their mind on their own failure and will have no effect.
}

\feat{Mentally Scarred}{headology-wound}{10}{
	\featref{curse}
}{
	You have mastered a more specific form of curse---a curse of physical wounding.
	
	If you can convince someone that they are wounded, they develop the wounds they believe they have.
	This directly causes {\damage}---not a {\damagetest}---appropriate to the kind of wound they develop.
}

\feat{Mind over Magic}{foil-magic}{15}{
	\noprereq
}{
	For all the magic circles and burning incense, magic ultimately comes from the mind.
	Not only do you know this, but you know \emph{how to exploit it}.
	
	If you can convince a practitioner of magic that their magic won't work, then it won't.
}

\feat{Doubt \& Despair}{foil-magic-2}{25}{
	\featref{foil-magic}
}{
	Under your tender care, even the smallest seed of doubt can flourish into a blossoming tree of failure.
	
	If you can make a practitioner so much as doubt the efficacy of their magic, or their own ability to work it, then the magic will either fail to work or, at the GM's option, backfire.
}

\feat{Mind Like a Razor}{headology-weapons-improvise}{10}{
	\featref{willing-tools-improvise}
}{
	If you can convince your foes that what you wield is a weapon, their flesh will believe you.
	You may treat an item you wield or throw as a \weaponref{knife}, \weaponref{hand-weapon} or \weaponref{thrown-weapon} (depending on its size and whether you're throwing it) if you can convince the target that it can cut (or otherwise deal damage) like one.
	A demonstration against an inanimate object, or another foe, will often suffice.
	Even your bare hands can cut like \weaponrefplural{knife} if you convince your foes that they can.
}

\feat{Change Blindness}{headology-stealth}{10}{
	\noprereq
}{
	You may hide in plain sight by leveraging the fact that people don't \emph{expect} to see you there.
	This uses a \testtype{charm}{stealth} Test.
	You must remain silent and quite still, though you may creep around slowly.
	
	In order to make use of this feat, anyone you are hiding from must have no reason to expect to see you, or anyone.
	If they see much out of place---a drawer opened or a vase knocked over---they might look for whoever did it and will immediately spot you.
	Furthermore, you can only use it if the people you are hiding from have some degree of familiarity with the location; they must have seen it before, at least.
	Somebody entering a room for the very first time doesn't know what to expect and will see it as it is, you included.
	
	Lastly, somebody seeing a group or crowd of people has no reason not to expect other people with them.
	This feat does not allow you to hide in such a situation, unless everyone in the group has the feat.
	%TODO: Is there a feat that helps blending in with a crowd?
}

\feat{Elsewhere}{headology-stealth-2}{15}{
	\featref{headology-stealth}
}{
	While \featref{headology-stealth} lets you hide from people who aren't expecting \emph{anyone}, you've now figured out how to hide from people who aren't expecting \emph{you}.
	As long as someone is convinced \emph{you} won't be somewhere---for instance, you've told them you'll be somewhere else---they won't see you there.
	Note that it is not enough for them not to expect you there---except as falls under the purview of \featref{headology-stealth}---they must expect you not to be there.
	
	This only holds up as long as you aren't too too intrusive.
	For example, you shouldn't pass in front of something they are paying attention to, make any loud noises, or open any doors they are looking at.
	However, you might be able to get away with moving things around.
	Even if somebody notices that something has been moved, they ought not to suspect \emph{you} to have done it, as long as they still believe you are somewhere else.
	Tests to avoid being noticed, if it is in doubt, use \testtype{charm}{stealth}.
	
	Furthermore, this feat does allow you to go unnoticed in a crowd, as long as the person watching has good reason to believe you won't there.
}

\feat{You Shall Not Pass!}{headology-barrier}{20}{
	\noprereq
}{
	You may erect barriers inside people's heads, allowing them to project them into reality.
	If you convince someone that they cannot pass some barrier, they become unable to.
	Even if they are thrown bodily against the barrier, they will bounce off it.
	This does not prevent them throwing stones, poking a stick, using magic across the barrier, or the like.
	
	The barrier can be of any shape or nature that you can convince the target of.
	For example, you might draw a line in the sand, convince them that they cannot enter a house, or tell them that they cannot touch you.
}

\feat{Fake Sympathy}{headology-sympathetic-magic}{25}{
	\skillref[1]{sympathetic-magic},
	any feat giving a use for {\symlinks}
}{
	Although you know how to perform \discref{sympathetic-magic}, you've also figured out how to skimp on the magic and just use \discref{headology}.
	You may establish a fake {\symlink} just by convincing the target that you have established one.
	They need not understand the actual mechanisms of \discref{sympathetic-magic}---in fact, it's probably better if they don't---they just need to know that by affecting the {\symbol}, you can affect them.
	Establishing this fake link does not require the usual Test, only any Tests to convince the target.
	It does not count towards your maximum number of {\symlinks}
	It lasts as long as the target continues to believe it does---as such, it is not subject to {\stress}.
	
	You may transmit any effects along this fake link that you could along a normal {\symlink}---anything you possess the appropriate \discref{sympathetic-magic} feat for.
	However, you may only do so by showing the target what you are doing, and even explaining it if necessary.
	For example, \featref{sympathetic-speak} is useless: if the target cannot hear the sounds anyway, they don't know what to expect, and receive nothing.
	
	This {\symlink} doesn't actually exist in any sense, so you cannot modify it in any way you could modify a normal {\symlink}.
	However, nor can anybody else, and it is not impeded by anything that would impede a normal {\symlink}, unless the target is aware of and believes in such impedance.
}

\feat{Placebo}{headology-brewing}{15}{
	\noprereq
}{
	Often, the promise of a cure is more important than the cure itself.
	You can save a lot of time brewing this way, if you just talk to people.
	
	If you know how to make a brew, and have a mixture of approximately the right size, consistency, and colour, you might be able to use that instead.
	If you can convince someone that what they're taking will have the effect of that brew, then it acts as that brew for them.
	This works not only with brews that you have a feat to make, but also the same minor remedies that you might otherwise make with a \skillref{brewing} Test.
	However, if a brew requires a feat to make, and you don't have that feat, this won't work.
	
	This only works if they are convinced at the time they take the brew; it can't work retroactively.
	As such, it's not all that much use for poisoning people.
}

\feat{Retroactive Placebo}{headology-brewing-2}{15}{
	\featref{headology-brewing}
}{
	If you try to convince someone that you've poisoned their wine, they're hardly likely to drink it.
	But if you convince someone that you'd poisoned the wine they've just drunk, they might well drop dead.
	
	You may use \featref{headology-brewing} even if you convince someone \emph{after} they take the brew.
	Bear in mind that, obviously, they're unlikely to believe you if there's no way you could have touched the mixture they drank.
	
	The time taken for the brew to kick in is counted from when they took it, not when you convince them.
	As such, the effect will often kick in immediately after you convince them.
	It doesn't matter if this means it kicks in late, as long as they can believe they've been resisting it, or it's rather slow-acting.
	Convincing someone that yesterday's poison is only now affecting them might be difficult, though.
	
	This also functions for \featref{headology-brewing-antidote} and \featref{headology-brewing-antidote-2}; you can convince people that a mixture was an {\antidote} after they take it.
}

\feat{Poison is in the Mind}{headology-brewing-antidote}{10}{
	\featref{headology-brewing}
}{
	Sometimes it's useful to end the effect of a potion without giving away that it was fake all along.
	In this case, you can give someone an {\antidote}.
	Just as fake as the original, of course.
	
	If you can convince someone that a mixture they take is an {\antidote}, it functions as one.
	It counteracts whichever mixtures you convince them that it will.
	However, it can \emph{only} counteract mixtures that were applied using \featref{headology-brewing} in the first place; it is not effective against any real brew.
}

\feat{Placebo Panacea}{headology-brewing-antidote-2}{20}{
	\featref{headology-brewing-antidote}
}{
	You can counteract even the deadliest poisons with plain water, if your powers of persuasion are up to scratch.
	When you use \featref{headology-brewing-antidote}, the fake {\antidote} may counteract \emph{any} brew, even a real one.
}


Today was not going well for Linda Greene.
It had started out alright.
A brisk walk in the frosty air at sunrise, a quick trip up to the castle to drop off a couple of poultices for the servants there.
The cook had even given her a big side of braised ham for her help.
But things had gone downhill pretty quickly when the warty old crone had strolled in and started turning people into frogs.

Now here she was, speeding over the mountaintops, hair and cloak whipped back by the frozen wind, and a crazy old hag hot on her tail.
The crone had a wicked-looking knife clutched between her teeth.
Been screaming that she was going to gut the king with it, or somthing unpleasant like that.
Well, the king was safe for now, even if he was croaking rather indignantly.
Linda had stuffed him down her blouse so his now-cold-blooded majesty wouldn't freeze in the mountain air.
It did explain the indignancy, perhaps, but Linda had bigger problems on her mind.
The hag was gaining on her.

Linda leant right forwards and threw the stick into a dive.
She picked up speed as she shed altitude, but the hag quickly followed suit.
Her feet brushed the snow as she skimmed down the far side of the mountain

\dots %TODO: Write the middle of the chase.

She wasn't the best flyer in the world, she knew that.
She wasn't even the best in the kingdom; young Wren up Salwich way could fly circles around her.
And this hag, too, was clearly better than her.
But the problem with being the best was that there were some things you didn't actually get to practice that much.
Some things that Linda, who was the first to admit that her reach often exceeded her grasp, got to practice all too often.

%TODO: Stall the sticks, and have Linda recover. The hag falls into a snowdrift.


\chapter{Broomcraft}
\chaplabel{broomcraft}

A broom is primarily a witch's method of getting from A to B: from village to village, out into a distant forest, or all the way up the city.
It's not the easiest mode of transport, and it can be quite terrifying at first, but a witch can pick up the rudiments in a week or two's practice.
This is as far as most witches go.
But some, with enough practice, skill and flamboyance, can turn it into a real art.

\section{Laws of Aviation}

For an unpracticed witch, there are a lot limitations to broomstick flying.
After all, she is sitting on a thin stick floating hundreds of metres in the air.
First and foremost, it is easiest to balance on a broomstick if one sits side-saddle, and this is all an unpracticed witch is capable of.
This does, however, make it a lot harder to turn, and to fly at high speeds.
Barrel rolls are right out.

\subsection{Taking Off}

Getting the broomstick off the ground in the first place is no easy task.
A broom needs a running start before the magic will catch, and even then it isn't consistent.
The witch must hold the broomstick level as she runs along the ground, then jump on it quickly when it starts.

Attempting to start a broom requires an Action and a 15 metre run-up.
A character must move this distance in a straight line on one Turn, and may Dash as part of the broom-starting action if necessary. %TODO: Ensure Dashing is in the rules.
They must also succeed on a TN 12 Grace + Flying Test or the broom fails to start.
As normal, the Test is not required if there is no time pressure, as the witch may run up and down as many times as necessary until the broom starts.

The Test to start a broom may be more difficult in adverse conditions; the following table provides suggestions for the TN of such Tests.
It is possible to achieve the necessary run-up through falling, although such a thing is \emph{very} difficult and the consequences for failure are obviously drastic.

\begin{simpletable}{rX}
	\toprule
	TN & Conditions\\
	\midrule
	12 & Nominal.\\
	15 & Blowing a gale.\\
	18 & In a bog.\\
	21 & While falling.\\
	\bottomrule
\end{simpletable}

\subsection{Lift}

A broomstick can carry one witch, and about as much equipment as she could easily walk around with on the ground.
It can also carry a familiar, as long as it's of reasonable size.
A cat is fine, a beagle is borderline, a wolfhound is right out.

A little bit of extra weight, or something inconveniently large, makes the broomstick unwieldy.
Tests to take off or perform manoeuvres are more difficult, and the broom's maximum speed may be reduced.
A lot of extra weight, such as a passenger, makes proper flight impossible.
The broomstick cannot take off, and cannot remain in level flight.
It might still be possible to bring it down and land safely, with an appropriate Test.



\section{Feats}

\feat{Ride Astride}{astride}{
	None
}{
	By sitting astride your broom, instead of side-saddle, you can go faster and turn more sharply without falling off.
	It's harder to balance, but you've got the hang of it now.
	
	%TODO
}

\feat{Chocks Away}{astride}{
	None
}{
	There's a simple knack to starting a broom, and you've got it down pat now.
	You don't need a Test to start a broom under normal conditions (although you still need the run-up), and the TN of any Test to start the broom under difficult conditions is reduced by 3.
}

\feat{Broom Whisperer}{untrained-broom}{
	\skillref[1]{flying}
}{
	You've got the knack of flying for yourself now, and don't need a broom to be trained to fly it.
	You can even train a broom this way, although without one of its own to learn from the process takes about 24 hours of flight time.
}


\chapter{Sympathetic Magic}
\disclabel{sympathetic-magic}{Sympathist}{Sympathists}

\section{Sympathetic Links \& Symbols}
\seclabel{sympathetic-links}

Central to the practice of \discref{sympathetic-magic} is the creation and manipulation of {\symbols}.
A {\symbol} is a representation of a creature or object, and by affecting the {\symbol} a witch may cause a mirroring effect upon the target.
Not every \materialref{poppet} or \materialref{effigy} is automatically a {\symbol}.
It must by magically bound to the target by a {\symlink}.

A novice witch can only maintain one {\symlink} at a time.
It's not that maintaining one is particularly arduous; once established, a {\symlink} remains in place indefinitely, as long as the target is not resisting it.
Rather, two {\symlinks} tend to tangle themselves up, like pieces of string left together in a drawer.
Soon enough, both are totally useless and they have to be cut to separate them.

A {\symlink} by itself does nothing, but a \practitioner{sympathetic-magic} soon learns to use it to transmit numerous things: sensations, physical effects and more.
A {\symlink} doesn't always transmit everything it is capable of transmitting: only what the witch who established it wants it to.
The witch can change what the link transmits at any point she chooses, regardless of proximity to the {\symbol} or the target.
However, she has no particular sense of what is being transmitted by the link, and must watch the {\symbol} or the target if she wants to know.
As such, leaving {\symbols} lying around is a slightly dangerous proposition.

\subsection{Establishing a Sympathetic Link}

The simplest method for establishing a {\symlink} actually relies upon a trick of \discref{headology}.
The target must be \emph{expecting} the link, allowing the witch the opportunity to fasten it in place.
As such, at first, the witch can only establish {\symlinks} with people as the target, using a \materialref{poppet} or \materialref{effigy} as the {\symbol}.

Establishing the link requires an {\action}.
The target must see the {\symbol}, and the witch must declare to the target that she is binding them together.
Many witches adopt a standard incantation for this, often some piece of mumbo jumbo that suits the mystique they wish to cultivate.
The important thing is that the target understands the intent---that they are \emph{convinced} by it is not so important as in ``true'' \discref{headology}.

The target's expectation provides a hook that the witch may fasten the {\symlink} to.
If the target welcomes the {\symlink}, this is easy---it is established automatically and remains in place indefinitely.
Otherwise, establishing the link requires a \testtype{wit}{sympathetic-magic} Test {\opposed} by the target's \testtype{will}{sympathetic-magic}.

\subsection{Severing a Sympathetic Link}
\seclabel{break-sympathetic-link}

A witch can sever any {\symlink} she has established as an {\action}, or as part of establishing any new {\symlink}.
Additionally, a {\symlink} is severed if the {\symbol} or target are destroyed, or die.

Otherwise, a {\symlink} to an object, a willing creature, or a creature who is unaware they are the target of a {\symlink} at all, will persist indefinitely.
However, a {\symlink} to a creature that knows it is the target of a link, and does not wish to be, will be dislodged over time.
It automatically breaks after a minute, but can be broken sooner if it is {\stressed}.
This applies even if the creature previously accepted the link, but now wants rid of it.

Some uses of a {\symlink} will cause it considerable {\stress}, giving an unwilling creature another change to dislodge the link.
In this case, repeat the Test used to establish a link---your \testtype{wit}{sympathetic-magic} {\opposed} by the target's \testtype{will}{sympathetic-magic}.
If the target wins the Test, the {\symlink} is broken.
Actions that {\stress} a link will say so in their relevant feats.

\section{Feats}

\feat{Stable Sympathy}{symlink-stable}{20}{
	\skillref[1]{sympathetic-magic}
}{
	An unwilling target will soon throw off a {\symlink}, but you've learned to stabilise your links against this, leaving them fastened strong in the face of adversity.
	However, this requires some preparation.
	
	By using an \materialref{effigy} in the likeness of the target as the {\symbol}, the {\symlink} does not expire over time, even when resisted.
	However, this does not allow it to resist {\stress}.
	This still requires the usual Test to establish the link in the first place.
}

\feat{Taglock Binding}{symlink-taglock}{20}{
	\skillref[1]{sympathetic-magic}
}{
	Normally, the hook to fasten a {\symlink} in place is provided by the target's \emph{expectation} of a link.
	This is the simplest and strongest way, but not the only one.
	
	You can establish {\symlinks} to creatures, using a \materialref{taglock}, and a \materialref{poppet} or \materialref{effigy} as the {\symbol}.
	Establishing the {\symlink} uses an {\action}, while touching the \materialref{taglock} and the {\symbol}.
	However, {\symlinks} fastened in this way are weaker.
	Anything that would {\stress} the link---or destroy it, as with \featref{sympathetic-damage}---simply snaps the {\symlink} without taking effect.
	Remember, however, that you can always choose not to try and transmit anything that would {\stress}, and hence break, the link.
}

\feat{Twin Links}{symlink-extra}{20}{
	\skillref[1]{sympathetic-magic}
}{
	You may maintain two {\symlinks} simultaneously.
}

\feat{Sympathetic Jerk}{sympathetic-puppet}{15}{
	None
}{
	An expert \practitioner{sympathetic-magic} can make their target dance on the puppet strings of their {\symlink}.
	You aren't there yet, but you've taken the first step.
	
	You cannot control your target's movements, but you---or someone else holding the {\symbol}---can \emph{disrupt} them by jerking the {\symbolpossessive} limb the wrong way at the opportune time.
	If the target is just walking and talking normally, this doesn't do more than faintly disturb them.
	But if they are performing something highly physical or precise---running, jumping, aiming a weapon, or sewing, for example---it can severely disrupt them.
	Jerking the correct limb at the correct time requires knowing what the target is doing, or at least being able to take a very good guess.
	Normally, this means being able to see them.
	
	Typically, you can use this by taking the \actionref{ready} {\action} in order to disrupt the target's next {\action}, while holding their {\symbol}.
	Common disruptions include making them miss an \actionref{attack}, or making them trip and fall prone when jumping or taking the \actionref{dash} {\action}.
	The GM ultimately decides the result of any disruption.
	Disruptions like those listed above do not require a Test, but if the outcome is in doubt, the GM may call for an {\opposedtest}.
	This typically uses \testtype{wit}{sympathetic-magic} for the witch, and might use something like \testtype{grace}{athletics} or \testtype{grace}{weaponry} for the target.
}

\feat{Sympathetic Puppet}{sympathetic-puppet-2}{25}{
	\skillref[1]{sympathetic-magic},
	\featref{sympathetic-puppet}
}{
	You can control someone's actions through a {\symlink}.
	Only intermittently, and not precisely, but that doesn't make it much less terrifying.
	
	As an {\action}, someone can puppet a target by manipulating its linked {\symbolpossessive} limbs.
	The manipulator takes a physical {\action} on behalf of the target, which may be moving up to its \statref{speed} using the \actionref{dash} {\action}.
	This also deprives the target of their {\action} on their next {\turn}---unless that {\action} would be purely non-physical---although they may still make their usual movement.
	
	Using this {\stresses} the {\symlink}.
	
	Puppetry is quite difficult to do precisely.
	You can control limbs, and you can even open and close the hands and jaw, if the {\symbol} has the appropriate anatomy to manipulate.
	But speaking is impossible, and any work with the fingers requires you to manipulate the {\symbolpossessive} fingers with the same precision---a difficult proposition using your own bulky fingers.
	
	The manipulator suffers a \negative{6} penalty to any \emph{physical} Tests they must make on the target's behalf.
	These Tests typically use \attref{grace}, to finely manipulate the {\symbol}, and whichever skill would be used for performing the {\action} normally.
	However, \skillrefspeciality{performance}{Puppeteer} can be used in place of the normal skill.
}

\feat{Sympathetic Destruction}{sympathetic-damage}{20}{
	None
}{
	When a {\symbol} is destroyed, you can send its death throes lashing along the {\symlink}, tearing at its target.
	Roll a {\damagetest} against the target, using \testtype{wit}{sympathetic-magic}.
	This works against objects, as well as creatures.
	
	Tearing a {\symbol} apart typically requires an {\action}, though you might find a faster way to destroy it.
	The destruction of the {\symbol} obviously terminates the {\symlink}.
}

\feat{Sympathetic Stabbing}{sympathetic-damage-2}{15}{
	\skillref[1]{sympathetic-magic},
	\featref{sympathetic-damage}
}{
	You no longer need to destroy a {\symbol} outright to wound the target.
	When a {\symbol} is significantly damaged in some way---sticking a pin in it is traditional---you may roll a {\damagetest} against the target, using \testtype{wit}{sympathetic-magic}.
	This works against objects, as well as creatures.
	Using this effect {\stresses} the {\symlink}.
	
	Attacking a {\symbol} to activate this should typically require an {\action}, though you might find a faster way to damage it.
}

\feat{Sympathetic Buoyancy}{sympathetic-weight}{10}{
	None
}{
	The mass of a {\symbol} affects the mass of its target: a stone or iron \materialref{poppet} will make a person heavier while a wood or cloth one will make them lighter.
	Not hugely so---no more than about \SI{25}{\percent}---but enough to make a person easily float or sink, and to aid or hinder jumping and climbing.
	%TODO: Mechanical effects on jumping, etc.
	
	This effect can be used on objects as well as creatures, making them easier or harder to lift and carry.
}

\feat{Sympathetic Sleep}{sympathetic-sleep}{10}{
	None
}{
	A {\symbol} can rest in place of its target, allowing the target to work through most of the night.
	The rest, the {\symbol} needs to be tucked into a small bed, with soft bedding, a pillow, and sheets.
	It needs to be in a quiet, dim location, and generally to be in conditions where a person could easily sleep.
	The {\symbol} cannot be used for any other \discref{sympathetic-magic} while it is resting.
	
	As long as the {\symbol} rests for at least 8 hours each day, the target can get by on only 1 hour of sleep each day without any ill effects.
	However, the target does not recover from {\damage} and {\exhaustion} as a result of this rest.
}

\feat{Sympathetic Insomnia}{sympathetic-sleep-deprive}{15}{
	\skillref[1]{sympathetic-magic},
	\featref{sympathetic-sleep},
	\featref{symlink-stable}
}{
	By keeping a {\symbol} awake, you can deprive its target of restful sleep.
	If the {\symbol} is subjected to loud noises, bright lights, stony bedding, or other significant discomforts while the target sleeps, the sleep will be fitful and restless.
	The sleep does not help them recover from {\damage} or {\exhaustion} (although they may still benefit from a day of rest).
	If this goes on for several nights, they may begin suffering {\exhaustion} due to sleep deprivation.
}

\feat{Sympathetic Narcolepsy}{sympathetic-sleep-cause}{15}{
	\skillref[1]{sympathetic-magic},
	\featref{sympathetic-sleep},
	\featref{symlink-stable}
}{
	\featref{sympathetic-sleep} lets a {\symbol} sleep instead of the target.
	You've reversed this, and may instead let the {\symbol} send the target to sleep.
	
	If you tuck a {\symbol} in, as you would for \featref{sympathetic-sleep}, then you may cause it to bring on tiredness in the target.
	This does not kick in for a minute, while the {\symbol} falls asleep.
	After this minute, make a \testtype{wit}{sympathetic-magic} {\opposed} by the target's \attref{will} Test.
	If you succeed, the target falls into a deep sleep.
	They cannot be roused for 8 hours (so long as the {\symbol} continues to sleep), but benefit as though they were sleeping naturally.
	
	Succeed or fail, this will not work on the same target again for another 24 hours.
	They've either slept of the tiredness, or fought through it.
}

\feat{Sympathetic Warmth}{sympathetic-heat}{10}{
	None
}{
	The temperature of a {\symbol} affects the temperature of its target.
	Uncomfortable temperatures remain comfortable as long as the {\symbol} is at a comfortable temperature, and comfortable temperatures become uncomfortable if the {\symbol} is warmed or chilled.
	This effect cannot create dangerous temperatures---hot enough to cause heat stroke or cold enough to cause hypothermia---but can counteract them if the {\symbol} is inversely heated or cooled.
	Temperatures sufficiently extreme to cause {\damage}, such as fire or anything that would directly freeze the flesh, are outside the reach of this effect.
}

\feat{Sympathetic Combustion}{sympathetic-fire}{15}{
	\skillref[1]{sympathetic-magic},
	\featref{sympathetic-damage},
	\featref{sympathetic-heat}
}{
	When you burn someone in effigy, they really burn.
	If a {\symbol} is destroyed by fire, and you use \featref{sympathetic-damage}, the target also catches fire.
	A person ignited this way begins at \dice{3} {\fire}.
}

\feat{Sympathetic Malady}{sympathetic-attribute-reduce}{10}{
	None
}{
	You may afflict a target with various maladies by though a {\symlink}.
	You may reduce one of their attributes by 1 point by causing some appropriate affliction to the {\symbol}.
	For instance, you could reduce the target's \attref{grace} by binding their {\symbolpossessive} arms and legs, their \attref{heed} by blindfolding their {\symbol}, or their \attref{charm} by giving their {\symbol} some obvious disfigurement.
	A target may only be subject to one of these effects at a time, per witch who is affecting them.
}

\feat{Sympathetic Communication}{sympathetic-speak}{20}{
	\skillref[1]{sympathetic-magic}
}{
	You can send sounds along a {\symlink}, like a string telephone.
	A creature can hear sounds that originate near its {\symbol}, as long as it is conscious and not deafened.
	It can avoid this by plugging its ears, although this obviously leaves it deaf to its own surroundings as well.
	The {\symbol} has a very short range of hearing; speaking through it essentially requires picking it up and holding it near the mouth.
}

\feat{Sympathetic Pestering}{sympathetic-speak-2}{15}{
	\skillref[1]{sympathetic-magic},
	\featref{sympathetic-speak},
	\featref{sympathetic-sleep-deprive}
}{
	When sending sounds along a {\symlink} using \featref{sympathetic-speak}, you may send them directly into the target's mind, bypassing its ears.
	The target hears them even if it is deaf, or has its ears plugged.
	You may even be able to wake the target up with loud enough sounds, if it is asleep.
}

\feat{Sympathetic Ventriloquism}{sympathetic-puppet-speak}{10}{
	\skillref[2]{sympathetic-magic},
	\featref{sympathetic-puppet-2},
	\featref{sympathetic-speak-2}
}{
	Puppeteering the vocal cords requires a lot more precision than swinging the limbs around.
	However, it doesn't take as much force---using this effect does not {\stress} the {\symlink}.
	
	While a {\symbolpossessive} jaw is flapped around, the target will speak anything said into the {\symbolpossessive} ear.
	This obviously requires that the {\symbol} possesses an appropriate jaw.
	The target speaks in its own voice, so an animal cannot be made to speak particularly well.
	
	This does not prevent the target from talking whenever this is not being actively used, so you have to force the target to talk constantly if you want to prevent it getting a word in edgeways.
}

\feat{Sympathetic Knot}{sympathetic-knot}{15}{
	\skillref[1]{sympathetic-magic},
	\featref{symlink-extra}
}{
	Normally when {\symlinks} get tangled, it renders both useless.
	However, if you knot them together intentionally, carefully, you can take advantage of it.
	
	You can knot together two or more {\symlinks} of the same kind---to creatures or to objects---as an {\action}.
	This requires that you are touching at least one end of each {\symlink} to be involved in the knot.
	For example, knotting together two {\symlinks} from \materialrefplural{poppet} to people requires you to be touching both \materialrefplural{poppet}, both people, or the \materialref{poppet} from one link and the person from the other.
	
	You can also undo a knot as an {\action}, but again you must be touching at least one end of every {\symlink} in the knot---you can only undo knots in their entirety, and not remove just one {\symlink}.
	Similarly, severing any {\symlink} in the knot severs all of them.
	You can only knot or unknot your own {\symlinks}.
	
	While two {\symlinks} are knotted, anything transmitted by any {\symbol} in the knot affects every target in the knot.
	You may still control what each {\symbol} transmits, but it always transmits to all targets.
}

\feat{Unbarred Sympathy}{sympathetic-ignore-barrier}{15}{
	\skillref[2]{sympathetic-magic}
}{
	Most barriers that interfere with magical effects don't break a {\symlink}, they just prevent it transmitting.
	But a finger on a string doesn't stop it from vibrating; it just restricts it.
	You can circumvent it if you know how.
	
	Barriers created by a \featref{circle-contain}, \featref{circle-exclude}, \featref{circle-contain-exclude}, or the like no longer impede transmission by your {\symlinks}.
	You still can't establish a {\symlink} that would be blocked by such a barrier, however.
}

\feat{Threading the Barrier}{sympathetic-ignore-barrier-2}{10}{
	\skillref[3]{sympathetic-magic},
	\featref{sympathetic-ignore-barrier}
}{
	If air can pass a magical barrier, why not a {\symlink}.
	It's just like threading a needle: it takes a bit of dexterity and your eyesight better be good, but it's hardly \emph{impossible}.
	
	You may establish a {\symlink} even through the barrier created by a \featref{circle-contain}, \featref{circle-exclude}, \featref{circle-contain-exclude}, or the like.
	%You can't always do it first time, however, and the GM may require a Test if you are in a hurry.
}


\end{document}
