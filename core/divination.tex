\chapter{Divination}
\disclabel{divination}{Seer}{Seers}

\section{Feats}

\feat{Perfect Positioning}{divination-self}{10}{
	None
}{
	While most divination deals in seeing the future, the past, or distant locations, there is one very useful trick that simply allows you to see the \emph{here} and \emph{now}.
	
	You always know where and when you are, and which way you are facing.
	You can't get lost---you always know which direction any place you've been before lies in, although not necessarily the best way to get there.
	You know what time it is, and how long you've been sleeping whenever you awake.
	
	However, this does not function in the {\mentalrealm}.
}

\feat{Taglock Identification}{divination-taglock-identify}{10}{
	None
}{
	You can touch a \materialref{taglock} and detect who it originates from.
	If you have met the target, you can identify them infallibly.
	
	If you have never met the target, you must make a \testtype{heed}{divination} Test, with higher results giving more information about the target.
	You can only get general information about the target this way, such as height, build, sex, appearance, and occupation.
	You can't get any information about their location, or even whether they are still alive.
}

\feat{Taglock Tracing}{divination-taglock-location}{10}{
	\featref{divination-taglock-identify},
	\featref{divination-self}
}{
	When you touch a \materialref{taglock}, you can find the person or creature it originates from.
	You know their location.
	This comes in the form of distance and direction---you can navigate to them easily, but that they are, for example, in the castle requires you to know the location of the castle, and figure out that's the location they're in.
	If they are moving, you can feel their position changing as they do.
}

\feat{Scrying}{scrying}{10}{
	None
}{
	You can use a \materialref{crystal-ball} to scry.
	You can scry on any location within 1 kilometre, estimating distance and direction, or visualising a place you've seen.
	This essentially forms a sensor, hanging in the air at the target location---although the ``sensor'' is invisible, intangible, and doesn't actually exist in any real sense.
	While scrying, you can move the sensor at a strolling pace, in all three dimensions and unimpeded by walls.
	
	It takes about a minute of concentration on the \materialref{crystal-ball} for the image to form.
	Anyone looking into the ball sees out from the sensor, and can look around by looking into the ball at different angles.
	However, you cannot see anything more than a dozen metres from the sensor; the image becomes too distorted.
	You only get vision; no other senses.
	
	Scrying requires continuous concentration---you must remain within a metre of the \materialref{crystal-ball}, and can't manage any more significant actions than talking or idly knitting.
	You may only maintain one scrying sensor at a time.
}

\feat{Mirror Scrying}{scrying-mirror}{15}{
	\skillref[1]{divination},
	\featref{scrying}
}{
	You may use a mirror instead of a \materialref{crystal-ball} for \featref{scrying}.
	Being non-spherical, the angle of view is restricted.
	You have to turn the mirror you are scrying through around in order to see the other way.
}

\feat{Eyeball Scrying}{scrying-no-material}{15}{
	\skillref[2]{divination},
	\featref{scrying-mirror}
}{
	By a sufficient stretch of the imagination, an eyeball is nearly a \materialref{crystal-ball}.
	It takes a little trick to look out from the inside of them, rather than in from the outside, but you've made it work.
	
	You can use \featref{scrying} without a \materialref{crystal-ball} or mirror, by using your eyeballs.
	You are blinded to your own surroundings while doing so.
	Additionally, this works only for you.
	Other people cannot see in on your scrying.
}

\feat{Taglock Scrying}{scrying-person}{15}{
	\skillref[1]{divination},
	\featref{scrying},
	\featref{divination-taglock-location}
}{
	You may scry on a particular person, finding them using a \materialref{taglock} you hold as you begin \featref{scrying}.
	The scrying sensor appears within 2 metres of them, regardless of whether they lie within your usual scrying range.
	You can move it around as normal within this 2 metre radius of them, and it is dragged along behind them no matter how fast they move.
	You can only move the sensor further away from them if it is within your normal scrying range, but it stops following them if you do so.
}

\feat{Danger Sense}{divination-initiative}{20}{
	\skillref[1]{divination}
}{
	You can't quite see the future, as such, but you can tell when \emph{something} is about to happen.
	You might not know what it is, but it won't catch you entirely unawares.
	You may use your \skillref{divination} skill in place of any other skill when rolling {\initiative}, and may even use it when the {\initiative} Test would otherwise use no skill at all.
}

\feat{Perfect Prediction}{divination-dodge}{20}{
	\skillref[1]{divination},
	\featref{divination-initiative}
}{
	With your eyes closed and all your concentration turned to it, you can enhance your premonitions of danger with perfect clarity.
	Activating this effect requires an {\action}, and lasts until the beginning of your next {\turn}, or until you open your eyes.
	For the duration, you automatically evade any \actionref{attack} or other harmful effect that could reasonably be evaded, unless it hits you with a critical success.
	
	This only allows you to sense anything which may harm you, restrain you, or the like.
	With your eyes closed, you likely unaware of many other things.
	
	Furthermore, maintaining this for long periods is tiring.
	The GM may call for Tests to avoid {\exhaustion} after a minute or more of use.
}
