\chapter{Divination}
\disclabel{divination}{Seer}{Seers}

\section{Foretelling}
\seclabel{foretelling}

For many witches, telling the future is the most enticing part of \discref{divination}.
It is, however, the trickiest.

In truth, nobody has ever managed to see their own future.
To know one's own fate, but to do \emph{nothing} to avoid it, is a level of self-discipline beyond any \practitioner{divination} the world has yet seen.
And without that self-discipline, the future is simply inscrutable---it seems almost to be defending itself against change.

But witches wouldn't be witches if they hadn't found a way to cheat.
You can't see your own future, but you can see the next closest thing.
So foretelling, such as it is practiced, is a matter of forking the stream of time, of tossing a rock into the flow and sending it thundering down two different channels.
There is a moment in every foretelling ritual, or spell, where the \practitioner{divination} goes two ways.
On one branch, she completes the ritual.
On the other, she stops abruptly halfway through, stands up, and goes about her life none the wiser.
It is the world of the second \practitioner{divination} that the first \practitioner{divination} sees.

Most \practitioners{divination} treat this as a metaphor, of course.
They've never \emph{actually} abandoned the ritual halfway through, no matter how many times they've seen themselves do it.
But some do wonder if those copies of themselves exist out there somewhere, even the ones they've seen die.

\subsection{The World Foretold}

All in all, a \practitioner{divination} performing foretelling sees visions of the world as it would turn out if she did not perform this foretelling.
So long as she does not interfere, she may assume that things will proceed, for the most part, the same.
But, as she changes more things, the worlds will diverge further, until her predictions become useless.

Furthermore, it is worth noting that sitting still in a dark room, doing nothing, doesn't mean a \practitioner{divination} is not interfering.
In the world where she performed no divination, the one she has peered in on, she likely acted, going out and doing whatever she normally did.
By sitting still in this world, she leaves out the results of those actions, probably rendering her predictions inaccurate quite quickly.

As such, the best way to keep predictions accurate for as long as possible is to act as though she never made them in the first place, going about as though she were blind to the future.
It will never work indefinitely, but she might be able to keep it up for a while.

\subsection{Foretelling \& the Metagame}

Telling the future is tricky not only for a witch, in character, but also for the GM running the game.
Giving good visions requires the GM to predict how events would play out, assuming the \practitioner{divination} had no foreknowledge.
This requires more planning than some GMs might want to do.

Furthermore, it often requires predicting how the player's characters would act, which can cause problems.
The GM is free to ask the players themselves, if that won't give away any plot details that the vision itself won't.
It can even be fun to play out the world where the \practitioner{divination} did not perform her foretelling, before jumping back to the point of the foretelling with the \practitionerpossessive{divination} new knowledge and playing again from there.
That can grow tedious, of course---it should be used sparingly, and with the agreement of everyone at the table.

Most importantly, both the players and the GM should be understanding that telling the future is tricky business.
Everyone should be willing to forgive mistakes, and to try and resolve everything coherently.
And if everything is just too much, the GM can request that players simply avoid the more troublesome feats.
Even then, feats such as \featref{divination-initiative} and \featref{divination-dodge} allow for witches who can see the future without requiring the GM to invent visions.

\subsection{Foretelling as a Plot Device}

Prophecies and visions are an ancient staple of myths and legends, as well as contemporary literature.
GMs should feel free to use foretelling as a device in creating plots for the game.
This may use the visions provided by a player character's feats, other omens or visions imparted upon a \practitioner{divination}, or a non-player \practitioner{divination}.

One aspect to play with as part of a plot is the assumption that no \practitioner{divination} can see their own future.
Just because no \practitioner{divination} has had the self-discipline to manage it yet, doesn't mean that it is impossible.
Any plot violating this assumption must be very carefully executed, to avoid taking agency away from the players, but can be very effective if it is done well.
Just remember that witches are notorious cheats---even predestined events are only fixed to the extent that they've been seen, and can always be staged.

\section{Feats}

\feat{Darkvision}{darkvision}{15}{
	None
}{
	You no longer need light, so fickle and often absent, to see.
	You can see perfectly in darkness, as though everything were lit by full daylight.
	This works even if you are standing in light and looking into darkness.
}

\feat{Unclouded Vision}{xray-vision}{10}{
	\skillref[1]{divination},
	\featref{darkvision}
}{
	Translucent substances, such as fog, murky water, or sandstorms, do not obscure or distort your vision.
	Intermediate cases, such as a leafy canopy or dense cobwebs, can be see through with a \testtype{heed}{divination} Test.
}

\feat{Perfect Positioning}{divination-self}{10}{
	None
}{
	While most \discref{divination} deals in seeing the future, the past, or distant locations, there is one very useful trick that simply allows you to see the \emph{here} and \emph{now}.
	
	You always know where and when you are, and which way you are facing.
	You can't get lost---you always know which direction any place you've been before lies in, although not necessarily the best way to get there.
	You know what time it is, and how long you've been sleeping whenever you awake.
	
	However, this does not function in the {\mentalrealm}.
}

\feat{Taglock Identification}{divination-taglock-identify}{10}{
	None
}{
	You can touch a \materialref{taglock} and detect who it originates from.
	If you have met the target, you can identify them infallibly.
	
	If you have never met the target, you must make a \testtype{heed}{divination} Test, with higher results giving more information about the target.
	You can only get general information about the target this way, such as height, build, sex, appearance, and occupation.
	You can't get any information about their location, or even whether they are still alive.
}

\feat{Taglock Tracing}{divination-taglock-location}{10}{
	\featref{divination-taglock-identify},
	\featref{divination-self}
}{
	When you touch a \materialref{taglock}, you can find the person or creature it originates from.
	You know their location.
	This comes in the form of distance and direction---you can navigate to them easily, but that they are, for example, in the castle requires you to know the location of the castle, and figure out that's the location they're in.
	If they are moving, you can feel their position changing as they do.
}

\feat{Scrying}{scrying}{10}{
	None
}{
	You can use a \materialref{crystal-ball} to scry.
	You can scry on any location within 1 kilometre, but must be able to estimate distance and direction the place you are scrying.
	This is easy enough for any place you've taken a simple route to or from, but it might be hard to scry on the middle of a maze using this.
	\featref{divination-self} allows you to perfectly place any location you have been, and \featref{divination-taglock-location} lets you place a person.
	
	Scrying essentially forms a sensor, hanging in the air at the target location---although the ``sensor'' is invisible, intangible, and doesn't actually exist in any real sense.
	While scrying, you can move the sensor at a strolling pace, in all three dimensions and unimpeded by walls.
	It must stay within your 1 kilometre scrying radius.
	
	It takes about a minute of concentration on the \materialref{crystal-ball} for the image to form.
	Anyone looking into the ball sees out from the sensor, and can look around by looking into the ball at different angles.
	However, you cannot see anything more than a dozen metres from the sensor; the image becomes too distorted.
	You only get vision; no other senses.
	However, you still benefit from anything that enhances your vision, such as \featref{darkvision}.
	
	Scrying requires continuous concentration---you must remain within a metre of the \materialref{crystal-ball}, and can't manage any more significant actions than talking or idly knitting.
	You may only maintain one scrying sensor at a time.
}

\feat{Mirror Scrying}{scrying-mirror}{15}{
	\skillref[1]{divination},
	\featref{scrying}
}{
	You may use a mirror instead of a \materialref{crystal-ball} for \featref{scrying}.
	Being non-spherical, the angle of view is restricted.
	You have to turn the mirror you are scrying through around in order to see the other way.
}

\feat{Eyeball Scrying}{scrying-eyes}{15}{
	\skillref[2]{divination},
	\featref{scrying-mirror}
}{
	By a sufficient stretch of the imagination, an eyeball is nearly a \materialref{crystal-ball}.
	It takes a little trick to look out from the inside of them, rather than in from the outside, but you've made it work.
	
	You can use \featref{scrying} without a \materialref{crystal-ball} or mirror, by using your eyeballs.
	You can, and should, leave them in your head for this.
	This works only for you; other people cannot see in on your scrying.
	
	You are blinded to your own surroundings while using your eyes in this way.
	Additionally, unless you also have \featref{scrying-unattended}, you must keep most of your concentration on the scrying, and cannot do much more than stroll around a bit.
}

\feat{Taglock Scrying}{scrying-person}{15}{
	\skillref[1]{divination},
	\featref{scrying},
	\featref{divination-taglock-location}
}{
	You may scry on a particular person, finding them using a \materialref{taglock} you hold as you begin \featref{scrying}.
	The scrying sensor appears within 2 metres of them, regardless of whether they lie within your usual scrying range.
	You can move it around as normal within this 2 metre radius of them, and it is dragged along behind them no matter how fast they move.
	You can only move the sensor further away from them if it is within your normal scrying range, but it stops following them if you do so.
}

\feat{Bird's-Eye Scrying}{scrying-sight-range}{15}{
	\skillref[1]{divination},
	\featref{scrying}
}{
	You've learned to counteract the distortion caused through your scrying apparatus.
	When you are \featref{scrying}, your vision is no longer restricted to within a dozen metres of the sensor.
	You may see just as well as if you were at the location of the sensor yourself.
	
	With a sufficiently aerial view, you might even see for several kilometres, while keeping the sensor within 1 kilometre.
	However, you have no more than your usual ability to resolve detail at that range.
}

\feat{Far Scrying}{scrying-range}{20}{
	\skillref[2]{divination},
	\featref{scrying}
}{
	When you are \featref{scrying}, your range is no longer limited to 1 kilometre---it is now unlimited.
	Placing a location at long distances becomes increasingly inaccurate, making \featref{divination-self} even more useful.
}

\feat{Roaming Scrying}{scrying-speed}{10}{
	\skillref[1]{divination},
	\featref{scrying}
}{
	You've become adept at refocusing your scrying sensor, letting you move it far faster than the slow walk you could manage before.
	While \featref{scrying}, you can move the sensor fast---fast enough to keep up with someone running, or even, with a Test, someone on a broomstick.
	You can also use an {\action} to refocus the sensor to a different location entirely, following the usual rules for positioning a scrying sensor in the first place.
}

\feat{Fast Scrying}{scrying-start-speed}{20}{
	\skillref[2]{divination},
	\featref{scrying-speed}
}{
	You can focus a scrying sensor in an instant.
	You may begin \featref{scrying} as an {\action}, instead of requiring a minute.
}

\feat{Effortless Scrying}{scrying-start-speed-2}{20}{
	\skillref[3]{divination},
	\featref{scrying-start-speed}
}{
	Scrying has become as natural to you as blinking.
	You may begin \featref{scrying}---or refocus the sensor as per \featref{scrying-speed}---at any time, without using an {\action}.
}

\feat{Unattended Scrying}{scrying-unattended}{10}{
	\featref{scrying}
}{
	Once you've fired up a \materialref{crystal-ball}, you can leave it running while you go and do something else.
	You no longer need continuous concentration to maintain \featref{scrying}, and can even leave the image unattended, possibly watched by someone else.
	However, you still need to be near the \materialref{crystal-ball} to form the image, or to move the sensor.
	Forming the image requires concentration, but moving it is as easy as glancing around.
	If you have \featref{scrying-speed}, refocusing entirely is still an {\action}.
	
	You can still only maintain one instance of scrying at a time.
	The image dissolves if you begin a new instance of scrying, and you can also dispel it at any time you choose.
	
	You can use this with a mirror as well as a \materialref{crystal-ball} if you have \featref{scrying-mirror}, and with your own eyes if you have \featref{scrying-eyes}.
	In this latter case, other people still cannot look in on your scrying, but it allows you to act normally while scrying.
}

\feat{Multiple Scrying}{scrying-multiple}{10}{
	\skillref[1]{divination},
	\featref{scrying-unattended},
	\featref{scrying-mirror}
}{
	The is no longer any limit to the number of \featref{scrying} images you can maintain at a time.
	However, you still need one \materialref{crystal-ball} or mirror per image.
	If you have \featref{scrying-eyes}, one instance of scrying still requires both your eyeballs, and renders you blind to your surroundings.
}

\feat{One-Eyed Scrying}{scrying-multiple-eyes}{15}{
	\skillref[2]{divination},
	\featref{scrying-multiple},
	\featref{scrying-eyes}
}{
	\featref{scrying} with your eyeballs no longer requires both of them, but only one.
	As such, you are no longer blinded to your own surroundings while scrying with your eyes.
	Alternatively, you may scry on two different places at the same time, one with each eye, becoming blinded to your own surroundings as normal.
}

\feat{Eavesdropper's Scry}{scrying-sound}{20}{
	\featref{scrying}
}{
	You may hear sounds from a \featref{scrying} sensor.
	These sounds emanate from the \materialref{crystal-ball} or mirror used for scrying, to be heard by everyone around.
	However, if you are using \featref{scrying-eyes}, only you can hear them.
	You may disable and resume the transmission of sound at will, while you are near the scrying surface.
	
	You can only hear sounds originating in the same dozen-metre radius as you can see, unless you also have \featref{scrying-sight-range}.
}

\feat{Someone Watching}{gaze-detection}{10}{
	None
}{
	People have a fine sense of when they're being stared at, but you've honed yours to an art.
	You always know when you're being watched.
	You can get a sense of how much attention is being paid to you---whether it's just someone watching out of the corner of their eye, someone staring right at you, or the rapt attention of dozens---but you don't know \emph{who} is watching.
}

\feat{Eyes on You}{gaze-detection-2}{10}{
	\skillref[1]{divination},
	\featref{gaze-detection}
}{
	Feeling everybody's eyes on you all the time can be quite unnerving, but you've deemed it to be a price worth paying.
	You can always tell how many people (or animals) are watching you, how much attention each is paying, and where each is watching from.
	If someone is watching by \featref{scrying}, you can tell they are scrying, but you detect the location of the sensor rather than the person themself.
}

\feat{Hindsight}{hindsight}{10}{
	None
}{
	Seeing the past is far easier than seeing the future, although you don't have any chance of changing it.
	
	You may enter hindsight at any time, without an {\action}.
	You begin to see, hear, and otherwise sense in reverse, running backwards in time from the point you begin hindsight.
	You sense things from where you are currently standing, so, for example, you could go back to see what happened in a room before you entered it.
	
	You sense everything in reverse, but at normal speed.
	For example, if you want to see what happened ten minutes before you began hindsight, you must wait ten minutes to rewind to that point.
	Speech is hard to interpret in reverse, requiring a \attref{wit} Test for anything but the most simplest phrases.
	You may end your hindsight at any time, and your senses immediately return to the present.
	You may maintain hindsight for as long as you can stay awake, but it ends if you fall asleep or unconscious.
	
	You may act normally while using hindsight, however your cannot sense your present surroundings, only the past.
	However, you may still feel things that you touch in the present, and conversely cannot touch anything in the past.
	This makes it quite disturbing when someone in the past walks straight through where you are currently standing.
	Additionally, hindsight breaks down and ends if you move too far from where you began it; you can manage a couple of metres without trouble, or more with a Test.
	
	While using hindsight, you benefit to any enhancements to your senses, such as \featref{darkvision}.
	However, you can only see \featref{scrying} images that existed in the time and place you are looking at.
	This means that you simply cannot use it in conjunction with \featref{scrying-eyes}.
}

\feat{Rapid Rewind}{hindsight-speed}{15}{
	\skillref[1]{divination},
	\featref{hindsight}
}{
	Sometimes, you just can't arrive at the scene of the crime for quite a while after it happened, and don't have the time to waste sending your eyes that far back.
	Thankfully, you've learned to pick up the pace.
	
	When using \featref{hindsight}, you may accelerate your perception backwards in time, up to 20 times the speed.
	This lets you go backwards an hour in just three minutes, or most of a day in an hour.
	If you also have \featref{hindsight-foresight-bidirectional}, you can go up to this speed forwards or backwards, though never ahead of the present time.
}

\feat{Lightning Rewind}{hindsight-speed-2}{15}{
	\skillref[2]{divination},
	\featref{hindsight-speed}
}{
	You may accelerate your \featref{hindsight} to ludicrous speed.
	This follows the same rules as \featref{hindsight-speed}, except that you may perceive time at up to a million times the normal speed, looking backwards at more than ten days per second, or a century in less than an hour.
}

\feat{Foresight}{foresight}{20}{
	None
}{
	Seeing the future is more difficult than seeing the past, but you've got it figured out.
	The ritual to begin a vision requires 15 minutes of concentration, while remaining in one place.
	You select a time in the future---no more than a day ahead---to view.
	You must specify a time, for example, 30 minutes or 8 hours ahead; finding a specific event requires dead reckoning.
	
	The vision works much like \featref{hindsight}: you see and hear the area around where you currently stand, blind and deaf to the present time, and cannot move more than a couple of metres without disrupting the vision.
	However, it plays \emph{forward} from the time you selected.
	Furthermore, it is unstable; the vision only lasts for 1 minute before it breaks down and ends.
	
	The vision follows the normal rules for {\foretelling}; the world you see is the world as it would have been if you did not have this vision.
}

\feat{Roaming Visions}{hindsight-foresight-move}{15}{
	\featref{hindsight} or \featref{foresight}
}{
	Moving around---any distance---no longer disrupts your \featref{hindsight} or \featref{foresight}.
}

\feat{Time Scrying}{hindsight-foresight-scrying}{25}{
	\skillref[1]{divination},
	\featref{hindsight-foresight-move},
	\featref{scrying}
}{
	You may use \featref{hindsight} or \featref{foresight} on your \featref{scrying} images, making them watch backwards in time, or watch some time in the future.
	Starting \featref{foresight} on a scrying surface requires you to concentrate on the scrying surface for as long as it would normally take you to begin \featref{foresight}.
	
	If you also have \featref{scrying-unattended}, you can even leave the image running using either of them.
	However, changing the speed or pausing it (if you have feats to do so), requires you to be near the scrying surface, just like moving the sensor.
	\featref{foresight} will still expire at the usual time.
}

\feat{Freeze Frame}{hindsight-foresight-pause}{10}{
	\featref{hindsight}
}{
	You may pause your \featref{hindsight}, freezing everything in place so that you can look around leisurely.
	You may change at will between using hindsight normally (seeing backwards) and paused, but ending hindsight still jumps you back the present.
	You may also just slow it down, watching things backwards in slow-motion.
	
	By initiating and immediately pausing hindsight, you may also freeze-frame the present moment.
	Note that time will still continue around you; only your perception is frozen.
	
	You may also pause or slow down your \featref{foresight} visions, if you have that feat.
	The duration of such visions is limited by the time you \emph{see}, rather than time that passes for you.
	So you may spend several hours examining a paused vision, even if you cannot see anything that occurs more than 1 minute after the vision starts.
}

\feat{Fore-Hindsight}{hindsight-foresight-bidirectional}{10}{
	\skillref[1]{divination},
	\featref{hindsight-foresight-pause}
}{
	Much as you may pause your \featref{hindsight}, you may play it forwards again.
	This makes interpreting speech much easier, among other things.
	
	You may change at will between using your hindsight forwards, backwards, or pausing it, going forwards or backwards at any speed up to normal speed.
	You may never get ahead of the present time using this.
	
	You gain the reverse benefit on you \featref{foresight} visions, if you have that feat.
	You may play them forwards or backwards, as long you remain within the time frame defined by when the vision began, and its maximum duration.
}

\feat{Danger Sense}{divination-initiative}{20}{
	\skillref[1]{divination},
	\featref{foresight}
}{
	Sensing the future is easier if you don't seek the clear visions of \featref{foresight}, but settle for a vague premonition that \emph{something} is about to happen.
	You might not know what it is, but it won't catch you entirely unawares.
	You may use your \skillref{divination} skill in place of any other skill when rolling {\initiative}, and may even use it when the {\initiative} Test would otherwise use no skill at all.
}

\feat{Predictive Evasion}{divination-dodge}{20}{
	\skillref[1]{divination},
	\featref{divination-initiative}
}{
	With your eyes closed and all your concentration turned to it, you can enhance your premonitions of danger with perfect clarity.
	Activating this effect requires an {\action}, and lasts until the beginning of your next {\turn}, or until you open your eyes.
	For the duration, you automatically evade any \actionref{attack} or other harmful effect that could reasonably be evaded, unless it hits you with a critical success.
	
	This only allows you to sense anything which may harm you, restrain you, or the like.
	With your eyes closed, you are likely unaware of many other things.
	
	Furthermore, maintaining this for long periods is tiring.
	The GM may call for Tests to avoid {\exhaustion} after a minute or more of use.
}

\feat{Unblinded Prediction}{divination-dodge-see}{15}{
	\skillref[2]{divination},
	\featref{divination-dodge}
}{
	You can interpret your premonitions of danger even while distracted by visual input.
	You do not need to close your eyes to use \featref{divination-dodge}.
}

\feat{Perfect Premonition}{divination-dodge-2}{10}{
	\skillref[3]{divination},
	\featref{divination-dodge}
}{
	Foreseeing danger from even further ahead, you \emph{never} fail to evade it.
	Even critical successes do not hit you while you use \featref{divination-dodge}.
}
