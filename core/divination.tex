\chapter{Divination}
\disclabel{divination}{Seer}{Seers}

\section{Feats}

\feat{Perfect Positioning}{divination-self}{10}{
	None
}{
	While most divination deals in seeing the future, the past, or distant locations, there is one very useful trick that simply allows you to see the \emph{here} and \emph{now}.
	
	You always know where and when you are, and which way you are facing.
	You can't get lost---you always know which direction any place you've been before lies in, although not necessarily the best way to get there.
	You know what time it is, and how long you've been sleeping whenever you awake.
	
	However, this does not function in the {\mentalrealm}.
}

\feat{Taglock Identification}{divination-taglock-identify}{10}{
	None
}{
	You can touch a \materialref{taglock} and detect who it originates from.
	If you have met the target, you can identify them infallibly.
	
	If you have never met the target, you must make a \testtype{heed}{divination} Test, with higher results giving more information about the target.
	You can only get general information about the target this way, such as height, build, sex, appearance, and occupation.
	You can't get any information about their location, or even whether they are still alive.
}

\feat{Taglock Tracing}{divination-taglock-location}{10}{
	\featref{divination-taglock-identify},
	\featref{divination-self}
}{
	When you touch a \materialref{taglock}, you can find the person or creature it originates from.
	You know their location.
	This comes in the form of distance and direction---you can navigate to them easily, but that they are, for example, in the castle requires you to know the location of the castle, and figure out that's the location they're in.
	If they are moving, you can feel their position changing as they do.
}

\feat{Scrying}{scrying}{10}{
	None
}{
	You can use a \materialref{crystal-ball} to scry.
	You can scry on any location within 1 kilometre, but must be able to estimate distance and direction the place you are scrying.
	This is easy enough for any place you've taken a simple route to or from, but it might be hard to scry on the middle of a maze using this.
	\featref{divination-self} allows you to perfectly place any location you have been, and \featref{divination-taglock-location} lets you place a person.
	
	Scrying essentially forms a sensor, hanging in the air at the target location---although the ``sensor'' is invisible, intangible, and doesn't actually exist in any real sense.
	While scrying, you can move the sensor at a strolling pace, in all three dimensions and unimpeded by walls.
	It must stay within your 1 kilometre scrying radius.
	
	It takes about a minute of concentration on the \materialref{crystal-ball} for the image to form.
	Anyone looking into the ball sees out from the sensor, and can look around by looking into the ball at different angles.
	However, you cannot see anything more than a dozen metres from the sensor; the image becomes too distorted.
	You only get vision; no other senses.
	
	Scrying requires continuous concentration---you must remain within a metre of the \materialref{crystal-ball}, and can't manage any more significant actions than talking or idly knitting.
	You may only maintain one scrying sensor at a time.
}

\feat{Mirror Scrying}{scrying-mirror}{15}{
	\skillref[1]{divination},
	\featref{scrying}
}{
	You may use a mirror instead of a \materialref{crystal-ball} for \featref{scrying}.
	Being non-spherical, the angle of view is restricted.
	You have to turn the mirror you are scrying through around in order to see the other way.
}

\feat{Eyeball Scrying}{scrying-eyes}{15}{
	\skillref[2]{divination},
	\featref{scrying-mirror}
}{
	By a sufficient stretch of the imagination, an eyeball is nearly a \materialref{crystal-ball}.
	It takes a little trick to look out from the inside of them, rather than in from the outside, but you've made it work.
	
	You can use \featref{scrying} without a \materialref{crystal-ball} or mirror, by using your eyeballs.
	You can, and should, leave them in your head for this.
	This works only for you; other people cannot see in on your scrying.
	
	You are blinded to your own surroundings while using your eyes in this way.
	Additionally, unless you also have \featref{scrying-unattended}, you must keep most of your concentration on the scrying, and cannot do much more than stroll around a bit.
}

\feat{Taglock Scrying}{scrying-person}{15}{
	\skillref[1]{divination},
	\featref{scrying},
	\featref{divination-taglock-location}
}{
	You may scry on a particular person, finding them using a \materialref{taglock} you hold as you begin \featref{scrying}.
	The scrying sensor appears within 2 metres of them, regardless of whether they lie within your usual scrying range.
	You can move it around as normal within this 2 metre radius of them, and it is dragged along behind them no matter how fast they move.
	You can only move the sensor further away from them if it is within your normal scrying range, but it stops following them if you do so.
}

\feat{Bird's-Eye Scrying}{scrying-sight-range}{15}{
	\skillref[1]{divination},
	\featref{scrying}
}{
	You've learned to counteract the distortion caused through your scrying apparatus.
	When you are \featref{scrying}, your vision is no longer restricted to within a dozen metres of the sensor.
	You may see just as well as if you were at the location of the sensor yourself.
	
	With a sufficiently aerial view, you might even see for several kilometres, while keeping the sensor within 1 kilometre.
	However, you have no more than your usual ability to resolve detail at that range.
}

\feat{Far Scrying}{scrying-range}{20}{
	\skillref[2]{divination},
	\featref{scrying}
}{
	When you are \featref{scrying}, your range is no longer limited to 1 kilometre---it is now unlimited.
	Placing a location at long distances becomes increasingly inaccurate, making \featref{divination-self} even more useful.
}

\feat{Roaming Scrying}{scrying-speed}{10}{
	\skillref[1]{divination},
	\featref{scrying}
}{
	You've become adept at refocussing your scrying sensor, letting you move it far faster than the slow walk you could manage before.
	While \featref{scrying}, you can move the sensor fast---fast enough to keep up with someone running, or even someone on a broomstick.
	You can also use an {\action} to refocus the sensor to a different location entirely, following the usual rules for positioning a scrying sensor in the first place.
}

\feat{Fast Scrying}{scrying-speed-2}{20}{
	\skillref[2]{divination},
	\featref{scrying-speed}
}{
	You can focus a scrying sensor in an instant.
	You may begin \featref{scrying} as an {\action}, instead of requiring a minute.
}

\feat{Unattended Scrying}{scrying-unattended}{10}{
	\featref{scrying}
}{
	Once you've fired up a \materialref{crystal-ball}, you can leave it running while you go and do something else.
	You no longer need continuous concentration to maintain \featref{scrying}, and can leave even leave the image unattended, possibly watched by someone else.
	However, you still need to be near the \materialref{crystal-ball} and concentrating to form the image in the first, and to move the sensor.
	
	You can still only maintain one instance of scrying at a time.
	The image dissolves if you begin a new instance of scrying, and you can also dispel it at any time you choose.
	
	You can use this with a mirror as well as a \materialref{crystal-ball} if you have \featref{scrying-mirror}, and with your own eyes if you have \featref{scrying-eyes}.
	In this latter case, other people still cannot look in on your scrying, but it allows you to act normally while scrying.
}

\feat{Multiple Scrying}{scrying-multiple}{10}{
	\skillref[1]{divination},
	\featref{scrying-unattended},
	\featref{scrying-mirror}
}{
	The is no longer any limit to the number of \featref{scrying} images you can maintain at a time.
	However, you still need one \materialref{crystal-ball} or mirror per image.
	If you have \featref{scrying-eyes}, one instance of scrying still requires both your eyeballs, and renders you blind to your surroundings.
}

\feat{Danger Sense}{divination-initiative}{20}{
	\skillref[1]{divination}
}{
	You can't quite see the future, as such, but you can tell when \emph{something} is about to happen.
	You might not know what it is, but it won't catch you entirely unawares.
	You may use your \skillref{divination} skill in place of any other skill when rolling {\initiative}, and may even use it when the {\initiative} Test would otherwise use no skill at all.
}

\feat{Perfect Prediction}{divination-dodge}{20}{
	\skillref[1]{divination},
	\featref{divination-initiative}
}{
	With your eyes closed and all your concentration turned to it, you can enhance your premonitions of danger with perfect clarity.
	Activating this effect requires an {\action}, and lasts until the beginning of your next {\turn}, or until you open your eyes.
	For the duration, you automatically evade any \actionref{attack} or other harmful effect that could reasonably be evaded, unless it hits you with a critical success.
	
	This only allows you to sense anything which may harm you, restrain you, or the like.
	With your eyes closed, you are likely unaware of many other things.
	
	Furthermore, maintaining this for long periods is tiring.
	The GM may call for Tests to avoid {\exhaustion} after a minute or more of use.
}
