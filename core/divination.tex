\discipline{Divination}{divination}{Seer}{Seers}

\dropcapdiscref{divination} is the art of seeing, hearing, and otherwise sensing in ways that are not normally possible.
Many witches consider it to be a very narrow and limited discipline.
In many ways, they are right.
It certainly lacks the versatility of \discref{willing}.
In fact, it lacks \discrefpossessive{willing} ability to affect the world at all.

Rather, \practitioners{divination}---as the practitioners of \discref{divination} are called---tend to rely on enhanced knowledge and awareness.
It's all very well, they say, having the power to crush a man beneath a ton of rock, but which man to crush?
Was he the one who did the crime?
And where's he run off to now?
With {\foretelling}, it may be possible to stop him before he even commits the crime.
After all, an ounce of prevention is worth a pound of cure, and a single arrow can change the world if you know just where to put it.

Despite its seemingly narrow focus, \discref{divination} still covers a fairly wide gamut of abilities.
The simplest just enhance an existing sense, allowing a witch to see in the dark, for example.
Some, such as \featref{divination-self}, grant an entirely new sense.
Perhaps the most interesting, however, are \featref{scrying}, \featref{hindsight}, and \featref{foresight}.
These allow the witch to see distant locations, the past, and even the future.

In terms of the necessary equipment, \discref{divination} sits between the extremes of \discref{willing} and \discref{ritual-magic}.
Much of it, particularly simple enhancements to a witch's senses, requires nothing beyond a witch's eyes or ears, and sometimes not even those.
In fact, many of these abilities require essentially no ongoing effort from the witch; once she knows the trick of it, it's as easy as opening her eyes.
Some \discref{divination}, however---particularly sensing distant locations---requires a lot more concentration and equipment.
\featref{scrying} requires a surface to look through, and several abilities require an object to focus upon, to find the right time or location to sense.

\attref{heed} is the most important attribute in \discref{divination}.
The discipline is entirely about extending one's awareness, and a witch who is blind to her own surroundings in the first place is going to have a hard time looking any further afield.

\section{Foretelling}
\seclabel{foretelling}

For many witches, telling the future is the most enticing part of \discref{divination}.
It is, however, the trickiest.

In truth, nobody has ever managed to see their own future.
To know one's own fate, but to do \emph{nothing} to avoid it, is a level of self-discipline beyond any \practitioner{divination} the world has yet seen.
And without that self-discipline, the future is simply inscrutable---it seems almost to be defending itself against change.

But witches wouldn't be witches if they hadn't found a way to cheat.
You can't see your own future, but you can see the next closest thing.
So foretelling, such as it is practiced, is a matter of forking the stream of time, of tossing a rock into the flow and sending it thundering down two different channels.
There is a moment in every foretelling ritual, or spell, where the \practitioner{divination} goes two ways.
On one branch, she completes the ritual.
On the other, she stops abruptly halfway through, stands up, and goes about her life none the wiser.
It is the world of the second \practitioner{divination} that the first \practitioner{divination} sees.

Most \practitioners{divination} treat this as a metaphor, of course.
They've never \emph{actually} abandoned the ritual halfway through, no matter how many times they've seen themselves do it.
But some do wonder if those copies of themselves exist out there somewhere, even the ones they've seen die.

\subsection{The World Foretold}

All in all, a \practitioner{divination} performing foretelling sees visions of the world as it would turn out if she did not perform this foretelling.
So long as she does not interfere, she may assume that things will proceed, for the most part, the same.
But, as she changes more things, the worlds will diverge further, until her predictions become useless.

Furthermore, it is worth noting that sitting still in a dark room, doing nothing, doesn't mean a \practitioner{divination} is not interfering.
In the world where she performed no divination, the one she has peered in on, she likely acted, going out and doing whatever she normally did.
By sitting still in this world, she leaves out the results of those actions, probably rendering her predictions inaccurate quite quickly.

As such, the best way to keep predictions accurate for as long as possible is to act as though she never made them in the first place, going about as though she were blind to the future.
It will never work indefinitely, but she might be able to keep it up for a while.

\subsection{Foretelling \& the Metagame}

Telling the future is tricky not only for a witch, in character, but also for the GM running the game.
Giving good visions requires the GM to predict how events would play out, assuming the \practitioner{divination} had no foreknowledge.
This requires more planning than some GMs might want to do.

Furthermore, it often requires predicting how the player's characters would act, which can cause problems.
The GM is free to ask the players themselves, if that won't give away any plot details that the vision itself won't.
It can even be fun to play out the world where the \practitioner{divination} did not perform her foretelling, before jumping back to the point of the foretelling with the \practitionerpossessive{divination} new knowledge and playing again from there.
That can grow tedious, of course---it should be used sparingly, and with the agreement of everyone at the table.

Most importantly, both the players and the GM should be understanding that telling the future is tricky business.
Everyone should be willing to forgive mistakes, and to try and resolve everything coherently.
And if everything is just too much, the GM can request that players simply avoid the more troublesome feats.
Even then, feats such as \featref{divination-initiative} and \featref{divination-dodge} allow for witches who can see the future without requiring the GM to invent visions.

\subsection{Foretelling as a Plot Device}

Prophecies and visions are an ancient staple of myths and legends, as well as contemporary literature.
GMs should feel free to use foretelling as a device in creating plots for the game.
This may use the visions provided by a player character's feats, other omens, or visions imparted upon a \practitioner{divination}, or a non-player \practitioner{divination}.

One aspect to play with as part of a plot is the assumption that no \practitioner{divination} can see their own future.
Just because no \practitioner{divination} has had the self-discipline to manage it yet, doesn't mean that it is impossible.
Any plot violating this assumption must be very carefully executed, to avoid taking agency away from the players, but can be very effective if it is done well.
Just remember that witches are notorious cheats---even predestined events are only fixed to the extent that they've been seen, and can always be staged.

\section{Visions of the Past and Future}
\seclabel{visions}

The counterpart to seeing the future, {\foretelling}, is seeing the past; often called hindsight.
This carries some important difference to {\foretelling}.
For one thing; it's easier.
The past has already happened; it's right there for looking at.
Secondly, it's always accurate.
Looking at the future inevitably changes it, so that what you see isn't quite what will wind up happening.
Looking at the past, however, does nothing to change it.
The other side of this, of course, is that it's too late to change it, even if you want to.

Despite the differences, however, {\visions} of the past and the future have a lot in common.
This section lays out the mechanics of such {\visions}, as used by the \featref{hindsight} and \featref{foresight} feats.

Firstly, the feat that grants you the {\vision} defines the period of time that the {\vision} covers.
For \featref{hindsight}, this period of time is always the entire past, from the present moment, as far back as you can reach.
\featref{foresight}, however, is less stable.
Its {\visions} begin at a particular time---some time in the future--- and run for a fixed length of time after that.
You can end them early, but you can never run them beyond that time.

For a novice witch, time within a {\vision} simply plays from one end to the other, at normal speed.
With \featref{foresight}, time plays forward, from the beginning of the {\vision} to the end.
With \featref{hindsight}, time plays backwards, from the present moment until the witch stops maintaining the {\vision}.
More experienced witches will learn to break these rules, of course.

While in a {\vision}, you see, hear, and otherwise sense around the current location of your body.
However, your senses detect what was, or will be, happening at the time you are looking in on.
For example, with \featref{hindsight}, you could enter a room, then look backwards to see what happened in a room before you entered it.

{\visions} affect all senses except touch.
You see, hear, smell, and taste the time you are looking in on.
You benefit from any enhancements to these senses, such as \featref{darkvision}, and even from entirely new senses, such as that granted by \featref{death-detection}.
You cannot use {\visions} within the {\mentalrealm}---or even sense the {\mentalrealm} at all---without the feat \featref{vision-projection}.

Your sense of touch, however, remains in the present time.
You cannot touch anything from the past or future, and people in these times can even walk straight through you---quite a disturbing experience.
A {\vision} renders you blind, deaf, and otherwise insensate to your current surroundings; only touch keeps you connected to the present.
Someone trying to bring you out of a {\vision} is advised to tap you on the shoulder; you won't sense anything but a touch.

You may end a {\vision} at any time you choose, at which point your senses immediately snap back to the present.
The {\vision} also ends if you fall asleep, fall unconscious, or, in the case of \featref{foresight}, you reach the end of its allotted period of time.

You may act normally during a {\vision}; knitting, talking, even pacing.
You cannot sense what you're doing except through touch, which can make things quite difficult.
Walls don't tend to move too much, however---as long as you are looking at the immediate past or future---so you can hopefully avoid walking into those.

Although you can move about during a {\vision}, a novice cannot hold a {\vision} together it they move too far from where they began it.
The {\vision} breaks down and ends if you move too far from where you began it; you can manage a couple of metres without trouble, or more with a Test.
\featref{vision-move} lifts this limitation.

The limitation of only being able to see the area around where you are currently standing is quite a major one, but there are a couple of things you can do to lift it, making use of \featref{scrying}.
While in a {\vision}, you can see any scrying images that existed in the time and place you are looking at.
With \featref{foresight} in particular, you might be able to look at yourself scrying in the future.
This is far harder to arrange with \featref{hindsight}, however, and does not work at all with \featref{scrying-eyes}.
The more reliable solution is to use the feat \featref{vision-scrying}.

\section{Feats}

\feat{Dimvision}{darkvision}{15}{
	\noprereq
}{
	Your eyes, magically assisted, can pick up the barest traces of light.
	You suffer no penalties in low-light conditions, though you are as blind as anyone in complete darkness.
	This works even if you are standing in light and looking into darkness.
}

\feat{Darkvision}{darkvision-2}{10}{
	\skillref[1]{divination},
	\featref{darkvision}
}{
	You no longer need light, so fickle and often absent, to see.
	This works as \featref{darkvision}, except it also works in complete darkness.
}

\feat{Unclouded Vision}{xray-vision}{10}{
	\skillref[1]{divination},
	\featref{darkvision}
}{
	Translucent substances, such as fog, murky water, or sandstorms, do not obscure or distort your vision.
	Intermediate cases, such as a leafy canopy or dense cobwebs, can be seen through with a \testtype{heed}{divination} Test.
}

\feat{Blindsight}{blindsight}{15}{
	\skillref[2]{divination},
	\featref{darkvision-2}
}{
	You no longer need to use your eyes to see.
	You cannot be blinded, or have your vision impaired.
	
	This covers mundane afflictions, such as grit in your eyes, or having your eyes gouged out, as well as effects such as a \featref{smell-potion}, or being transformed into a blind animal.
	Other divination feats, however, such as \featref{scrying-eyes} and \featref{divination-dodge}, blind you at deeper level.
	This offers you no protection against them.
	%TODO: If I add a headology feat where you can convince someone they are blind, this offers no immunity to that. Although you'll be much harder to convince, knowing that you can't be blinded.
	
	Likewise, this doesn't make you immune to a blindfold, or a bag over your head.
	You're still seeing out from your eyes, or your eye sockets if those are gone.
}

\feat{Eyes in the Back of Your Head}{360-vision}{15}{
	\skillref[1]{divination}
}{
	You can see behind you.
	In fact you can see all around yourself; 360 degrees, plus up and down.
	This makes you far harder to sneak up on, and lets you observe someone without looking in their direction.
	The GM may give a bonus to \skillref{perception} Tests relying on spotting something that you might not be looking at.
	
	This feat does not literally give you any extra eyes, and only works as long as you can see normally.
	However, it also works when looking through a \featref{scrying} surface, allowing you to see out in all directions from the sensor.
}

\feat{Perfect Positioning}{divination-self}{10}{
	\noprereq
}{
	While much of \discref{divination} deals in sensing the future, the past, or distant locations, there is one very useful trick that simply allows you to sense, precisely, the \emph{here} and \emph{now}.
	
	You always know where and when you are, and which way you are facing.
	You can't get lost---you always know which direction any place you've been before lies in, although not necessarily the best way to get there.
	You know what time it is, and how long you've been sleeping whenever you awake.
	
	However, this does not function in the {\mentalrealm}.
}

\feat{Taglock Identification}{divination-taglock-identify}{10}{
	\noprereq
}{
	You can touch a \materialref{taglock} and detect who it originates from.
	If you have met the target, you can identify them infallibly.
	
	If you have never met the target, you must make a \testtype{heed}{divination} Test, with higher results giving more information about the target.
	You can only get general information about the target this way, such as height, build, sex, appearance, and occupation.
	You can't get any information about their location, or even whether they are still alive.
}

\feat{Taglock Tracing}{divination-taglock-location}{10}{
	\featref{divination-taglock-identify},
	\featref{divination-self}
}{
	When you touch a \materialref{taglock}, you can find the person or creature it originates from.
	You know their location.
	This comes in the form of distance and direction---you can navigate to them easily, but that they are, for example, in the castle requires you to know the location of the castle, and figure out that's the location they're in.
	If they are moving, you can feel their position changing as they do.
}

\feat{Scrying}{scrying}{10}{
	\noprereq
}{
	You can use a \materialref{crystal-ball} to scry.
	You can scry on any location within 1 kilometre, but must be able to estimate distance and direction the place you are scrying.
	This is easy enough for any place you've taken a simple route to or from, but it might be hard to scry on the middle of a maze using this.
	\featref{divination-self} allows you to perfectly place any location you have been, and \featref{divination-taglock-location} lets you place a person.
	
	Scrying essentially forms a sensor, hanging in the air at the target location---although the ``sensor'' is invisible, intangible, and doesn't actually exist in any real sense.
	While scrying, you can move the sensor at a strolling pace, in all three dimensions and unimpeded by walls.
	It must stay within your 1 kilometre scrying radius.
	
	It takes about a minute of concentration on the \materialref{crystal-ball} for the image to form.
	Anyone looking into the ball sees out from the sensor, and can look around by looking into the ball at different angles.
	However, you cannot see anything more than a dozen metres from the sensor; the image becomes too distorted.
	You only get vision; no other senses.
	However, you still benefit from anything that enhances your vision, such as \featref{darkvision}.
	
	Scrying requires continuous concentration---you must remain within a metre of the \materialref{crystal-ball}, and can't manage any more significant actions than talking or idly knitting.
	You may only maintain one scrying sensor at a time.
}

\feat{Mirror Scrying}{scrying-mirror}{15}{
	\skillref[1]{divination},
	\featref{scrying}
}{
	You may use a mirror instead of a \materialref{crystal-ball} for \featref{scrying}.
	Being non-spherical, the angle of view is restricted.
	You have to turn the mirror you are scrying through around in order to see the other way.
}

\feat{Eyeball Scrying}{scrying-eyes}{15}{
	\skillref[2]{divination},
	\featref{scrying-mirror}
}{
	By a sufficient stretch of the imagination, an eyeball is nearly a \materialref{crystal-ball}.
	It takes a little trick to look out from the inside of them, rather than in from the outside, but you've made it work.
	
	You can use \featref{scrying} without a \materialref{crystal-ball} or mirror, by using your eyeballs.
	You can, and should, leave them in your head for this.
	This works only for you; other people cannot see in on your scrying.
	
	You are blinded to your own surroundings while using your eyes in this way.
	Additionally, unless you also have \featref{scrying-unattended}, you must keep most of your concentration on the scrying, and cannot do much more than stroll around a bit.
}

\feat{Taglock Scrying}{scrying-person}{15}{
	\skillref[1]{divination},
	\featref{scrying},
	\featref{divination-taglock-location}
}{
	You may scry on a particular person, finding them using a \materialref{taglock} you hold as you begin \featref{scrying}.
	The scrying sensor appears within 2 metres of them, regardless of whether they lie within your usual scrying range.
	You can move it around as normal within this 2 metre radius of them, and it is dragged along behind them no matter how fast they move.
	You can only move the sensor further away from them if it is within your normal scrying range, but it stops following them if you do so.
}

\feat{Bird's-Eye Scrying}{scrying-sight-range}{15}{
	\skillref[1]{divination},
	\featref{scrying}
}{
	You've learned to counteract the distortion caused through your scrying apparatus.
	When you are \featref{scrying}, your vision is no longer restricted to within a dozen metres of the sensor.
	You may see just as well as if you were at the location of the sensor yourself.
	
	With a sufficiently aerial view, you might even see for several kilometres, while keeping the sensor within 1 kilometre.
	However, you have no more than your usual ability to resolve detail at that range.
}

\feat{Far Scrying}{scrying-range}{20}{
	\skillref[2]{divination},
	\featref{scrying}
}{
	When you are \featref{scrying}, your range is no longer limited to 1 kilometre---it is now unlimited.
	Placing a location at long distances becomes increasingly inaccurate, making \featref{divination-self} even more useful.
}

\feat{Roaming Scrying}{scrying-speed}{10}{
	\skillref[1]{divination},
	\featref{scrying}
}{
	You've become adept at refocusing your scrying sensor, letting you move it far faster than the slow walk you could manage before.
	While \featref{scrying}, you can move the sensor fast---fast enough to keep up with someone running, or even, with a Test, someone on a broomstick.
	You can also use an {\action} to refocus the sensor to a different location entirely, following the usual rules for positioning a scrying sensor in the first place.
}

\feat{Fast Scrying}{scrying-start-speed}{20}{
	\skillref[2]{divination},
	\featref{scrying-speed}
}{
	You can focus a scrying sensor in an instant.
	You may begin \featref{scrying} as an {\action}, instead of requiring a minute.
}

\feat{Effortless Scrying}{scrying-start-speed-2}{20}{
	\skillref[3]{divination},
	\featref{scrying-start-speed}
}{
	Scrying has become as natural to you as blinking.
	You may begin \featref{scrying}---or refocus the sensor as per \featref{scrying-speed}---at any time, without using an {\action}.
}

\feat{Unattended Scrying}{scrying-unattended}{10}{
	\featref{scrying}
}{
	Once you've fired up a \materialref{crystal-ball}, you can leave it running while you go and do something else.
	You no longer need continuous concentration to maintain \featref{scrying}, and can even leave the image unattended, possibly watched by someone else.
	However, you still need to be near the \materialref{crystal-ball} to form the image in the first place, or to move the sensor.
	Forming the image requires concentration, but moving it is as easy as glancing around.
	If you have \featref{scrying-speed}, refocusing entirely is still an {\action}.
	
	You can still only maintain one instance of scrying at a time.
	The image dissolves if you begin a new instance of scrying, and you can also dispel it at any time you choose.
	The image also dissolves if you die.
	
	You can use this with a mirror as well as a \materialref{crystal-ball} if you have \featref{scrying-mirror}, and with your own eyes if you have \featref{scrying-eyes}.
	In this latter case, other people still cannot look in on your scrying, but it allows you to act normally while scrying.
}

\feat{Multiple Scrying}{scrying-multiple}{10}{
	\skillref[1]{divination},
	\featref{scrying-unattended},
	\featref{scrying-mirror}
}{
	The is no longer any limit to the number of \featref{scrying} images you can maintain at a time.
	However, you still need one \materialref{crystal-ball} or mirror per image.
	If you have \featref{scrying-eyes}, one instance of scrying still requires both your eyeballs, and renders you blind to your surroundings.
}

\feat{One-Eyed Scrying}{scrying-multiple-eyes}{15}{
	\skillref[2]{divination},
	\featref{scrying-multiple},
	\featref{scrying-eyes}
}{
	\featref{scrying} with your eyeballs no longer requires both of them, but only one.
	As such, you are no longer blinded to your own surroundings while scrying with your eyes.
	Alternatively, you may scry on two different places at the same time, one with each eye, becoming blinded to your own surroundings as normal.
}

\feat{Eavesdropper's Scry}{scrying-sound}{20}{
	\featref{scrying}
}{
	You may hear sounds from a \featref{scrying} sensor.
	These sounds emanate from the \materialref{crystal-ball} or mirror used for scrying, to be heard by everyone around.
	However, if you are using \featref{scrying-eyes}, only you can hear them.
	You may disable and resume the transmission of sound at will, while you are near the scrying surface.
	
	You can only hear sounds originating in the same dozen-metre radius as you can see, unless you also have \featref{scrying-sight-range}.
}

\feat{Someone Watching}{gaze-detection}{10}{
	\noprereq
}{
	People have a fine sense of when they're being stared at, but you've honed yours to an art.
	You always know when you're being watched.
	You can get a sense of how much attention is being paid to you---whether it's just someone watching out of the corner of their eye, someone staring right at you, or the rapt attention of dozens---but you don't know \emph{who} is watching.
}

\feat{Eyes on You}{gaze-detection-2}{10}{
	\skillref[1]{divination},
	\featref{gaze-detection}
}{
	Feeling everybody's eyes on you all the time can be quite unnerving, but you've deemed it to be a price worth paying.
	You can always tell how many people (or animals) are watching you, how much attention each is paying, and where each is watching from.
	If someone is watching by \featref{scrying}, you can tell they are scrying, but you detect the location of the sensor rather than the person themself.
}

\feat{Hindsight}{hindsight}{10}{
	\noprereq
}{
	Seeing the past is far easier than seeing the future, although you don't have any chance of changing it.
	
	You may enter hindsight at any time, without an {\action}.
	This gives you a {\vision} of the past, running backwards in time from the point you begin hindsight.
	
	You sense everything in reverse, but at normal speed.
	For example, if you want to see what happened ten minutes before you began the {\vision}, you must wait ten minutes to rewind to that point.
	Speech is hard to interpret in reverse, requiring a \attref{wit} Test for anything but the most simplest phrases.
}

\feat{Foresight}{foresight}{20}{
	\noprereq
}{
	Seeing the future is more difficult than seeing the past, but you've got it figured out.
	The ritual to begin a {\vision} of the future requires 15 minutes of concentration, while remaining in one place.
	The {\vision} follows the normal rules for {\foretelling}; the world you see is the world as it would have been if you did not have this {\vision}.
	
	You select a time in the future---no more than a day ahead---to view.
	You must specify a time, for example, 30 minutes or 8 hours ahead; finding a specific event requires dead reckoning.
	The {\vision} is unstable and brief, it covers the period of time only 1 minute forward from time you specified, and ends when you reach the end of this time.
}

\feat{Rapid Rewind}{vision-speed}{15}{
	\skillref[1]{divination},
	\featref{hindsight} or \featref{foresight}
}{
	Sometimes, you just can't arrive at the scene of the crime for quite a while after it happened, and don't have the time to waste sending your eyes that far back.
	Thankfully, you've learned to pick up the pace.
	
	You may accelerate your {\visions}, up to 20 times the speed.
	This lets you view an hour in just three minutes, or most of a day in an hour.
	You can switch between rapid speed and normal speed at will, to watch the interesting bits and skip the useless parts.
	
	This goes in the normal direction for the {\vision}: forward for \featref{foresight} and backwards for \featref{hindsight}.
	It is more useful for \featref{hindsight}, of course, as it allows you to go further back.
	If you also have \featref{vision-bidirectional}, you can go up to this speed forwards or backwards, though never outside of the {\visionpossessive} alloted time.
}

\feat{Lightning Rewind}{vision-speed-2}{15}{
	\skillref[2]{divination},
	\featref{vision-speed}
}{
	You may accelerate your {\visions} to ludicrous speed.
	This follows the same rules as \featref{vision-speed}, except that you may perceive time at up to a million times the normal speed; more than ten days per second, or a century in less than an hour.
}

\feat{Roaming Visions}{vision-move}{15}{
	\featref{hindsight} or \featref{foresight}
}{
	Moving around---any distance---no longer disrupts your {\visions}.
}

\feat{Freeze Frame}{vision-pause}{10}{
	\featref{hindsight} or \featref{foresight}
}{
	You may pause your {\visions}, freezing everything in place so that you can look around leisurely.
	You may change at will between watching the {\vision} normally and paused, but ending the {\vision} still jumps you back the present.
	You may also just slow it down, watching things in slow-motion, although still in the same direction.
	
	If you have \featref{hindsight}, you may initiate and immediately pause it to freeze-frame the present moment.
	Note that time will still continue around you; only your perception is frozen.
	
	The duration of \featref{foresight} {\visions} is limited by the time you \emph{see}, rather than time that passes for you.
	So you may spend several hours examining a paused {\vision}, even if you cannot see anything that occurs more than 1 minute after the {\vision} starts.
	You have a moment's warning when the {\vision} is about to expire, allowing you to pause to examine the very end of the {\vision}, before it fades.
}

\feat{Fore-Hindsight}{vision-bidirectional}{10}{
	\skillref[1]{divination},
	\featref{vision-pause}
}{
	Much as you may pause your {\visions}, you can play them the other way.
	This makes interpreting speech from \featref{hindsight} much easier, among other things.
	
	You may change at will between watching your {\visions} forwards, backwards, or pausing them, going forwards or backwards at any speed up to normal speed.
	You must always stay within the {\visionpossessive} allotted time, however.
	The same as with \featref{vision-pause}, you have a moment's warning when the allotted time is about to expire, allowing you to turn and watch back the other way.
}

\feat{Time Scrying}{vision-scrying}{25}{
	\skillref[1]{divination},
	\featref{vision-move},
	\featref{scrying}
}{
	You may use \featref{hindsight} or \featref{foresight}---assuming you have them---on your \featref{scrying} images, making them watch backwards in time, or watch some time in the future.
	Starting \featref{foresight} on a scrying surface requires you to concentrate on the scrying surface for as long as it would normally take you to begin \featref{foresight}.
	
	If you also have \featref{scrying-unattended}, you can even leave the image running using either of them.
	However, changing the speed or pausing it (if you have feats to do so), requires you to be near the scrying surface, just like moving the sensor.
	\featref{foresight} will still expire at the usual time.
}

\feat{Haruspex}{foretelling-entrails}{10}{
	\featref{foresight}
}{
	By spilling the entrails of a creature at least the size of a cat, you can foretell great violence.
	You must spill these entrails, fresh from the creature's body, as part of beginning \featref{foresight}.
	If you do so, then instead of specifying a time for the resulting {\vision}, you see a moment of violence.
	
	This searches for the most violent moment within the times \featref{foresight} would normally allow you to view, as determined by the GM.
	The {\vision} begins around the start of the violence, but is always positioned such that it encompasses the most violent part.
	You may search a shorter time than you are able to, if you wish.
	For example, you may choose to look only for the most violent moment within the next hour.
	
	This only searches for violence in your immediate surroundings---such that you would be able to sense it in the {\vision} from where you stand now.
	In combination with \featref{vision-scrying}, it searches around the current location of the scrying sensor.
}

\feat{Tarot}{foretelling-chance}{25}{
	\skillref[1]{divination},
	\featref{foresight}
}{
	Simply guessing at the time you want to foresee is rather inaccurate, and likely to leave you with a {\vision} of a whole lot of nothing.
	By laying out \materialrefplural{tarot} while you begin \featref{foresight}, you may try to aim your {\vision} at a specific event.
	
	You must specify the event you are looking for, in as much detail as you can.
	For example, you might look for ``when our coven comes back out of this cave'', ``the next time someone enters this pub'', or ``the next time I snap my fingers''.
	The event you specify must be something you would be able to sense in the {\vision}, either from where you stand now, or---in combination with \featref{vision-scrying}---from the current location of the scrying sensor.
	
	\materialrefplural{tarot} are an instrument of chance, and using this feat \emph{always} requires a Test.
	The GM sets the {\tn}, based on the specifity of the event you are looking for.
	The more you can specify about it---the more accurately you can visualise it---the easier the Test is.
	For example, ``the next time I snap my fingers'' specifies very little about most of the scene, while ``when our coven comes back out of this cave'' specifies several people, and what they are doing.
	The Test's difficulty, as usual, is raised further by rushing your magic, or using substandard materials.
	
	If you succeed on the Test, and the event you specified occurs within the time \featref{foresight} would normally allow you to view, then you receive a {\vision} of the event, of the usual length.
	If you fail the Test, or if the event would not occur where you could sense it within the time you can view, then you receive a {\vision} of a random point within the time you searched.
	You have no indication of which has occurred, beyond whether you see the event within the {\vision}.
	If you wish, you can shorten the period of time you are searching; for example, you can look for it only within the next hour.
}

\feat{Danger Sense}{divination-initiative}{20}{
	\skillref[1]{divination},
	\featref{foresight}
}{
	Sensing the future is easier if you don't seek the clear {\visions} of \featref{foresight}, but settle for a vague premonition that \emph{something} is about to happen.
	You might not know what it is, but it won't catch you entirely unawares.
	You may use your \skillref{divination} skill in place of any other skill when rolling {\initiative}, and may even use it when the {\initiative} Test would otherwise use no skill at all.
}

\feat{Predictive Evasion}{divination-dodge}{20}{
	\skillref[1]{divination},
	\featref{divination-initiative}
}{
	With your eyes closed and all your concentration turned to it, you can enhance your premonitions of danger with perfect clarity.
	Activating this effect requires an {\action}, and lasts until the beginning of your next {\turn}, or until you open your eyes.
	For the duration, you automatically evade any \actionref{attack} or other harmful effect that could reasonably be evaded, unless it hits you with a critical success.
	
	This only allows you to sense anything which may harm you, restrain you, or the like.
	With your eyes closed, you are likely unaware of many other things.
	
	Furthermore, maintaining this for long periods is tiring.
	The GM may call for Tests to avoid {\exhaustion} after a minute or more of use.
}

\feat{Unblinded Prediction}{divination-dodge-see}{15}{
	\skillref[2]{divination},
	\featref{divination-dodge}
}{
	You can interpret your premonitions of danger even while distracted by visual input.
	You do not need to close your eyes to use \featref{divination-dodge}.
}

\feat{Perfect Premonition}{divination-dodge-2}{10}{
	\skillref[3]{divination},
	\featref{divination-dodge}
}{
	Foreseeing danger from even further ahead, you \emph{never} fail to evade it.
	Even critical successes do not hit you while you use \featref{divination-dodge}.
}

\feat{Scrying the Minds}{scrying-projection}{10}{
	\skillref[1]{divination},
	\skillref[1]{projection},
	\featref{scrying-unattended},
	\featref{projection-start}
}{
	While in the {\mentalrealm}, you can extend your sense through \featref{scrying} surfaces; yours or other people's.
	If you have \featref{projection-mental-realm-sense}, you can do so even while inhabiting a body.
	You can only sense minds within a dozen metres of the scrying sensor, regardless of how far you could normally sense in the {\mentalrealm}, unless you have \featref{scrying-sight-range}.
	
	While in the {\mentalrealm}, it is trivial to jump from the scrying surface to the place it is looking at.
	You have no help in going the other way, however; you can't even detect the scrying sensor at the other end.
	So it pays to ensure you have some other way back.
	
	You cannot use this through any scrying surface that is seeing the past or future, by \featref{vision-scrying}, unless you also have \featref{vision-projection}.
}

\feat{Mind-Controlled Scrying}{scrying-projection-move}{10}{
	\skillref[1]{divination},
	\skillref[1]{projection},
	\featref{scrying-projection}
}{
	You can move your own \featref{scrying} sensors from within the {\mentalrealm}, as long as you are near enough to the scrying surface.
	This affords you all the same control you would have if you were near the scrying surface physically---you can move the sensor at strolling speed, move it faster or refocus it if you have \featref{scrying-speed}, and so on.
	If you also have \featref{vision-projection} and \featref{vision-scrying}, you can also move the sensor through time from within the {\mentalrealm}, following the usual rules.
}

\feat{Temporal-Mentality}{vision-projection}{10}{
	\skillref[1]{divination},
	\skillref[1]{projection},
	\featref{hindsight} or \featref{foresight},
	\featref{projection-start}
}{
	You can use \featref{hindsight} or \featref{foresight}---assuming you have them---from within the {\mentalrealm}, if you have them, sensing the minds in the past or future.
	If you also have \featref{projection-mental-realm-sense}, you can sense the {\mentalrealm} in the past or future even while using \featref{hindsight} or \featref{foresight} from within a body.
	However, you cannot form {\interfaces} with minds in any time but the present.
	Furthermore, you cannot enter or leave the {\mentalrealm} during the {\vision} without disrupting it, causing it to end.
}

\feat{Temporal-Mental Leap}{vision-projection-start}{5}{
	\skillref[1]{divination},
	\skillref[2]{projection},
	\featref{vision-projection},
	\featref{vision-move},
	\featref{projection-start-3}
}{
	You may enter or leave the {\mentalrealm} while using \featref{hindsight} or \featref{foresight}, without disrupting the {\vision}.
}

\feat{Temporal-Mental Reading}{vision-projection-interface}{15}{
	\skillref[2]{divination},
	\skillref[1]{projection},
	\featref{vision-projection}
}{
	While you are using \featref{vision-projection}, you may form {\subtleinterfaces} with minds in the past or future.
	You can use these {\interfaces} along with feats such as \featref{projection-read-mind} and \featref{projection-read-senses-unwilling}, to gain information.
	Obviously, you cannot use them with feats such as \featref{projection-speak-unwilling} or \featref{projection-merge}, that would affect; you cannot change the past, or {\visions} of the future---only observe.
	
	You can use this to bypass a target's \featref{projection-interface-sense}.
	If they have \featref{projection-interface-block}, however, you cannot establish an {\interface} with them, unless they were intentionally leaving their mind open at the time.
}
