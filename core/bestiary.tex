\chapter{Bestiary}
\chaplabel{bestiary}

\section{Tame Animals}

Mankind has tamed a number of creatures, and even trained them to obey commands.
The following animals are suitable choices for an \featref{animal-companion}.

Some of these animals can certainly been found in the wild as well; these are simply those that are most commonly trained.

\creature{Dog}{dog}{
	\atttable{1}{1}{\negative 3}{\negative 2}{3}{2}{\negative 1}{2}
}{
	\speed{12}
}{
	\skillref[2]{intimidation}, \skillref[2]{perception}, \skillref[1]{weaponry}
}{
	Dogs have been changed extensively by their domestication, diversified into many breeds.
	They can be trained quite heavily, and even the average man's working dog will respond to a few commands.
	Certain breeds might have abilities beyond those listed here.
}{
	\ability{Bite}{
		The dog rolls 5 dice for unarmed {\damagetests}.
	}
}

\creature{Raptor (Eagle/Falcon/Hawk)}{raptor}{
	\atttable{\negative 1}{3}{\negative 3}{\negative 2}{2}{3}{\negative 3}{2}
}{
	\speed{2}, \flyspeed{20}
}{
	\skillref[2]{flying}, \skillref[2]{perception}, \skillref[1]{weaponry}
}{
	A raptor is a buzzard, eagle, falcon, harrier, hawk, kite or osprey; a bird of prey.
	They are often trained as sport hunters among the nobility, but can actually catch useful food for a skilled falconer.
}{
	\ability{Eagle Eyes}{
		The raptor rolls an extra die on \skillref{perception} Tests to see things at a long distance.
	}
	
	\ability{Beak \& Talons}{
		The raptor rolls 3 dice for unarmed {\damagetests}, or 4 dice when striking from a dive.
	}
}

\section{Wild Animals}

In the untamed wilderness, a witch might come across many mundane creatures, and even these can pose a major threat.
To a well-prepared witch, however, they might be food, or even friends.
The following animals are suitable choices for a \featref{animal-companion-wild}.

If a GM wants to use an animal that is not listed here, it can be very easy to adapt the statistics of a familiar.
Simply reduce its \attref{ken}, \attref{wit}, and possibly \attref{charm}, and remove any inappropriate skills.



\section{Extraordinary Herbs}

Many \herbtypeplural{5} possess some level of mobility or the ability to defend themselves; enough to qualify them as a creature.
A witch with \skillref[3]{botany} is assumed to have enough tricks and experience to keep these herbs under control, but someone else wandering into her garden might be less lucky.
The GM can also use these as inspiration for an adventure, or perhaps a short scene when a witch leaves her garden untended for too long, and has to bring it back into shape.
