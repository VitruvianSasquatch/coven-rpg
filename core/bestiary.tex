\chapter{Bestiary}
\chaplabel{bestiary}

\section{Creature Statistics}

\subsection{Missing Attributes}

Some creatures, by their nature, lack some attributes altogether.
They automatically fail any Test that would require this attribute.
If a creature lacks a \attref{grace} or \attref{heed} score, it has no \statref{dr}; an \actionref{attack} automatically hits it.
If it lacks a \attref{will} score, calculate its \statref{st} as though its \attref{will} were 0.
However, in this case, it is immune to {\shock}; it is only destroyed when its \statref{st} is reduced to 0.
If it lacks a \attref{might} score, it is very fragile, and destroying it is trivial.

As always, the GM should also employ some common sense in dealing with what may or may not affect a creature.
A creature lacking \attref{ken}, \attref{wit}, and \attref{will} scores has no mind; it won't be affected by talking to it, or by most \discref{projection}.
Most potions will have no affect on non-living creatures, and may not function properly against plants.
And so on---exceptions to the usual rules abound among the more mythical creatures of the world.

\section{Tame Animals}

Mankind has tamed a number of creatures, and even trained them to obey commands.
The following animals are suitable choices for an \featref{animal-companion}.

Some of these animals can certainly been found in the wild as well; these are simply those that are most commonly trained.

\creature{Dog}{Dogs}{dog}{
	\atttable{1}{1}{\negative 3}{\negative 2}{3}{2}{1}{2}
}{
	\speed{12}
}{
	\skillref[2]{intimidation}, \skillref[2]{perception}, \skillref[1]{weaponry}
}{
	\creaturerefplural{dog} have been changed extensively by their domestication, diversified into many breeds.
	They can be trained quite heavily, and even the average man's working \creatureref{dog} will respond to a few commands.
	Certain breeds might have abilities beyond those listed here.
}{
	\ability{Bite}{
		The \creatureref{dog} rolls 5 dice for \weaponref{unarmed} {\damagetests}.
	}
}

\creature{Horse}{Horses}{horse}{
	\atttable{5}{0}{\negative 3}{\negative 3}{3}{1}{\negative 3}{2}
}{
	\speed{25}
}{
	\skillref[1]{athletics}
}{
	\creaturerefplural{horse} possess great strength and speed, ensuring their widespread use---as mounts, pack animals, plough-pullers, and even meat.
	Most villages have at least a few trained \creaturerefplural{horse} around.
}{
	\ability{Hooves}{
		The \creatureref{horse} rolls 3 dice for \weaponref{unarmed} {\damagetests}.
	}
}

\creature[Raptor (Eagle/Falcon/Hawk)]{Raptor}{Raptors}{raptor}{
	\atttable{\negative 1}{3}{\negative 3}{\negative 2}{2}{3}{\negative 3}{2}
}{
	\speed{2}, \flyspeed{20}
}{
	\skillref[2]{flying}, \skillref[2]{perception}, \skillref[1]{weaponry}
}{
	A \creatureref{raptor} is a buzzard, eagle, falcon, harrier, hawk, kite or osprey; a bird of prey.
	They are often trained as sport hunters among the nobility, but can actually catch useful food for a skilled falconer.
}{
	\ability{Eagle Eyes}{
		The \creatureref{raptor} rolls an extra die on \skillref{perception} Tests to see things at a long distance.
	}
	
	\ability{Beak \& Talons}{
		The \creatureref{raptor} rolls 3 dice for \weaponref{unarmed} {\damagetests}, or 4 dice when striking from a dive.
	}
}

\section{Wild Animals}

In the untamed wilderness, a witch might come across many mundane creatures, and even these can pose a major threat.
To a well-prepared witch, however, they might be food, or even friends.
The following animals are suitable choices for a \featref{animal-companion-feral}.

If a GM wants to use an animal that is not listed here, it can be very easy to adapt the statistics of a familiar.
Simply reduce its \attref{ken}, \attref{wit}, and possibly \attref{charm}, and remove any inappropriate skills.

\creature{Bear}{Bears}{bear}{
	\atttable{5}{0}{\negative 3}{\negative 3}{4}{2}{\negative 3}{3}
}{
	\speed{15}
}{
	\skillref[1]{athletics}, \skillref[2]{intimidation}, \skillref[1]{perception}, \skillref[1]{weaponry}
}{
	There is very little more terrifying than a fast-oncoming \creatureref{bear}.
	Thankfully, they tend to be non-violent creatures unless provoked.
	But never threaten a mother \creaturerefpossessive{bear} cubs.
}{
	\ability{Bite \& Claws}{
		The \creatureref{bear} rolls 5 dice for \weaponref{unarmed} {\damagetests}.
	}
}
