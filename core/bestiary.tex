\chapter{Bestiary}
\chaplabel{bestiary}

Some creatures, by their nature, lack some attributes altogether.
They automatically fail any Test that would require this attribute.
If a creature lacks a \attref{grace} or \attref{heed} score, it has no \statref{dr}; an \actionref{attack} automatically hits it.
If it lacks a \attref{will} score, calculate its \statref{st} as though its \attref{will} were 0.
However, in this case, it is immune to {\shock}; it is only destroyed when its \statref{st} is reduced to 0.
If it lacks a \attref{might} score, it is very fragile, and destroying it is trivial.

As always, the GM should also employ some common sense in dealing with what may or may not affect a creature.
A creature lacking \attref{ken}, \attref{wit}, \attref{will} scores has no mind; it won't be affected by talking to it, or by most \discref{projection}.
Most potions will have no affect on non-living creatures, and may not function properly against plants.
And so on---exceptions to the usual rules abound among the more mythical creatures of the world.

\section{Tame Animals}

Mankind has tamed a number of creatures, and even trained them to obey commands.
The following animals are suitable choices for an \featref{animal-companion}.

Some of these animals can certainly been found in the wild as well; these are simply those that are most commonly trained.

\creature{Dog}{Dogs}{dog}{
	\atttable{1}{1}{\negative 3}{\negative 2}{3}{2}{1}{2}
}{
	\speed{12}
}{
	\skillref[2]{intimidation}, \skillref[2]{perception}, \skillref[1]{weaponry}
}{
	Dogs have been changed extensively by their domestication, diversified into many breeds.
	They can be trained quite heavily, and even the average man's working dog will respond to a few commands.
	Certain breeds might have abilities beyond those listed here.
}{
	\ability{Bite}{
		The dog rolls 5 dice for unarmed {\damagetests}.
	}
}

\creature{Horse}{Horses}{horse}{
	\atttable{5}{0}{\negative 3}{\negative 3}{3}{1}{\negative 3}{2}
}{
	\speed{25}
}{
	\skillref[1]{athletics}
}{
	Horses possess great strength and speed, ensuring their widespread use---as mounts, pack animals, plough-pullers, and even meat.
	Most villages have at least a few trained horses around.
}{
	\ability{Hooves}{
		The horse rolls 3 dice for unarmed {\damagetests}.
	}
}

\creature[Raptor (Eagle/Falcon/Hawk)]{Raptor}{Raptors}{raptor}{
	\atttable{\negative 1}{3}{\negative 3}{\negative 2}{2}{3}{\negative 3}{2}
}{
	\speed{2}, \flyspeed{20}
}{
	\skillref[2]{flying}, \skillref[2]{perception}, \skillref[1]{weaponry}
}{
	A raptor is a buzzard, eagle, falcon, harrier, hawk, kite or osprey; a bird of prey.
	They are often trained as sport hunters among the nobility, but can actually catch useful food for a skilled falconer.
}{
	\ability{Eagle Eyes}{
		The raptor rolls an extra die on \skillref{perception} Tests to see things at a long distance.
	}
	
	\ability{Beak \& Talons}{
		The raptor rolls 3 dice for unarmed {\damagetests}, or 4 dice when striking from a dive.
	}
}

\section{Wild Animals}

In the untamed wilderness, a witch might come across many mundane creatures, and even these can pose a major threat.
To a well-prepared witch, however, they might be food, or even friends.
The following animals are suitable choices for a \featref{animal-companion-wild}.

If a GM wants to use an animal that is not listed here, it can be very easy to adapt the statistics of a familiar.
Simply reduce its \attref{ken}, \attref{wit}, and possibly \attref{charm}, and remove any inappropriate skills.

\creature{Bear}{Bears}{bear}{
	\atttable{5}{0}{\negative 3}{\negative 3}{4}{2}{\negative 3}{3}
}{
	\speed{15}
}{
	\skillref[1]{athletics}, \skillref[2]{intimidation}, \skillref[1]{perception}, \skillref[1]{weaponry}
}{
	There is very little more terrifying than a fast-oncoming bear.
	Thankfully, they tend to be non-violent creatures unless provoked.
	But never threaten a mother bear's cubs.
}{
	\ability{Bite \& Claws}{
		The bear rolls 5 dice for unarmed {\damagetests}.
	}
}

\section{Extraordinary Herbs}

Many \herbtypeplural{5} possess some level of mobility or the ability to defend themselves; enough to qualify them as a creature.
A witch with \skillref[3]{botany} is assumed to have enough tricks and experience to keep these herbs under control, but someone else wandering into her garden might be less lucky.
The GM can also use these as inspiration for an adventure, or perhaps a short scene when a witch leaves her garden untended for too long, and has to bring it back into shape.

\creature{Grand Wormwood}{Grand Wormwoods}{grand-wormwood}{
	\atttable{4}{\nostat}{0}{0}{4}{\negative 2}{2}{5}
}{
	\speed{0}
}{
	\skillref[1]{botany}, \skillref[2]{intimidation}, \skillref[3]{persuasion}
}{
	Regular wormwood is a mere \herbtype{3}.
	However, in the right conditions it can be encouraged to grow stronger, fuller, and smarter, becoming the lordly and demanding \creatureref{grand-wormwood}.
	
	To cultivate \creatureref{grand-wormwood}, one must plant a ring of small wormwood plants around a previously established wormwood bush.
	These are to become its subjects.
	Then, one must honour the central bush, treating it like a king, ensuring it knows it is the grandest wormwood of all.
	It must receive the finest fertiliser, and regular watering.
	Bowing to the bush on every interaction with it certainly helps, and some gardeners even go so far as to sacrifice other plants or small animals to it, and lay them upon its roots.
	The wormwood's subject bushes suffer stunted growth as their energy goes to feed their king, and they need occasional replacement as they wither and die.
	
	A \creatureref{grand-wormwood} bush is distinguishable by the fact that it grows far taller, just over the height of a man, and develops a clear crown of leaves at its peak.
	Such distinguishing features are not particularly necessary, however, due to the sheer \attref{presence} of the plant---anyone who passes near it can simply feel that it is nearby, and can't help but find their eye drawn to admiring it.
	The weak-willed feel themselves compelled to serve the bush, humbled by its magnificence.
	A gardener must be very careful when cultivating the \creatureref{grand-wormwood}.
	Give it too little attention, and it shall wither away, regressing to regular wormwood.
	But a gardener who gives it too much soon finds herself a willing slave to the bush, spending every day bringing it more fertiliser, more sacrifices, and grovelling before it.
	
	The greatest threat of all comes from the interaction of two \creatureref{grand-wormwood} bushes.
	Each has been raised to know that it is the king of the wormwoods, the grandest in all the lands.
	Sensing another \creatureref{grand-wormwood}---and it might sense one from several gardens away---crushes this notion.
	In a lucky case, the weaker-willed bush simply dies of shame.
	Normally, however, everything ends in blood, sap and tears.
	Great swathes of countryside have been known to burn in the ensuing battles.
}{
	\ability{Wooden}{
		The \creatureref{grand-wormwood} has 4 \statref{res}.
		It cannot move---not even a limb---or speak.
	}
	
	\ability{Lord's Servants}{
		A creature encountering the \creatureref{grand-wormwood} for the first time, or serving it in some fashion, must succeed on a {\tn} 12 \testtype{will}{botany} Test or become the bush's servant.
		The creature serves the bush willingly and utterly, even to the complete neglect of its own well-being.
		It may repeat the Test when it first grows thirsty, when it first grows hungry, and when it first suffers sleep deprivation, breaking free on a success.
		It it fails each of these, it serves the bush until the neglect of its own health kills it.
		Others may attempt to talk or smack it out of its servitude, and isolation from the bush can help.
	}
	
	\ability{One Lord}{
		A \creatureref{grand-wormwood} bush can sense the existence of other \creatureref{grand-wormwood} bushes within about a kilometre.
		If it detects one, and both survive the initial shame, it enters a homicidal rage.
		The {\tn} to resist its Lord's Servants ability increases to 18, and it turns all its servants to eliminating the opposing \creatureref{grand-wormwood}.
		It cannot survive in this state for more than a few days.
		If both opposing bushes still stand at this time, they spontaneously combust.
	}
}

\creature{Ironwood}{Ironwoods}{ironwood}{
	\atttable{8}{\nostat}{\nostat}{\nostat}{\nostat}{\nostat}{\nostat}{\nostat}
}{
	\speed{0}
}{
	None
}{
	The \creatureref{ironwood} is a mighty tree.
	A fully grown specimen is over 100 metres tall, with a crown just as far across.
	It grows only in soils rich in iron ores, absorbing the metal to strengthen its wood and support its tremendous bulk.
	The wood is as hard as iron; it must be forged, not carved.
	
	Many doubt that the \creatureref{ironwood} is any more than a mindless plant, but such people have never tried to fell one.
	Anyone who takes an axe to its trunk suffers immediate retaliation.
	A single acorn---the size and weight of an anvil---falls from the canopy to land unerringly where they were standing.
	A fully grown tree has hundreds of these acorns, enough to crush even a large crew of loggers.
}{
	\ability{Iron}{
		The \creatureref{ironwood} is immune to mundane weapons, and to fire.
		Extraordinary methods are required to fell it.
		It cannot move---not even a limb---or speak.
	}
	
	\ability{Acorns Like Anvils}{
		Any attack against the \creatureref{ironwood} made by a creature under its broad canopy provokes an automatic and immediate response; an acorn dropped from its high branches.
		The \creatureref{ironwood} rolls \dice{3} to hit, and \dice[8]{6} for the {\damagetest}.
	}
}

\creature{Moly}{Molies}{moly}{
	\atttable{\nostat}{\nostat}{3}{5}{5}{3}{3}{3}
}{
	\speed{0}
}{
	\skillref[3]{botany}, \skillref[3]{projection}
}{
	The \creatureref{moly} (rhymes with holy, not holly) is a fairly small herb, much like a daffodil in size and appearance.
	Each plant grows a single stalk, with a few leaves and a single flower on top.
	However, its flowers are purest white, and its roots are deepest black.
	For most of its life, the \creatureref{moly} is a mundane plant.
	In dispersing its seeds, however, it shows great and terrifying magic.
	
	A \creatureref{moly} has a mind.
	This mind doesn't get up to much; the plant cannot move or see, so it has very little to think about.
	The mind is only detectable by \discref{projection}, a subtle presence in the {\mentalrealm}.
	It is not until the plant dies that the mind comes into its own.
	
	The \creatureref{moly} is naturally skilled in \discref{projection}.
	At the moment of the plant's death---when it is eaten, uprooted, or cut---the mind cuts free into the {\mentalrealm}.
	There, it {\possesses} the body of the creature that killed it.
	Using the {\possessed} creature, the \creatureref{moly} eats its old body, if it hasn't already.
	It then seeks out a suitable fertile location, and kills itself.
	The seeds within its belly soon germinate, and the next generation of \creaturerefplural{moly} flourish, using the creature's rotting corpse as compost.
	
	With the \creaturerefpossessive{moly} overwhelming ability in \discref{projection}, the only reliable way to harvest the plant and survive is to use another creature.
	Most witches will use a rat, or some sort of insect.
	The creature kills the plant and becomes {\possessed}, and the witch must snatch away the \creatureref{moly} before it is eaten.
	With no seeds to disperse, the {\possessed} creature is aimless, and soon dies.
}{
	\ability{Plant}{
		The \creaturerefpossessive{moly} plant body cannot move---not even a limb---or speak.
		While {\possessing} another creature, it can move, and even speak, just as normal.
	}
	
	\ability{Projection}{
		The \creaturerefpossessive{moly} mind automatically enters the {\mentalrealm} when its plant body dies.
		It cannot otherwise enter the {\mentalrealm} in any way.
		It needs no {\lifeline}, but can only survive a few minutes in the {\mentalrealm}.
		The \creatureref{moly} can, and will, \featref{projection-start-other-3} and {\possess} the creature that killed its plant body.
	}
}
