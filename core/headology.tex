\chapter{Headology}
\disclabel{headology}{Headologist}{Headologists}

\section{Feats}

\feat{Curse}{curse}{15}{
	None
}{
	It's a well known fact that someone who believes they will fail is more likely to do so.
	It doesn't take a drop of magic to make that true, but not everyone knows how to leverage it.
	You do.
	
	If someone believes that you have cursed them, or even if you can convince them that they have been cursed by something else, they suffer bad luck.
	Whenever they make a Test, dice that roll a 3 count towards a critical failure.
	The GM is also encouraged to make their critical failures a little more dire.
	This bad luck persists as long as the supposed curse is present in their minds; it might help to remind them now and again.
	
	This only applies if they believe they are under a rather broad curse, or specifically a bad luck curse.
	An overly specific curse---for example, ``May your crops wither in your fields,'' or ``May your nose fall from your face''---does nothing to focus their mind on their own failure and will have no effect.
}

\feat{Mind over Magic}{foil-magic}{15}{
	None
}{
	For all the magic circles and burning incense, magic ultimately comes from the mind.
	Not only do you know this, but you know \emph{how to exploit it}.
	
	If you can convince a practitioner of magic that their magic won't work, then it won't.
}

\feat{Doubt \& Despair}{foil-magic-2}{25}{
	\featref{foil-magic}
}{
	Under your tender care, even the smallest seed of doubt can flourish into a blossoming tree of failure.
	
	If you can make a practitioner so much as doubt the efficacy of their magic, or their own ability to work it, then the magic will either fail to work or, at the GM's option, backfire.
}

\feat{Mind Like a Razor}{headology-weapons-improvise}{10}{
	\featref{willing-tools-improvise}
}{
	If you can convince your foes that what you wield is a weapon, their flesh will believe you.
	You may treat an item you wield or throw as a \weaponref{knife}, \weaponref{hand-weapon} or \weaponref{thrown-weapon} (depending on its size and whether you're throwing it) if you can convince the target that it can cut (or otherwise deal damage) like one.
	A demonstration against an inanimate object, or another foe, will often suffice.
	Even your bare hands can cut like \weaponrefplural{knife} if you convince your foes that they can.
}

\feat{Change Blindness}{headology-stealth}{10}{
	None
}{
	You may hide in plain sight by leveraging the fact that people don't \emph{expect} to see you there.
	This uses a \testtype{charm}{stealth} Test.
	You must remain silent and quite still, though you may creep around slowly.
	
	In order to make use of this feat, anyone you are hiding from must have no reason to expect to see you, or anyone.
	If they see much out of place---a drawer opened or a vase knocked over---they might look for whoever did it and will immediately spot you.
	Furthermore, you can only use it if the people you are hiding from have some degree of familiarity with the location; they must have seen it before, at least.
	Somebody entering a room for the very first time doesn't know what to expect and will see it as it is, you included.
	
	Lastly, somebody seeing a group or crowd of people has no reason not to expect other people with them.
	This feat does not allow you to hide in such a situation, unless everyone in the group has the feat.
	%TODO: Is there a feat that helps blending in with a crowd?
}

\feat{Elsewhere}{headology-stealth-2}{15}{
	\featref{headology-stealth}
}{
	While \featref{headology-stealth} lets you hide from people who aren't expecting \emph{anyone}, you've now figured out how to hide from people who aren't expecting \emph{you}.
	As long as someone is convinced \emph{you} won't be somewhere---for instance, you've told them you'll be somewhere else---they won't see you there.
	Note that it is not enough for them not to expect you there---except as falls under the perview of \featref{headology-stealth}---they must expect you not to be there.
	
	This still requires a \testtype{charm}{stealth} Test, and you can't be too intrusive.
	For example, you shouldn't pass in front of something they are paying attention to, make any loud noises, or open any doors they are looking at.
	However, you might even be able to get away with moving things around.
	Even if somebody notices that something has been moved, they ought not to suspect \emph{you} to have done it, as long as they still believe you are somewhere else.
	
	Furthermore, this feat does allow you to go unnoticed in a crowd, as long as the person watching has good reason to believe you won't there.
}

\feat{Fake Sympathy}{headology-sympathetic-magic}{25}{
	\skillref[1]{sympathetic-magic},
	any feat giving a use for {\symlinks}
}{
	Although you know how to perform \discref{sympathetic-magic}, you've also figured out how to skimp on the magic and just use \discref{headology}.
	You may establish a fake {\symlink} just be convincing the target that you have established one.
	They need not understand the actual mechanisms of \discref{sympathetic-magic}---in fact, it's probably better if they don't---they just need to know that by affecting the {\symbol}, you can affect them.
	Establishing this fake link does not require the usual Test, only any Tests to convince the target.
	It does not count towards your maximum number of {\symlinks}
	It lasts as long as the target continues to believe it does---as such, it is not subject to {\stress}.
	
	You may transmit any effects along this fake link that you could along a normal {\symlink}---anything you possess the appropriate \discref{sympathetic-magic} feat for.
	However, you may only do so by showing the target what you are doing, and even explaining it if necessary.
	For example, \featref{sympathetic-speak} is useless: if the target cannot hear the sounds anyway, they don't know what to expect, and receive nothing.
	
	This {\symlink} doesn't actually exist in any sense, so you cannot modify it in any way you could modify a normal {\symlink}.
	However, nor can anybody else, and it is not impeded by anything that would impede a normal {\symlink}, unless the target is aware of and believes in such impedance.
}

\feat{Placebo}{headology-brewing}{15}{
	Any \discref{brewing} feat
}{
	Often, the promise of a cure is more important than the cure itself.
	You can save a lot of time brewing this way, if you just talk to people.
	
	If you know how to make a brew, and have a mixture of approximately the right size, consistency, and colour, you might be able to use that instead.
	If you can convince someone that what they're taking will have the effect of that brew, then it acts as that brew for them.
	This works not only with brews that you have a feat to make, but also the same minor remedies that you might otherwise make with a \skillref{brewing} Test.
	However, if a brew requires a feat to make, and you don't have that feat, this won't work.
}
