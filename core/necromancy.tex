\chapter{Necromancy}
\disclabel{necromancy}{Necromancer}{Necromancers}

\section{Reanimation and Resurrection}

Many \practitioners{necromancy} draw a distinction between reanimation and resurrection.
Reanimation is a crude process, somewhat akin to \discref{golemancy}.
It's nothing more than the application of raw animating force to a corpse, to stand it up and get it shuffling around again.
The creature retains its instincts, its muscle memory and the like, but that's as much through what is left of its biology as it is through the will that animated it.
The results of reanimation are known as the undead.

Resurrection, by contrast, brings a creature back back to life, in full.
If the creature had a soul, it is restored to the body.
There may be a few ill effects of the process, not to mention whatever killed it in the first place, but these can typically be recovered from.
For all intents and purposes, the creature is just as much alive as it was in the first place.

In theory, at least.

The trouble is that nobody has ever achieved true resurrection.
Dozens of witches and hundreds of charlatans have all claimed to.
Many have even come incredibly close, but there has always been some slight snag.
The search, of course, continues, but many have given up all hope that it is possible.

\section{Phylacteries}
\seclabel{phylacteries}

Resurrection, even to the limited extent that it is possible, requires the return of the creature's soul.
While an experienced \practitioner{necromancy} might reach through the Veil between worlds to pluck the soul from whatever afterlife it may be enjoying, it can be a lot easier to keep the soul shackled to the mortal realm.
Such is the purpose of a {\phylactery}.

A {\phylactery} is an object into which a shard of a person's soul has been bound, enchanted so that the rest of the soul will join it when it would otherwise pass on.
{\phylacteries} must be created from a clay jar, at least the size of a fist but possibly larger.
They are no more robust than the jars they are created from, and their destruction frees the shard of soul within.
The destruction of a {\phylactery} is always felt by the person whose soul it contains, just as the death of a familiar, but otherwise carries no ill effects.
If the {\phylactery} is destroyed after the person has died and their entire soul has passed into it, their soul is released to pass on to the afterlife.

A person can have no more than one {\phylactery} at a time, and the previous one must always be destroyed before a new one can be created.
Likewise, a single object cannot be the {\phylactery} for more than one person at a time.

It should be noted that a {\phylactery} does nothing to \emph{prevent} a person's death; it only makes it easier to restore them afterwards.
However, as long as a witch's soul remains in the {\phylactery} and does not depart this realm, the witch's death does not kill her familiar.

\section{The Lurching Dead}

The reanimation of dead creatures, as practiced by so many \practitioners{necromancy}, is a process filled with limitations and drawbacks.
Many of these can be overcome by an experienced \practitioner{necromancy}, but, for the beginner, the rules of such creatures are presented here.

Reanimation comes in many forms: \undeadrefplural{zombie}, \undeadrefplural{skeleton} and more.
Each comes with its own associated changes to the statistics of a creature, listed in the following sections.
Many changes, however, are shared between all reanimated creatures and are listed here.

Most reanimations do nothing to heal {\damage} to the corpse: both {\damage} suffered before and during death, and any further {\damage} done to the corpse since then.
If this reduces its \statref{st} to 0 or below, the corpse is too mangled to successfully reanimate.
Some reanimations also require an unrotted corpse.
It usually takes a little over a week before a corpse becomes too rotted for such a reanimation, although temperature and moisture can alter this.
Corpses can be preserved by {\embalming}.

A reanimated creature loses its memory and identity.
It remembers general skills such as how to hunt, but forgets such information as the location of its den, and loses any mannerisms that distinguished it in life.

Most reanimated creatures also lack many biological processes that they had in life.
They do not need to breathe, eat, drink or sleep.
They are immune to poisons, diseases and the like.
Additionally, they cannot heal themselves or be healed, and are unaffected by beneficial potions and such.
Lastly, they cannot produce any venom or other such substances, so any benefit of a venomous bite, sting or so on is lost.

For 24 hours after reanimation, the reanimated creature is under your control.
You can assert direct control over it, or give it general instructions which it will carry out to the best of its ability.
You control it mentally, without need for verbal instructions or gestures, however you can only provide instructions while you can see, hear or otherwise sense it.
You can only have control over one reanimated corpse at a time; reanimating another one will free the previous as though the 24 hours had expired.

When your control over the creature expires, it regains free will and begins to act as an animal of its kind normally would.
However, it is ravenously and insatiably hungry.
As such, it is generally considered good practice to put the creature down before this occurs.

\subsection{Zombies}
\undeadlabel{Zombie}{Zombies}{zombie}

A \undeadref{zombie} is the simplest reanimation possible; the corpse, fully clothed in its own flesh, is simply stood up and walked around as it is.
Is it a clumsy creature, with most of the mind rotted away as well, and it only grows worse as the corpse rots further.

A corpse reanimated as a \undeadref{zombie} loses 2 points from all attributes except \attref{might} and \attref{will}.
Its \statref{speed} is halved.
Its \statref{st} increases by 2, however.
If it could fly, it is now too clumsy to do so.

A corpse reanimated as a \undeadref{zombie} is not healed of any {\damage}.
The reanimation also requires that the corpse is unrotted.
Furthermore, it does nothing to slow the rot.
A \undeadref{zombie} that rots too far loses animation.

\subsection{Ghouls}
\undeadlabel{Ghoul}{Ghouls}{ghoul}

A \undeadref{ghoul} retains greater mental and physical faculties than a \undeadref{zombie}, but this comes at a dangerous price.
A \undeadref{ghoul} is sustained only by consuming the flesh of its own kind.
For example, a rabbit \undeadref{ghoul} must consume rabbit flesh, and a human \undeadref{ghoul} must consume human flesh.

A corpse reanimated as a \undeadref{ghoul} loses 2 points from its \attref{ken}, \attref{wit}, and \attref{charm} scores.
It retains the ability to fly, if it could in life.

A corpse reanimated as a \undeadref{ghoul} is not healed of any {\damage}.
However, an animated \undeadref{ghoul} may heal {\damage} by consuming the flesh of its own kind.
An entire corpse is sufficient to restore {\damage} equal to its maximum \statref{st}, with smaller portions restoring proportionally smaller amounts.
A ghoul can consume an entire corpse in less than a minute, and there is no end to its hunger: it could consume corpses for hours on end without being sated.

Reanimating a corpse as a \undeadref{ghoul} also requires that it is unrotted.
While animated and fed at least one full corpse each week, however, a \undeadref{ghoul} does not continue to rot.

Lastly, a \undeadrefpossessive{ghoul} unnatural hunger makes it harder to control than most undead.
It must be fed a full corpse at least once a week, or it always breaks free of the \practitionerpossessive{necromancy} control.
Even a \undeadref{ghoul} reanimated as a \undeadref{souled} is not immune to this: it goes insane with hunger after a week, and does not return to sanity until it has fed again.

\subsection{Draugar}
\undeadlabel{Draugr}{Draugar}{draugr}

With all the moisture drawn out of a corpse, it is not only prevented from rotting, but can also be freed from the bloated clumsiness that afflicts \undeadrefplural{zombie}.
The result is a \undeadref{draugr} (plural \undeadrefplural{draugr}).
The better preservation also grants it a better memory and senses.

A corpse reanimated as a \undeadref{draugr} loses 2 points from its \attref{wit}, \attref{charm}, and \attref{presence} scores.
It retains the ability to fly, if it could in life.

A corpse reanimated as a \undeadref{draugr} is not healed of any {\damage}.
The reanimation requires that the corpse is unrotted, and it also must be {\embalmed} by desiccation (drying out).
This usually is usually done using salt, but can happen naturally to creatures that die in deserts.

To remain animated, the \undeadref{draugr} must be kept dry; water bloats the corpse, starts it rotting again, and immediately ends its animation.
It might manage a 30 second sprint through light rain, but heavier rain is too much.
Given a heavy leather coat, it might just about be able to travel in rain, but the moisture will still get to it in a couple of hours.
\undeadrefplural{draugr} are often used to guard ancient tombs, sealed inside where water cannot intrude.

\subsection{Skeletons}
\undeadlabel{Skeleton}{Skeletons}{skeleton}

A \undeadref{skeleton} is the result of reanimating only the bones of a creature, the flesh rotted or carved away.
The bones arrange themselves in the air, supported by nothing but the will of the animating witch, and the creature's conviction in its own shape.
The result is a creature far less clumsy than a zombie, but not so resilient.

A corpse reanimated as a \undeadref{skeleton} loses 2 points from all attributes except \attref{grace} and \attref{will}.
Its \statref{st} is also reduced by 2, in addition to the loss from the reduced \attref{might}.
The mere bones of wings are not sufficient to allow it to fly, if it previously could.
It also sinks in water, but may move along the bottom.

Requiring only the bones, a \undeadref{skeleton} is not affected by most {\damage} sustained by the corpse.
Only a critical success on a {\damagetest}, or an intentional effort after death, will typically have broken any bones.
Likewise, it is not affected by {\damage} in the course of its undeath; any blow insufficient to scatter it across the floor is insufficient to scratch its bones.

A \undeadref{skeleton} lasts a long time without decomposing; at least a decade, and even longer if kept dry.

\subsection{Living Fossils}
\undeadlabel{Living Fossil}{Living Fossils}{fossil}

A \undeadref{fossil} is much like a \undeadref{skeleton}, except that the bones have been mineralised, impregnated with stone.
The essence of earth permeates the creature, strengthening it.

A \undeadref{fossil} uses the same rules as a \undeadref{skeleton}, except that its \attref{might} is not reduced.
Additionally, fossilised bones do not decay, lasting millennia and more.

\subsection{Haunts}
\undeadlabel{Haunt}{Haunts}{souled}
%TODO: Souled, Haunts or something else?

A \undeadref{souled} is the result of necromancy that is beginning to lift itself from mere reanimation towards the ideal of resurrection.
It is the result of imbuing a soul into a more conventionally reanimated undead such as a \undeadref{zombie} or \undeadref{skeleton}.
It is subject to the usual modifications to its statistics, as appropriate to the kind of reanimation.

However, a \undeadref{souled} retains its memories, identity and free will, and is not subject to the usual hunger.
It is not controlled by the \practitioner{necromancy} who reanimated it.
Furthermore, its \attref{might} and \attref{grace} are the only attributes subject to change; the other six are always unchanged.
It retains all its skills and feats.
It is still subject to all the benefits and detriments of its loss of biological processes, such as immunity to suffocation, disease and potions.
Lastly, it is still subject to usual rules for {\damage} and rotting, so may require \featref{undead-repair}.

Names for \undeadrefplural{souled} vary considerably, with many \undeadrefplural{souled} themselves finding the term unpleasant.
They may refer to themselves as the Souled, or using some other name.

\section{Embalming}
\seclabel{embalming}

Decomposition can be such a pain for a \practitioner{necromancy}, putting valuable corpses to waste.
Most corpses barely last more than a week before they are too rotted to make some kinds of undead, such as \undeadrefplural{zombie} and \undeadrefplural{ghoul}.
A \practitioner{necromancy} can always strip away the flesh and raise \undeadrefplural{skeleton}, but these might not suit her needs.
Instead, she might turn to {\embalming}.

Anyone can attempt to {\embalm} a corpse.
The process is a mixture of surgery, and treatment with substances that slow decomposition.
Various substances can be used, with varying effectiveness.
Soaking a corpse in strong alcohol can preserve it for a month or more.
Drying it with salt can preserve it indefinitely, as long as it is not wetted again.
Very long periods of preservation can be achieved with dedicated \featref{embalming-fluid}.

{\embalming} a corpse typically takes a few hours, and requires a \testtypespeciality{ken}{crafting}{Embalmer} Test.
Failure means that the corpse, or some parts of it, won't be preserved, or at least won't last as long as they could.
Particularly bad results can cause {\damage} to corpse.

{\embalming} does nothing to repair {\damage} to the corpse, or to reduce rotting that has already occurred.
It only slows or prevents further rotting.

\section{Feats}

\feat{Raise Zombie}{animate-zombie}{20}{
	None
}{
	You can restore a terrible facsimile of life to the bodies of deceased animals, reanimating it as a \undeadref{zombie}.
	For now, you are limited to animals at least as large as a mouse, and no larger than a medium-size dog such as a bloodhound.
	You can't manage a human or any animal that has been a familiar, due to interference from the link with its soul.
	
	\materials{An animal corpse, a \circleref{small}, a lit candle which the ritual extinguishes}
	
	The reanimation ritual takes five minutes, and must be performed in the dark.
}

\feat{Raise Skeleton}{animate-skeleton}{15}{
	\featref{animate-zombie}
}{
	After a few reanimations, most \undeadrefplural{zombie} are starting to come apart at the seams a bit.
	There comes a time when it's easier just to strip all the flesh off and make the bones stand up by themselves.
	You may reanimate the bones of an animal corpse as a \undeadref{skeleton}, subject to the same limitations as \featref{animate-zombie}.
	
	\materials{The bones of an animal corpse (with the flesh removed), a \circleref{small}, a lit candle which the ritual extinguishes}
	
	The reanimation ritual takes five minutes, and must be performed in the dark.
}

\feat{Raise Ghoul}{animate-ghoul}{20}{
	\skillref[1]{necromancy},
	\featref{animate-zombie}
}{
	\undeadrefplural{ghoul} are faster and scarier than \undeadrefplural{zombie}, but also \emph{hungrier}.
	You may reanimate an animal corpse as a \undeadref{ghoul}, subject to the same limitations as \featref{animate-zombie}.
	
	\materials{An animal corpse, an additional corpse to be consumed by the \undeadref{ghoul}, a \circleref{small}, a lit candle which the ritual extinguishes}
	
	The reanimation ritual takes five minutes, and must be performed in the dark.
	At the conclusion of the ritual, the newly-arisen \undeadref{ghoul} must immediately be fed a complete corpse---of the same kind of animal as the ghoul---or it does not fall under the \practitionerpossessive{necromancy} control.
}

\feat{Raise Fossil}{animate-fossil}{20}{
	\skillref[1]{necromancy},
	\featref{animate-skeleton}
}{
	Fossilisation is naturally a slow process, but a dedicated \practitioner{necromancy} can accelerate the process.
	You may reanimate the bones of an animal corpse as a \undeadref{fossil}, subject to the same limitations as \featref{animate-zombie}.
	
	\materials{The bones of an animal corpse (with or without flesh) buried in a bog, a \circleref{small}, a small heap of finely crushed rock, a lit candle which the ritual extinguishes}
	
	Beginning the reanimation ritual requires five minutes, but the \undeadref{fossil} does not rise for 24 hours.
	For the entire 24 hours, the candle must remain lit, the \materialref{ritual-circle} must remain intact, and the area must remain dark.
	The witch need not be present for the whole duration, however.
	
	Over the course of the 24 hour period, the rock dust is drawn into the bog and incorporated into the bones, and any remaining flesh rots away.
	At the conclusion, the \undeadref{fossil} is animated and claws its way to the surface.
}

\feat{Maintain Control}{undead-maintain-control}{10}{
	\featref{animate-zombie}
}{
	You can reassert control over a reanimated creature you already control, resetting the time before your control expires.
	
	\materials{A reanimated creature under your control, a \materialref{ritual-circle} of the same size required to initially animate the creature, a lit candle which the ritual extinguishes}
	
	The ritual takes five minutes, and must be performed in the dark.
	The reanimated creature must remain within the \materialref{ritual-circle} for the duration.
}

\feat{Deanimate}{deanimate}{10}{
	\featref{animate-zombie}
}{
	You can withdraw the animating force from a creature you have reanimated, returning it to death.
	
	\materials{A reanimated creature under your control, a \materialref{ritual-circle} of the same size required to initially animate the creature, an unlit candle which the ritual lights}
	
	The ritual takes five minutes, and must be performed in a brightly lit location.
	The reanimated creature must remain within the \materialref{ritual-circle} for the duration.
}

\feat{Stitches}{undead-repair}{10}{
	\featref{animate-zombie}
}{
	Many reanimations and resurrections are ineffective on corpses which are too badly damaged.
	You've figured out a way around that, with the right repairs.
	This usually involves stitching the missing bits back on, or gluing some bones.
	
	The repair and reanimation requires a Test, with the {\tn} determined by how badly damaged the corpse is, using your choice of \skillref{necromancy} or \skillref{healing}.
	A successful Test repairs at least enough {\damage} to restore the creature's \statref{st} to 1, and a high result may repair even more.
	You may only do this as part of a reanimation; you cannot repair a corpse while it is animated.
}

\feat{Scraps}{undead-repair-2}{10}{
	\skillref[1]{necromancy},
	\skillref[1]{healing},
	\featref{undead-repair}
}{
	You can do more than stitch a damaged corpse back together; you can stitch \emph{several} corpses together.
	You may assemble a corpse for reanimation out of parts from different corpses, stitched or otherwise attached together.
	A corpse assembled out of several different, but individually intact, parts can be much healthier than a corpse with several damaged and repaired parts.
	
	The pieces must all come from creatures of the same kind, and must be assembled to form a complete creature of that kind.
	The repair and reanimation still requires a Test, using your choice of \skillref{necromancy} or \skillref{healing}.
}

\feat{Darning}{undead-repair-animated}{15}{
	\skillref[2]{necromancy},
	\skillref[2]{healing},
	\featref{undead-repair}
}{
	You may make repairs to a corpse even while it is currently animated.
	Any Tests made to do so use \skillref{healing}.
	You may even reattach severed parts, though these must be the original parts unless you also have \featref{undead-repair-2}.
}

\feat{Major Undead}{undead-larger}{20}{
	\skillref[1]{necromancy},
	\featref{animate-zombie}
}{
	Larger bodies need more force to reanimate, but it's force you've learned to provide.
	When you perform a ritual to reanimate a creature, you may use a \circleref{medium} instead of a \circleref{small}, in order to ignore the upper size limit on the creature.
}

\feat{Sever Soul}{undead-human}{20}{
	\skillref[1]{necromancy},
	\featref{undead-larger}
}{
	Reanimating a creature that once possessed a soul has previously proven impossible, due to interference from the residual link.
	You've learned to sever these links, and hence reanimate these creatures.
	
	You may reanimate a human, or an animal that once a familiar, using any of the reanimation rituals provided by
	\featref{animate-zombie}, \featref{animate-skeleton} or \featref{animate-ghoul},
	provided you know them.
	You must use an iron blade as part of the ritual, to sever the link.
	Reanimating a human requires a \circleref{medium}, as per \featref{undead-larger}.
}

\feat{Undead Familiar}{reanimate-familiar}{10}{
	\featref{animate-zombie}
}{
	While a soul normally interferes with reanimating a creature, you've begun to figure out how to use it to your advantage, beginning on the path towards resurrection.
	Unfortunately, you can't actually summon any souls back to their bodies yet.
	Not to worry, though, for you have quite ready access to one soul in particular: your familiar's, so inextricably bound to your own.
	
	If your familiar dies and you can recover the corpse, you can reanimate it, paying no XP cost beside that required to purchase this feat in the first place.
	You may use any of the reanimation rituals provided by {\souledrituals}, provided you know them.
	It becomes a kind of \undeadref{souled} appropriate to the reanimation ritual used.
	
	Reanimating a familiar in this way does not prevent recovering it through the usual repetition of the binding ritual later (see \secref{familiar-injury-death}), although the normal XP cost must still be paid each time that method used.
}

\feat{Phylactery}{phylactery}{10}{
	\skillref[1]{necromancy},
	\featref{undead-larger},
	\featref{reanimate-familiar}
}{
	You've learned to restore your familiar's soul to its body in the event of its death.
	The next step is simply to perform the same procedure upon \emph{yourself}.
	This is complicated by the fact that you are dead, of course, so you ought to have a very good plan in place for pulling this off.
	Examples include a resurrection pact with a trustworthy friend who knows this same procedure, having your familiar do it (\featref{phylactery-familiar}), or ensuring you can stick around to do it yourself (\featref{projection-lifeline-phylactery}).
	
	Firstly, this feat allows you to extract a sliver of your own soul and place it in a {\phylactery}.
	The ritual to do so requires an hour, and must be performed in a dark place.
	It costs 10 XP each time you perform the ritual, as you extract another sliver of your soul.
	
	\materials{The clay jar to become the {\phylactery}, a drop of your own blood, \herb[deadly nightshade]{belladonna}{2}, a \circleref{medium}}
	
	Secondly, you have learned to use a {\phylactery} in a reanimation ritual.
	This must be the {\phylactery} containing the soul of the person whose body is being reanimated, but you may do this with anybody's {\phylactery}, not just your own.
	
	You may use any of the reanimation rituals provided by {\souledrituals}, provided you know them.
	The {\phylactery} takes the place of the candle in the ritual, which typically requires a \circleref{medium}, as per \featref{undead-larger}.
	The person is reanimated as a kind of \undeadref{souled} appropriate to the reanimation ritual used.
}

\feat{Familiar Resurrection}{phylactery-familiar}{10}{
	\skillref[1]{necromancy},
	\featref{phylactery}
}{
	Reanimating yourself is hard, what with being dead and all.
	So you've taught your familiar to do it for you.
	
	Your familiar can perform the reanimation ritual granted by \featref{phylactery}.
	It is only the link between the soul in your {\phylactery} and the sliver of the same soul in your familiar that affords it the magical intuition to do this, so it can only perform the ritual in order to reanimate \emph{you}.
}

\feat{Self-Sacrifice}{phylactery-lethal}{10}{
	\skillref[1]{necromancy},
	\featref{phylactery}
}{
	Dividing your soul is always costly, making the typical method of creating a {\phylactery} rather draining.
	Fortunately, it is possible to create a {\phylactery} without dividing your soul.
	It's simple, really---you just shift your \emph{entire} soul into the {\phylactery} at once.
	This is, of course, lethal.
	
	\materials{The clay jar to become the {\phylactery}, a deadly dose of \herb[deadly nightshade]{belladonna}{2}, a \circleref{medium}}
	
	The ritual requires only a minute, and must be performed in a dark place.
	It has no XP cost.
	It kills your body and traps your entire soul in the {\phylactery}.
	
	If you have \featref{projection-start-2} or \featref{projection-spike}, and an alternative {\lifeline}, you can leave your body in time to survive the ritual.
	If you have \featref{projection-lifeline-phylactery}, you may even transition seamlessly to using the {\lifeline} provided by the {\phylactery} you are creating.
}

\feat{Touching the Veil}{death-detection}{10}{
	None
}{
	When a soul departs our world for the next, its passage disrupts the Veil between worlds.
	A witch who knows what to look for can feel this disruption.
	
	You can feel where people have died, though this sense is damped by both distance and time.
	If you pass through the actual position of the death, you'll notice for up to about two weeks after it occurred.
	You automatically sense a death in the same room only for a few days after it's happened, and in the same house for only about a day.
	However, a Test can reveal slightly older or more distant deaths, if you are searching for them.
	You can't gain any information about the identity of the victim or the cause of death.
	
	Locations of mass or repeated death can leave their traces lingering for much longer.
	The site of a battlefield or sacrificial altar may be felt for many years after.
}

\feat{Medium}{medium}{10}{
	\featref{projection-start}
}{
	It is possible for the souls of the dead to {\possess} the bodies of the living, although most souls are not strong enough to force their way in.
	A specially prepared mind and body may invite them in, however.
	
	You may enter a mediumship trance by consuming \herb[stinking nightshade]{black henbane}{2} and meditating for a minute in a dark place.
	Upon doing so, you enter the {\mentalrealm} and may act as normal there.
	The trance lasts a few minutes, during which you cannot return to your body.
	
	While your body remains in trance, it is vulnerable to {\possession} by any nearby souls of the dead.
	These can include the souls of those who have died nearby, family and friends of you or other nearby people, those who took some special interest in you (such as your mortal enemies), or sometimes even randomly passing souls.
	A soul is aware of your identity, and must choose to {\possess} your body.
	
	The soul gains indefinite control of your body, using the normal rules for {\possession}.
	This makes {\possession} by a malevolent soul a very risky prospect.
	The soul retains all the memories it had in life, and memories of any experiences it has had on the mortal plane since then, but has no recollection of the afterlife.
	It is aware that it has died, and that some time has passed since it died, but has no idea how much.
	
	If you hope to gain information from the {\possessing} soul, you are advised to have an assistant to ask it questions, or at least to leave it a piece of paper with some questions and a quill.
	Mediums interested in limiting the harm a malevolent soul can wreak may be interested in a \featref{circle-contain}.
}

\feat{Piercing the Veil}{medium-death-location}{10}{
	\featref{medium},
	\featref{death-detection}
}{
	The point where a soul departed this world is the easiest point for it to return, and with a prepared mind you may reach out and offer it an invitation.
	If you enter a \featref{medium} trance at the location of a creature's death, as detected by \featref{death-detection}, you may offer that creature's soul an invitation to possess you.
	If it elects not to {\possess} you, you may end the trance early, before another soul has a chance to {\possess} you.
}
