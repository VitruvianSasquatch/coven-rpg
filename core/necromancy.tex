\discipline{Necromancy}{necromancy}{Necromancer}{Necromancers}

\section{Reanimation and Resurrection}

Many \practitioners{necromancy} draw a distinction between reanimation and resurrection.
Reanimation is a crude process, somewhat akin to \discref{golemancy}.
It's nothing more than the application of raw animating force to a corpse, to stand it up and get it shuffling around again.
The creature retains its instincts, its muscle memory and the like, but that's as much through what is left of its biology as it is through the will that animated it.
The results of reanimation are known as the undead.

Resurrection, by contrast, brings a creature back back to life, in full.
If the creature had a soul, it is restored to the body.
There may be a few ill effects of the process, not to mention whatever killed it in the first place, but these can typically be recovered from.
For all intents and purposes, the creature is just as much alive as it was in the first place.

In theory, at least.

The trouble is that nobody has ever achieved true resurrection.
Dozens of witches and hundreds of charlatans have all claimed to.
Many have even come incredibly close, but there has always been some slight snag.
The search, of course, continues, but many have given up all hope that it is possible.

\section{Reanimation Rituals \& the Undead}
\seclabel{reanimation-rituals}

Undead, the products of reanimation, come in many different forms: \undeadrefplural{zombie}, \undeadrefplural{skeleton}, \undeadrefplural{ghoul}, and more.
The rituals to reanimate these creatures are just as numerous, but they all share mark{\'e}d similarities.
And many \practitioners{necromancy}, who consider these rituals to form the heart and soul of the discipline, learn a lot of techniques for improving them.

For convenience, the feats which modify these rituals refer to them collectively as the {\reanimationrituals}, and the feats which provide them are listed here.
\begin{itemize}
	\item \featref{animate-zombie}
	\item \featref{animate-skeleton}
	\item \featref{animate-ghoul}
	\item \featref{animate-draugr}
	\item \featref{animate-sea-draugr}
	\item \featref{animate-fossil}
	\item \featref{animate-fire-skeleton}
	\item \featref{animate-shade}
	\item \featref{animate-wraith}
\end{itemize}
Obviously, you cannot perform any variant of a ritual unless you have the feat to perform that ritual in the first place.

\subsection{The Lurching Dead}

The statistics and capabilities of an undead creature are based upon the statistics and capabilities of the creature whose corpse it is raised from.
Each variety of undead comes with its own associated changes to these statistics, listed in the following sections.
Many changes, however, are shared between all reanimated creatures, and are listed here.

Most reanimations do nothing to heal {\damage} to the corpse: both {\damage} suffered before and during death, and any further {\damage} done to the corpse since then.
If this reduces its \statref{shock-threshold} to 0 or below, the corpse is too mangled to successfully reanimate.
Some reanimations also require an unrotted corpse.
It usually takes a little over a week before a corpse becomes too rotted for such a reanimation, although temperature and moisture can alter this.
Corpses can be preserved by {\embalming}.

A reanimated creature loses its memory and identity.
It remembers general skills such as how to hunt, but forgets such information as the location of its den, and loses any mannerisms that distinguished it in life.

Most reanimated creatures also lack many biological processes that they had in life.
They do not need to breathe, eat, drink, or sleep.
They are immune to poisons, diseases and the like.
Additionally, they cannot heal themselves or be healed, and are unaffected by potions and such.
They do not suffer from {\shock}, and are simply deanimated if they suffer a {\damagetest} equalling or exceeding their \statref{shock-threshold}.
Lastly, they cannot produce any venom or other such substances, so any benefit of a venomous bite, sting, or the like is lost.

For 24 hours after reanimation, the reanimated creature is under your control.
You can assert direct control over it, or give it general instructions which it will carry out to the best of its ability.
You control it mentally, without need for verbal instructions or gestures, however you can only provide instructions while you can see, hear, or otherwise sense it.
A witch who can raise an undead creature learns to control one, and only one, reanimated corpse at a time.
Reanimating a second one will free the previous as though the 24 hours had expired.
All your undead are also freed from your control when you die.

When your control over the creature expires, it regains free will.
It begins to act as an animal of its kind normally would, seeking whichever food it would normally eat.
However, it is eternally, ravenously, and insatiably hungry.
It's generally considered good practice to put the creature down before this occurs.

\undead{Zombie}{Zombies}{zombie}{
	A \undeadref{zombie} is the simplest reanimation possible; the corpse, fully clothed in its own flesh, is simply stood up and walked around as it is.
	Is it a clumsy creature, with most of the mind rotted away as well, and it only grows worse as the corpse rots further.
	
	A corpse reanimated as a \undeadref{zombie} loses 2 points from all attributes except \attref{might} and \attref{will}.
	Its \statref{speed} is halved.
	Its \statref{shock-threshold} increases by 2, however.
	If it could fly, it is now too clumsy to do so.
	
	A corpse reanimated as a \undeadref{zombie} is not healed of any {\damage}.
	The reanimation also requires that the corpse is unrotted.
	Furthermore, it does nothing to slow the rot.
	A \undeadref{zombie} that rots too far loses animation.
}

\undead{Ghoul}{Ghouls}{ghoul}{
	A \undeadref{ghoul} retains greater mental and physical faculties than a \undeadref{zombie}, but this comes at a dangerous price.
	A \undeadref{ghoul} is sustained only by consuming the flesh of its own kind.
	For example, a rabbit \undeadref{ghoul} must consume rabbit flesh, and a human \undeadref{ghoul} must consume human flesh.
	
	A corpse reanimated as a \undeadref{ghoul} loses 2 points from its \attref{ken}, \attref{wit}, and \attref{charm} scores.
	It retains the ability to fly, if it could in life.
	
	A corpse reanimated as a \undeadref{ghoul} is not healed of any {\damage}.
	However, an animated \undeadref{ghoul} may heal {\damage} by consuming the flesh of its own kind.
	An entire corpse is sufficient to restore {\damage} equal to its maximum \statref{shock-threshold}, with smaller portions restoring proportionally smaller amounts.
	A ghoul can consume an entire corpse in less than a minute, and there is no end to its hunger: it could consume corpses for hours on end without being sated.
	\undeadrefplural{ghoul} created from partial corpses, such as with \featref{undead-head}, need only eat their own mass to count it as a full corpse.
	
	Reanimating a corpse as a \undeadref{ghoul} also requires that it is unrotted.
	While animated and fed at least one full corpse each week, however, a \undeadref{ghoul} does not continue to rot.
	
	Lastly, a \undeadrefpossessive{ghoul} unnatural hunger makes it harder to control than most undead.
	It must be fed a full corpse at least once a week, or it always breaks free of the \practitionerpossessive{necromancy} control.
	Even a \undeadref{ghoul} reanimated as a \undeadref{souled} is not immune to this: it goes insane with hunger after a week, and does not return to sanity until it has fed again.
}

\undead{Draugr}{Draugar}{draugr}{
	With all the moisture drawn out of a corpse, it is not only prevented from rotting, but can also be freed from the bloated clumsiness that afflicts \undeadrefplural{zombie}.
	The result is a \undeadref{draugr} (plural \undeadrefplural{draugr}).
	The better preservation also grants it a better memory and senses.
	
	A corpse reanimated as a \undeadref{draugr} loses 2 points from its \attref{wit}, \attref{charm}, and \attref{presence} scores.
	It retains the ability to fly, if it could in life.
	
	A corpse reanimated as a \undeadref{draugr} is not healed of any {\damage}.
	The reanimation requires that the corpse is unrotted, and it also must be {\embalmed} by desiccation (drying out).
	This usually is usually done using salt, but can happen naturally to creatures that die in deserts.
	
	To remain animated, the \undeadref{draugr} must be kept dry; water bloats the corpse, starts it rotting again, and immediately ends its animation.
	It might manage a 30 second sprint through light rain, but heavier rain is too much.
	Given a heavy leather coat, it might just about be able to travel through rain, but the moisture will still get to it in a couple of hours.
	The \undeadref{draugr} is aware of this limitation, and will avoid moisture when not under a \practitionerpossessive{necromancy} direct control.
	
	\undeadrefplural{draugr} are often used to guard ancient tombs, sealed inside where water cannot intrude.
}

\undead{Sea-Draugr}{Sea-Draugar}{sea-draugr}{
	The \undeadrefplural{sea-draugr} are, in many ways, the complete opposite of the regular \undeadrefplural{draugr}.
	Creatures of seas and lakes, they can only be animated from the corpses of those who died by drowning.
	They revel in their water-bloated flesh, lurking beneath the surface and dragging their prey to join them in their watery grave.
	
	A \undeadref{sea-draugr} is subject to the same rules as a regular \undeadref{draugr}.
	It also gains a swimming speed equal to its land speed, or half its flying speed, whichever is greater.
	
	However, instead of remaining dry, an \undeadref{sea-draugr} must remain soaked.
	It functions best when immersed in water, and begins to weaken about five minutes after it emerges.
	After about ten minutes, it dries out too much and completely loses animation.
	
	Regular wetting can extend this time; it might get half an hour in rain, or even an indefinite time if the rain is sufficiently torrential.
	\emph{Continuous} attention using \featref{willing-water-vapour}, or a couple of uses of \featref{willing-water-vapour-2} every 5 minutes, also suffices.
	However, it must be completely immersed for at least 8 hours each day.
	Just like a \undeadref{draugr}, a \undeadref{sea-draugr} is aware of this limitation, and will seek out water.
	
	As long as the corpse remains animated as a \undeadref{sea-draugr}, and sufficiently wetted, it will not continue to rot.
}

\undead{Skeleton}{Skeletons}{skeleton}{
	A \undeadref{skeleton} is the result of reanimating only the bones of a creature, the flesh rotted or carved away.
	The bones arrange themselves in the air, supported by nothing but the will of the animating witch, and the creature's conviction in its own shape.
	The result is a creature far less clumsy than a zombie, but not so resilient.
	
	A corpse reanimated as a \undeadref{skeleton} loses 2 points from all attributes except \attref{grace} and \attref{will}.
	Its \statref{shock-threshold} is also reduced by 2, in addition to the loss from the reduced \attref{might}.
	The mere bones of wings are not sufficient to allow it to fly, if it previously could.
	It also sinks in water, but may move along the bottom.
	
	Requiring only the bones, a \undeadref{skeleton} is not affected by most {\damage} sustained by the corpse.
	Only a critical success on a {\damagetest}, or an intentional effort after death, will typically have broken any bones.
	Likewise, it is not affected by {\damage} in the course of its undeath; any blow insufficient to scatter it across the floor is insufficient to scratch its bones.
	
	A \undeadref{skeleton} lasts a long time without decomposing; at least a decade, and even longer if kept dry.
}

\undead{Living Fossil}{Living Fossils}{fossil}{
	A \undeadref{fossil} is much like a \undeadref{skeleton}, except that the bones have been mineralised, impregnated with stone.
	The essence of earth permeates the creature, strengthening it.
	
	A \undeadref{fossil} uses the same rules as a \undeadref{skeleton}, except that its \attref{might} is not reduced.
	Additionally, fossilised bones do not decay, lasting millennia and more.
}

\undead{Blazing Skeleton}{Blazing Skeletons}{fire-skeleton}{
	A \undeadref{fire-skeleton} can only be made from the charred bones of a creature that died burning.
	Flames race across its bones as it walks, and its eye sockets blaze like the sun.
	It spreads destruction wherever it steps, leaving fire and ash in its wake.
	
	A \undeadref{fire-skeleton} mostly uses the same rules as a \undeadref{skeleton}, with a handful of differences.
	Firstly, the bones used to animate must be charred by fire.
	This should not be enough to break or destroy the bones, but damage caused by fire does not count as {\damage} against the animated \undeadref{fire-skeleton}.
	
	Secondly, the \undeadref{fire-skeleton} always burns, as long as it is animated.
	It burns without fuel, and without damaging itself---in fact, it is immune to all harm from fire and heat.
	The fire goes out if the \undeadref{fire-skeleton} loses animation.
	Conversely, the \undeadref{fire-skeleton} loses animation if the fire is extinguished, such as by being immersed in water.
	The fire is somewhat robust, however; it can survive moderate rain, simply causing the droplets to boil away.
	
	The \undeadrefpossessive{fire-skeleton} flames produce heat, and can combust things, just like normal flame.
	They will ignite most combustible materials they touch.
	A \undeadrefpossessive{fire-skeleton} \weaponref{unarmed} attacks also {\ignite} the target, at \dice{2}, or add 1 die of {\fire} to a target who is already burning.
}

\undead{Shade}{Shades}{shade}{
	\undeadrefplural{shade} bridge the gap between ghosts, and the other corporeal undead.
	They are formed from a creature's body, but always appear to be wreathed in shadow.
	Every part of them is dark: black or grey.
	Their facial features are indistinct, or even absent.
	
	Although it is formed from a creature's body, a \undeadref{shade} is insubstantial.
	It often finds its fingers passing straight through objects, like shadows flitting over them.
	Although this makes it harder to affect the world, it also affords the \undeadref{shade} a degree of protection.
	Swords can pass right through it, without even disturbing it.
	
	Light, however, brings the \undeadref{shade} form, clarity.
	This makes it vulnerable.
	Worse still, sunlight can burn it away entirely, destroying it.
	
	A \undeadref{shade} loses 2 points from its \attref{ken}, \attref{charm}, and \attref{presence} scores.
	Furthermore, its insubstantial nature causes it to lose 5 points from its \attref{might} score.
	However, it suffers no penalties to vision in low-light conditions, or even complete darkness.
	And, in complete darkness, it is immune to all physical harm.
	
	Light, even dim light, makes the \undeadref{shade} vulnerable again.
	If the light falls only on part of its body, only that part is vulnerable.
	Sunlight, however, is worse.
	The \undeadref{shade} suffers a \dice{5} {\damagetest} every {\round} that it is exposed to direct sunlight.
	Reducing exposure can reduce the number of dice rolled for the {\damagetest}, but even if it wrapped entirely in thick black cloth, leaving just its eyes exposed so that it might see causes it to suffer a \dice{1} {\damagetest} every round.
	
	Reanimating a corpse as a \undeadref{shade} requires that it is unrotted, but it does not continue to rot while it is animated.
	When the \undeadref{shade} is deanimated, it leaves the corpse fully corporeal again, albeit with a slightly dark pallor, and still affected by any {\damage} the \undeadref{shade} suffered.
	If the \undeadref{shade} is deanimated in sunlight, however, it burns away entirely, leaving no corpse---not even ash.
}

\undead{Wraith}{Wraiths}{wraith}{
	A \undeadref{wraith} is the invention of a foul \practitioner{necromancy} from a bygone era.
	She sacrificing dozens of people to the darkness of a \creatureref{stygian-nightshade}, then recovered their flayed corpses for reanimation.
	The result was a variety of \undeadref{shade} that carried the \creaturerefpossessive{stygian-nightshade} wicked claws, able to rend flesh despite their intangibility.
	The process has been refined since, and needs nothing more than a sprig of \creatureref{stygian-nightshade}.
	However, it still only works on the corpses of creatures that died violent deaths.
	
	A \undeadref{wraith} appears just like a \undeadref{shade}, and uses all the same rules, except for two differences.
	Firstly, the \undeadref{wraith} can affect the physical world with full force; it loses no \attref{might}.
	
	Secondly, it sprouts wicked claws of stygian darkness, increasing the number of dice it rolls for \weaponref{unarmed} {\damagetests}.
	A creature without an effective attack gains one, and rolls 2 dice, while a creature with an existing attack rolls at least 3 dice.
	A human, or other creature with proper hands, rolls 5 dice.
}

\undead{Haunt}{Haunts}{souled}{
	%TODO: Souled, Haunts or something else?
	A \undeadref{souled} is the result of necromancy that is beginning to lift itself from mere reanimation towards the ideal of resurrection.
	It is the result of imbuing a soul into a more conventionally reanimated undead such as a \undeadref{zombie} or \undeadref{skeleton}.
	It is subject to the usual modifications to its statistics, as appropriate to the kind of reanimation.
	
	However, a \undeadref{souled} retains its memories, identity and free will, and is not subject to the usual hunger.
	It is not controlled by the \practitioner{necromancy} who reanimated it.
	Furthermore, its \attref{might} and \attref{grace} are the only attributes subject to change; the other six are always unchanged.
	It retains all its skills and feats.
	It is still subject to all the benefits and detriments of its loss of biological processes, such as immunity to suffocation, disease and potions.
	Lastly, it is still subject to usual rules for {\damage} and rotting, so may require \featref{undead-repair}.
	
	Names for \undeadrefplural{souled} vary considerably, with many \undeadrefplural{souled} themselves finding the term unpleasant.
	They may refer to themselves as the Souled, or using some other name.
}

\section{Embalming}
\seclabel{embalming}

Decomposition can be such a pain for a \practitioner{necromancy}, putting valuable corpses to waste.
Most corpses barely last more than a week before they are too rotted to make some kinds of undead, such as \undeadrefplural{zombie} and \undeadrefplural{ghoul}.
A \practitioner{necromancy} can always strip away the flesh and raise \undeadrefplural{skeleton}, but these might not suit her needs.
Instead, she might turn to {\embalming}.

Anyone can attempt to {\embalm} a corpse.
The process is a mixture of surgery, and treatment with substances that slow decomposition.
Various substances can be used, with varying effectiveness.
Soaking a corpse in strong alcohol can preserve it for a month or more.
Drying it with salt can preserve it indefinitely, as long as it is not wetted again.
Very long periods of preservation can be achieved with dedicated \featref{embalming-fluid}.

{\embalming} a corpse typically takes a few hours, and requires a \testtypespeciality{ken}{crafting}{Embalmer} Test.
Failure means that the corpse, or some parts of it, won't be preserved, or at least won't last as long as they could.
Particularly bad results can cause {\damage} to corpse.

{\embalming} does nothing to repair {\damage} to the corpse, or to reduce rotting that has already occurred.
It only slows or prevents further rotting.

\section{Phylacteries}
\seclabel{phylacteries}

Resurrection, even to the limited extent that it is possible, requires the return of the creature's soul.
While an experienced \practitioner{necromancy} might reach through the Veil between worlds to pluck the soul from whatever afterlife it may be enjoying, it can be a lot easier to keep the soul shackled to the mortal realm.
Such is the purpose of a {\phylactery}.

A {\phylactery} is an object into which a shard of a person's soul has been bound, enchanted so that the rest of the soul will join it when it would otherwise pass on.
{\phylacteries} must be created from a clay jar, at least the size of a fist but possibly larger.
They are no more robust than the jars they are created from, and their destruction frees the shard of soul within.
The destruction of a {\phylactery} is always felt by the person whose soul it contains, just as the death of a familiar, but otherwise carries no ill effects.
If the {\phylactery} is destroyed after the person has died and their entire soul has passed into it, their soul is released to pass on to the afterlife.

A person can have no more than one {\phylactery} at a time, and the previous one must always be destroyed before a new one can be created.
Likewise, a single object cannot be the {\phylactery} for more than one person at a time.

It should be noted that a {\phylactery} does nothing to \emph{prevent} a person's death; it only makes it easier to restore them afterwards.
However, as long as a witch's soul remains in the {\phylactery} and does not depart this realm, the witch's death does not kill her familiar.

\section{Feats}

\feat{Raise Zombie}{animate-zombie}{20}{
	\noprereq
}{
	You can restore a terrible facsimile of life to the bodies of deceased animals, reanimating it as a \undeadref{zombie}.
	For now, you are limited to animals at least as large as a mouse, and no larger than a medium-size dog such as a bloodhound.
	You can't manage a human, or any animal that has been a familiar, due to interference from the link with its soul.
	
	\materials{An animal corpse, a \circleref{small}, a lit candle which the ritual extinguishes}
	
	The reanimation ritual takes five minutes, and must be performed in the dark.
}

\feat{Raise Skeleton}{animate-skeleton}{15}{
	\featref{animate-zombie}
}{
	After a few reanimations, most \undeadrefplural{zombie} are starting to come apart at the seams a bit.
	There comes a time when it's easier just to strip all the flesh off and make the bones stand up by themselves.
	You may reanimate the bones of an animal corpse as a \undeadref{skeleton}, subject to the same limitations as \featref{animate-zombie}.
	
	\materials{The bones of an animal corpse (with the flesh removed), a \circleref{small}, a lit candle which the ritual extinguishes}
	
	The reanimation ritual takes five minutes, and must be performed in the dark.
}

\feat{Raise Ghoul}{animate-ghoul}{20}{
	\skillref[1]{necromancy},
	\featref{animate-zombie}
}{
	\undeadrefplural{ghoul} are faster and scarier than \undeadrefplural{zombie}, but also \emph{hungrier}.
	You may reanimate an animal corpse as a \undeadref{ghoul}, subject to the same limitations as \featref{animate-zombie}.
	
	\materials{An animal corpse, an additional corpse to be consumed by the \undeadref{ghoul}, a \circleref{small}, a lit candle which the ritual extinguishes}
	
	The reanimation ritual takes five minutes, and must be performed in the dark.
	At the conclusion of the ritual, the newly-arisen \undeadref{ghoul} must immediately be fed a complete corpse---of the same kind of animal as the ghoul---or it does not fall under the \practitionerpossessive{necromancy} control.
}

\feat{Raise Draugr}{animate-draugr}{20}{
	\skillref[1]{necromancy},
	\featref{animate-zombie}
}{
	By animating corpses as \undeadrefplural{draugr}, you can keep them around longer than mere \undeadrefplural{zombie}.
	You may reanimate a desiccated animal corpse as a \undeadref{draugr}, subject to the same limitations as \featref{animate-zombie}.
	
	\materials{An animal corpse {\embalmed} by desiccation (drying out), a \circleref{small}, a lit candle which the ritual extinguishes}
	
	The reanimation ritual takes five minutes, and must be performed in the dark.
}

\feat{Raise Sea-Draugr}{animate-sea-draugr}{10}{
	\skillref[1]{necromancy},
	\featref{animate-draugr}
}{
	A small variation on the ritual to animate \undeadrefplural{draugr} lets you create the opposite.
	You may reanimate a drowned, soaked animal corpse as a \undeadref{sea-draugr}, subject to the same limitations as \featref{animate-zombie}.
	
	\materials{A water-soaked animal corpse that died by drowning, a \circleref{small}, a lit candle which the ritual extinguishes}
	
	The reanimation ritual takes five minutes, and must be performed in the dark.
	The corpse must be kept soaked during the ritual, so keep a few buckets of water handy.
	And make sure the \materialref{ritual-circle} won't be washed away.
}

\feat{Raise Fossil}{animate-fossil}{20}{
	\skillref[1]{necromancy},
	\featref{animate-skeleton}
}{
	Fossilisation is naturally a slow process, but a dedicated \practitioner{necromancy} can accelerate the process.
	You may reanimate the bones of an animal corpse as a \undeadref{fossil}, subject to the same limitations as \featref{animate-zombie}.
	
	\materials{The bones of an animal corpse (with or without flesh) buried in a bog, a \circleref{small}, a small heap of finely crushed rock, a lit candle which the ritual extinguishes}
	
	Beginning the reanimation ritual requires five minutes, but the \undeadref{fossil} does not rise for 24 hours.
	For the entire 24 hours, the candle must remain lit, the \materialref{ritual-circle} must remain intact, and the area must remain dark.
	The witch need not be present for the whole duration, however.
	
	Over the course of the 24 hour period, the rock dust is drawn into the bog and incorporated into the bones, and any remaining flesh rots away.
	At the conclusion, the \undeadref{fossil} is animated and claws its way to the surface.
}

\feat{Raise Blazing Skeleton}{animate-fire-skeleton}{20}{
	\skillref[2]{necromancy},
	\featref{raise-skeleton}
}{
	Playing with fire is dangerous, and playing with \undeadrefplural{fire-skeleton} is even worse.
	But you've decided it's worth the risk.
	You may reanimate the charred bones of an animal that died burning as a \undeadref{fire-skeleton}, subject to the same limitations as \featref{animate-zombie}.
	
	\materials{The charred bones of an animal that died burning (with the flesh removed), a \circleref{small}, a lit candle which the ritual extinguishes}
	
	The reanimation ritual takes five minutes, and must be performed in the dark.
	The ritual ignites the bones, but it requires fire to do it.
	Normally the candle suffices, but if the candle is substituted for a {\phylactery}, a flame must still be provided.
}

\feat{Raise Shade}{animate-shade}{20}{
	\skillref[1]{necromancy},
	\featref{animate-zombie}
}{
	A \undeadref{shade} is a valuable tool in a \practitionerpossessive{necromancy} arsenal, silent and deadly in the dark.
	You may reanimate an animal corpse as a \undeadref{shade}, subject to the same limitations as \featref{animate-zombie}.
	
	\materials{An animal corpse, a \circleref{small}, a lit candle which the ritual extinguishes}
	
	The reanimation ritual takes five minutes, and must be performed in the dark, \emph{at night}.
}

\feat{Raise Wraith}{animate-wraith}{20}{
	\skillref[2]{necromancy},
	\featref{animate-shade}
}{
	Using the victim of a violent death, and a sprig of \creatureref{stygian-nightshade} placed in its mouth, you can raise a powerful and violent variety of \undeadref{shade}: a \undeadref{wraith}.
	You may reanimate an animal corpse as a \undeadref{wraith}, subject to the same limitations as \featref{animate-zombie}.
	
	\materials{The corpse of an animal which died violently, \herbcreature{stygian-nightshade}{5}, a \circleref{small}, a lit candle which the ritual extinguishes}
	
	The reanimation ritual takes five minutes, and must be performed in the dark, \emph{at night}.
}

\feat{Maintain Control}{undead-control}{10}{
	\featref{animate-zombie}
}{
	You can reassert control over a reanimated creature you already control, resetting the time before your control expires.
	
	\materials{A reanimated creature under your control, a \materialref{ritual-circle} of the same size required to initially animate the creature, a lit candle which the ritual extinguishes}
	
	The ritual takes five minutes, and must be performed in the dark.
	The reanimated creature must remain within the \materialref{ritual-circle} for the duration.
}

\feat{Assert Control}{undead-control-2}{15}{
	\skillref[1]{necromancy},
	\featref{undead-control}
}{
	You can use use the \featref{undead-control} ritual on an uncontrolled undead in order to bring it under your control.
	However, you cannot use this against undead that cannot normally be subject a \practitionerpossessive{necromancy} control, such as \undeadrefplural{souled}.
	Undead you take control of this way still count against the maximum number of undead you can control, and exceeding this limit will free an earlier undead from your control, just as raising a new one would.
}

\feat{Steal Control}{undead-control-3}{15}{
	\skillref[2]{necromancy},
	\featref{undead-control-2}
}{
	You can use the \featref{undead-control} ritual on an undead controlled by another \practitioner{necromancy}, breaking their hold and bringing it under your control.
	This is subject to the same limitations as \featref{undead-control-2}; you cannot use it on an undead that cannot be controlled by a \practitioner{necromancy}, and the undead counts against your control limit.
}

\feat{Deanimate}{deanimate}{10}{
	\featref{animate-zombie}
}{
	You can withdraw the animating force from a creature you have reanimated, returning it to death.
	
	\materials{A reanimated creature under your control, a \materialref{ritual-circle} of the same size required to initially animate the creature, an unlit candle which the ritual lights}
	
	The ritual takes five minutes, and must be performed in a brightly lit location.
	The reanimated creature must remain within the \materialref{ritual-circle} for the duration.
}

\feat{Offensive Deanimation}{deanimate-2}{20}{
	\featref{deanimate}
}{
	You have learned to draw the animation out of an undead whether you put it there or not.
	You can use the \featref{undead-control} ritual against any undead, be it under your control, under another \practitionerpossessive{necromancy} control, uncontrolled, or even a \undeadref{souled}.
}

\feat{Precision Control}{control-deanimate-small}{10}{
	\skillref[1]{necromancy},
	\featref{undead-large},
	\featref{undead-control} or \featref{deanimate}
}{
	You can use a \circleref{small} for the \featref{undead-control} or \featref{deanimate} rituals---assuming you have the feat to perform the ritual at all---regardless of the size of circle you would require to animate the creature in the first place.
	Note, however, that the undead must fit inside the circle, so you will need slightly bigger than a \circleref{small} for an elephant, or the like.
}

\feat{Rapid Control}{control-deanimate-fast}{20}{
	\skillref[2]{necromancy},
	\featref{undead-control} or \featref{deanimate}
}{
	You have become far faster at manipulating an undead's animating force---it can hardly be called a ritual anymore, though it still requires a \materialref{ritual-circle}.
	You can perform the \featref{undead-control} or \featref{deanimate} rituals in just one {\action}.
}

\feat{Stitches}{undead-repair}{10}{
	\featref{animate-zombie}
}{
	Many reanimations and resurrections are ineffective on corpses which are too badly damaged.
	By sealing wounds, stitching severed parts back on, and gluing bones together, you can solve this.
	Any parts you reattach must come from the original creature.
	
	The repair and reanimation requires a Test, with the {\tn} determined by how badly damaged the corpse is, using your choice of \skillref{necromancy} or \skillref{healing}.
	A successful Test repairs at least enough {\damage} to restore the creature's \statref{shock-threshold} to 1, and a high result may repair even more.
	You must perform the repairs while the corpse is dead; you cannot repair it while it is animated.
}

\feat{Scraps}{undead-repair-2}{10}{
	\skillref[1]{necromancy},
	\skillref[1]{healing},
	\featref{undead-repair}
}{
	You can do more than stitch a damaged corpse back together; you can stitch \emph{several} corpses together.
	When using \featref{undead-repair}, you may assemble the corpse to be animated out of parts from different corpses.
	A corpse assembled out of several individually intact parts can be healthier than a single, damaged corpse.
	
	The pieces must all come from creatures of the same kind; all from humans, all from dogs, and so on.
	They must be assembled to form a creature of that kind; you cannot make a six-legged dog.
}

\feat{Chimera}{undead-repair-3}{25}{
	\skillref[3]{necromancy},
	\skillref[2]{healing},
	\featref{undead-repair-2}
}{
	You have mastered the art of assembling corpses, creating horrifying, chimeric monstrosities.
	When using \featref{undead-repair-2}, the parts needn't all come from the same kind of animal.
	They needn't form a normal creature, either; you could stitch extra legs on to a dog.
	
	The creature can typically use replacement anatomy easily; for example, if you replace a human's arms with a bear's.
	New anatomy, however---an extra pair of limbs, for example---may take several hours, or even days to learn.
	Neither a human with an animal mouth, nor an animal with a human mouth, can speak properly.
	
	The GM may invent a set of statistics for the resulting creature, based upon the component creatures and modified, as usual, by its kind of undeath.
}

\feat{Darning}{undead-repair-active}{15}{
	\skillref[1]{necromancy},
	\skillref[2]{healing},
	\featref{undead-repair}
}{
	You may make repairs to a corpse even while it is currently animated.
	Any Tests made to do so use \skillref{healing}.
	You may even reattach severed parts, though these must be the original parts unless you also have \featref{undead-repair-2}.
	\featref{undead-repair-3} even allows you to attach parts from different kinds of creature.
}

\feat{Knitted Resurrection}{undead-repair-active-2}{15}{
	\skillref[2]{necromancy},
	\skillref[3]{healing},
	\featref{undead-repair-active}
}{
	You have discovered a means to resurrect dead tissue by attaching it to living tissue.
	This allows you to reattach severed parts to a person or animal.
	Any Tests made to do so use \skillref{healing}.
	
	This doesn't do much, if anything, to heal {\damage}; no more than normal surgery.
	The reattached parts, however, become living parts of the creature, for all intents and purposes.
	Some rot---up to about a week---is tolerable, though disgusting, and will be healed naturally after reattachment.
	
	The reattached parts must be the original parts, unless you also have \featref{undead-repair-2}.
	If you do have \featref{undead-repair-2}, however, you may replace failed organs, or broken limbs, with healthy ones from another creature.
	The target must remain alive throughout the entire process, so replacing a heart is incredibly difficult, and replacing a brain is impossible.
	\featref{undead-repair-3} even allows you to attach parts from different kinds of creature.
}

\feat{Major Undead}{undead-large}{20}{
	\skillref[1]{necromancy},
	\featref{animate-zombie}
}{
	Larger bodies need more force to reanimate, but it's force you've learned to provide.
	When you perform a {\reanimationritual}, you may use a \circleref{medium} instead of a \circleref{small}, in order to ignore the upper size limit on the creature.
	You still cannot reanimate a creature with a soul, such as a human; you need \featref{undead-human} to do so.
}

\feat{Undead Head}{undead-head}{10}{
	\featref{animate-zombie}
}{
	Rather than bothering to reanimate larger creatures, you can just reanimate parts of them.
	The head, specifically, the seat of consciousness.
	
	You may reanimate a creature's severed head using a {\reanimationritual}.
	The usual restrictions apply; for example, you cannot reanimate a human unless you also have \featref{undead-human}.
	However, when evaluating whether you need \featref{undead-large} and a \circleref{medium}, consider only the size of the creature's head, not the whole creature.
	As such, any head short of an elephant's only needs a \circleref{small}.
	
	The resulting creature has all the limitations you would expect from a severed head.
	It can't move, and can only bite people who put their hands in its mouth.
	It has no \attref{might} or \attref{grace} scores for most purposes, though it retains its \attref{might} score for calculating its \statref{shock-threshold}, and for biting.
	It can still see, hear, and so on, and vocalise or speak as it could in life.
}

\feat{Sever Soul}{undead-human}{20}{
	\skillref[1]{necromancy},
	\featref{undead-large} or \featref{undead-head}
}{
	Reanimating a creature that once possessed a soul has previously proven impossible, due to interference from the residual link.
	You've learned to sever these links, and hence reanimate these creatures.
	
	You may reanimate a human, or an animal that was once a familiar, using a {\reanimationritual}.
	You must use an iron blade as part of the ritual, to sever the link.
	
	Reanimating an entire human requires a \circleref{medium}, and \featref{undead-large}.
	Reanimating just a human head, using \featref{undead-head}, does not.
	
	A reanimated familiar has lost the link to its witch, and is now just a normal animal of its kind.
	See \featref{reanimate-familiar} to reanimate your familiar without losing this link.
}

\feat{Undead Familiar}{reanimate-familiar}{10}{
	\featref{animate-zombie}
}{
	While a soul normally interferes with reanimating a creature, you've begun to figure out how to use it to your advantage, beginning on the path towards resurrection.
	Unfortunately, you can't actually summon any souls back to their bodies yet.
	Not to worry, though, for you have quite ready access to one soul in particular: your familiar's, so inextricably bound to your own.
	
	If your familiar dies and you can recover the corpse, you can reanimate it, paying no XP cost beside that required to purchase this feat in the first place.
	You may use any {\reanimationritual}, and it becomes the appropriate kind of \undeadref{souled}.
	
	Reanimating a familiar in this way does not prevent recovering it through the usual repetition of the binding ritual later (see the section \secref{familiar-injury-death}), although the normal XP cost must still be paid each time that method used.
}

\feat{Phylactery}{phylactery}{10}{
	\skillref[1]{necromancy},
	\featref{reanimate-familiar},
	\featref{undead-large} or \featref{undead-head}
}{
	You've learned to restore your familiar's soul to its body in the event of its death.
	The next step is simply to perform the same procedure upon \emph{yourself}.
	This is complicated by the fact that you are dead, of course, so you ought to have a very good plan in place for pulling this off.
	Examples include a resurrection pact with a trustworthy friend who knows this same procedure, having your familiar do it (\featref{phylactery-familiar}), or ensuring you can stick around to do it yourself (\featref{projection-lifeline-phylactery}).
	
	Firstly, this feat allows you to extract a sliver of your own soul and place it in a {\phylactery}.
	The ritual to do so requires an hour, and must be performed in a dark place.
	It costs 10 XP each time you perform the ritual, as you extract another sliver of your soul.
	
	\materials{The clay jar to become the {\phylactery}, a drop of your own blood, \herb[deadly nightshade]{belladonna}{2}, a \circleref{medium}}
	
	Secondly, you have learned to use a {\phylactery} in a reanimation ritual.
	This must be the {\phylactery} containing the soul of the person whose body is being reanimated, but you may do this with anybody's {\phylactery}, not just your own.
	
	You may use any {\reanimationritual}, reanimating the person as the appropriate kind of \undeadref{souled}.
	The {\phylactery} takes the place of the candle in the ritual.
	The ritual typically requires a \circleref{medium} and \featref{undead-large}---unless you take advantage of \featref{undead-head}.
}

\feat{Familiar Resurrection}{phylactery-familiar}{10}{
	\skillref[1]{necromancy},
	\featref{phylactery}
}{
	Reanimating yourself is hard, what with being dead and all.
	So you've taught your familiar to do it for you.
	
	Your familiar can perform the {\reanimationritual} variant granted by \featref{phylactery}.
	It is only the link between the soul in your {\phylactery} and the sliver of the same soul in your familiar that affords it the magical intuition to do this, so it can only perform the ritual in order to reanimate \emph{you}.
}

\feat{Self-Sacrifice}{phylactery-lethal}{10}{
	\skillref[1]{necromancy},
	\featref{phylactery}
}{
	Dividing your soul is always costly, making the typical method of creating a {\phylactery} rather draining.
	Fortunately, it is possible to create a {\phylactery} without dividing your soul.
	It's simple, really---you just shift your \emph{entire} soul into the {\phylactery} at once.
	This is, of course, lethal.
	
	\materials{The clay jar to become the {\phylactery}, a deadly dose of \herb[deadly nightshade]{belladonna}{2}, a \circleref{medium}}
	
	The ritual requires only a minute, and must be performed in a dark place.
	It has no XP cost.
	It kills your body and traps your entire soul in the {\phylactery}.
	
	If you have \featref{projection-start-2} or a \featref{projection-potion}, and an alternative {\lifeline}, you can leave your body in time that your mind survives the ritual.
	If you have \featref{projection-lifeline-phylactery}, you may even transition seamlessly to using the {\lifeline} provided by the {\phylactery} you are creating.
}

\feat{Touching the Veil}{death-detection}{10}{
	\noprereq
}{
	When a soul departs our world for the next, its passage disrupts the Veil between worlds.
	A witch who knows what to look for can feel this disruption.
	
	You can feel where people have died, though this sense is damped by both distance and time.
	If you pass through the actual position of the death, you'll notice for up to about two weeks after it occurred.
	You automatically sense a death in the same room only for a few days after it's happened, and in the same house for only about a day.
	However, a Test can reveal slightly older or more distant deaths, if you are searching for them.
	You can't gain any information about the identity of the victim or the cause of death.
	
	Locations of mass or repeated death can leave their traces lingering for much longer.
	The site of a battlefield or sacrificial altar may be felt for many years after.
}

\feat{Medium}{medium}{10}{
	\featref{projection-start}
}{
	It is possible for the souls of the dead to {\possess} the bodies of the living, although most souls are not strong enough to force their way in.
	A specially prepared mind and body may invite them in, however.
	
	You may enter a mediumship trance by consuming \herb[stinking nightshade]{black henbane}{2} and meditating for a minute in a dark place.
	Upon doing so, you enter the {\mentalrealm} and may act as normal there.
	The trance lasts until your mind returns to your body, which it may do as normal.
	
	While your body remains in trance, it is vulnerable to {\possession} by any nearby souls of the dead.
	These can include the souls of those who have died nearby, family and friends of you or other nearby people, those who took some special interest in you (such as your mortal enemies), or sometimes even randomly passing souls.
	A soul is aware of your identity, and must choose to {\possess} your body.
	
	The soul gains indefinite control of your body, using the normal rules for {\possession}.
	This makes {\possession} by a malevolent soul a very risky prospect.
	The soul retains all the memories it had in life, and memories of any experiences it has had on the mortal plane since then, but has no recollection of the afterlife.
	It is aware that it has died, and that some time has passed since it died, but has no idea how much.
	
	If you hope to gain information from the {\possessing} soul, you are advised to have an assistant to ask it questions, or at least to leave it a piece of paper with some questions and a quill.
	Mediums interested in limiting the harm a malevolent soul can wreak may be interested in a \featref{circle-contain}.
}

\feat{Piercing the Veil}{medium-death-location}{10}{
	\featref{medium},
	\featref{death-detection}
}{
	The point where a soul departed this world is the easiest point for it to return, and with a prepared mind you may reach out and offer it an invitation.
	If you enter a \featref{medium} trance at the location of a creature's death, as detected by \featref{death-detection}, you may offer that creature's soul an invitation to possess you.
	If it elects not to {\possess} you, you may end the trance early, before another soul has a chance to {\possess} you.
}
