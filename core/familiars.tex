\chapter{Familiars}
\chaplabel{familiars}

A wizened old woman leans back in her rocking chair, eyes closed.
A white cat lies curled in her lap, its own eyes also shut, purring as she rubs its chin.

A handsome, tanned woman stands on the peak of a grassy hill, arm held aloft.
A falcon dives from above, alighting on her thick leather glove.
It casts its eyes north-west, then knowingly back at the witch.
With a sly grin, she casts the bird back into the air and strides downhill after it.

A bright-eyed girl, no more than thirteen, stands beside a bubbling cauldron, carefully teasing the seeds from a pine cone with a small knife.
``Sage leaf next, Harold?''
She looks up at the frog on the kitchen bench, as it croaks and nudgs one of the piles of herbs that surrounds it.
``Ohh, right. Rosemary. Of course{\dots}''
The girl shakes her head and tuts to herself as she counts out seven leaves into her hand and drops them in the cauldron.
Harold peers over from the bench, keeping a close eye on the brew as it slowly turns a deep blue.

\section{No Mere Beast}

A witch's familiar is no mere animal.
It is a fusion of summoned spirit, tamed beast, and a tiny sliver of soul from the witch herself.
Obtaining a familiar is one of the first steps for any witch-in-training, and the familiar often aids in the witch in her subsequent learning.

Familiars are intelligent creatures, in some cases even more intelligent than the witches they are bound to.
They understand language, though the limits of animal form mean that most are incapable of speech.
Despite this, the bond that a witch shares with her familiar allow them to communicate.
With simple looks and gestures a familiar can communicate great meaning to its witch, communicating as effectively as if through speech.
This ability does not extend to other witches, and especially not to layfolk, who may require a Test to interpret a familiar's communication.
Pointing and beckoning are typically fairly unambiguous, however.

A witch's communication with her familiar allows her to lean on its expertise when her own is lacking.
A witch may use her familiar's ranks in a skill in place of her own, as long as the Test takes at least a minute, and she can confer with her familiar through the duration.

\section{Binding a Familiar}

Binding a familiar takes place in a simple ritual, achievable by even the most novice witch, though often performed under direct tutelage.
The spirit to be bound to the familiar can be obtained in a number of ways: a lesser demon under contract, an amenable nature spirit or a spirit summoned from beyond.
It is not uncommon for a witch to use the spirit from the familiar of her teacher's teacher, or of an ancestor if witchcraft runs in her family.
The animal to become the familiar must be tamed by the witch, at least enough that it willing remains by her side throughout the ritual.
Many witches find this to be the hardest part of the ritual, and it means that some animals make for rather rare familiars.
Lastly, the witch must offer up a sliver of her own soul, to seal the bond.
She does so by feeding the familiar animal a drop of her own blood.

Upon completion of the ritual, the spirit and animal are fused to form a new entity, the familiar.
It takes on personality elements from both and the body of the animal.
Slight changes to its physical form often manifest, however, such as a coat that always remains strangely glossy, a slight chill to the touch, or sharper, whiter teeth.
Changes in eye colour are especially common.
Lastly, the sliver of the witch's soul included in the creation of a familiar also influences its personality.
It ensures that, although a witch and her familiar may not always get along, and may certainly disagree on the best way to go about something, they will always have one another's best interests at heart.

\section{Creating a Familiar}

From the perspective of character creation, there are many things to bear in mind when creating a familiar.
While the familiar is unlikely to take the foreground as much as the witch herself, they are still a character in their own right, and should be designed as such.

The most important decision is the form the familiar will take, the animal they were created from.
This determines the familiar's attributes, skills and abilities.
Note that familiars, as non-human characters, may have attributes below the human 0 to 5 range.

Beside its game statistics, it is also important to get an idea of your familiar as a character.
Try answering some of the following questions.

\begin{itemize}
	\item What is your familiar's name?
	\item Is your familiar male or female?
		Do you not know?
	\item At what stage in her life, and her training, did your witch bind her familiar?
	\item Which sort of spirit was used in the binding? %TODO: Remove this if it becomes mechanical.
	\item Do your witch and her familiar get along?
		Do they engage in playful banter?
		Philosophical debate?
	\item Does your familiar have any quirks, physical or mental?
\end{itemize}

Lastly, it is important to decide whether each familiar will be played by the player or the GM.
Both are valid, but if the GM is playing familiars they should typically act in their witch's best interests.

\section{Familiar Injury and Death}
\seclabel{familiar-injury-death}

Familiars suffer {\damage}, {\shock} and death just like other characters.
A witch is always aware when her familiar dies, feeling it as a searing pain in her very soul.

It is possible to recover a deceased familiar through a slight variant on the original binding ritual.
The familiar's spirit comes willingly, but another animal of the same kind must be provided.
The familiar, once reformed, may take either the new animal's appearance or its original one.

Repeating the ritual takes another sliver of the witch's soul, provided through another drop of blood.
As such, recovering a deceased familiar costs 10 XP every time.

\section{Familiar Animals}

A list of the types of animal available as familiars is presented below, along with the attributes, skills and abilities of the familiar.
Besides the abilities listed below, the players and GM are encouraged to apply common sense.
For instance, familiars lack thumbs and will struggle with door handles, and a weasel can squeeze through a smaller hole than a hound.

If you would like your familiar to be an animal not presented on the list below, discuss your option with your GM.
It might be possible to design a new familiar for you to use, or to use the statistics of a familiar presented here to represent something else.
Note that familiars are fairly small animals; the exclusion of anything larger than a medium-sized dog is intentional.

Many types of familiar, more powerful ones, come with an associated XP cost.
This is deducted from the witch's starting XP.
Some types of familiar also come with options which may be purchased for an additional XP cost.
These represent inherent differences in the animal used and must be purchased at the same time your familiar is created.
You may only select one option; they are mutually exclusive.

Lastly, bear in mind that some feats that can be purchased later depend upon particular types of familiar, and your familiar's later development is limited by its form.
As such, it can be worth taking a quick look at other feats you may be interested in taking when selecting your familiar.
%TODO: If those are all in a discipline chapter on familiars, direct people there.

\familiar{Cat}{cat}{15}{
	\atttable{\negative 1}{3}{2}{2}{2}{2}{3}{1}
}{
	\skillref[1]{athletics}, \skillref[1]{deception}, \skillref[1]{perception}, \skillref[2]{stealth}
}{
	Graceful and charming on the outside, cats can be incredibly sly and manipulative underneath.
	Just like many witches.
}{
	\familiarability{Natural Acrobat}{
		The cat rolls an extra die on Tests to retain its balance, land on its feet, or avoid damage from falling.
		%TODO: Reduce fall damage in some more definite fashion?
	}
	
	\familiarability{Claws}{
		The cat's unarmed attacks deal 4 dice of damage.
	}
}{}

\familiar{Dog}{dog}{15}{
	\atttable{1}{1}{1}{1}{3}{2}{0}{2}
}{
	\skillref[2]{perception}, \skillref[1]{weaponry}
}{
	A man's best friend, and often a witch's too.
	Dogs are a diverse lot, including hunting dogs, sheepdogs, sled dogs and more.
}{
	\familiarability{Bite}{
		The dog's unarmed attacks deal 5 dice of damage.
	}
}{
	\familiaroption{Scenthound}{5}{
		The dog rolls an extra die on \skillref{perception} skills relying on smell.
	}
}

\familiar{Ferret/Stoat/Weasel}{mustelid}{15}{
	\atttable{\negative 2}{3}{1}{1}{2}{2}{2}{0}
}{
	\skillref[1]{athletics}, \skillref[2]{stealth}, \skillref[1]{weaponry}
}{
	A ferret, stoat, weasel, polecat, ermine, mink or marten.
	Despite their small size, these creatures are ferocious predators.
	Their long, narrow bodies allow them to invade the burrows of much smaller animals, or the trousers of their witch's unfortunate foes.
}{
	\familiarability{Bite}{
		The ferret's unarmed attacks deal 4 dice of damage.
	}
	
	\familiarability{Slippery}{
		The ferret's \statref{dr} is increased by 2.
	}
}{}

\familiar{Frog/Toad}{frog}{5}{
	\atttable{\negative 2}{\negative 1}{1}{1}{2}{0}{\negative 1}{\negative 1}
}{
	\skillref[1]{brewing}
}{
	Frogs and toads make excellent companions to brewing witches, due to their natural affinity with water.
	Particularly with some of the stuff that gets into the murkier ponds around{\dots}
	
	It is important to try and keep their skin moist, but maybe refrain from dropping them in the cauldron.
}{
	\familiarability{Amphibian}{
		The frog can breathe underwater.
	}
	
	\familiarability{Leapfrog}{
		The frog can jump at least 3 metres from a standing start.
		It rolls an additional die on Tests made to jump.
	}
	
	%TODO: Something about keeping the skin moist? Maybe when there are exhaustion/fatigue rules.
}{}

\familiar{Raptor (Eagle/Falcon/Hawk)}{raptor}{25}{
	\atttable{\negative 1}{3}{1}{2}{2}{3}{\negative 1}{2}
}{
	\skillref[2]{flying}, \skillref[2]{perception}, \skillref[1]{weaponry}
}{
	A raptor is a buzzard, eagle, falcon, harrier, hawk, kite or osprey; a bird of prey.
	They are excellent fliers, have keen eyesight, and nobody would want to tangle with their wicked beak and talons.
}{
	\familiarability{Eagle Eyes}{
		The raptor rolls an extra die on \skillref{perception} Tests to see things at a long distance.
	}
	
	\familiarability{Beak \& Talons}{
		The raptor's unarmed attacks deal 4 dice of damage.
	}
}{}

\familiar{Rat/Mouse}{rat}{0}{
	\atttable{\negative 2}{1}{1}{1}{2}{1}{\negative 2}{\negative 1}
}{
	\skillref[1]{stealth}
}{
	The rat is a rather widely reviled animal, but it's certainly easy for a new witch looking for a familiar to find one.
	And it can get into smaller places than a cat or bird, which often proves helpful.
}{
	\familiarability{Filth-Liver}{
		The rat rolls an extra die on Tests to resist poison or disease.
	}
	
	\familiarability{Keen Smell}{
		The rat rolls an extra die on \skillref{perception} skills relying on smell.
	}
}{}

%\subsection{Beaver}

%Placeholder.

%\subsection{Bat}

%Placeholder.

%\subsection{Badger}

%Placeholder.

%\subsection{Cat}

%Placeholder.

%\subsection{Chicken}

%Placeholder.

%\subsection{Crow/Raven}

%Placeholder.

%\subsection{Dog}

%Placeholder.

%\subsection{Dove}

%Placeholder.

%\subsection{Duck}

%Placeholder.

%\subsection{Ferret/Stoat/Weasel}

%Placeholder.

%\subsection{Fox}

%Placeholder.

%\subsection{Frog/Toad}

%Placeholder.

%\subsection{Gecko}

%Placeholder.

%\subsection{Goose}

%Placeholder.

%\subsection{Hamster/Gerbil/Guinea Pig}

%Placeholder.

%\subsection{Hedgehog/Porcupine}

%Placeholder.

%\subsection{Lemming}

%Placeholder.

%\subsection{Lizard}

%Placeholder.

%\subsection{Mole}

%Placeholder.

%\subsection{Otter}

%Placeholder.

%\subsection{Owl}

%Placeholder.

%\subsection{Parrot}

%Placeholder.

%\subsection{Pigeon}

%Placeholder.

%\subsection{Rabbit/Hare}

%Placeholder.

%\subsection{Raptor}

%Placeholder.

%\subsection{Rat/Mouse}

%Placeholder.

%\subsection{Salamander/Newt}

%Placeholder.

%\subsection{Seabird}

%Placeholder.

%\subsection{Snake (Constrictor)}

%Placeholder.

%\subsection{Snake (Venomous)}

%Placeholder.

%\subsection{Songbird}

%Placeholder.

%\subsection{Spider}

%Placeholder.

%\subsection{Squirrel}

%Placeholder.

%\subsection{Swan}

%Placeholder.

%\subsection{Tortoise}

%Placeholder.

%\subsection{Vole/Shrew/Gopher}

%Placeholder.

%\subsection{Woodpecker}

%Placeholder.
