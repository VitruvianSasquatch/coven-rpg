\chapter{Familiars}
\chaplabel{familiars}

\dropcap{A} wizened old woman leans back in her rocking chair, eyes closed.
A white cat lies curled in her lap, its own eyes also shut, purring as she rubs its chin.

A handsome, tanned woman stands on the peak of a grassy hill, arm held aloft.
A falcon dives from above, alighting on her thick leather glove.
It casts its eyes north-west, then knowingly back at the witch.
With a sly grin, she throws the bird back into the air and strides downhill after it.

A bright-eyed girl, no more than thirteen, stands beside a bubbling cauldron, carefully teasing the seeds from a pine cone with a small knife.
``Sage leaf next, Harold?''
She looks up at the frog on the kitchen bench, as it croaks and nudges one of the piles of herbs that surrounds it.
``Ohh, right. Rosemary. Of course{\dots}''
The girl shakes her head and tuts to herself as she counts out seven leaves into her hand and drops them in the cauldron.
Harold peers over from the bench, keeping a close eye on the brew as it slowly turns a deep blue.

\section{No Mere Beast}

A witch's familiar is no mere animal.
It is a fusion of a tamed beast, and a tiny sliver of soul from the witch herself.
Obtaining a familiar is one of the first steps for any witch-in-training, and the familiar often aids in the witch in her subsequent learning.

Familiars are intelligent creatures, in some cases even more intelligent than the witches they are bound to.
They understand language, though the limits of animal form mean that most are incapable of speech.
Despite this, the bond that a witch shares with her familiar allow them to communicate.
With simple looks and gestures a familiar can communicate great meaning to its witch, communicating as effectively as if through speech.
This ability does not extend to other witches, and especially not to layfolk, who may require a Test to interpret a familiar's communication.
Pointing and beckoning are typically fairly unambiguous, however.

A witch's communication with her familiar allows her to lean on its expertise when her own is lacking.
A witch may use her familiar's ranks in a skill in place of her own, as long as the Test takes at least a minute, and she can confer with her familiar through the duration.

\section{Binding a Familiar}

Binding a familiar takes place in a simple ritual: achievable by even the most novice witch, though often performed under direct tutelage.
It requires a \circleref{small}, and takes an hour to perform.
The animal to become the familiar must be tamed by the witch beforehand, at least enough that it willingly remains by her side throughout the ritual.
Many witches find this to be the hardest part of binding their familiar, and it means that some animals make for rather rare familiars.
Lastly, the witch must offer up a sliver of her own {\soul}, to seal the bond.
She does so by feeding the familiar animal a drop of her own blood.

Upon completion of the ritual, the animal and the {\soul}-sliver are fused to form a new entity, the familiar.
It retains many traits from the animal, but becomes decidedly more human in personality.
Slight changes to its physical form often manifest, such as a coat that always remains strangely glossy, a slight chill to the touch, or sharper, whiter teeth.
Changes in eye colour are especially common.
Lastly, the sliver of the witch's {\soul} included in the creation of a familiar also influences its personality.
It ensures that, although a witch and her familiar may not always get along, and may certainly disagree on the best way to go about something, they will always have each other's best interests at heart.

\section{Creating a Familiar}

From the perspective of character creation, there are many things to bear in mind when creating a familiar.
While the familiar is unlikely to take the foreground as much as the witch herself, they are still a character in their own right, and should be designed as such.

The most important decision is the form the familiar will take, the animal they were created from.
This determines the familiar's attributes, skills and abilities.
Note that familiars, as non-human characters, may have attributes below the human 0 to 5 range.

Beside its game statistics, it is also important to get an idea of your familiar as a character.
Try answering some of the following questions.

\begin{itemize}
	\item What is your familiar's name?
	\item Is your familiar male or female?
		Do you not know?
	\item At what stage in her life, and her training, did your witch bind her familiar?
	\item Do your witch and her familiar get along?
		Do they engage in playful banter?
		Philosophical debate?
	\item Does your familiar have any quirks, physical or mental?
\end{itemize}

Lastly, it is important to decide whether each familiar will be played by the player or the GM.
Both are valid, but if the GM is playing familiars they should typically act in their witch's best interests.

\section{Familiar Injury and Death}
\seclabel{familiar-injury-death}

Familiars suffer {\damage}, {\shock} and death just like other characters.
A witch is always aware when her familiar dies, feeling it as a searing pain in her very soul.
It is even worse the other way around, however.
If a witch dies and her {\soul} departs the world, it tears free the shard of {\soul} bound in the familiar, killing it.

It is possible to recover a deceased familiar.
This requires the witch to tame another animal of the same kind, and repeat the original binding ritual upon it.
The familiar's {\soul} and personality are entirely restored, and it may take either the new animal's appearance or its original one.

Repeating the ritual takes another sliver of the witch's {\soul}, provided through another drop of blood.
As such, recovering a deceased familiar costs 10 XP every time.

Familiars tend to age more slowly than a normal animal of their kind would, growing old and dying at the same time as their witch does.

\section{Familiar Fighting}

Some witches will be inclined to have their familiars fight for them.
This is perfectly valid; some familiars will even make better fighters than their witches.
In {\structuredtime}, familiars take their {\turn} at the same time as their witch, using their witch's {\initiative} score.

Familiars cannot normally use weapons, what with lacking hands.
Some can make {\unarmed} attacks, and the number of dice they roll for their {\damagetests} are given in their list of abilities.
If this is not given, they have no effective attacks.

``No effective attacks'' does not mean that they cannot fight at all---a \familiarref{rat} still has teeth.
It simply means that any fight between one of these familiars and one \emph{with} effective attacks is a foregone conclusion.
If two creatures without effective attacks end up in a fight, it may be prudent to reduce several minutes of fighting to one opposed Test, or a handful of them.

\section{Familiar Animals}

A list of the types of animal available as familiars is presented below, along with the attributes, skills and abilities of the familiar.
Besides the abilities listed below, the players and GM are encouraged to apply common sense.
For instance, familiars lack thumbs and will struggle with door handles, and a weasel can squeeze through a smaller hole than a hound.

If you would like your familiar to be an animal not presented on the list below, discuss your option with your GM.
It might be possible to design a new familiar for you to use, or to use the statistics of a familiar presented here to represent something else.
Note that familiars are fairly small animals; the exclusion of anything larger than a medium-sized dog is intentional.

Many types of familiar---more powerful ones---come with an associated XP cost.
This is deducted from the witch's starting XP.
Some types of familiar also come with options which may be purchased for an additional XP cost.
These represent inherent differences in the animal used and must be purchased at the same time your familiar is created.
You may only select one option; they are mutually exclusive.

Lastly, bear in mind that some feats that can be purchased later depend upon particular types of familiar, and your familiar's later development is limited by its form.
As such, it can be worth taking a quick look at other feats you may be interested in taking when selecting your familiar.
%TODO: If those are all in a discipline chapter on familiars, direct people there.

\familiar{Bat}{Bats}{bat}{15}{
	\atttable{\negative 2}{2}{1}{2}{2}{2}{0}{0}
}{
	\speed{2}, \flyspeed{15}
}{
	\skillref[1]{divination}, \skillref[1]{flying}, \skillref[1]{perception}, \skillref[1]{stealth}
}{
	Lurkers in darkness, and nocturnal fliers, \familiarrefplural{bat} fit right in with a certain type of witch.
	Their echolocation not only makes them excellent scouts in dark caves, but also lends them a slight natural talent with \discref{divination}.
}{
	\ability{Echolocation}{
		The \familiarref{bat} can sense perfectly in darkness, or even when blinded, using its echolocation.
		This works within about 50 metres.
		The sounds it produces are beyond the hearing range of humans, birds, fish, and amphibians, but can be detected by smaller mammals, such as cats, dogs and rats, as well as some insects.
	}
	
	\familiaroption{Vampire Bat}{5}{
		The \familiarref{bat} gains \skillref[1]{weaponry}, and it rolls 1 die for {\unarmed} {\damagetests}.
		Its bite is painless, and can go unnoticed by the victim.
		A full feed takes about 30 minutes, but it draws enough \materialref{blood} to use as a \materialref{taglock} in just one {\action}---even if it deals no {\damage}.
		It can regurgitate this \materialref{blood} any time in the next few hours.
	}
}

\familiar{Cat}{Cats}{cat}{20}{
	\atttable{\negative 2}{3}{2}{2}{2}{2}{3}{1}
}{
	\speed{10}
}{
	\skillref[1]{athletics}, \skillref[1]{deception}, \skillref[1]{perception}, \skillref[2]{socialising}, \skillref[2]{stealth}
}{
	Graceful and charming on the outside, \familiarrefplural{cat} can be incredibly sly and manipulative underneath.
	Just like many witches.
}{
	\ability{Natural Acrobat}{
		The \familiarref{cat} rolls an extra die on Tests to jump, retain its balance, land on its feet, or avoid {\damage} from falling.
		%TODO: Reduce fall damage in some more definite fashion?
	}
	
	\ability{Cat's Eyes}{
		The \familiarref{cat} can see excellently in the dark.
		It suffers no penalties in low-light conditions, though it is as blind as anyone in complete darkness.
	}
	
	\ability{Claws}{
		The \familiarref{cat} rolls 2 dice for {\unarmed} {\damagetests}.
	}
}

\familiar[Crow/Raven/Magpie]{Crow}{Crows}{crow}{10}{
	\atttable{\negative 2}{2}{2}{3}{2}{2}{1}{1}
}{
	\speed{2}, \flyspeed{15}
}{
	\skillref[1]{divination}, \skillref[1]{flying}, \skillref[1]{necromancy}, \skillref[1]{perception}
}{
	\capital{\familiarrefplural{crow}} and other corvids are the smartest birds, and among the smartest animals of all.
	Many people see their appearance as an omen, typically of ill fortune, or death.
	If they're followed by a witch, this might even be the case.
}{
	\ability{Thief of Glitter}{
		The \familiarref{crow} rolls an extra die on \skillref{perception} Tests to spot shiny objects.
	}
}

\familiar{Dog}{Dogs}{dog}{20}{
	\atttable{1}{1}{1}{1}{3}{2}{1}{2}
}{
	\speed{12}
}{
	\skillref[2]{intimidation}, \skillref[2]{perception}, \skillref[1]{weaponry}
}{
	A man's best friend, and often a witch's too.
	\capital{\familiarrefplural{dog}} are a diverse lot, including hunting dogs, sheepdogs, sled dogs and more.
}{
	\ability{Bite}{
		The \familiarref{dog} rolls 5 dice for {\unarmed} {\damagetests}.
	}
	
	\familiaroption{Scenthound}{5}{
		The \familiarref{dog} rolls an extra die on \skillref{perception} Tests relying on smell.
	}
	
	\familiaroption{Greyhound}{5}{
		The \familiarref{dog} has a \statref{speed} of \speed{20}.
	}
	
	\familiaroption{Sheepdog}{5}{
		The \familiarref{dog} has \skillref[2]{animals}.
	}
}

\familiar[Ferret/Stoat/Weasel]{Ferret}{Ferrets}{ferret}{15}{
	\atttable{\negative 2}{3}{1}{1}{2}{2}{1}{0}
}{
	\speed{8}
}{
	\skillref[1]{athletics}, \skillref[2]{stealth}, \skillref[1]{weaponry}
}{
	A \familiarref{ferret}, stoat, weasel, polecat, ermine, mink, or marten.
	Despite their small size, these creatures are ferocious predators.
	Their long, narrow bodies allow them to invade the burrows of much smaller animals, or the trousers of their witch's unfortunate foes.
}{
	\ability{Bite}{
		The \familiarref{ferret} rolls 2 dice for {\unarmed} {\damagetests}.
	}
	
	\ability{Slippery}{
		The \familiarrefpossessive{ferret} \statref{dodge-rating} is increased by 2.
	}
}

\familiar[Frog/Toad]{Frog}{Frogs}{frog}{0}{
	\atttable{\negative 3}{\negative 1}{1}{1}{2}{0}{\negative 1}{\negative 1}
}{
	\speed{4}, \swimspeed{4}
}{
	\skillref[1]{brewing}
}{
	\capital{\familiarrefplural{frog}} and toads make excellent companions to brewing witches, due to their natural affinity with water.
	Particularly with some of the stuff that gets into the murkier ponds around{\dots}
	
	It is important to try and keep their skin moist, but maybe refrain from dropping them in the cauldron.
}{
	\ability{Amphibian}{
		The \familiarref{frog} can breathe underwater.
	}
	
	\ability{Leapfrog}{
		The \familiarref{frog} can jump at least 3 metres from a standing start.
		It rolls an additional die on Tests made to jump.
	}
	
	%TODO: Something about keeping the skin moist? Maybe when there are exhaustion/fatigue rules.
}

\familiar{Owl}{Owls}{owl}{15}{
	\atttable{\negative 2}{2}{3}{2}{2}{3}{1}{1}
}{
	\speed{2}, \flyspeed{12}
}{
	\skillref[1]{flying}, \skillref[2]{perception}, \skillref[1]{stealth}
}{
	The great wisdom of \familiarrefplural{owl} makes them well suited to those witches who enjoy a spot of intellectual conversation, and can't find any other humans who seem to be up to it.
	They also make excellent nocturnal scouts, and even hunters.
}{
	\ability{Night Eyes}{
		The \familiarref{owl} can see excellently in the dark.
		It suffers no penalties in low-light conditions, though it is as blind as anyone in complete darkness.
	}
	
	\ability{Swooping Talons}{
		The \familiarref{owl} rolls 2 dice for {\unarmed} {\damagetests}, or 3 dice when striking from a dive.
	}
}

\familiar[Raptor (Eagle/Falcon/Hawk)]{Raptor}{Raptors}{raptor}{25}{
	\atttable{\negative 1}{3}{1}{2}{2}{3}{\negative 1}{2}
}{
	\speed{2}, \flyspeed{20}
}{
	\skillref[2]{flying}, \skillref[2]{perception}, \skillref[1]{weaponry}
}{
	\capital{\familiarrefplural{raptor}} include buzzards, eagles, falcons, harriers, hawks, kites, and osprey; birds of prey.
	They are excellent fliers, have keen eyesight, and nobody would want to tangle with their wicked beak and talons.
	Among the nobility, falconry is largely a status symbol, but a witch with a \familiarref{raptor} for a familiar has herself a great asset.
}{
	\ability{Eagle Eyes}{
		The \familiarref{raptor} rolls an extra die on \skillref{perception} Tests to see things at a long distance.
	}
	
	\ability{Beak \& Talons}{
		The \familiarref{raptor} rolls 3 dice for {\unarmed} {\damagetests}, or 4 dice when striking from a dive.
	}
}

\familiar[Rat/Mouse]{Rat}{Rats}{rat}{0}{
	\atttable{\negative 3}{1}{1}{1}{2}{1}{\negative 2}{\negative 1}
}{
	\speed{8}
}{
	\skillref[1]{stealth}
}{
	The \familiarref{rat} is a rather widely reviled animal, but it's certainly easy for a new witch looking for a familiar to find one.
	And it can get into smaller places than a \familiarref{cat} or \familiarref{owl}, which often proves helpful.
}{
	\ability{Filth-Liver}{
		The \familiarref{rat} rolls an extra die on Tests to resist poison or disease.
	}
	
	\ability{Keen Smell}{
		The \familiarref{rat} rolls an extra die on \skillref{perception} Tests relying on smell.
	}
}

\familiar{Songbird}{Songbirds}{songbird}{5}{
	\atttable{\negative 3}{2}{1}{2}{2}{1}{2}{0}
}{
	\speed{2}, \flyspeed{15}
}{
	\skillref[1]{flying}, \skillrefspeciality[2]{performance}{Singing}
}{
	\capital{\familiarrefplural{songbird}} include sparrows, larks, robins, wrens, thrushes, warblers, nightingales, and countless other types of bird.
	They fill the forests and woods, and a witch seeking one needs only to follow the sound of singing.
}{
	\ability{Songspeak}{
		The \familiarref{songbird} may use its song---which can carry for a few hundred metres---to communicate with you, just as effectively as through speech.
		If it learns to communicate with any other creatures, such as through \featref{familiar-language}, \featref{familiar-language-2}, \featref{familiar-language-3}, or \featref{familiar-language-animals}, it can use its song for that too.
	}
}

\familiar{Spider}{Spiders}{spider}{5}{
	\atttable{\negative 5}{2}{1}{1}{1}{2}{\negative 2}{\negative 1}
}{
	\speed{2}
}{
	\skillref[1]{intimidation}, \skillref[2]{stealth}
}{
	Crawling up walls, hanging from ceilings, and so tiny as to avoid notice, \familiarrefplural{spider} make excellent spies.
	They also terrify some people, which can often prove handy.
}{
	\ability{Web Spinner}{
		Given about an hour, the \familiarref{spider} can spin a cobweb.
		This does nothing to creatures of much size, but traps insects and the like on contact.
		Particularly dense webbing can also obstruct vision.
	}
	
	\ability{Spider Climb}{
		The \familiarref{spider} can move at full speed over walls and ceilings, at no risk of falling.
		The \familiarref{spider} can also hang from a surface on a single strand of web silk.
		It spools out this strand, or climbs up it, using its usual \statref{speed}.
	}
	
	\familiaroption{Venomous}{5}{
		The \familiarref{spider} may inject venom with a bite.
		The bite is not immediately harmful, and in some cases, may go unnoticed.
		Intense pain at the site of the bite begins after about 5 minutes.
		Sickness, including nausea, vomiting, and weakness, develops over the next hour, and may last several days.
		Victims must make a \attref{might} Test to determine the severity and duration of symptoms.
		Critical failure on this Test leads to death.
	}
	
	\familiaroption{Silk Sailor}{5}{
		The \familiarref{spider} is light enough to be carried on the wind, ballooning on a single thread of gossamer.
		It gains \skillref[1]{flying}.
		As an {\action}, it can spool out this gossamer thread, and take off.
		It will only fly in moderate or strong winds, and cannot control the direction of its flight; it goes where the wind blows.
		However, it can travel many hundreds of miles this way, moving quite quickly in strong enough winds.
	}
}

%\subsection{Beaver}

%Placeholder.

%\subsection{Chicken}

%Placeholder.

%\subsection{Dove}

%Placeholder.

%\subsection{Gecko}

%Placeholder.

%\subsection{Goose}

%Placeholder.

%\subsection{Lemming}

%Placeholder.

%\subsection{Lizard}

%Placeholder.

%\subsection{Mole}

%Placeholder.

%\subsection{Otter}

%Placeholder.

%\subsection{Parrot}

%Placeholder.

%\subsection{Pigeon}

%Placeholder.

%\subsection{Praying Mantis}

%Placeholder.

%\subsection{Rabbit/Hare}

%Placeholder.

%\subsection{Salamander/Newt}

%Placeholder.

%\subsection{Seabird}

%Placeholder.

%\subsection{Squirrel}

%Placeholder.

%\subsection{Swan}

%Placeholder.

%\subsection{Tortoise}

%Placeholder.

%\subsection{Vole/Shrew/Gopher}

%Placeholder.

%\subsection{Woodpecker}

%Placeholder.
