\chapter{Druidcraft}
\chaplabel{druidcraft}

\section{Feats}

\feat{Animal Companion}{animal-companion}{25}{
	\skillref[1]{animals}
}{
	You have a highly-trained a loyal companion that will follow you anywhere.
	This might be a hound, hawk, horse or any other easily trainable animal the GM approves.
	%TODO: Links to statblocks.
	If it is killed or otherwise lost, it takes several weeks to train another creature to the same level.
	
	You don't share any special bond with this animal like you do with your familiar, only a bond established through training and friendship.
	You've trained it with enough commands to get it to do what you want under normal circumstances, as long as you continue to treat it well.
	
	Note that you do need this feat to have a pet, or even a riding horse.
	This feat is only necessary to have an animal that will follow you unquestioningly into life-threatening scenarios.
}

\feat{Twin Companions}{animal-companion-2}{25}{
	\skillref[2]{animals},
	\featref{animal-companion}
}{
	You have trained an additional \featref{animal-companion}, allowing you to have two at once.
	They may be the same kind of animal, or different kinds.
}

\feat{Beastmaster}{animal-companion-3}{25}{
	\skillref[3]{animals},
	\featref{animal-companion-2}
}{
	A third \featref{animal-companion} rounds out your pack.
}

\feat{Wild Companion}{animal-companion-wild}{15}{
	\skillref[2]{animals},
	\featref{animal-companion}
}{
	Anyone can train a hound, a hawk, or a horse.
	But a bear?
	A goat?
	A \emph{cat}?
	It takes quite someone to tame such a beast.
	
	You may make an \featref{animal-companion} from even animals that cannot easily trained; any animal the GM approves.
	You may only have one such \featref{animal-companion}; keeping several playing nicely together and under control is still beyond you.
}

\feat{Beast Whisperer}{animal-language}{15}{
	\skillref[1]{animals}
}{
	Through observations of you own interactions with your familiar, you've learned to establish a similar, albeit lesser, form of communication with other animals.
	Establishing such a language takes at least a week's training, although you can train a handful of similar animals at once.
	Such training is only possible with an animal that is trained or domesticated outside of this, or at least in the process of becoming so.
	An \featref{animal-companion} is automatically considered to have this training.
	
	The ``language'' this establishes is rather rudimentary.
	It's a mixture of gestures, sounds and body language, and it takes quite some time to convey any more complexity.
	A short, shouted command still works as normal, but anything longer takes at least twice as long to convey as it would through speech---possibly more than ten times as long for something rather complicated.
	It works both ways, allowing you to communicate to the animal, and also allowing the animal to convey information to you.
	
	This communication does nothing to improve the animal's intelligence.
	Many animals can follow a multi-stage series of instructions, and even relay information such as what they saw and where they've been, but abstract reasoning is beyond them.
	This feature also gives an animal no particular inclination to follow instructions from you, if it did not already have it.
}

\feat{Clever Boy}{animal-intelligence}{20}{
	\skillref[2]{animals},
	\featref{animal-language}
}{
	A witch who treats her animals like people might come to find her animals behaving a little like people.
	An animal that you have trained---besides a familiar, which is already far smarter than any mere beast---seems more intelligent.
	%If it was already a fairly intelligent creature, it might even become as intelligent as a rather dim-witted human.
	It gains a \positive{1} bonus to its \attref{ken} and \attref{wit} scores.
	It becomes able to understand more complicated concepts, perhaps even rudimentary abstract reasoning, and its memory improves.
	This benefit applies automatically to an \featref{animal-companion}, but also applies to other animals that the witch has played a major part in training.
}

\feat{Forest Tongue}{animal-language-2}{15}{
	\skillref[2]{animals},
	\featref{animal-language}
}{
	As you learn to think like an animal, you're becoming better at talking like one too.
	You may communicate with even untrained animals through the same ``language'' provided by \featref{animal-language}.
	This requires that the animal is receptive and paying attention to you.
}

\feat{Critter-Chatter}{animal-language-speed}{20}{
	\skillref[2]{animals},
	\featref{animal-language}
}{
	With more expressive body language and better impressions of animal noises, you can speed up your communications.
	You can use the ``language'' provided by \featref{animal-language} to convey information at about the same speed as speech.
}

\feat{Bark and Squawk}{animal-language-3}{20}{
	\skillref[3]{animals},
	\featref{animal-language-2},
	\featref{animal-language-speed}
}{
	By barking, roaring and squawking loudly enough, you can convey your meaning to even an oncoming bear.
	You might not convince him to stop, but you can certainly \emph{try}.
	You can use the ``language'' provided by \featref{animal-language} to communicate with even unreceptive animals, if you can attract their attention.
}
