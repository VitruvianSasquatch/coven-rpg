\chapter{Druidcraft}
\chaplabel{druidcraft}

\section{Feats}

\feat{Beast Whisperer}{animal-language}{15}{
	\skillref[1]{animals}
}{
	Through observations of you own interactions with your familiar, you've learned to establish a similar, albeit lesser, form of communication with other animals.
	Establishing such a language takes at least a week's training, although you can train a handful of similar animals at once.
	Such training is only possible with an animal that is trained or domesticated outside of this, or at least in the process of becoming so.
	
	The ``language'' this establishes is rather rudimentary.
	It's a mixture of gestures, sounds and body language, and it takes quite some time to convey any more complexity.
	A short, shouted command still works as normal, but anything longer takes at least twice as long to convey as it would through speech---possibly more than ten times as long for something rather complicated.
	It works both ways, allowing you to communicate to the animal, and also allowing the animal to convey information to you.
	
	This communication does nothing to improve the animal's intelligence.
	Many animals can follow a multi-stage series of instructions, and even relay information such as what they saw and where they've been, but abstract reasoning is beyond them.
	This feature also gives an animal no particular inclination to follow instructions from you, if it did not already have it.
}

\feat{Uplifting Training}{animal-intelligence}{20}{
	\skillref[2]{animals},
	\featref{animal-language}
}{
	A witch who treats her animals like people might come to find her animals behaving a little like people.
	An animal that you have trained---besides a familiar, which is already far smarter than any mere beast---seems more intelligent.
	%If it was already a fairly intelligent creature, it might even become as intelligent as a rather dim-witted human.
	It gains a \positive{1} bonus to its \attref{ken} and \attref{wit} scores.
	It becomes able to understand more complicated concepts, perhaps even rudimentary abstract reasoning, and its memory improves.
}

\feat{Bark and Squark}{animal-language-2}{15}{
	\skillref[2]{animals},
	\featref{animal-language}
}{
	As you learn to think like an animal, you're becoming better at talking like one too.
	You may communicate with even untrained animals through the same ``language'' provided by \featref{animal-language}.
	This requires that the animal is receptive and paying attention to you.
}
