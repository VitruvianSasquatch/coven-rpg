\discipline{Druidcraft}{druidcraft}{Druid}{Druids}

\section{The Wild Side}
\seclabel{druidcraft-wild}

Some \practitioners{druidcraft} seek to harness the power of nature.
They tame animals, and cultivate herbs.
Their abilities are often entirely non-magical, springing entirely from a deep understanding of the plants and animals they tend.

But some believe that \emph{harnessing} nature is anathema to the very soul of \discref{druidcraft}.
They prefer to \emph{unleash} nature's power; its wrath.
Some of them even would see civilisation fall; see the forests reclaim the world.
For these \practitioners{druidcraft}, the hand of civilisation---taming, or cultivation---inhibits their magic.
They can only work with {\wild} plants and animals.

As with so many things, whether or a plant or animal counts as {\wild} is not strictly defined.
Guidelines are presented below, but the GM is the final arbiter.
And, if the \practitioner{druidcraft} in question can make a good enough case, the GM might allow a Test to make her magic work.

For a plant, {\wildness} begins when its seed is planted.
If it was planted by \emph{intention}, typically by a human or another intelligent creature, then it isn't {\wild}.
But this can change.
A plant that is tended regularly, even if its seed originally fell naturally, loses its {\wildness}.
Conversely, a seed that was planted intentionally, but now hasn't been tended for years, might become {\wild}.

As an exception, an \herbtype{5} that is conscious or intelligent---one with a score in any non-physical attribute---is \emph{never} considered {\wild}.

Whether or not an animal is {\wild} is determined more by its upbringing and experiences than by the circumstances of its birth.
Mostly, it is determined by its reactions to humans---or other intelligent creatures.
An animal that retains its natural reaction to people---be that fear or predatory instinct---is still {\wild}.
But if it is accustomed to people, seeing them as masters, suppliers of food, or is even something to ignore, that animal is not {\wild}.

A familiar or an \featref{animal-companion}---even a \featref{animal-companion-feral}---is never a {\wild} animal.
An undead creature is no longer an animal at all.

\capital{\wild} magic can prove a little self-defeating, with both animals and plants.
The more that a \practitioner{druidcraft} uses her magic on a {\wild} animal or plant, the more accustomed to humans it becomes.
If she keeps tending to the same animal or plant, it might no longer be {\wild}.
She should be careful not to let this happen.

\section{Feats}

\feat{Animal Companion}{animal-companion}{25}{
	\skillref[1]{animals}
}{
	You have a highly-trained a loyal companion that will follow you anywhere.
	This might be a \creatureref{dog}, \creatureref{raptor}, \creatureref{horse}, or any other easily trainable animal the GM approves.
	If it is killed or otherwise lost, it takes several weeks to train another creature to the same level.
	
	You don't share any magical bond with this animal like you do with your familiar, only a bond established through training and friendship.
	You've trained it with enough commands to get it to do what you want under normal circumstances, as long as you continue to treat it well.
	
	Note that you do not need this feat to have a pet, or even a riding horse.
	This feat is only necessary to have an animal that will follow you unquestioningly into life-threatening scenarios.
}

\feat{Twin Companions}{animal-companion-2}{25}{
	\skillref[2]{animals},
	\featref{animal-companion}
}{
	You have trained an additional \featref{animal-companion}, allowing you to have two at once.
	They may be the same kind of animal, or different kinds.
}

\feat{Beastmaster}{animal-companion-3}{25}{
	\skillref[3]{animals},
	\featref{animal-companion-2}
}{
	A third \featref{animal-companion} rounds out your pack.
}

\feat{Feral Companion}{animal-companion-feral}{15}{
	\skillref[2]{animals},
	\featref{animal-companion}
}{
	Anyone can train a hound, a hawk, or a horse.
	But a bear?
	A goat?
	A \emph{cat}?
	It takes quite someone to tame such a beast.
	
	You may make an \featref{animal-companion} from even animals that cannot easily trained; any animal the GM approves.
	You may only have one such \featref{animal-companion}; keeping several playing nicely together and under control is still beyond you.
}

\feat{Feral Beastmaster}{animal-companion-feral-2}{15}{
	\skillref[3]{animals},
	\featref{animal-companion-feral},
	\featref{animal-companion-2}
}{
	A sloth of bears?
	An army of frogs?
	A herd of cats?
	No menagerie is beyond your ability to tame.
	
	You may make \emph{every} \featref{animal-companion} a \featref{animal-companion-feral}, if you wish.
}

\feat{Beast Whisperer}{animal-language}{10}{
	\skillref[1]{animals}
}{
	Through observations of you own interactions with your familiar, you've learned to establish a similar, albeit lesser, form of communication with other animals.
	Establishing such a language takes at least a week's training, although you can train a handful of similar animals at once.
	Such training is only possible with an animal that is trained or domesticated outside of this, or at least in the process of becoming so.
	An \featref{animal-companion} is automatically considered to have this training.
	
	The ``language'' this establishes is rather rudimentary.
	It's a mixture of gestures, sounds, and body language, and it takes quite some time to convey any complexity.
	A short, shouted command still works as normal, but anything longer takes at least twice as long to convey as it would through speech---possibly more than ten times as long for something rather complicated.
	It works both ways, allowing you to communicate to the animal, and also allowing the animal to convey information to you.
	
	This communication does nothing to improve the animal's intelligence.
	Many animals can follow a multi-stage series of instructions, and even relay information such as what they saw and where they've been, but abstract reasoning is beyond them.
	This feature also gives an animal no particular inclination to follow instructions from you, if it did not already have it.
}

\feat{Beast Tongue}{animal-language-2}{15}{
	\skillref[2]{animals},
	\featref{animal-language}
}{
	As you learn to think like an animal, you're becoming better at talking like one too.
	You may communicate with even untrained animals through the same ``language'' provided by \featref{animal-language}.
	This requires that the animal is receptive and paying attention to you.
}

\feat{Critter-Chatter}{animal-language-speed}{15}{
	\skillref[2]{animals},
	\featref{animal-language}
}{
	With more expressive body language and better impressions of animal noises, you can speed up your communications.
	You can use the ``language'' provided by \featref{animal-language} to convey information at about the same speed as speech.
}

\feat{Bark and Squawk}{animal-language-3}{10}{
	\skillref[3]{animals},
	\featref{animal-language-2},
	\featref{animal-language-speed}
}{
	By barking, roaring, and squawking loudly enough, you can convey your meaning to even an oncoming bear.
	You might not convince him to stop, but you can certainly \emph{try}.
	You can use the ``language'' provided by \featref{animal-language} to communicate with even unreceptive animals, if you can attract their attention.
}

\feat{Clever Boy}{animal-intelligence}{20}{
	\skillref[2]{animals},
	\featref{animal-companion},
	\featref{animal-language}
}{
	A witch who treats her animals like people might come to find her animals behaving a little like people.
	An animal that you have trained---besides a familiar, which is already far smarter than any mere beast---seems more intelligent.
	It gains a \positive{1} bonus to its \attref{ken} and \attref{wit} scores.
	It becomes able to understand more complicated concepts, perhaps even rudimentary abstract reasoning, and its memory improves.
	This benefit applies automatically to an \featref{animal-companion}, but also applies to other animals that the witch has played a major part in training.
}

\feat{Walkies}{animal-speed}{10}{
	\skillref[1]{animals},
	\featref{animal-companion}
}{
	Regular walks and runs with your animals keeps the muscles trained.
	An animal you have trained---except your familiar---adds 2 to its \statref{speed}.
	This applies to walking, swimming, flying or any other method the animal possesses, although it requires that the animal is able to move in the first place, and cannot do more than double the existing speed.
	This benefit applies automatically to an \featref{animal-companion}, but also applies to other animals that the witch has played a major part in training.
}

\feat{Run Like the Wind}{animal-speed-2}{15}{
	\skillref[3]{animals},
	\featref{animal-speed}
}{
	Your beasts are champion runners, fliers or swimmers, capable of incredible turns of speed.
	When you apply the bonus of \featref{animal-speed}, the animal's \statref{speed} increases by 2, or half its original speed, whichever is higher.
}

\feat{Forestwalker}{plant-movement}{10}{
	\skillref[1]{botany}
}{
	You have an agreement of sorts with the {\wild} plants of the world.
	They won't hurt you if you don't hurt them.
	
	You are unaffected by {\difficultterrain} or other impediment of movement caused by {\wild} plants.
	This doesn't go quite so far as to let you walk through solid trees, but you can run through a bramble bush just fine.
	Furthermore, their natural defences, such as a bramble's thorns or a nettle's sting, don't affect you.
	Lastly, the fact that you don't disturb the plants can make you far harder to track when travelling in a forest.
	
	This only works as long as you refrain from damaging the plants---start hacking at a bramble patch and it will tear at you just as much as anyone else.
	The effects of plants you ingest, or use in \discref{brewing}, are similarly unchanged.
}

\feat{Treespeaker}{plant-speak}{10}{
	\skillref[1]{botany}
}{
	Speaking with plants isn't like speaking with animals.
	Plants have no minds; there's nothing to speak to.
	But with many plants, trees, bushes, and grasses bound by their roots, whispering together as the wind blows through their canopy---a forest as a whole forms a creature, of sorts.
	It feels things moving through it---rustling its leaves, snapping its twigs---and the disturbances ripple through the plants.
	And when a tree burns, the whole forest cries out in anguish.
	
	You may listen to {\wild} plants around you, reading the visible and audible signs they give.
	This allows you to make a \testtype{heed}{botany} Test to pick up on disturbances that have affected the plants: animals passing through, people camping, crash-landings, fires, and so on.
	A thorough search takes a couple of minutes, but obvious signs might be detected in an {\action}.
	
	The talk of the plants is affected by time and distance, making anything more than a kilometre away or more than an hour ago quite difficult to detect.
	It is also far easier to pick up on these signs where the vegetation is denser; a forest is easy, but grassland is incredibly difficult.
	Lastly, any significant damage to plants is far easier to pick up on.
	Someone who takes branches to make a shelter leaves clear evidence, and logging echoes far and wide.
	Fire is by far the easiest to detect; signs of a large conflagration might echo to the next forest over, and be seen even by a witch who isn't paying attention.
}

\feat{Wild Sprout}{plant-grow}{10}{
	\skillref[1]{botany}
}{
	Touching a {\wild} plant, you can stimulate a part of it to grow quickly.
	You can make it sprout a leaf or twig, make a flower bloom, or make it grow a small bunch of berries.
	It takes about a minute to grow a part, and you much touch the plant for the duration.
	
	It will only grow parts of the plant that could normally grow---you can't grow blackberries on an apple tree, only apples.
	It can grow unseasonable parts, however---leaves on a tree in winter, or blackberries in spring.
}

\feat{Druidic Grove}{plant-grove}{15}{
	\skillref[1]{botany}
}{
	\materials{A \materialref{stone-circle} with plants growing within}
	
	You can pour your power into the soil, creating a grove where plants flourish.
	The ritual to do so takes an hour, and the effect lasts for one lunar month.
	
	For the duration, plants within the \materialref{stone-circle} grow and thrive without any tending.
	The area can be used as a {\garden}, without requiring any time tending the \materialrefplural{herb} that grow within.
	Only \materialrefplural{herb} that you could normally grow using your \skillref{botany} skill are affected.
	If you have \skillref[3]{botany}, however, even the needs of \herbtypeplural{5} are fulfilled by the grove.
}

\feat{A Stand of Trees}{tree-standing-stone}{15}{
	\skillref[2]{botany},
	\skillref[1]{ritual-magic},
	\featref{plant-movement}
}{
	The trees will be stone for you, if you ask them nicely enough.
	You may treat any {\wild} tree taller than a man as a \materialref{standing-stone}, without a Test.
	Similarly, a ring of trees, or a clearing ringed by trees, acts as a \materialref{stone-circle}.
	Treating a forest without a clearing as a \materialref{stone-circle} still requires a Test---it's not obviously a circle, as such.
}

\section{Familiar Feats}

Just as \emph{you} can learn and improve, so can your familiar.
These feats represent that ability, and are also how you can increase your familiar's attributes and skills.

You still purchase these feats from your normal XP, so helping your familiar to develop comes at some cost to your own improvement.
The price is often well worth it, however; your familiar is a constant and loyal companion, and can contribute greatly in many situations.
Besides, you might learn a thing or two yourself while teaching your familiar.

Some of these feats require a particular kind of familiar, while others require your familiar to have particular skills.
This might affect your choice of familiar, so it can be worth reading these feats before selecting a familiar.
However, you will sometimes be able to train an unskilled familiar, in order to acquire a particular feat, so this needn't entirely dictate your choice.

\feat{Familiar Attribute}{familiar-attribute}{15}{
	\noprereq
}{
	Through regular training with your familiar, you've managed to improve its natural abilities.
	Increase one of your familiar's attributes by 1.
	You may take this feat multiple times, but you can only increase each attribute once.
}

\feat{Familiar Skill}{familiar-general-skill}{15}{
	\noprereq
}{
	You've taught your familiar a new skill, or helped it to improve an existing one.
	It gains 1 rank in a {\generalskill}.
	You may take this feat multiple times, but you can only increase each skill once.
}

\feat{Familiar Speciality}{familiar-speciality-skill}{15}{
	\noprereq
}{
	Some animals have a natural ability with a particular craft or vocation; an ability that is often carried to a familiar of that type.
	You've nurtured and developed that talent in your familiar.
	It gains 1 rank in a {\specialityskill}, in which it already had at least 1 rank.
	You may take this feat multiple times, but you can only increase each skill once.
}

\feat{Familiar Discipline}{familiar-discipline-skill}{25}{
	\noprereq
}{
	Not many animals carry a natural talent for magic, but your familiar does.
	You've honed this talent, bringing it to heights that some witches never even reach.
	Your familiar gains 1 rank in a {\disciplineskill}, in which it already had at least 1 rank.
	You may take this feat multiple times, but you can only increase each skill once.
}

\feat{Familiar Familiarity}{familiar-language}{5}{
	\noprereq
}{
	Each ``language'' that a witch shares with her familiar is unique, relying on the bond between their souls.
	However, all these ``languages'' share something in common, and, with a bit of work, your familiar has picked up some of this.
	
	Your familiar can now communicate with any other witch's familiar, just as quickly and easily as it can communicate with you.
	This works in both directions, with your familiar both ``speaking'' and ``listening''.
}

\feat{Familiar Witchspeak}{familiar-language-2}{5}{
	\featref{familiar-language}
}{
	A few more language lessons between you and your familiar have taught your familiar to communicate with other witches; and you to communicate with other familiars.
	This works just as quickly and easily as communication between you and your own familiar, and in both directions.
	
	Your familiar can only communicate with other witches who have---or once had---a familiar of their own.
	However, it can also communicate anyone who has somehow had a familiar, but is not a witch.
}

\feat{Familiar Layspeak}{familiar-language-3}{15}{
	\skillreffamiliar[1]{socialising},
	\featref{familiar-language-2}
}{
	Communicating with layfolk is \emph{far} harder than communicating with witches; most people aren't even expecting an animal to talk to them!
	But, as long as your familiar can draw their attention, it can make them understand.
	This works just like communication between \emph{you} and your familiar.
	However, it can sometimes leave layfolk a little confused as to \emph{why} they can understand this animal's gestures; they just can.
}

\feat{Familiar Beastspeak}{familiar-language-animals}{10}{
	\skillref[1]{animals},
	\skillreffamiliar[1]{animals},
	\featref{familiar-language},
	\featref{animal-language}
}{
	Just as you have learned to communicate with animals, you have taught your familiar to.
	It ought to be easier---it's nearly an animal itself, after all!
	
	Your familiar gains the benefit of \featref{animal-language}, although it communicates with the animals \emph{you} have trained---it cannot teach the ``language'' itself.
	If you also have, \featref{animal-language-2}, \featref{animal-language-speed}, or \featref{animal-language-3}, your familiar gains the benefit of those as well.
}
