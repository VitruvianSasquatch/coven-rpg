\chapter{Ritual Magic}
\chaplabel{ritual-magic}

\section{Feats}

\feat{Circle of Containment}{circle-contain}{15}{
	None
}{
	\materials{A \circleref{medium} (no larger), a pinch of sugar, a white-handled knife}
	%Sugar draws things in.
	
	Performing this rite takes about fifteen minutes, involving tracing the knife around the perimeter and sprinkling the sugar.
	You may perform the rite from inside or outside the \materialref{ritual-circle}.
	
	At the completion of the rite, the \materialref{ritual-circle} is sealed.
	Objects and creatures inside the \materialref{ritual-circle} cannot pass outside of it, and no magic from inside may affect the outside.
	The barrier extends even into the {\mentalrealm}.
	Light and sound from inside can still leave, however.
	Air may diffuse slowly across the boundary, preventing suffocation or problems with pressure.
	Furthermore, nothing inside the circle may disturb the \materialref{ritual-circle} itself.
	
	This effect lasts 24 hours, but ends early if the \materialref{ritual-circle} is broken.
	The rite to renew the effect cannot be performed while the effect remains in place.
}

\feat{Circle of Exclusion}{circle-exclude}{15}{
	None
}{
	\materials{A \circleref{medium} (no larger), a pinch of salt, a black-handled knife}
	%Salt is a preservative, driving foulness away.
	
	This functions as \featref{circle-contain}, except the other way.
	Objects, creatures, and magical effects can pass out of the \materialref{ritual-circle}, but not into it, and the \materialref{ritual-circle} may be disturbed only from the inside.
}

\feat{Circle of Severance}{circle-contain-exclude}{10}{
	\featref{circle-contain} and \featref{circle-exclude}
	%The unconventional use of "and" is intentional here, in light of the fact that so many subsequent feats use "or".
}{
	\materials{A \circleref{medium} (no larger), a pinch of sugar, a pinch of salt, a double-edged knife}
	
	This functions as \featref{circle-contain}, except in both directions.
	Objects, creatures and magical effects cannot pass into or out of the \materialref{ritual-circle}, and the \materialref{ritual-circle} cannot be disturbed in any fashion.
}

\feat{Renew Barrier}{circle-barrier-renew}{10}{
	\featref{circle-contain} or \featref{circle-exclude}
}{
	You may repeat the rite to establish a \featref{circle-contain}, \featref{circle-exclude} or \featref{circle-contain-exclude} while the effect of one is ongoing, assuming you have the feat to establish one in the first place.
	You may do so from inside or outside of the \materialref{ritual-circle}.
	The new duration runs from the completion of the repeated rite, though this cannot be used to reduce the remaining duration.
}

\feat{Stabilise Barrier}{circle-barrier-duration}{20}{
	\skillref[1]{ritual-magic},
	\featref{circle-barrier-renew}
}{
	You have learned to tweak the stability of your circles' barriers, making them endure longer, or collapse faster.
	When you establish or renew a \featref{circle-contain}, \featref{circle-exclude} or \featref{circle-contain-exclude}, you may select its duration.
	You can make it last mere minutes, many years, or even indefinitely.
	
	A \featref{circle-contain-exclude} is more unstable than the other kinds, however.
	It's duration cannot exceed the natural 24 hours.
}

\feat{Miniature Barrier}{circle-barrier-small}{15}{
	\featref{circle-contain} or \featref{circle-exclude}
}{
	You may create a \featref{circle-contain}, \featref{circle-exclude} or \featref{circle-contain-exclude} using a \circleref{small}, assuming you have the feat to establish one at all.
}

\feat{Maximise Barrier}{circle-barrier-large}{15}{
	\featref{circle-contain} or \featref{circle-exclude}
}{
	You may create a \featref{circle-contain}, \featref{circle-exclude} or \featref{circle-contain-exclude} using a larger \materialref{ritual-circle}, assuming you have the feat to establish one at all.
	There are no upper limits to the size of the \materialref{ritual-circle} you may use, except those imposed by practicality.
	However, establishing or renewing the effect still requires tracing the perimeter with the relevant knife and sprinkling the correct substance.
	Very large perimeters may require longer than 15 minutes, and a large quantity of substance to sprinkle.
}

\feat{Barrier Augmentation}{circle-barrier-augment}{15}{
	\skillref[1]{ritual-magic},
	\featref{circle-contain} or \featref{circle-exclude}
}{
	The \featref{circle-contain} and \featref{circle-exclude} are very simple \materialrefplural{ritual-circle}.
	So simple, in fact, that it is possible to combine them with another circle.
	You may scribe two \materialrefplural{ritual-circle} of the same size in the same place, provided one is a \featref{circle-contain} or \featref{circle-exclude}.
	This requires the usual investment of time for both circles, independently.
	Additionally, the \featref{circle-contain} or \featref{circle-exclude} provides the same protection to the other \materialref{ritual-circle} as it provides to itself, while it is active.
	
	However, the \featref{circle-contain}, \featref{circle-exclude} and \featref{circle-contain-exclude} and all incompatible, and no two can ever be scribed in the same place.
}

\feat{Ritual Fire}{ritual-fire}{10}{
	None
}{
	\materials{A \circleref{small} (no larger), a flame}
	
	This ritual takes about a minute, and lights a tiny flame into a merry campfire.
	The fire is large enough to fill a \circleref{small}, and produces enough heat to cook food, melt snow or keep a group of campers warm.
	It burns without fuel even in the coldest of conditions, though will gladly consume fuel thrown into it, or spread to nearby flammable material if it is available.
	
	The fire cannot be naturally extinguished and lasts 8 hours, or until the \materialref{ritual-circle} is broken.
	Any fires lit from the ritual fire and burning on regular fuel will continue afterwards.
}

\feat{Ritual Forge}{ritual-fire-2}{10}{
	\featref{ritual-fire}
}{
	You may use a \circleref{medium} (no larger) to create a \featref{ritual-fire}.
	When you do so, the fire fills the larger circle.
	It provides enough heat to keep a sizeable crowd warm on a winter night, and furthermore, burns hot enough to be used as an iron forge.
}

\feat{Pillar of Flame}{ritual-fire-burst}{15}{
	\skillref[2]{ritual-magic},
	\featref{ritual-fire}
}{
	Instead of allowing your \featref{ritual-fire} to burn its heat out over many hours, you may release it all in one moment of glorious conflagration.
	As an {\action}, you may throw \herb{dried corn kernels}{2} onto a \featref{ritual-fire}.
	On the following {\round}, the fire erupts in a massive pillar of flame.
	This extinguishes the \featref{ritual-fire}.
	
	The pillar has enough heat to reduce wood to ash, and enough force to blow masonry apart.
	Any creature caught in the pillar comes to a quick end.
	However, the blast is contained entirely within the radius of the \materialref{ritual-circle}; those outside feel nothing more than an uncomfortable wave of heat.
	
	The pillar from a \featref{ritual-fire} reaches 10 metres high: enough to blow through a couple of storeys of a house.
	The pillar from a \featref{ritual-fire-2} reaches about 100 metres, and can be seen from more than \SI{30}{\kilo\metre} away, particularly at night.
}

\feat{Rite of Transposition}{ritual-teleport}{15}{
	None
}{
	When even a broomstick isn't fast enough, there are yet faster ways to travel.
	It is possible to connect a pair of locations, many miles apart, and then to take a single step from one to the other.
	Joining two points like this requires them to be well-anchored, however, with a pair of structures to fix them properly in space.
	For a novice, only \materialrefplural{stone-circle} will suffice.
	
	\materials{A \materialref{stone-circle} as the point of departure, a \materialref{stone-circle} as the point of arrival, an iron rod driven halfway into the ground at the point of departure}
	
	The ritual takes an hour, and at its conclusion the witch and her familiar may step from the point of departure to the point of arrival.
	Anything they are wearing or carrying is transported with them, except other creatures.
	Upon arrival, the iron rod used in the ritual is buried halfway in the ground at the destination, now upside down.
	
	The ritual is performed at the departure point, but the witch must be able to picture the arrival point clearly.
	As such, she must have seen it at some point previously.
	The \materialref{stone-circle} at the arrival point must still be standing.
	If it is not, the witch receives no warning until she attempts the passage at the conclusion of the ritual.
	Then, she must make a Test to avoid the disastrous consequences of an incomplete transposition.
}

\feat{Rapid Transposition}{ritual-teleport-speed}{10}{
	\skillref[1]{ritual-magic},
	\featref{ritual-teleport}
}{
	You may perform the \featref{ritual-teleport} in just five minutes.
}

\feat{Transpose Company}{ritual-teleport-others}{15}{
	\skillref[1]{ritual-magic},
	\featref{ritual-teleport}
}{
	Travelling by yourself is lonely, but you've got the hang of bringing company.
	When you perform the \featref{ritual-teleport}, you may leave the connection open.
	For the next {\round}, others may take the same step from the point of departure to the point of arrival, also bringing everything they carry.
	Objects must be carried by just one person; if someone passes through carrying just one end of an object, the object doesn't usually survive having its two ends several miles apart.
	
	The connection can be severed early by pulling the iron rod out of the ground, at either end.
	It winds up at whichever end it is pulled out from, although remains only at the departure end if the connection closes naturally.
}

\feat{Portal}{ritual-teleport-portal}{10}{
	\skillref[2]{ritual-magic},
	\featref{ritual-teleport-others}
}{
	Bringing company on your trips is all well and good, but sometimes your luggage is too heavy to carry.
	When you perform the \featref{ritual-teleport}, you may open the connection as a portal.
	It appears as a shimmering hole in the air, with the other side visible but heavily distorted.
	Creatures can pass from one side to the other as normal, but large objects---anything that will fit in the \materialref{stone-circle}---may also be carried, pushed or rolled through.
	The connection still only remains open for the usual length of time, however, and anything that is only halfway through when it closes suffers violent disassembly.
}

\feat{Rite of Return}{ritual-teleport-return}{10}{
	\skillref[1]{ritual-magic},
	\featref{ritual-teleport-others}
}{
	When you leave open the connection in a \featref{ritual-teleport}, you may make it bidirectional.
	If you do so and the connection closes naturally, the iron rod is left in the ground at both ends.
	There is still only one, however, and pulling it out at one end causes the ground to swallow it at the other.
}

\feat{Extended Transposition}{ritual-teleport-duration}{10}{
	\skillref[2]{ritual-magic},
	\featref{ritual-teleport-others}
}{
	%Slight changes to the \featref{ritual-teleport} allow you to leave the connection open longer.
	When you leave the connection open in a \featref{ritual-teleport}, you may leave it open for an hour.
	It still ends early if the rod is pulled out.
	Each \materialref{stone-circle} may only support one connection at a time.
}

\feat{Indefinite Transposition}{ritual-teleport-duration-2}{20}{
	\skillref[3]{ritual-magic},
	\featref{ritual-teleport-duration}
}{
	When you recklessly punch holes in the fabric of space, they stay punched.
	When you leave the connection open in a \featref{ritual-teleport}, you may leave it open indefinitely.
	It still ends early if the rod is pulled out.
}
