\chapter{Ritual Magic}
\chaplabel{ritual-magic}

\section{Ritual Circle Augmentation}
\seclabel{ritual-circle-augmentation}

Normally, it is impossible to draw overlapping or concentric \materialrefplural{ritual-circle}.
However, some \materialrefplural{ritual-circle} are simple and adaptable enough that they can be used to {\augment} an existing circle.
Sometimes this changes the properties of the existing circle, sometimes it simply serves as two independent but coinciding circles.
Various feats will grant the ability to perform such {\augmentation}, and lay out its effects.

{\augmenting} a circle takes as long as scribing another circle of the same size as the one being {\augmented}.
Each circle may only have one {\augmentation}; one more leaves too many overlapping lines and renders the entire mess dysfunctional.

\section{Feats}

\feat{Circle of Containment}{circle-contain}{15}{
	None
}{
	\materials{A \circleref{medium} (no larger), a pinch of sugar, a single-edged knife}
	%Sugar draws things in.
	
	Performing this rite takes about fifteen minutes, involving tracing the knife around the perimeter and sprinkling the sugar.
	You may perform the rite from inside or outside the \materialref{ritual-circle}.
	
	At the completion of the rite, the \materialref{ritual-circle} is sealed.
	Objects and creatures inside the \materialref{ritual-circle} cannot pass outside of it, and no magic from inside may affect the outside.
	The barrier extends even into the {\mentalrealm}.
	Light and sound from inside can still leave, however.
	Air may diffuse slowly across the boundary, preventing suffocation or problems with pressure.
	Furthermore, nothing inside the circle may disturb the \materialref{ritual-circle} itself.
	
	This effect lasts 24 hours, but ends early if the \materialref{ritual-circle} is broken.
	The rite to renew the effect cannot be performed while the effect remains in place.
}

\feat{Circle of Exclusion}{circle-exclude}{15}{
	None
}{
	\materials{A \circleref{medium} (no larger), a pinch of salt, a single-edged knife}
	%Salt is a preservative, driving foulness away.
	
	This functions as \featref{circle-contain}, except the other way.
	Objects, creatures, and magical effects can pass out of the \materialref{ritual-circle}, but not into it, and the \materialref{ritual-circle} may be disturbed only from the inside.
}

\feat{Circle of Severance}{circle-contain-exclude}{10}{
	\featref{circle-contain} and \featref{circle-exclude}
	%The unconventional use of "and" is intentional here, in light of the fact that so many subsequent feats use "or".
}{
	\materials{A \circleref{medium} (no larger), a pinch of sugar, a pinch of salt, a double-edged knife}
	
	This functions as \featref{circle-contain}, except in both directions.
	Objects, creatures and magical effects cannot pass into or out of the \materialref{ritual-circle}, and the \materialref{ritual-circle} cannot be disturbed in any fashion.
}

\feat{Renew Barrier}{circle-barrier-renew}{10}{
	\featref{circle-contain} or \featref{circle-exclude}
}{
	You may repeat the rite to establish a \featref{circle-contain}, \featref{circle-exclude} or \featref{circle-contain-exclude} while the effect of one is ongoing, assuming you have the feat to establish one in the first place.
	You may do so from inside or outside of the \materialref{ritual-circle}.
	The new duration runs from the completion of the repeated rite, though this cannot be used to reduce the remaining duration.
}

\feat{Stabilise Barrier}{circle-barrier-duration}{20}{
	\skillref[1]{ritual-magic},
	\featref{circle-barrier-renew}
}{
	You have learned to tweak the stability of your circles' barriers, making them endure longer, or collapse faster.
	When you establish or renew a \featref{circle-contain}, \featref{circle-exclude} or \featref{circle-contain-exclude}, you may select its duration.
	You can make it last mere minutes, many years, or even indefinitely.
	
	A \featref{circle-contain-exclude} is more unstable than the other kinds, however.
	It's duration cannot exceed the natural 24 hours.
}

\feat{Miniature Barrier}{circle-barrier-small}{15}{
	\featref{circle-contain} or \featref{circle-exclude}
}{
	You may create a \featref{circle-contain}, \featref{circle-exclude} or \featref{circle-contain-exclude} using a \circleref{small}, assuming you have the feat to establish one at all.
}

\feat{Maximise Barrier}{circle-barrier-large}{15}{
	\featref{circle-contain} or \featref{circle-exclude}
}{
	You may create a \featref{circle-contain}, \featref{circle-exclude} or \featref{circle-contain-exclude} using a larger \materialref{ritual-circle}, assuming you have the feat to establish one at all.
	There are no upper limits to the size of the \materialref{ritual-circle} you may use, except those imposed by practicality.
	However, establishing or renewing the effect still requires tracing the perimeter with the relevant knife and sprinkling the correct substance.
	Very large perimeters may require longer than 15 minutes, and a large quantity of substance to sprinkle.
}

\feat{Barrier Augmentation}{circle-barrier-augment}{15}{
	\skillref[1]{ritual-magic},
	\featref{circle-contain} or \featref{circle-exclude}
}{
	The \featref{circle-contain} and \featref{circle-exclude} are very simple \materialrefplural{ritual-circle}, and it is a simple matter to adapt them as {\augmentations} to existing circles.
	You may scribe the \featref{circle-contain} or \featref{circle-exclude} as an {\augmentation} to an existing \materialref{ritual-circle}, provided you can use these circles in the size of the circle you are {\augmenting}.
	The {\augmentation} functions as a normal circle, and the rite to active it can be performed independently of any rites using the {\augmented} circle.
	Additionally, the {\augmentation} provides the same protection to the {\augmented} circle as it provides to itself, as long as it is active.
	
	The \featref{circle-contain}, \featref{circle-exclude} and \featref{circle-contain-exclude} are all incompatible, and these {\augmentations} cannot be used upon these circles.
}

\feat{Ritual Fire}{ritual-fire}{10}{
	None
}{
	\materials{A \circleref{small} (no larger), a flame}
	
	This ritual takes about a minute, and lights a tiny flame into a merry campfire.
	The fire is large enough to fill a \circleref{small}, and produces enough heat to cook food, melt snow or keep a group of campers warm.
	It burns without fuel even in the coldest of conditions, though will gladly consume fuel thrown into it, or spread to nearby flammable material if it is available.
	
	The fire cannot be naturally extinguished and lasts 8 hours, or until the \materialref{ritual-circle} is broken.
	Any fires lit from the ritual fire and burning on regular fuel will continue afterwards.
}

\feat{Ritual Forge}{ritual-fire-2}{10}{
	\featref{ritual-fire}
}{
	You may use a \circleref{medium} (no larger) to create a \featref{ritual-fire}.
	When you do so, the fire fills the larger circle.
	It provides enough heat to keep a sizeable crowd warm on a winter night, and furthermore, burns hot enough to be used as an iron forge.
}

\feat{Pillar of Flame}{ritual-fire-burst}{15}{
	\skillref[2]{ritual-magic},
	\featref{ritual-fire}
}{
	Instead of allowing your \featref{ritual-fire} to burn its heat out over many hours, you may release it all in one moment of glorious conflagration.
	As an {\action}, you may throw \herb{dried corn kernels}{2} onto a \featref{ritual-fire}.
	On the following {\round}, the fire erupts in a massive pillar of flame.
	This extinguishes the \featref{ritual-fire}.
	
	The pillar has enough heat to reduce wood to ash, and enough force to blow masonry apart.
	Any creature caught in the pillar comes to a quick end.
	However, the blast is contained entirely within the radius of the \materialref{ritual-circle}; those outside feel nothing more than an uncomfortable wave of heat.
	
	The pillar from a \featref{ritual-fire} reaches 10 metres high: enough to blow through a couple of storeys of a house.
	The pillar from a \featref{ritual-fire-2} reaches about 100 metres, and can be seen from more than \SI{30}{\kilo\metre} away, particularly at night.
}

\feat{Rite of Transposition}{ritual-teleport}{15}{
	None
}{
	When even a broomstick isn't fast enough, there are yet faster ways to travel.
	It is possible to connect a pair of locations, many miles apart, and then to take a single step from one to the other.
	Joining two points like this requires them to be well-anchored, however, with a pair of structures to fix them properly in space.
	For a novice, only \materialrefplural{stone-circle} will suffice.
	
	\materials{A \materialref{stone-circle} as the point of departure, a \materialref{stone-circle} as the point of arrival, an iron rod driven halfway into the ground at the point of departure}
	
	The ritual takes an hour, and at its conclusion the witch and her familiar may step from the point of departure to the point of arrival.
	Anything they are wearing or carrying is transported with them, except other creatures.
	Upon arrival, the iron rod used in the ritual is buried halfway in the ground at the destination, now upside down.
	
	The ritual is performed at the departure point, but the witch must be able to picture the arrival point clearly.
	As such, she must have seen it at some point previously.
	The \materialref{stone-circle} at the arrival point must still be standing.
	If it is not, the witch receives no warning until she attempts the passage at the conclusion of the ritual.
	Then, she must make a Test to avoid the disastrous consequences of an incomplete transposition.
}

\feat{Rapid Transposition}{ritual-teleport-speed}{10}{
	\skillref[1]{ritual-magic},
	\featref{ritual-teleport}
}{
	You may perform the \featref{ritual-teleport} in just five minutes.
}

\feat{Transpose Company}{ritual-teleport-others}{15}{
	\skillref[1]{ritual-magic},
	\featref{ritual-teleport}
}{
	Travelling by yourself is lonely, but you've got the hang of bringing company.
	When you perform the \featref{ritual-teleport}, you may leave the connection open.
	You need not pass through yourself.
	For the next {\round}, others may take the step from the point of departure to the point of arrival, also bringing everything they carry.
	Objects must be carried by just one person; if someone passes through carrying just one end of an object, the object doesn't usually survive having its two ends several miles apart.
	
	The connection can be severed early by pulling the iron rod out of the ground, at either end.
	It winds up at whichever end it is pulled out from, although remains only at the departure end if the connection closes naturally.
}

\feat{Portal}{ritual-teleport-portal}{10}{
	\skillref[2]{ritual-magic},
	\featref{ritual-teleport-others}
}{
	Bringing company on your trips is all well and good, but sometimes your luggage is too heavy to carry.
	When you perform the \featref{ritual-teleport}, you may open the connection as a portal.
	It appears as a shimmering hole in the air, with the other side visible but heavily distorted.
	Creatures can pass from one side to the other as normal, but large objects---anything that will fit in the \materialref{stone-circle}---may also be carried, pushed or rolled through.
	The connection still only remains open for the usual length of time, however, and anything that is only halfway through when it closes suffers violent disassembly.
}

\feat{Rite of Return}{ritual-teleport-return}{10}{
	\skillref[1]{ritual-magic},
	\featref{ritual-teleport-others}
}{
	When you leave open the connection in a \featref{ritual-teleport}, you may make it bidirectional.
	If you do so and the connection closes naturally, the iron rod is left in the ground at both ends.
	There is still only one, however, and pulling it out at one end causes the ground to swallow it at the other.
}

\feat{Extended Transposition}{ritual-teleport-duration}{10}{
	\skillref[2]{ritual-magic},
	\featref{ritual-teleport-others}
}{
	%Slight changes to the \featref{ritual-teleport} allow you to leave the connection open longer.
	When you leave the connection open in a \featref{ritual-teleport}, you may leave it open for an hour.
	It still ends early if the rod is pulled out.
	Each \materialref{stone-circle}---or \materialref{standing-stone}, using \featref{ritual-teleport-no-circle-arrive} and \featref{ritual-teleport-no-circle-depart}---may only support one connection at a time.
}

\feat{Indefinite Transposition}{ritual-teleport-duration-2}{20}{
	\skillref[3]{ritual-magic},
	\featref{ritual-teleport-duration}
}{
	When you recklessly punch holes in the fabric of space, they stay punched.
	When you leave the connection open in a \featref{ritual-teleport}, you may leave it open indefinitely.
	It still ends early if the rod is pulled out.
}

\feat{Immediate Transposition}{ritual-teleport-speed-2}{15}{
	\skillref[3]{ritual-magic},
	\featref{ritual-teleport-speed}
}{
	You've figured out which parts of the \featref{ritual-teleport} are actually the most important, and you've reduced it to a mere few flicks of the wrist as you drive the rod into the ground.
	You may perform the \featref{ritual-teleport} as an {\action}.
	Furthermore, you may pass through and remove the rod from the ground again, or even \actionref{ready} to remove the rod from the ground, as part of the same {\action}.
}

\feat{Arrival Point}{ritual-teleport-no-circle-arrive}{20}{
	\skillref[2]{ritual-magic},
	\featref{ritual-teleport}
}{
	The \featref{ritual-teleport} needs solid structures at both ends to anchor the connection.
	One \materialrefplural{standing-stone} doesn't have the mass that a ring of them does, but it's still significant enough to receive a connection.
	
	You may use a \materialref{standing-stone} instead of a \materialref{stone-circle} for the \emph{arrival} point of the \featref{ritual-teleport}.
	You can still create a bidirectional connection, if you have \featref{ritual-teleport-return}, but must create it from the \materialref{stone-circle}.
	You can use \materialref{standing-stone} at both ends if you also have \featref{ritual-teleport-no-circle-depart}.
}

\feat{Departure Point}{ritual-teleport-no-circle-depart}{20}{
	\skillref[2]{ritual-magic},
	\featref{ritual-teleport}
}{
	Performing a ritual without a circle is pretty funny business, but it still seems to work if you run around a \materialref{standing-stone}.
	
	You may use a \materialref{standing-stone} instead of a \materialref{stone-circle} for the \emph{departure} point of the \featref{ritual-teleport}.
	You can still create a bidirectional connection, if you have \featref{ritual-teleport-return}, but must create it from the \materialref{standing-stone}.
	You can use \materialref{standing-stone} at both ends if you also have \featref{ritual-teleport-no-circle-arrive}.
}

\feat{Rite of Invisibility}{ritual-hide}{10}{
	None
}{
	\materials{A \circleref{medium}, a cloak}
	
	Performing this rite takes a minute, and requires you to stand in the circle and don the cloak.
	At the conclusion, you---along with the cloak, and your hat---become invisible as long as you remain in the unbroken circle wearing the cloak.
	You may also conceal other objects beneath the cloak, but any objects outside remain visible.
	You can still see yourself and your equipment.
}

\feat{Bestow Invisibility}{ritual-hide-others}{10}{
	\featref{ritual-hide}
}{
	You may perform the \featref{ritual-hide} upon other people and creatures.
	This requires one cloak per person, but several people can share a circle.
	People affected by the ritual, within the same circle, can see each other just as they can see themselves.
}

\feat{Rite of Concealment}{ritual-hide-all}{10}{
	\skillref[1]{ritual-magic},
	\featref{ritual-hide-others}
}{
	\materials{A \circleref{medium} (no larger), a sheet}
	
	Performing this rite takes a minute, and requires throwing the sheet over something inside.
	The sheet itself, and anything covered by it, becomes invisible as long as it is within the unbroken circle.
	The ground beneath, including the \materialref{ritual-circle} itself, remains visible.
	
	Creatures and people underneath the sheet can still see as normal within it.
	However, their vision of anything outside is obstructed by the sheet itself, and lifting it to peek outside reveals them.
}

\feat{Self-Concealing Circle}{ritual-hide-circle}{20}{
	\skillref[1]{ritual-magic},
	\featref{ritual-hide}
}{
	Turning a person invisible is of rather limited use as long as they are standing in a perfectly visible magical circle.
	You've finally figured out a way around that.
	When you perform the \featref{ritual-hide} or \featref{ritual-hide-all}, you may also turn the \materialref{ritual-circle} itself invisible for as long as the ritual lasts.
}

\feat{Combined Concealment}{ritual-hide-combine}{10}{
	\skillref[1]{ritual-magic},
	\featref{ritual-hide-all}
}{
	You can leverage the similarity between the \featref{ritual-hide} and the \featref{ritual-hide-all}, and may use a \materialref{ritual-circle} scribed in either design for either rite.
	Notably, this lets you use the same \materialref{ritual-circle} for both rites simultaneously.
	This does not affect the size of the design you can use, only the design.
}

\feat{Total Concealment}{ritual-hide-other-senses}{10}{
	\featref{ritual-hide}
}{
	When you perform the \featref{ritual-hide} or \featref{ritual-hide-all}, you may extend the effect to senses other than sound.
	Smells and sounds emitted by the affected creatures and objects are not detectable to anyone they are invisible to.
	Other creatures under the effect of the rite, or inside the sheet, respectively, can still hear and smell them, just as they can see them.
	This does not extend to touch; everyone and everything still exists and there is nothing you can do to prevent people bumping into them.
	
	A creature under the effect of the \featref{ritual-hide} may intentionally circumvent this effect when they speak, sing, whistle, or the like.
}

\feat{Hiding Nook}{ritual-hide-small}{10}{
	\featref{ritual-hide}
}{
	You may use a \circleref{small} for the \featref{ritual-hide} and \featref{ritual-hide-all}.
	It is incredibly difficult to fit than four people in a single \circleref{small}, and even then they'd better be very close friends.
}

\feat{Hiding Houses}{ritual-hide-large}{15}{
	\skillref[1]{ritual-magic},
	\featref{ritual-hide-all}
}{
	You may perform the \featref{ritual-hide-all} using a larger \materialref{ritual-circle}, though you may soon find yourself needing a very large sheet.
}

\feat{Concealing Augmentation}{ritual-hide-augment}{20}{
	\skillref[1]{ritual-magic},
	\featref{ritual-hide-circle}
}{
	Applying what you've learned from the \featref{ritual-hide}, you've learned to conceal your other \materialrefplural{ritual-circle}.
	You have learned an {\augmentation} that hides the {\augmented} circle, and itself, from sight.
	
	The {\augmented} circle does not function while it is hidden in this way---visibility is important to \materialrefplural{ritual-circle}.
	But it is a simple matter to break the {\augmentation} with a scratch through one line, and then to reconnect the same line when necessary.
}
