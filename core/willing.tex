\chapter{Willing}
\chaplabel{willing}

\discref{willing} is the most raw and versatile application of a witch's magic.
Known to many layfolk as sorcery or spellcraft, it is the art of making something true simply by willing it hard enough.
Most \discref{willing} is performed without any of the accoutrements that accompany other forms of magic, and it doesn't follow the prescribed formulae of rites and brews.
This makes it the weakest form of magic in some ways, but its flexibility and ease of access more than make up for it.
So much so that every witch knows at least the basics.

Like any witchcraft, \discref{willing} is something anyone can do if they know how.
But there is a knack to it.
It requires that the witch not only \emph{want} something to be the case, but \emph{believe} that it already is.
That she outright refuses to accept any possibility that it might not, in fact, be the case.
It involves willfully deceiving not only oneself, but also the very universe.
Most people would never even think to try it, but it is among the first things any aspiring witch must learn.

The line between \discref{willing} and \discref{headology} can be a little blurred, at times.
Both have the ability to make things true by making people believe them.
Many Willers say that the difference is that \discref{willing} affects the real world, while \discref{headology} only affects other people's minds.
The Headologists point out that other people are just as much a part of the real world as any old rock is.
Some Headologists say that the difference is that \discref{headology} is about convincing other people, while \discref{willing} is about convincing yourself.
The Willers point out that it's about more than convincing yourself, it's about convincing the world.
And that includes other people.
A few say that there's no real difference at all, that it's just two ways of thinking about the same thing.
These tend to be the witches who are obnoxiously good at both, and everyone else pointedly ignores them.

One interesting property of \discref{willing} is that it cannot affect other people or animals, although you can affect yourself.
It takes more than force of will to convince someone that they're a different shape; usually this entails talking to them.
This doesn't stop people getting knocked off their feet by a gust of wind, or crushed by a falling tree, however.
Witches interested in affecting people more directly are encouraged to pursue \discref{headology}.
%Or swordplay.

\section{Willing Tests}

\discref{willing} is not about memorising rites or recipes, nor about complexities and intricacies; \attref{ken} and \attref{wit} are unimportant to most applications of \discref{willing}.
Rather, \discref{willing} is about shunting your own stubbornness and conviction up against the fabric of reality until it gives in; it depends upon raw force of \attref{will}.

Similarly, skill in \skillref{willing} represents very little in the way of knowledge, making it even more useless to a non-practitioner than most discipline skills.
Rather, the skill primarily represents a witch's ability to convince herself of things that are not yet true, in order that she may make them so.
An unskilled witch has difficulty with this, and it takes some effort to acheive even a broad, imprecise effect.
The best Willers, however, can effortlessly visualise even the faintest detail of their desired reality.

As such, the \skillref{willing} skill represent not only a witch's ability to rush or stretch her magic, but her accuracy with it.
She might use it to thread a needle, form an intricate shape, or change something subtly to avoid notice.
She may also use her \skillref{willing} skill in place of \skillref{weaponry} when attacking with an object she is controlling through \discref{willing}.

\section{Feats}

\feat{Basic Willing}{willing-basic}{10}{
	None
}{
	You can perform very basic acts of \discref{willing} upon things you can touch, given a bit of time to focus your mind and an obvious physical cue.
	Examples include:
	\begin{itemize}
		\item Lighting tinder or a candle without a spark, by cupping your hands around it and blowing on it.
		%\item Colouring or mildly flavouring a small pot of water by stirring it.
		\item Scratching writing into stone using just a fingernail.
		\item Rubbing stains out of clothing using your bare hands.
		\item Combing your hair with just your fingers.
	\end{itemize}
	The amount of time required to produce an effect depends on the desired outcome, but should be more than an {\action} without a Test.
	This ability cannot produce a lasting effect by itself.
	You can light a fire, because that sustains itself once ignited, but you cannot create, destroy or melt a pebble.
}

\feat{Kindling}{willing-fire}{15}{
	\featref{willing-basic}
}{
	You've practiced \discref{willing} a fire to life, and it's getting a lot easier for you.
	You can now ignite a fire within a dozen metres as an {\action}, with nothing more than a quick glare.
	You no longer require tinder, but still need something a fire can catch on fairly easily, such as twigs, cloth or dry leaves.
	Lighting a log or floorboards is still beyond you.
	
	The fire begins small, so will be extinguished by rain or a moderate wind before it can catch.
	A person whose clothes are ignited with this begins at \dice{1} {\fire}.
}

\feat{Fan the Flames}{willing-fire-2}{15}{
	\skillref[1]{willing},
	\featref{willing-fire}
}{
	You can use your will as a bellows, blowing a fire hotter and brighter.
	As an {\action}, you may double the size of an existing fire within a dozen metres of you.
	However, this is less effective on large fires: you can ignite more than about a cubic metre of a material in one {\action}.
	Using this against a person who is on {\fire} increases their {\fire} by 1 die.
	
	Additionally, through continuous concentration, you may double the heat and brightness of an existing fire (up to a cubic metre of it).
	This does double the rate at which it consumes fuel, however.
	A campfire affected in such a way is hot enough to forge iron with.
}

\feat{Firestarter}{willing-fire-3}{20}{
	\skillref[2]{willing},
	\featref{willing-fire-2}
}{
	By fanning the flames of a fire as you light it, you can burn bigger things.
	When you use \featref{willing-fire}, you can ignite the fire immediately on an object such as a log, or floorboards.
	The flame begins larger, enough to withstand drizzling rain or a moderate wind.
	A person whose clothes are ignited with this begins at \dice{2} {\fire}.
}

\feat{Extinguish}{willing-extinguish}{15}{
	\skillref[1]{willing},
	\featref{willing-fire}
}{
	Your experience working with fire allows you to extinguish them as easily as you light them.
	As an {\action}, you can extinguish up to a cubic metre of burning material within a dozen metres.
	The embers are still left hot to the touch, but not particularly dangerous.
}

\feat{Flamewalker}{willing-extinguish-2}{25}{
	\skillref[2]{willing},
	\featref{willing-extinguish}
}{
	You can extinguish fire near you---very near you---with only a modicum of concentration.
	As long as you are conscious, you are immune to the detrimental effects of heat and fire.
	This effect extends to your clothes, and most stuff you're carrying as long as it's not too large and doesn't extend too far from you.
}

\feat{Heatsink}{willing-extinguish-3}{15}{
	\skillref[3]{willing},
	\featref{willing-extinguish-2}
}{
	You can suck the heat from the air far and wide around you.
	As an {\action}, you can extinguish all fire within a dozen metres of you.
}

\feat{A Tool for the Job}{willing-tools-improvise}{20}{
	\featref{willing-basic}
}{
	Sometimes, the easiest way to convince someone of something is the hit them with a big stick until they agree with you.
	The world itself is no different.
	You've learned to make \discref{willing} easier using physical tools, even if they aren't the \emph{right} tools.
	
	Most simply, this means axes and knives cut just as well as ever in your hands, even if they've lost their edge.
	But you can take it even further, cutting carrots or trees with nothing more than an appropriately shaped stick.
	You can make any similarly-shaped object behave as the appropriate tool for a job.
	For a worse approximation, this may require a Test.
	A solid branch with a flat, sort of axe head shaped bit on the end will do a fine job of cutting down a tree.
	A solid branch without such an attachment would require a Test.
	A limp reed is going to be a real stretch.
	
	Such tools still obey the usual rules of \discref{willing}, and are of no additional use as weapons against people and animals.
	See \featref{headology-weapons-improvise} if you want weapons too.
}

\feat{A Hefty Tool}{willing-tools-effective}{15}{
	\skillref[1]{willing},
	\featref{willing-basic}
}{
	You can make an appropriate tool more effective when you use it.
	Or an inappropriate tool, with \featref{willing-tools-improvise}.
	
	Tools are several times more effective when you use them.
	For example, when you use a spade it always lifts clumps of dirt several times the size of the spade's head.
	You can bring down a tree that you can barely wrap your arms around with only 4 or 5 swings of an axe.
	You can bail out a rowboat with only a few scoops of a bucket.
	
	This only works as long as you are still using the tool.
	For instance, you cannot store any more water than normal in a bucket unless you are carrying it.
}

\feat{Bubbling Brook}{willing-water}{10}{
	\featref{willing-basic}
}{
	Water is considered by many to be an element of change.
	You've certainly figured out how to change it.
	While touching water, you can move it around with your mind.
	You can make it flow, swirl, form into fairly elaborate shapes, or even float into the air.
	
	You can only affect the water while it remains one continguous mass, which you must be touching.
	Afterwards, it flows normally again.
	You can only affect a couple of buckets-full at a time, and can't stretch it out over more than a couple of metres.
	You also can't move the water fast enough to hurt anybody.
	You can move other liquids if they are primarily water, such as wine, blood or most potions.
	As always with \discref{willing}, you cannot affect liquids inside a living person.
}

\feat{Water Walk}{water-walk}{20}{
	\skillref[1]{willing},
	\featref{willing-water}
}{
	You can walk on water, or any other liquid you could affect with \featref{willing-water}.
	This takes great concentration, and you cannot take an {\action} and move on the water's surface in the same {\turn}.
	You may take an {\action} if you stand still on the water, however.
	
	If the water is flowing, you will be carried with it.
	Staying upright on fast flowing or turbulent water may require a Test, and the effect requires you to stay on your feet; falling prone will cause you to fall into the water.
	You may take use an entire {\turn} to clamber onto the water, if you are swimming at the surface.
}

\feat{River Run}{water-walk-2}{15}{
	\skillref[2]{willing},
	\featref{water-walk}
}{
	Walking on water has become second nature to you.
	You may take {\actions} while moving.
	Additionally, flow and turbulence pose you no threat.
	You may treat water you are standing on as though it were not flowing, and you can remain on the water's surface even when prone.
	Lastly, climbing upright onto the water while swimming at the surface is treated as though you are merely standing from being prone.
}

\feat{Condensation}{willing-water-vapour}{10}{
	\featref{willing-water}
}{
	The air is filled with water, and the skilled may draw it out to form liquid.
	You can draw it out within a couple of metres, into a container, spilling it on the ground, or holding it using \featref{willing-water}.
	
	Under normal conditions, you can produce about a litre a minute this way.
	However, this will be faster in a swamp or slower in dry air.
	In some situations, such as a desert or a burning building, the GM may call for a Test to gather enough water to be useful at all.
	
	You may also evaporate liquid water into the air in the same way, at the same rate.
	As such, you can perform \discref{brewing} that would normally require a \mixcreationref{still} by hand, although it still takes a couple of hours.
}

\feat{Illuminuous}{willing-light}{10}{
	\featref{willing-basic}
}{
	Your bright or gloomy moods are more literal than most, as you Will your surroundings brighter or darker.
	This effects a region centered on you and extends up to a few metres, enough to fill a room in a cottage.
	You can change the light level by about the amount that would be emitted by a few candles, just enough to read by.
	Light you cast this way has no apparent source; it simply suffuses the area.
	Maintaining this effect requires a minimum of concentration, and does not impede your other activities.
}

\feat{Rope Dance}{willing-rope}{15}{
	\featref{willing-basic}
}{
	Anyone can move a whip with their hand, but you can move one with your mind.
	That said, you still need to hold it{\dots}
	
	While holding a whip, rope, string, or thread up to 2 metres long, you can move it with your mind.
	You can barely lift any weight other than the rope itself; even a small knife tied to the end is a struggle.
	You can still yank on the rope or reel something in with your hands, of course.
	You do have very fine control, however, comparable to your manual dexterity.
	
	%TODO: Weapon damage? Add whips to the main table?
}

\feat{Know the Ropes}{willing-rope-2}{15}{
	\skillref[1]{willing},
	\featref{willing-rope}
}{
	As the finest rope-whipper in the West (or near enough), and you can thread a needle at twenty paces.
	
	When using \featref{willing-rope}, you can affect up to 20 metres of rope at a time.
	Additionally, you may divide this length between as many ropes as you can hold.
	This doesn't, however, grant you any extra ability to multitask, so this is about as practical as trying to use two whips at once.
}

\feat{Gust}{willing-wind}{20}{
	\featref{willing-basic}
}{
	Your mind can stir the air around you.
	You can create gusts of air within a dozen metres.
	Very minor and imprecise effects, like blowing hair or a cloak, don't require much effort, and can be done without an {\action}.
	If you concentrate as an {\action}, you can produce enough wind to send dishes flying, or to set a rock slowly rolling.
	With a Test, you might even produce enough of a gust to knock a person down.
	The gust must fairly localised; you can't shift enough air to blow at anything larger than a person.
}

\feat{Breath}{willing-breathing}{20}{
	\skillref[1]{willing},
	\featref{willing-wind}
}{
	You hold the wind within you.
	You are immune to suffocation and drowning.
}

\feat{Updraft}{willing-wind-self}{15}{
	\skillref[1]{willing},
	\featref{willing-wind}
}{
	By surrounding yourself in a localised updraft, you can leap higher and slow yourself as you fall.
	You may leap 3 times as far or high.
	As long as you are conscious, you may fall up to 5 metres safely, and subtract 5 metres from the distance fallen when suffering {\damage} as a result.
	%TODO: Check this after writing the falling rules.
}

\feat{Cushion of Air}{willing-wind-self-2}{20}{
	\skillref[2]{willing},
	\featref{willing-wind-self}
}{
	You can cushion your fall with the wind.
	As long as you are conscious, you do not suffer any {\damage} from falling.
}

\feat{Team Lift}{willing-wind-others}{15}{
	\skillref[2]{willing},
	\featref{willing-wind-self},
	\featref{willing-wind-aoe}
}{
	You may extend your own updraft to surround those around you.
	All creatures of your choice within a dozen metres may benefit from \featref{willing-wind-self}, and \featref{willing-wind-self-2}, if you have it.
}

\feat{Breeze}{willing-wind-aoe}{15}{
	\skillref[1]{willing},
	\featref{willing-wind}
}{
	\featref{willing-wind} produces only small gusts, to blow at one person or object.
	You can affect a larger area, altering the direction and strength of the wind everywhere within a dozen metres of you.
	You can't create more than a moderate wind with this effect; enough to pick up light objects and roll them away, and enough to be uncomfortable, but not enough to knock people down.
	You can also counteract a wind of up to the same strength, creating a region of dead calm.
	
	This effect acts in very broad strokes.
	It always affects a roughly spherical region around you---although you may reduce the radius---and affects the whole area in the same way.
	The effect requires minimal effort to maintain; you may begin or alter it on your {\turn} without requiring an {\action}.
}

\feat{Long-Winded}{willing-wind-range}{20}{
	\skillref[1]{willing},
	\featref{willing-wind}
}{
	The air stretches everywhere, always.
	The wind does not stop short after a few metres!
	There is no reason your wind should be so limited, either!
	
	You may ignore the range limitation for \featref{willing-wind} and \featref{willing-wind-others}.
	You may use these feats, if you have them, on anything you can see.
}

\feat{Wind}{willing-wind-aoe-2}{15}{
	\skillref[2]{willing},
	\featref{willing-wind-aoe},
	\featref{willing-wind-range}
}{
	You can affect the speed and direction of the wind over a large region.
	This follows the same rules as \featref{willing-wind-aoe}, however the effect may extend for many kilometres around you.
	
	Shifting such large volumes of air can take quite some time.
	A major change, such as reversing the direction, or changing a strong breeze to a dead calm, may require several {\rounds} to take effect.
}

\feat{Cloud}{willing-weather}{10}{
	\skillref[1]{willing},
	\featref{willing-wind-range},
	\featref{willing-water-vapour}
}{
	You can affect the weather, in small ways.
	You cannot change general weather patterns over an area, but you can create or disperse the odd cloud, and cause it to hold or drop its rain.
	These changes can be performed in the background, without requiring {\actions}.
	
	Your control is fine enough to pass a cloud in front of the sun at a certain moment, or conversely to break a hole in the cloud for the sun to shine through.
	You can create a dry patch in light rain, or a patch of light rain on an overcast day.
	Even in torrential rain, you can at least ease it slightly for a small group of people.
}

\feat{Weather}{willing-weather-2}{20}{
	\skillref[2]{willing},
	\featref{willing-weather},
	\featref{willing-wind-aoe-2}
}{
	You can alter the weather, even causing meteorological changes over whole regions.
	You can turn a clear day overcast, or even bring rain.
	You can banish the rain and cloud to bring the bright sun.
	You can turn a chilly day warm, or a balmy day parky.
	
	The weather must be seasonally appropriate; you can't make it snow outside of winter or baking hot outside of summer.
	You cannot create winds stronger than those achievable using \featref{willing-wind-aoe-2}.
	Thunderstorms prove particularly difficult to create; you might manage it if conditions are already relatively close, but you cannot create one from a clear or even merely overcast day.
	
	Changes also take a while to take effect.
	If there isn't a cloud in the sky, turning the day overcast might take fifteen minutes, and the rain mightn't start for half an hour.
	These changes can be performed in the background, without requiring {\actions}.
}

\feat{Storm}{willing-weather-3}{20}{
	\skillref[3]{willing},
	\featref{willing-weather-2}
}{
	You have become even more proficient at influencing the wind and weather.
	You can produce a thunderstorm from even the clearest of days inside half an hour.
	You can also create wind strong enough to blow over less robust trees, heavily impede walking and running, and possibly damage poorly-built buildings.
	However, you create this wind as a windstorm, and while you control its general direction you have very little influence over the finer details.
}

\feat{Lightning}{willing-lightning}{10}{
	\skillref[2]{willing},
	\featref{willing-weather}
}{
	When a thunderstorm brews, when the pressure aches in your bones, when the electricity can almost be heard humming in the air, then it takes the barest spark to send lightning racing from cloud to ground.
	
	As an {\action} in a thunderstorm, you can cause a lightning strike somewhere within several kilometres.
	You have only the barest control over the location of the lightning: a margin of error of a kilometre or two.
	Essentially, you have just enough control to determine whether the flash illuminates your face or casts you in silhouette.
	There is a limited potential within a thunderstorm's clouds, and creating lightning flashes more closely spaced than a minute is difficult.
}

\feat{Smite}{willing-lightning-2}{10}{
	\skillref[3]{willing},
	\featref{willing-lightning}
}{
	When you use \featref{willing-lightning}, you may aim your lightning with perfect precision as long as you are aiming at the highest point for a kilometre in any direction.
}

\feat{Megaphone}{willing-voice}{15}{
	\featref{willing-wind}
}{
	You can Will your voice to travel louder and further through the air around you.
	When you speak, sing, whistle or so on, you may make it up to twice as loud.
	However, you may also make it carry far further---up to several kilometres---even if you choose not to make it louder.
}

\feat{Ventriloquism}{willing-voice-2}{20}{
	\skillref[1]{willing},
	\featref{willing-voice}
}{
	You have developed enough control over the air that you can speak directly with your mind, without involving your mouth.
	Consequently, you need not speak from yourself, but can speak from anywhere you can see.
	In fact, you can even speak from several places simultaneously, and may whisper directly into several people's ears without the risk of anybody else hearing.
}

\feat{Silence}{willing-silence}{20}{
	\skillref[1]{willing},
	\featref{willing-wind}
}{
	Much as you can move the air around you, you can also still it, creating silence.
	You may prevent any sound escaping from yourself, silencing your voice, your breathing, you footsteps and so on.
	This only prevents sounds originating from right near you; you can silence a twig snapping under your boot, but not a vase you've knocked onto the floor.
}

\feat{Dead Air}{willing-silence-2}{15}{
	\skillref[1]{willing},
	\featref{willing-silence},
	\featref{willing-wind-aoe}
}{
	You may extend your bubble of silence to cover the area you may affect with \featref{willing-wind-aoe}.
	No sound can originate within, enter, or pass through this region.
}
