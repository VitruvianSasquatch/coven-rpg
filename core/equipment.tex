\chapter{Tools of the Craft}
\chaplabel{equipment}

\dropcap{Most} witches don't work naked, despite what some folks would have you believe.
In fact, many disciplines of magic require a fair bit of equipment; a lot of witches keep their pockets stuffed full of ritual components, and handy tools.

\section{Acquiring Equipment}

\titleemph{Coven} encourages a somewhat fast and loose approach to equipment.
Players aren't required to track every piece of string their character keeps in her pockets.
In fact, they should generally be assumed to have most of what they need for their magic, and other handy items and tools they might reasonably keep in their pockets.
\capital{\practitioners{ritual-magic}} are assumed to carry chalk; \practitioners{sympathetic-magic} normally have a few \materialrefplural{poppet}; most \practitioners{necromancy} carry a candle.
Furthermore, \practitioners{brewing} are assumed to have a few doses of any potion they know how to prepare on hand, most of the time.
If in doubt, the GM has final say over whether a character has a particular piece of equipment on hand, but should account for the character's personality in assessing this.

More substantial or important equipment should be kept track of, however, and the GM may need to be made aware of it.
It makes a lot of difference to how people react when someone is wearing a \weaponlink{sword}{hand-weapon}, or carrying a \weaponref{bow}.
And the GM may impose penalties on some Tests if you're lugging 50 kilos of grain around.

Even for these more substantial items, there is no cost to acquiring equipment at character creation.
Your witch has been knocking around for a few years, at least, and is assumed to have picked up everything she needs on a day-to-day basis.
More outlandish things should be checked with your GM, but even a decent \weaponlink{sword}{hand-weapon}, a \weaponref{bow} suited to her \attref{might}, a large \mixcreationref{cauldron}, and a \mixcreationref{still} are reasonable items to assume a witch would own, if she wants them.

There are a few items that should be established at the start of play, assuming any witches in the coven are interested in the relevant disciplines.
Does the coven have a \materialref{crystal-ball}, and are there any nearby \materialrefplural{standing-stone}, or \materialrefplural{stone-circle}?
These should be a matter of discussion between the GM and all the players.

There are also a couple of essential items which every witch is assumed to begin with, but which may be detailed and fleshed out as the player wishes: namely, a trained {\broomstick} and {\thehat}.
Lastly, a player should decide on the details and contents of their witch's {\cottage} and {\garden}.



\section{Barter}

Sometimes a witch needs to acquire new equipment during the course of the game.
If she doesn't have it in her pockets, she might have something at home.
She might find something suitable out in the wilderness, or she might be able to make it.
But, for some of the trappings of civilisation, she might be inclined to trade for them.

\titleemph{Coven}'s early medieval setting assumes a world not of currency, but of barter.
Village folk trade goods for goods; ten chickens for a cow, a new plough for a month's potatoes, shoeing a horse for some loaves of bread.
Witches mostly trade in favours, earned in cures and other services.
Cure someone's pig, ease their back ache, deliver their son safely; they'll be inclined to keep you well fed, or lend you what other aid they can.

A witch with a steading, who puts the effort in to help her people, earns enough favours to keep fed, and to pick up a few other items as she needs them.
If if she goes into the village looking for something, she can often find her someone who owes her one, or is willing to help to stay in her good graces.
As such, it isn't necessary to spend time playing out the acquisition of necessities, like food, and simple magical components.
However, for something more substantial, or urgent, the village folk might want a favour done in kind first.
Or they might take a little persuading.
The GM can use this to add a little extra challenge, or drama, as appropriate.



\section{Losing Equipment}

Some role-playing games treat a player character's equipment as sacrosanct.
Players have often paid a significant character creation cost for their equipment, and it could be unfair to take this from them.
\titleemph{Coven} intentionally avoids this idea; the players have paid nothing for their character's equipment, and shouldn't expect that they can always hold on to it.

As a GM, you should feel free to deprive the characters of their equipment whenever the plot, drama, or dice call for it.
Steal it, break it, drop it off a cliff; go wild.
The rules make generous provision for improvising alternative equipment, and finding other solutions to problems; make the players use these.
Keep them on their toes, always thinking, solving problems, improvising.
That's what being a witch is about, after all.
And if they come up with an interesting new use for an item, or a novel substitute for a magical component, let them have it.
But don't let them rely on it for everything.

This is not to say that a character cannot have equipment that they're attached to: their grandmother's \materialref{crystal-ball}, their father's \weaponlink{sword}{hand-weapon}, their wedding ring.
Nor should you deprive them of these things without reason.
But feel free to make them \emph{fight} for them.
Putting a precious object at risk is an excellent way to introduce more drama, to raise the stakes.
To make it personal.

This goes double for a witch's cottage.
While most village folk will respect a witch's cottage, her true enemies certainly won't.
Don't always give her a place to rest, to recover, or to prepare.
Sometimes, nowhere is safe.
Her cottage is her home turf, and she ought to have the advantage there, but she can still be forced to fight for it.



\section{Common Magical Components}

The various rites and magics of the various disciplines of witchcraft require too many different materials to enumerate here.
However, there a few components that make a regular appearance.
Some details of their acquisition, construction and use are given here.

\subsection{Ritual Circles}
\materiallabel{Ritual Circle}{Ritual Circles}{ritual-circle}
\circlelabel{Small}{small}
\circlelabel{Medium}{medium}
\circlelabel{Large}{large}

A \materialref{ritual-circle} describes any large arrangement of symbols or shapes required by a rite.
They are traditionally drawn on the floor in chalk, but other methods are far from uncommon; the visibility and accuracy are the only important aspects for most rites.
Some witches use paint for permanence, or even chisel their circles into stone.
Many a witch in a hurry has scratched their circles into the dirt with the toe of their boot.
Some witches even embroider their circles upon sheets of fabric that can be rolled up and laid down where needed.
However, a roll bearing even the smallest of circles is most of the height of a man.

Each rite requires a \materialref{ritual-circle} of a particular design.
The shape is different for every rite, but the same each time the rite is performed---even by a different witch.
As such, it is common practice to scribe a circle just once and use it to perform the rite many times.
\capital{\materialrefplural{ritual-circle}} are not even universally circular, although it is the most common shape and almost all have some sort of symmetry.
Squares, triangles and hexagons are not uncommon, and pentagrams are particularly common in certain disciplines.

A \materialref{ritual-circle} must be horizontal, and on a relatively flat surface.
It must be the right way up---on a floor, not a ceiling.
Obstructions inside it, such as large rocks, furniture, or the pillars of a building, mean that using it for magic requires an appropriate Test.
No two \materialrefplural{ritual-circle} may overlap in any fashion, or even encompass one another.

Rituals using a \materialref{ritual-circle} can be performed by a witch standing inside or outside the circle, unless specified otherwise.
If some particular object is clearly the focus of the ritual---such as a corpse being reanimated, or an object being destroyed---it must be inside the circle, and must remain there for the entire casting of the rite.

\capital{\materialrefplural{ritual-circle}} are classified primarily by their size.
A rite can be performed with a larger circle than it requires, unless specified otherwise.
Furthermore, the following list only describes the sizes typically \emph{required} by rites; circles of intermediate sizes, or even larger than \circlerefbare{large}, can be scribed, and might be useful in some situations.
Legends even tell of a witch that once scribed a \materialref{ritual-circle} around an entire castle.
\begin{itemize}
	\item A \circleref{small} can be scribed entirely in arm's reach while standing in one spot.
		It can comfortably be drawn in a couple of minutes.
	\item A \circleref{medium} is a few paces across.
		Most houses should have a room large enough to draw one in, if the furniture is moved.
		It can comfortably be drawn in a quarter of an hour.
	\item A \circleref{large} is at least two dozen paces across.
		A ballroom or village hall is probably the only place one could be drawn indoors, so most are drawn outside.
		At least a couple of hours are required to draw such a circle without haste.
\end{itemize}

\subsection{Standing Stones}
\materiallabel{Standing Stone}{Standing Stones}{standing-stone}

A \materialref{standing-stone} is simply a single, unbroken piece of stone, taller than a man and standing upright.
The shape is rather unimportant, as long as it is plainly recognisable as being upright---its longest dimension must be the vertical one.
These are naturally available in some locations, although more often they are lying down and might need a group of strong people to push them upright.
Elsewhere, they must be quarried out, or carried in.

\subsection{Megalithic Circles}
\materiallabel{Megalithic Circle}{Megalithic Circles}{stone-circle}

Some rites require a circle of\materialrefplural{standing-stone}, called a \materialref{stone-circle}.
Such a circle must be at least the size of a \circleref{large}, with at least a dozen stones.
The arrangement and shape of the stones is unimportant, as long as it is recognisable as a ring, and so the same circle can be used for all rites that require one.
Constructing a \materialref{stone-circle} is no easy task, typically requiring weeks of work by much of a village, even if the site is quite close to a stone quarry.

\subsection{Taglocks}
\materiallabel{Taglock}{Taglocks}{taglock}

A \materialref{taglock} is any part of a person's body, such as a piece of flesh, a strand of hair, a nail clipping, a drop of blood, or a gob of saliva.
It is often used to bind a spell to a particular target.
It can always be picked off a person---although taking a hair without being noticed might be difficult---but people often leave \materialrefplural{taglock} behind them, especially in places they frequent.
Finding a \materialref{taglock} in a place you suspect someone might have left on, such as their house or a bed they've slept in, typically uses \skillref{perception}.

\subsection{Poppets}
\materiallabel{Poppet}{Poppets}{poppet}

A \materialref{poppet} is an abstract representation of a person, although not a particular person.
Voodoo dolls are a typical example.
A \materialref{poppet} can be crafted from cloth, wood, clay, wax, or other suitable material.
It should be recognisable as a human, bearing four appropriately-arranged limbs, a head, and two eyes.
However, if it is to be used in \discref{sympathetic-magic} affecting a non-human creature, it should resemble whichever creature the magic is intended to affect.
A \materialref{poppet} should be at least a handspan tall, though can be much larger.

\subsection{Effigies}
\materiallabel{Effigy}{Effigies}{effigy}

An \materialref{effigy} is much like a \materialref{poppet} and follows all the rules for one, except that it represents a particular person and must be crafted in their likeness.
It can be used only to affect the person it resembles.
Ideally, an \materialref{effigy} should be recognisable to even passing acquaintances of the person it is supposed to represent.
Less recognisable \materialrefplural{effigy} will require an appropriate Test to be used for magic.

\subsection{Crystal Balls}
\materiallabel{Crystal Ball}{Crystal Balls}{crystal-ball}

A \materialref{crystal-ball} is a polished sphere made of glass or quartz crystal, often used in scrying.
\capital{\materialrefplural{crystal-ball}} are very difficult to produce or acquire.
A player must obtain permission from the GM to have one among their starting equipment, and the GM is within their rights to place some limitation upon it.
A witch just starting out in her own steading will likely have a substandard ball---scratched, or not quite spherical.
Even experienced witches often share one among the whole coven, and an apprentice likely has to borrow her teacher's.

Most witches leave their \materialref{crystal-ball} in their cottage, to prevent it coming to harm.
Acquiring a new ball, if one is smashed, can be quite a challenge---often an excellent opportunity for the GM to present an adventure.

Taking a \materialref{crystal-ball} into bright sunlight is dangerous, as it has a tendency to focus the light and set things on fire.
Most witches cover theirs with a dark cloth when it isn't in use, and scry in dimly-lit rooms.

\subsection{Tarot Cards}
\materiallabel{Tarot Card}{Tarot Cards}{tarot}

A deck of \materialrefplural{tarot} is much like a deck of standard playing cards, with the additional of Knights, joining the Kings, Queens, and Jacks, as well 22 other illustrated cards, such as The Fool, Justice, and The Tower; the Major Arcana.
Creating such a set of cards is not prohibitively expensive, but performing all the illustrations and so forth certainly requires a lot of time.
Any witch who requires such a deck will own one, but replacing it should it be lost can take quite a while.

In a pinch, it isn't too hard to use a regular deck of cards in place of \materialrefplural{tarot}, although it certainly requires a Test.
Some expert \practitioners{divination} have been known to use dice, or some other instrument of chance, in place of using cards at all.

\section{Herbs}
\materiallabel{Herb}{Herbs}{herb}

\capital{\materialrefplural{herb}} are an important component of most potions and poultices, as well as many spells.
Note that the term ``\materialref{herb}'' is used to encompass many things that are not technically herbs at all, such as fruit, fungi, tree bark and so on.

There are hundreds, perhaps thousands of different \materialrefplural{herb}, so instead of tracking every one, they are simply divided into categories based upon rarity.
Each potion or spell lists the highest rarity of \materialref{herb} that it requires.
Actual identities of the herbs may also be given, and can be used as guidelines to help improvise alternative spell components.

\capital{\materialrefplural{herb}} are classified as \herbtypebare{1}, \herbtypebare{2}, \herbtypebare{3}, \herbtypebare{4}, or \herbtypebare{5}.
This covers both how common the \materialref{herb} is in the wild, and how difficult it is to cultivate in a garden.

\subsection{Finding Herbs}

Finding \materialrefplural{herb} growing in the wild uses \testtype{heed}{botany}.
A successful Test to find \materialrefplural{herb} provides enough for a few potions or poultices, or a few performances of a rite.
Under normal circumstances, this should be enough for the task at hand, or to restock a witch's supply.
But if the witch is trying to brew a potion for everyone in a castle, this supply might not cut it.

\begin{itemize}
	\item
		\phantomsection\herblabel{Ubiquitous}{1}
		\capital{\herbtypeplural{1}} are incredibly easy to find and require no Test.
		They are primarily weeds that grow just about everywhere, whether people want them to or not.
		They should almost never take more than five minutes to find as long as you're outside, and even less if you're in a field or forest.
	\item
		\phantomsection\herblabel{Common}{2}
		\capital{\herbtypeplural{2}} typically require searching a few hedgerows.
		They'll certainly turn up in an hour, and can be found much faster with a relatively easy Test.
	\item
		\phantomsection\herblabel{Uncommon}{3}
		\capital{\herbtypeplural{3}} might require searching a large swathe of forest to even turn up one plant.
		This takes at least an hour, typically more, and requires a difficult Test.
	\item
		\phantomsection\herblabel{Rare}{4}
		\capital{\herbtypeplural{4}} might not be found even searching an entire forest.
		Performing such a search exhaustively is infeasible, but a skilled botanist knows how to look in the right places.
		Still, this can take an entire day and requires a very difficult Test.
	\item
		\phantomsection\herblabel{Extraordinary}{5}
		\capital{\herbtypeplural{5}} are not found in the wild under any but the most exceptional circumstances.
		They typically need to be traded from far-away places, sought out in mystic groves, or coaxed to grow under unusual conditions.
		The GM may make a quest out of finding them, in order that a witch can begin to cultivate them.
\end{itemize}

\subsection{Cultivating Herbs}
\seclabel{growing-herbs}

To get a larger and more reliable supply than she can find in the wilderness, a witch can grow \materialrefplural{herb} in a garden.
Guidelines on keeping a garden are given in the section \secref{gardens}.
However, there are additional skill requirements to grow some \materialrefplural{herb}.

Anyone can grow \herbtypeplural{2} or \herbtypeplural{1} in a garden.
As far as the weedy \herbtypeplural{1} go, most of the effort goes into keeping them under control.
Rarer \materialrefplural{herb} are harder to cultivate, however, requiring a witch---or a familiar---skilled in \skillref{botany}.
\capital{\herbtypeplural{5}}, in particular, are liable to attack or flee from a witch who doesn't care for them properly.

The following table gives the \skillref{botany} skill required to cultivate a \materialref{herb}.

\begin{simpletable}{llX}
	\toprule
	Rarity & Skill & Examples\\
	\midrule
	\capital{\herbtypebare{1}} & - & Dock, Nettle, Clover\\
	\capital{\herbtypebare{2}} & - & Lavender, Rosemary, Elderberry\\
	\capital{\herbtypebare{3}} & 1 & Tomato\\
	\capital{\herbtypebare{4}} & 2 & Truffle\\
	\capital{\herbtypebare{5}} & 3 & Mandrake, Triffid\\
	%TODO: Expand
	\bottomrule
\end{simpletable}



\section{Weapons}
\seclabel{weapons}

Weapons are divided into several broad categories.
Players are free to describe their character's weapons how they wish, within the bounds of reason, placing them in one of the categories.
Anything a character might find at hand and hit people with can also be placed into a category.

A weapon's accuracy is added as a flat bonus to rolls to hit, in place of an attribute.
A weapon's damage determines the number of dice rolled  for the {\damagetest} upon hitting.
The highest 3 dice are kept, as always, but the number of dice rolled are determined by the weapon instead of the wielder's skill.
The wielder's \attref{might} is added to the {\damagetest} for melee or thrown weapons, but not for \weaponrefplural{bow} or \weaponrefplural{blowgun}.
The bonus to the {\damagetest} for \weaponrefplural{bow} is determined by their draw weight, as explained in their section.
\capital{\weaponrefplural{blowgun}} receive no bonus to the {\damagetest} at all.

\notedtable{XllX[1.5]}{
	\toprule
	Weapon & Accuracy & Damage & Range \mbox{(metres)}\\
	\midrule
	\capital{\weaponref{unarmed}} & +2 & \dice{2}\tnote{*} & Melee\\
	\capital{\weaponref{club}} & +4 & \dice{4} & Melee\\
	\capital{\weaponref{knife}} & +2 & \dice{5} & Melee\\
	\capital{\weaponref{hand-weapon}} & +4 & \dice{5} & Melee\\
	\capital{\weaponref{thrown-rock}} & +0 & \dice{2} & $5 + 5\times\text{\attref{might}}$\\
	\capital{\weaponref{thrown-weapon}} & +0 & \dice{4} & $5 + 5\times\text{\attref{might}}$\\
	\capital{\weaponref{bow}} & +2 & \dice{5}\tnote{\dag} & $25 + 25 \times {}$ $\text{Draw Weight}$\\
	\capital{\weaponref{blowgun}} & +2 & \dice{3}\tnote{\ddag} & 15\\
	\bottomrule
}{
	\item[*] For a human; may differ for an animal.
	\item[\dag] Add the \weaponrefpossessive{bow} draw weight, instead of your \attref{might}.
	\item[\ddag] Do not add your \attref{might}.
}

\weapontype{Unarmed}{}{unarmed}{
	A punch, a kick, or a headbutt.
	This is any attack you make without any weapon at all.
	
	\dice{2} is for humans.
	Animals and familiars will list the number of dice for their {\unarmed} {\damagetests}, but use the same accuracy bonus unless this is also specified.
	If an animal or familiar's statistics do not mention its {\unarmed} attacks, assume that it has no effective attacks.
}

\weapontype{Club}{Clubs}{club}{
	A club, a walking stick, a chair, or a cauldron.
	A \weaponref{club} is just about anything you pick up and hit someone with.
}

\weapontype{Knife}{Knives}{knife}{
	A knife or dagger.
	Easily concealed, and a staple of blood witches.
	The short blade costs the wielder reach, but can do as much damage as a sword if you get the enemy in the tender parts.
}

\weapontype{Hand Weapon}{Hand Weapons}{hand-weapon}{
	A sword, an axe, a mace, a spear, a pike.
	This category covers most things actually designed as a weapon and larger than a knife.
}

\weapontype{Thrown Rock}{Thrown Rocks}{thrown-rock}{
	A genuine rock, but also a teapot, a boot or a frog.
	Anything you might pick up and throw.
	This includes weapons that aren't designed to be thrown, but get tossed anyway, such as swords.
}

\weapontype{Thrown Weapon}{Thrown Weapons}{thrown-weapon}{
	A spear, a knife, a hatchet.
	Any weapon you can throw that was actually designed for the purpose.
	Rocks from slingshots fall in this category too.
}

\weapontype{Bow}{Bows}{bow}{
	A bow and arrow.
	Also covers crossbows, if you include them in your setting.
	
	Each \weaponref{bow} has a draw weight, typically 1 for a shortbow, or 3 for a longbow.
	Bows can be custom-made with any draw weight from 0 to 5, however.
	
	When using a \weaponref{bow}, add the draw weight, not your \attref{might}, to the damage roll.
	However, your \attref{might} must equal or exceed the draw weight for you to be able to use the \weaponref{bow}.
}

\weapontype{Blowgun}{Blowguns}{blowgun}{
	A tube and dart, with the dart propelled by the power of your breath.
	Easier to conceal than a \weaponref{bow}, a \weaponref{blowgun} is often used to subtly deliver an \mixdeliveryref{injury} poison.
}



\section{The Broom}
\seclabel{broomsticks}

Sometimes, walking from one village to another just takes too long.
A lot of witches---to maintain their mystique or simply because the townsfolk wouldn't be happy otherwise---even choose to live quite a way from the nearest village.
Such circumstances make a broomstick an essential accessory for any witch.

Broomstick flight is no mean feat and while every witch picks up the rudiments, most can use it for nothing more than getting from A to B.
The broom needs a running start, has to be ridden sidesaddle, and has a turning circle several hundred metres across.
Detailed rules for flying a broomstick can be found in \chapref{flying}.

Before it can be used, a broomstick needs to be trained to to fly.
This requires someone to fly it around on another broomstick so that it can learn its craft from one of its fellows.
It must be held parallel to the broom being ridden, to ensure it learns to fly in the correct direction.
The process takes about eight hours.
These hours need not be consecutive, but should all be done within a couple of weeks.
Once trained, a broom retains its flight skill for a long time.
Taking it out for a few hours each year is enough to keep its hand in.

At character creation, every witch is assumed to own a trained broomstick one way or another.
It was probably trained using the broom of whoever taught her witchcraft, at least if she's still using their first broom.
It might feel like an old friend at this point, the witch familiar with every knot and notch in its handle.
A more careless witch might have gone through a few brooms during her career.



\section{The Cottage}
\seclabel{cottages}

\section{The Garden}
\seclabel{gardens}

Many witches keep a garden, for growing \materialrefplural{herb}, or keeping animals.
Normally this is outside the witch's cottage, but sometimes it is somewhere out in the wilderness, or beside the village green.
Sometimes, it would be more accurate to call it a farm than a garden.
A witch might even keep more than one garden, if she has the time---perhaps one at her own cottage, and one at a fellow witch's place.

A witch's garden does more than simply provide a little flavour to a character; it provides the natural components she needs to work her magic.
The most obvious of these are \materialrefplural{herb}, as described in the relevant section.
However, many witches also keep bees for honey; chickens for eggs; sheep for wool; and cows, or goats, for milk.
Some even keep a horse, preferring it to a broomstick.

\subsection{Garden Supplies}

A garden is assumed to be accompanied by some storage of its products.
Milk doesn't have to come right out of the cow's udder every time you need it, and \materialrefplural{herb} that must be harvested at a particular time are available in storage when you need them.
This storage is assumed to be enough for most purposes---possibly even enough of a \materialref{herb} to brew one potion for everyone in a castle.
But it can still be overtaxed, and that sort of thing shouldn't be tried too often.

\subsection{Gardening Time}

Like most equipment, there are no hard and fast rules about what a witch may keep in her garden---except for the \skillref{botany} skill required to keep the more difficult \materialrefplural{herb} (see the section \secref{growing-herbs}).
But caring for a garden is a significant time investment.
The more a witch keeps in her garden, the less time she has to tend the rest of her steading, and to pursue her own goals.
A few guidelines for the time required are provided below.
The effects of this are left up the GM, but mostly they are intended to prevent things getting out of hand.
As long as things are kept reasonable, and the witch never has to be away for too long, it is perfectly valid not to bother tracking this time.

Keeping \materialrefplural{herb} requires effort on a weekly basis.
If left unattended for more than a week, it can require an even more effort to get it back into shape afterwards.
The particularly needy or unruly \materialrefplural{herb} might die off, or even escape, during this time.

The amount of effort varies by the maximum rarity of \materialref{herb} in the garden.
\capital{\herbtypeplural{1}} require no effort at all; these weeds will spring up on any patch of ground, uninvited.
Otherwise, it takes about 4 hours each week for \herbtypeplural{2}, 8 hours for \herbtypeplural{3}, 12 hours for \herbtypeplural{4}, and 16 hours for \herbtypeplural{5}.
These assume a few types of \materialref{herb} for each rarity---a garden with 30 or so different species might take even longer.

Animals require tending almost every day.
Typically about half an hour per day, per type of animal, though this may be reduced by a high \skillref{animals} skill.
This assumes only a few animals of each type---a whole herd of cows takes more time.
Animals left unattended for more than a day might starve, or escape.

A witch might employ a little help in tending her garden.
This could come from family members, or someone who owes her a huge favour.
The availability of such help is up the GM.
Her familiar may also help her, although only if it has the \skillref{botany} or \skillref{animals} skill, as appropriate.
It must have also enough \skillref{botany} skill to keep the relevant \materialrefplural{herb}, the same as a witch herself.
Do employ common sense here---a \familiarref{dog}, even a sheepdog, won't be helping with beekeeping.



\section{The Hat}
\seclabel{the-hat}

A witch's pointed hat is the most important of her tools, in many regards.
There are no particular rules about the hat; its effects are left up to the GM.
But it always has an effect on people.
It may make them angry, reverent, reassured, or afraid, but most importantly it makes sure they know that they are in the presence of a witch.

A witch's hat says a lot about her, particularly to other witches.
When you create your character, you can answer the following questions about your hat.

\begin{itemize}
	\item Did you make it yourself?
	\item How tall is it?
	\item Is it the traditional black, or some other colour?
	\item How long have you had it?
		Is it visibly worn?
		Well cared for?
	\item Is it plain, tastefully decorated, or covered in stars and sequins?
	\item Does it have any useful accessories?
		Pockets?
\end{itemize}

Many witches accompany their hats by a black cloak or other such attire.
Opinions on occult jewellery are mixed: some witches wear masses, others frown on it heavily.
