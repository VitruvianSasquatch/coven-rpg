\chapter{Tools of the Craft}
\chaplabel{equipment}

\section{The Hat}
\seclabel{the-hat}

A witch's pointed hat is the most important of her tools, in many regards.
There are no particular rules about the hat; its effects are left up to the GM.
But it always has an effect on people.
It may make them angry, reverent, reassured, or afraid, but most importantly it makes sure they know that they are in the presence of a witch.

A witch's hat says a lot about her, particularly to other witches.
When you create your character, you can answer the following questions about your hat.

\begin{itemize}
	\item Did you make it yourself?
	\item How tall is it?
	\item Is it the traditional black, or some other colour?
	\item How long have you had it?
		Is it visibly worn?
		Well cared for?
	\item Is it plain, tastefully decorated, or covered in stars and sequins?
	\item Does it have any useful accessories?
		Pockets?
\end{itemize}

Many witches accompany their hats by a black cloak or other such attire.
Opinions on occult jewellery are mixed: some witches wear masses, others frown on it heavily.



\section{Broomsticks}

Sometimes, walking from one village to another just takes too long.
A lot of witches---to maintain their mystique or simply because the townsfolk wouldn't be happy otherwise---even choose to live quite a way from the nearest village.
Such circumstances make a broomstick an essential accessory for any witch.

Broomstick flight is no mean feat and while every witch picks up the rudiments, most can use it for nothing more than getting from A to B.
The broom needs a running start, has to be ridden sidesaddle, and has a turning circle several hundred metres across.
Detailed rules for flying a broomstick can be found in \chapref{broomcraft}.

Before it can be used, a broomstick needs to be trained to to fly.
This requires someone to fly it around on another broomstick so that it can learn its craft from one of its fellows.
It must be held parallel to the broom being ridden, to ensure it learns to fly in the correct direction.
The process takes about eight hours.
These hours need not be consecutive, but should all be done within a couple of weeks.
Once trained, a broom retains its flight skill for a long time.
Taking it out for a few hours each year is enough to keep its hand in.

At character creation, every witch is assumed to own a trained broomstick one way or another.
It was probably trained using the broom of whoever taught her witchcraft, at least if she's still using their first broom.
It might feel like an old friend at this point, the witch familiar with every knot and notch in its handle.
A more careless witch might have gone through a few brooms during her career.



\section{Common Magical Components}

The various rites and magics of the various disciplines of witchcraft require too many different materials to enumerate here.
However, there a few components that make a regular appearance.
Some details of their acquisition, construction and use are given here.

\subsection{Ritual Circles}
\materiallabel{Ritual Circle}{Ritual Circles}{ritual-circle}
\circlelabel{Small}{small}
\circlelabel{Medium}{medium}
\circlelabel{Large}{large}

A \materialref{ritual-circle} describes any large arrangement of symbols or shapes required by a rite.
They are traditionally drawn on the floor in chalk, but other methods are far from uncommon; the visibility and accuracy are the only important aspects for most rites.
Some witches use paint for permanence, or even chisel their circles into stone.
Many a witch in a hurry has scratched their circles into the dirt with the toe of their boot.
Some witches even embroider their circles upon sheets of fabric that can be rolled up and laid down where needed.
However, a roll bearing even the smallest of circles is most of the height of a man.

Each rite requires a \materialref{ritual-circle} of a particular design, different for every rite, but the same each time the rite is performed, even by a different witch.
This means that scribing a circle just once and using it for many performances of the rite is a common practice.
\materialrefplural{ritual-circle} are not even universally circular, although it is the most common shape and almost all have some sort of symmetry.
Squares, triangles and hexagons are not uncommon, and pentagrams are particularly common in certain disciplines.

A \materialref{ritual-circle} must be horizontal, and on a relatively flat surface.
Obstructions inside it, such as large rocks, furniture, or the pillars of a building, mean that using it for magic requires an appropriate Test.
No two \materialrefplural{ritual-circle} may overlap in any fashion, or even encompass one another.

\materialrefplural{ritual-circle} are classified primarily by their size.
A rite can be performed with a larger circle than it requires, unless specified otherwise.
\begin{itemize}
	\item A \circleref{small} can be scribed entirely in arm's reach while standing in one spot.
		It can comfortably be drawn in a couple of minutes.
	\item A \circleref{medium} is a few paces across.
		Most houses should have a room large enough to draw one in, if the furniture is moved.
		It can comfortably be drawn in a quarter of an hour.
	\item A \circleref{large} is at least two dozen paces across.
		A ballroom or village hall is probably the only place one could be drawn indoors, so most are drawn outside.
		At least a couple of hours are required to draw such a circle without haste.
\end{itemize}

\subsection{Standing Stone}
\materiallabel{Standing Stone}{Standing Stones}{standing-stone}

A \materialref{standing-stone} is simply a single, unbroken piece of stone, taller than a man and standing upright.
The shape is rather unimportant, as long as it is plainly recognisable as being upright---its longest dimension must be the vertical one.
These are naturally available in some locations, although more often they are lying down and might need a group of strong people to push them upright.
Elsewhere, they must be quarried out, or carried in.

\subsection{Megalithic Circles}
\materiallabel{Megalithic Circle}{Megalithic Circles}{stone-circle}

Some rites require a circle of\materialrefplural{standing-stone}, called a \materialref{stone-circle}.
Such a circle must be at least the size of a \circleref{large}, with at least a dozen stones.
The arrangement and shape of the stones is unimportant, as long as it is recognisable as a ring, and so the same circle can be used for all rites that require one.
Constructing a \materialref{stone-circle} is no easy task, typically requiring weeks of work by much of a village, even if the site is quite close to a stone quarry.

\subsection{Taglocks}
\materiallabel{Taglock}{Taglocks}{taglock}

A \materialref{taglock} is any part of a person's body, such as a piece of flesh, a strand of hair, a nail clipping, a drop of blood, or a gob of saliva.
It is often used to bind a spell to a particular target.
It can always be picked off a person---although taking a hair without being noticed might be difficult---but people often leave \materialrefplural{taglock} behind them, especially in places they frequent.
Finding a \materialref{taglock} in a place you suspect someone might have left on, such as their house or a bed they've slept in, typically uses \skillref{perception}.

\subsection{Poppets}
\materiallabel{Poppet}{Poppets}{poppet}

A \materialref{poppet} is an abstract representation of a person, although not a particular person.
Voodoo dolls are a typical example.
A \materialref{poppet} can be crafted from cloth, wood, clay, wax, or other suitable material.
It should be recognisable as a human, bearing four appropriately-arranged limbs, a head, and two eyes.
However, if it is to be used in \discref{sympathetic-magic} affecting a non-human creature, it should resemble whichever creature the magic is intended to affect.
A \materialref{poppet} should be at least a handspan tall, though can be much larger.

\subsection{Effigies}
\materiallabel{Effigy}{Effigies}{effigy}

An \materialref{effigy} is much like a \materialref{poppet} and follows all the rules for one, except that it represents a particular person and must be crafted in their likeness.
It can be used only to affect the person it resembles.
Ideally, an \materialref{effigy} should be recognisable to even passing acquaintances of the person it is supposed to represent.
Less recognisable \materialrefplural{effigy} will require an appropriate Test to be used for magic.

\subsection{Crystal Ball}
\materiallabel{Crystal Ball}{Crystal Balls}{crystal-ball}

A \materialref{crystal-ball} is a polished sphere made of glass or quartz crystal, often used in scrying.
\materialrefplural{crystal-ball} are very difficult to produce or acquire.
A player must obtain permission from the GM to have one among their starting equipment, and the GM is within their rights to place some limitation upon it.
A witch just starting out in her own steading will likely have a substandard ball---scratched, or not quite spherical.
Even experienced witches often share one among the whole coven, and an apprentice likely has to borrow her teacher's.

Most witches leave their \materialref{crystal-ball} in their cottage, to prevent it coming to harm.
Acquiring a new ball, if one is smashed, can be quite a challenge---often an excellent opportunity for the GM to present an adventure.

Taking a \materialref{crystal-ball} into bright sunlight is dangerous, as it has a tendency to focus the light and set things on fire.
Most witches cover theirs with a dark cloth when it isn't in use, and scry in dimly-lit rooms.

\subsection{Blood}
\materiallabel{Blood}{}{blood}

Many spells call for \materialref{blood}, in varying quantities and from various creatures.
Extracting a mere drop of \materialref{blood} carries no ill effects.
Furthermore, a willing or restrained donor can provide \SI{100}{\milli\litre} of \materialref{blood} per point of \attref{might} with no ill effects, approximately once per week.
Above that, every \SI{100}{\milli\litre} of \materialref{blood} extracted deals 1 point of {\damage}.
Creatures damaged in combat, by edged weapons, will also spill blood; approximately \SI{100}{\milli\litre} per point of {\damage} dealt.
This blood will typically be lost in the dirt, however.

The above guidelines apply to humans, who typically contain about approximately 5 litres of blood in total.
Differently sized creatures will provide appropriately more or less blood.


\section{Herbs and Gardens}
\materiallabel{Herb}{Herbs}{herb}

Herbs are an important component of most potions and poultices, as well as many spells.
Note that the term ``herb'' is used to encompass many things that are not technically herbs at all, such as fruit, fungi, tree bark and so on.

There are hundreds, perhaps thousands of different herbs, so instead of tracking every one, they are simply divided into categories based upon rarity.
Each potion or spell lists the highest rarity of herb that it requires.
Actual identities of the herbs may also be given, and can be used as guidelines to help improvise alternative spell components.

Herbs are classified as \herbtypebare{1}, \herbtypebare{2}, \herbtypebare{3}, \herbtypebare{4}, or \herbtypebare{5}.
This covers both how common the herb is in the wild, and how difficult it is to cultivate in a garden.

\subsection{Finding Herbs}

Finding herbs growing in the wild uses \testtype{heed}{botany}.
A successful Test to find herbs provides enough for a few potions or poultices, or a few performances of a rite.
Under normal circumstances, this should be enough for the task at hand, or to restock a witch's supply.
But if the witch is trying to brew a potion for everyone in a castle, this supply might not cut it.

\begin{itemize}
	\item
		\phantomsection\herblabel{Ubiquitous}{1}
		\herbtypeplural{1} are incredibly easy to find and require no Test.
		They are primarily weeds that grow just about everywhere, whether people want them to or not.
		They should almost never take more than five minutes to find as long as you're outside, and even less if you're in a field or forest.
	\item
		\phantomsection\herblabel{Common}{2}
		\herbtypeplural{2} typically require searching a few hedgerows.
		They'll certainly turn up in an hour, and can be found much faster with a relatively easy Test.
	\item
		\phantomsection\herblabel{Uncommon}{3}
		\herbtypeplural{3} might require searching a large swathe of forest to even turn up one plant.
		This takes at least an hour, typically more, and requires a difficult Test.
	\item
		\phantomsection\herblabel{Rare}{4}
		\herbtypeplural{4} might not be found even searching an entire forest.
		Performing such a search exhaustively is infeasible, but a skilled botanist knows how to look in the right places.
		Still, this can take an entire day and requires a very difficult Test.
	\item
		\phantomsection\herblabel{Extraordinary}{5}
		\herbtypeplural{5} are not found in the wild under any but the most exceptional circumstances.
		They typically need to be traded from far-away places, sought out in mystic groves, or coaxed to grow under unusual conditions.
		The GM may make a quest out of finding them, in order that a witch can begin to cultivate them.
\end{itemize}

\subsection{Cultivating Herbs}

Many witches, especially brewers, keep a garden for growing herbs.
This is typically outside their cottage, but can be anywhere she likes.
However, tending a complete garden requires about eight hours of work every week, which makes maintaining more than one a rather time-consuming task.
If the garden is left unattended for more than a week, it can require an even more considerable effort to get it back into shape.
The particularly needy or unruly herbs might die off, or even escape, during this time.

A garden is assumed to be accompanied by some storage of the herbs, so even herbs that need to be harvested at a particular time are available when they are needed.
A garden can provide an even greater supply of herbs than a search in the wild can, and may be just about enough to brew one potion for everyone in a castle.
But it can still be overtaxed, and this sort of thing shouldn't be tried too often.

Anyone can grow \herbtypeplural{2} or \herbtypeplural{1} in a garden.
As far as the weedy \herbtypeplural{1} go, most of the effort goes into keeping them under control.
Rarer herbs are harder to cultivate, however, requiring a skilled botanist (or a botanist with a skilled familiar).
\herbtypeplural{5}, in particular, are liable to attack or flee from a witch who doesn't care for them properly.

The following table gives the Botany skill required to cultivate a herb.

\begin{simpletable}{llX}
	\toprule
	Rarity & Skill & Examples\\
	\midrule
	Ubiquitous & - & Dock, Nettle, Clover\\
	Common & - & Lavender, Rosemary, Elderberry\\
	Uncommon & 1 & Tomato\\
	Rare & 2 & Truffle\\
	Extraordinary & 3 & Mandrake, Triffid\\
	%TODO: Expand
	\bottomrule
\end{simpletable}



\section{Weapons}
\seclabel{weapons}

Weapons are divided into several broad categories.
Players are free to describe their character's weapons how they wish, within the bounds of reason, placing them in one of the categories.
Anything a character might find at hand and hit people with can also be placed into a category.

A weapon's accuracy is added a flat bonus to rolls to hit, in place of an attribute.
A weapon's damage determines the number of dice rolled upon hitting.
The highest 3 dice are kept, as always, but the number of dice rolled are determined by the weapon instead of the wielder's skill.
The wielder's \attref{might} is added to the damage roll for melee or thrown weapons, but not for \weaponrefplural{bow}.
The bonus to the damage roll for \weaponrefplural{bow} is determined by their draw weight, as explained in their section.

\notedtable{XllX[1.5]}{
	\toprule
	Weapon & Accuracy & Damage & Range \mbox{(metres)}\\
	\midrule
	Fist & +2 & \dice{2} & Melee\\
	Club & +4 & \dice{4} & Melee\\
	Knife & +2 & \dice{5} & Melee\\
	Hand Weapon & +4 & \dice{5} & Melee\\
	Thrown Rock & +0 & \dice{2} & $5 + 5\times\text{\attref{might}}$\\
	Thrown Weapon & +0 & \dice{4} & $5 + 5\times\text{\attref{might}}$\\
	Bow & +2 & \dice{5}\tnote{*} & $25 + 25 \times {}$ $\text{Draw Weight}$\\
	\bottomrule
}{
	\item[*] Add the \weaponrefpossessive{bow} draw weight, instead of your \attref{might}.
}

\subsubsection{Fist}
\weaponlabel{Fist}{Fists}{unarmed}
A punch, a kick, or a headbutt.
Covers any attack you make without any weapon at all.

Some animals and familiars will have a different number of dice for their unarmed attacks, but use the same accuracy bonus unless this is also specified.

\subsubsection{Club}
\weaponlabel{Club}{Clubs}{club}
A club, a walking stick, a chair, or a cauldron.
A club is just about anything you pick up and hit someone with.

\subsubsection{Knife}
\weaponlabel{Knife}{Knives}{knife}
A knife or dagger.
Easily concealed, and a staple of blood witches.
The short blade costs the wielder reach, but can do as much damage as a sword if you get the enemy in the tender parts.

\subsubsection{Hand Weapon}
\weaponlabel{Hand Weapon}{Hand Weapons}{hand-weapon}
A sword, an axe, a mace, a spear, a pike.
This category covers most things actually designed as a weapon and larger than a knife.

\subsubsection{Thrown Rock}
\weaponlabel{Thrown Rock}{Thrown Rocks}{thrown-rock}
A genuine rock, but also a teapot, a boot or a frog.
Anything you might pick up and throw.
This includes weapons that aren't designed to be thrown, but get tossed anyway, such as swords.

\subsubsection{Thrown Weapon}
\weaponlabel{Thrown Weapon}{Thrown Weapons}{thrown-weapon}
A spear, a knife, a hatchet.
Any weapon you can throw that was actually designed for the purpose.
Rocks from slingshots fall in this category too.

\subsection{Bow}
\weaponlabel{Bow}{Bows}{bow}
A bow and arrow.
Also covers crossbows, if you include them in your setting.

Each \weaponref{bow} has a draw weight, typically 1 for a shortbow, or 3 for a longbow.
Bows can be custom-made with any draw weight from 0 to 5, however.

When using a \weaponref{bow}, add the draw weight, not your \attref{might}, to the damage roll.
However, your \attref{might} must equal or exceed the draw weight for you to be able to use the \weaponref{bow}.
