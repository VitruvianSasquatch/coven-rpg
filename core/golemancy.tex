\discipline{Golemancy}{golemancy}{Golemancer}{Golemancers}

\section{Animating a Golem}

A golem must be animated as part of the creation of its body, and the witch doing the animation must be involved in its creation, even if she is not the primary craftswoman.
To animate a golem, a witch simply touches it and wills it life; many consider \discref{golemancy} to be a particularly specialised application of \discref{willing}.
Animating a golem always requires at least a minute, even if the golem's body can be crafted faster than that.

A novice \practitioner{golemancy}---one who can create a golem---has enough animating force to maintain one, and only one, golem.
If she animates a new golem, the previous golem immediately becomes inanimate.
A witch may also withdraw the animating force from a golem she has animated at any time, though if this is to be done urgently (perhaps the golem has gone rogue), the GM may require a Test.
Lastly, all a witch's golems become inanimate if she dies.

The crafting and animation, although strongly interlinked, are separate processes.
Tests related to the craftsmanship use \skillref{crafting} and an appropriate attribute.
Tests related to the golem's animation, such as giving it instructions, use \testtype{will}{golemancy}.
A witch can only animate a particular material into a golem if she has taken the appropriate feat.

\section{A Golem's Instructions}

A witch just beginning to dabble in \discref{golemancy} only has the skill to make very simple-minded, single-purpose creatures, although she will learn more nuance later.
These golems are imbued with a single instruction at the moment of their creation.
They will follow this until its completion, whereupon they will simply stand still and await destruction.
The instruction must be very simple, and the golem has minimal ability to improvise around it.
It should not have any conditional aspects, and the golem is unable to respond to any form of communication.
Example instruction are given below.

\begin{itemize}
	\item Deliver this note to the castle.
	\item Fetch my broom.
	\item Kill that man.
	\item Sweep the floor every evening.
	\item Extinguish any fires you see.
\end{itemize}

Additional information necessary to the completion of the task, such as the location of (and directions to) the castle, or the identity of an intended victim, may be imparted with the instruction.
The golem will trust this information and cannot adapt if it is wrong, for example if the victim has been disguised.
Furthermore, such information must be quite explicit.
For example, a golem instructed to ``attack intruders'' has no mechanism for distinguishing between invited guests and intruders.

Giving instructions with nuance, or instructions with multiple linked parts (such as ``go to the castle and kill the King'') requires a Test, with a {\tn} set by the GM based on the complexity of the instruction.
A failure either prevents the golem from animating or, at the GM's option, corrupts the instructions.

\section{A Golem's Statistics}

A golem's physical statistics are determined by the material and method of its construction, and are specified in the appropriate feat.
These include its \attref{might}, \attref{grace}, \statref{speed}, \statref{resilience} and \statref{shock-threshold}.
A golem whose \statref{shock-threshold} is met or exceeded by a {\damagetest} is immediately destroyed, instead of going into {\shock}.
The GM is also advised to apply common sense to other consequences of a golem's construction: for example, a clay or fabric golem will sink in water, a wood or wax golem will float, and a gingerbread golem will go soggy and fall apart.
All golems are immune to poisons and diseases, and unaffected by potions, poultices, and the like.

As for its other attributes, a golem lacks \attref{ken}, \attref{wit}, \attref{will}, \attref{charm} and \attref{presence} entirely; it automatically fails Tests that would require them.
It has 0 \attref{heed}.
However, a golem is unrelenting and untiring, and it has no mind to affect.
As such, it may be considered to automatically succeed at many Tests that would require \attref{will}.
Lastly, a golem has no ranks in any skills.

Initially, a witch can only animate small golems: about a handspan in height.
She doesn't have enough animating force to manage anything bigger, and anything smaller can't hold the magic required.
These golems do not have any effective attacks, and are too small to use weapons.

A golem knows no languages; it cannot read, write, or comprehend speech.
It cannot speak, and furthermore cannot vocalise in any fashion.
The sounds it can make are limited to such things as clapping its hands and stamping its feet.

A golem has senses as good as a human, although only if its craftsmanship gives it the appropriate anatomy.
For example, a gingerbread golem with two currants for eyes can see, but if baked without the currants it will be blind.
A clay golem can only smell if a nose is sculpted upon its face.

\section{Feats}

\feat{Gingerbread Golem}{gingerbread-golem}{15}{
	\noprereq
}{
	The simplest golems are not baked of clay, but of dough.
	When you bake a humanoid figure from gingerbread, you may animate it as a gingerbread golem.
	The entire preparation and baking process takes approximately half an hour, although an entire batch of golems can be crafted at once if the size of the oven allows.
	
	\materials{Flour, sugar, eggs, butter, \herb{ginger}{3}}
	
	A gingerbread golem has \negative{2} \attref{might}, 1 \attref{grace}, and 15 \statref{speed}.
	It has an effective \statref{shock-threshold} of 1; it is destroyed if it takes any {\damage}.
	
	Additionally, a gingerbread golem has a limited lifespan.
	After about a week, it grows stale and can no longer move.
	Moisture or water, even a couple of minutes in rain, will destroy it sooner.
}

\feat{Fabric Golem}{fabric-golem}{15}{
	\noprereq
}{
	You can make your dolls get up and move.
	When you stitch, knit, or crochet fabric or yarn to form a humanoid figure, and fill it with stuffing, you may animate it as a fabric golem.
	The crafting process typically takes longer than an hour---much longer if you knit it.
	
	A fabric golem has \negative{2} \attref{might}, 1 \attref{grace}, and 10 \statref{speed}.
	It has 1 \statref{resilience} and a \statref{shock-threshold} of 10.
	\capital{\damage} to the golem can be repaired with a needle and thread, requiring several minutes.
	
	Unlike a gingerbread golem, a fabric golem isn't \emph{destroyed} by water.
	But a waterlogged fabric golem can't move until it dries out.
}


\feat{Wood Golem}{wood-golem}{15}{
	\featref{gingerbread-golem} or \featref{fabric-golem}
}{
	Wood offers a far more robust golem than gingerbread or cloth.
	When you whittle or assemble a humanoid figure from wood, you may animate it as a golem.
	The time required to do this depends on the size of the golem and the piece of wood you are crafting from.
	Whittling a small golem from an approximately man-shaped piece of wood may take as little as ten minutes, but carving one from a solid chunk of log might take an hour or more.
	Carving a much larger one could take days, and it would likely be faster to assemble it from multiple pieces of wood.
	
	A wooden golem has \negative{1} \attref{might}, 1 \attref{grace}, and 10 \statref{speed}.
	It has 4 \statref{resilience} and a \statref{shock-threshold} of 14.
	\capital{\damage} to a wooden golem cannot be repaired.
}

\feat{Wax Golem}{wax-golem}{15}{
	\featref{gingerbread-golem} or \featref{fabric-golem}
}{
	Wax isn't quite so robust as wood, but it can be very quick to mould and repair.
	When you mould or cast a humanoid figure from tallow or beeswax, you may animate it as a golem.
	The wax or tallow needs to be warmed to be moulded.
	Leaving it in the sun on a warm summer's day provides about the temperature required, as does sitting near a fire.
	Once warmed, the golem can be moulded by hand in a couple of minutes.
	
	A wax golem has \negative{1} \attref{might}, 1 \attref{grace}, and 10 \statref{speed}.
	It has 2 \statref{resilience} and a \statref{shock-threshold} of 12.
	\capital{\damage} to the golem can be easily repaired by the application of a little more warm wax.
	
	Wax golems are susceptible to heat.
	A hot summer's day won't hurt, just make them a little softer, but coming too close to a fire or forge will leave them in a dribbly pool on the ground.
}

\feat{Clay Golem}{clay-golem}{15}{
	\skillref[1]{golemancy},
	\featref{wood-golem} or \featref{wax-golem}
}{
	Wood, wax, tallow, wool, and gingerbread contain at least traces of life, making them easier to animate.
	Clay has never known life at all, but you've finally figure out how to teach it.
	When you mould a humanoid figure from clay and fire it in a kiln, you may animate it as a golem.
	The firing process takes at least ten hours, but takes no longer for a larger golem.
	
	A clay golem has 0 \attref{might}, 1 \attref{grace}, and 10 \statref{speed}.
	It has 12 \statref{resilience} and a \statref{shock-threshold} of 16.
	\capital{\damage} to a clay golem can be repaired by filling the cracks with clay and refiring the golem.
	
	Clay golems are all but immune to heat.
	After all, they were fired to over \SI{1000}{\celsius} at their creation.
	Only rapid quenching from red-hot poses any threat at all.
}

\feat{Stone Golem}{stone-golem}{10}{
	\skillref[1]{golemancy},
	\featref{wood-golem} or \featref{wax-golem}
}{
	Stone is a mighty material, but not one known for its movement.
	Stone golems are similarly strong, but are simply animated statues---immobile.
	When you carve a humanoid figure from stone, you may animate it as a golem.
	Stone is tough to chisel, and producing such a figure typically takes hours.
	
	A stone golem has 2 \attref{might}, 0 \attref{grace}, 12 \statref{resilience}, and a \statref{shock-threshold} of 22.
	\capital{\damage} to a wooden golem cannot be repaired.
	
	A stone golem has a \statref{speed} of 0, and no \statref{dodge-rating}---any attack aimed at it automatically hits.
	It must have its feet planted on the ground or floor when it is animated, and they become fixed to this spot.
	The golem is even immune to being moved by others, for as long as it is animated.
}

\feat{Golem Programming}{golem-change-instructions}{15}{
	\featref{gingerbread-golem} or \featref{fabric-golem}
}{
	You can change the instruction imbued into one of your golems, allowing you to reuse the same golem for multiple tasks.
	Reprogramming a golem requires you to be touching it, and takes a minute of concentration.
	The normal restrictions apply to the new instruction, and the golem can only have one instruction at a time; adding a new instruction removes the previous one.
	You may only reprogram golems powered by your own animating force.
}

\feat{Delegated Programming}{golem-change-instructions-familiar}{10}{
	\featref{golem-change-instructions}
}{
	You've taught your familiar a few tricks of golemancy, and it may use the bond it shares with you to tap into your own animating force.
	Your familiar may reprogram golems, using the same rules as \featref{golem-change-instructions}.
	It reprograms your golems, however---it has no golems of its own.
	If imbuing the new instruction requires a Test, your familiar uses its own \testtype{will}{golemancy}, not yours.
}

\feat{Golem Reanimation}{golem-reanimate}{15}{
	\featref{golem-change-instructions}
}{
	Normally, a golem must be animated as its body is created.
	But, much as you've learned to change its instructions after its animation, you've figured out how to reanimate it after its creation.
	
	You can animate a golem's body at any time, even separately from its creation---although you must still have been involved in its creation.
	The body need not have been animated before, but can have been.
	Animating the golem takes about a minute, and requires you to touch it.
}

\feat{Golem of Another}{golem-reanimate-2}{10}{
	\skillref[1]{golemancy},
	\featref{golem-reanimate}
}{
	Animating a golem usually requires an intimate familiarity with its form, a familiarity that can only come from helping to craft its body.
	You've either learned to acquire that familiarity through inspection, or you can use brute will to animate it without that familiarity.
	
	You can animate a figure of an appropriate size, shape, and material as a golem, using \featref{golem-reanimate}, even if you weren't involved in its creation.
}

\feat{Advanced Instructions}{golem-advanced-instructions}{20}{
	\skillref[1]{golemancy},
	\featref{gingerbread-golem} or \featref{fabric-golem}
}{
	You can imbue your golems with more advanced instructions.
	The instructions can contain several steps, and conditional elements.
	On the whole, the golem can contain instructions that would take no more than a minute to convey by reasonably-paced speech.
	
	The golem still has next to no ability to improvise around the instructions, and will unreasoningly attempt to carry them out until it completes them or is destroyed.
	Information can still be imparted alongside the instructions, but it must still be explicitly.
	For example, an instruction to ``attack intruders'' will still fail, however the golem can now be instructed to ``attack anyone except me who enters this house'' or ``attack anyone who enters this house unless the door is unlocked with the key.''
	
	The golem can respond to some degree of communication, such as pointing and nodding, if explicitly instructed to.
	However, it still cannot \emph{comprehend} the communication.
	For example, it can follow an instruction to ``go where this man points,'' but not to ``follow this man's directions''.
}

\feat{Golem Language}{golem-understand-language}{20}{
	\skillref[1]{golemancy},
	\featref{golem-advanced-instructions},
	\featref{golem-change-instructions}
}{
	You imbue golems with your own understanding of language, allowing them to understand speech, as well as to read and write.
	This also includes some understanding of body language, though sarcasm, metaphor and the like continue to elude the golem.
	
	The golem still exclusively follows the instructions it has been imbued with, but you may now give it instructions such as ``follow my orders,'' and give further orders verbally.
	Verbal instructions must be just as explicit as imbued ones, however.
}

\feat{Golem Familiar Interpretation}{golem-understand-familiar}{10}{
	\skillref[1]{golemancy},
	\featref{golem-understand-language},
	\featref{golem-change-instructions-familiar}
}{
	Your golems gain the same ability to innately understand your familiar that you have, as effectively as though your familiar was speaking.
}

\feat{Golem Intelligence}{golem-intelligence}{20}{
	\skillref[1]{golemancy},
	\featref{golem-advanced-instructions},
	\featref{golem-change-instructions}
}{
	You may imbue your golems with some degree of intelligence.
	Although, to be honest, they're still a little dull.
	The golem gains \attref{ken}, \attref{wit}, \attref{charm} and \attref{presence} scores of 0, and may make Tests requiring these attributes.
	It can also perform a little improvisation around the best way to carry out its instructions.
	For example, if instructed to kill someone, it might pick up a club or sword instead of using its fists.
	It may finally be given instructions such as ``attack intruders,'' and will use its best judgement to determine whether someone is an intruder.
	However, the golem cannot disobey an instruction, even if doing so would be in your best interest.
}

\feat{Golem Speech}{golem-speak}{15}{
	\skillref[2]{golemancy},
	\featref{golem-understand-language},
	\featref{golem-intelligence}
}{
	If you create your golem's body with a mouth and a tongue, you may imbue it with the ability to speak.
	%And sing, though not necessarily well.
}

\feat{Twin Golems}{more-golems}{15}{
	\featref{gingerbread-golem} or \featref{fabric-golem}
}{
	You can muster enough animating force to maintain a second golem at the same time.
	If you try to animate a third golem, you may choose which existing golem becomes inanimate.
}

\feat{Golem Crew}{more-golems-2}{20}{
	\skillref[2]{golemancy},
	\featref{more-golems}
}{
	As you muster more animating force, your crew of golems grows.
	You can maintain three golems simultaneously.
}

\feat{Dwarf Golem}{medium-golem}{20}{
	\featref{wood-golem} or \featref{wax-golem}
}{
	You can muster enough animating force to create larger golems, about mid-thigh height.
	Such golems have their \attref{might} increased by 1, and their \attref{grace} decreased by 1.
	You can maintain only one golem of this size---or larger---regardless of how many golems you can maintain in total.
	
	Golems of this size can attack effectively.
	They use 2 dice for {\unarmed} {\damagetests}.
	They can even use weapons, if their instructions are sufficiently explicit about acquiring and using them.
	
	Obviously, crafting the body of a larger golem takes more material and normally more time.
	Acquiring an oven, kiln, or forge large enough can also present an obstacle for some kinds of golem.
}

\feat{Dwarf Crew}{more-medium-golems}{20}{
	\skillref[1]{golemancy},
	\featref{medium-golem},
	\featref{more-golems}
}{
	You're really getting the hang of animating a \featref{medium-golem} now.
	You are no longer limited to just one, and may maintain your extra golems at this size if you wish---the ones granted by \featref{more-golems}, and \featref{more-golems-2}, if you have it.
}

\feat{Life-Size Golem}{large-golem}{25}{
	\skillref[1]{golemancy},
	\featref{medium-golem}
}{
	It takes a lot of clay to make a golem the size of a man, but this pales in comparison to the force required to bring such a golem to life.
	You should know, because you can finally muster that much force.
	Such golems have their \attref{might} increased by 3, and their \attref{grace} decreased by 1.
	They can attack and use weapons, the same as a \featref{medium-golem}.
	
	You can only maintain one golem of this size---or larger---regardless of how many golems you can maintain in total.
	Furthermore, a golem of this size also counts as a \featref{medium-golem}, against the limit for golems of that size.
	So you cannot maintain both a \featref{large-golem} and a \featref{medium-golem} without the \featref{more-medium-golems} feat.
}

\feat{Twin Men}{more-large-golems}{15}{
	\skillref[2]{golemancy},
	\featref{large-golem},
	\featref{more-medium-golems}
}{
	It takes a lot of animating force to maintain a \featref{large-golem}, but you've finally mustered enough to maintain \emph{two}.
}

\feat{Golem Helpers}{more-small-golems}{15}{
	\skillref[1]{golemancy},
	\featref{more-golems}
}{
	You can maintain 2 hand-sized golems in addition to however many golems you can otherwise maintain.
	These must always be hand-sized golems, however; they are unaffected by \featref{more-medium-golems}.
}

\feat{Helper Army}{more-small-golems-2}{15}{
	\skillref[2]{golemancy},
	\featref{more-small-golems}
}{
	You can maintain 5 hand-sized golems in addition to however many golems you can otherwise maintain.
	These replace the additional 2 given by \featref{more-small-golems}.
}

\feat{Batch Baking}{more-gingerbread-golems}{20}{
	\skillref[2]{golemancy},
	\featref{gingerbread-golem},
	\featref{more-golems}
}{
	Nobody bakes just one gingerbread man, so why should you animate just one?
	You can maintain two gingerbread golems for the effort of one, at any size category.
	This even applies to extra golems granted by other feats---such as \featref{more-medium-golems}, \featref{more-large-golems}, or \featref{more-small-golems}.
}

\feat{Stone Giant}{giant-stone-golem}{25}{
	\skillref[2]{golemancy},
	\featref{stone-golem},
	\featref{large-golem}
}{
	Stone forms all the largest things of our world---hills, mountains, continents---and it can also form the largest golems.
	You can make a stone golem larger than a \featref{large-golem}; three times the height of a man.
	Such golems have their \attref{might} increased by 6, their \attref{grace} decreased by 2, and their \statref{shock-threshold} increased by 8.
	
	Creating such a golem is a truly colossal task, typically taking weeks.
	You can only maintain one golem of this size, regardless of how many golems you can maintain in total.
	Furthermore, a golem of this size also counts as a \featref{large-golem}, against the limit for golems of that size.
	So you cannot maintain both a \featref{giant-stone-golem} and a \featref{medium-golem} without the \featref{more-medium-golems} feat, or a \featref{giant-stone-golem} and a \featref{large-golem} without the \featref{more-large-golems} feat.
}

\feat{Stone Guards}{more-stone-golems}{15}{
	\skillref[1]{golemancy},
	\featref{stone-golem}
}{
	Stone golems don't require as much animating force as most, as they are not particularly animated.
	With a bit of practice, you can animate a couple with no effort at all---useful as guards for your cottage, perhaps.
	
	You can maintain 2 stone golems in addition to however many golems you can otherwise maintain.
	These golems may be of any size category, except a \featref{giant-stone-golem}, and don't count towards any of your normal maintenance limits for the upper size categories.
}

\feat{Stone Gargoyle}{stone-gargoyle-golem}{10}{
	\skillref[1]{golemancy},
	\featref{stone-golem}
}{
	You may add wings to the stone figures you use to create your stone golems.
	Doing so doesn't allow them to fly, but it at least lets them cling to walls.
	
	A winged stone golem can be affixed to a wall, ceiling, or other such surface, instead of the ground.
	The legs and wings are affixed, but it still has its arms free.
	The wall must be able to support the weight of that much stone, and the golem is deanimated if the wall collapses.
	However, it hangs on the wall by magic, as long as it is animated, and needs nothing to attach it there.
	
	Tradition dictates, and allows, that these gargoyles also bear grotesque features such as horns, fangs, and claws.
	A gargoyle with such features---and at least the size of a \featref{medium-golem}---rolls 5 dice for {\unarmed} {\damagetests}.
}

\feat{Horse Golem}{golem-horse}{15}{
	\skillref[2]{golemancy},
	\featref{gingerbread-golem} or \featref{fabric-golem}
}{
	While a humanoid shape is certainly traditional for golems, other shapes are not impossible.
	Instead of making your golems humanoid, you may sculpt or otherwise form them in the shape of horses.
	A horse golem has the normal statistics for its size and material, except that its \statref{speed} is doubled.
	
	Horse golems can be made in every size that a \practitioner{golemancy} can make humanoid golems.
	A horse golem is of an appropriate nature to be ridden by a humanoid golem of the same size category.
	Therefore, the horse variety of a \featref{large-golem} can be ridden by a human, as long as it has at least 2 \attref{might}.
	
	A horse cannot understand language, and similarly a horse golem can never benefit from \featref{golem-understand-language} or \featref{golem-speak}.
}

\feat{Slender Golem}{golem-slender}{10}{
	\featref{gingerbread-golem} or \featref{fabric-golem}
}{
	You can animate more slender golems, with increased agility.
	You must make this decision when you craft the golem.
	A slender golem gains 1 \attref{grace}, but loses 1 \attref{might}.
	Furthermore, you can make it using slightly less material than a regular golem.
}

\feat{Bulky Golem}{golem-bulky}{10}{
	\featref{gingerbread-golem} or \featref{fabric-golem}
}{
	You can animate more bulky golems, with increased strength.
	You must make this decision when you craft the golem.
	A bulky golem gains 1 \attref{might}, but loses 1 \attref{grace}.
	However, it requires slightly more material than a regular golem.
}

\feat{Matryoshka Golems}{golem-nesting}{15}{
	\skillref[1]{golemancy},
	\featref{medium-golem},
	\featref{golem-reanimate}
}{
	You may nest golems inside one another.
	You can put a typical hand-sized golem inside a \featref{medium-golem}, or a \featref{medium-golem} inside a \featref{large-golem}.
	You may even nest all three levels.
	
	You must actually craft these nested golems, which obviously means the outer golems must be hollow.
	This is easy to do with clay or fabric, harder with wood, and very difficult with wax or gingerbread.
	The innermost golem does not need to be hollow, however, and does not even need to be the same material as the outer golem---it must be a material you can animate, however.
	However, it should be the same shape; you can only nest a \featref{golem-horse} inside another \featref{golem-horse}, for instance.
	
	Only the current outermost golem is animated---and only that one counts towards your animation limit.
	However, if it is destroyed, the next layer of golem down is automatically and immediately animated.
	It inherits the instructions and knowledge of the outer golem, and may immediately continue from where the outer golem left off.
}

\feat{Golem Self-Crafting}{golem-crafting-skill}{10}{
	\skillref[1]{golemancy},
	\skillrefspeciality[1]{crafting}{\anyspeciality},
	\featref{golem-intelligence}
}{
	When you animate a golem, you imbue it with your knowledge of the craft used to create it.
	It gains one \skillref{crafting} speciality appropriate to creating golems of its type.
	For example, you might give a gingerbread golem \skillrefspeciality{crafting}{Cook}, or a clay golem \skillrefspeciality{crafting}{Potter}.
	A wood golem might gain \skillrefspeciality{crafting}{Woodcarver} or \skillrefspeciality{crafting}{Carpenter}, depending on how it was constructed.
	
	The golem gains all the ranks that \emph{you} have in the relevant skill.
	You cannot use this feat if you do not have any ranks in a relevant skill, but the golem can gain up to 3 ranks, if you have that many.
	
	Some golems can ever repair themselves using this skill, if their instructions allow it and they have appropriate materials.
	Alternatively, you could set them to creating additional golem bodies, to animate with \featref{golem-reanimate-2}.
}

\feat{Remote Access}{golem-change-instructions-projection}{15}{
	\skillref[1]{golemancy},
	\featref{projection-golemancy},
	\featref{golem-change-instructions}
}{
	You can do more than just find your golems in the mental realm: you can alter them.
	You can reprogram a golem while you have an {\overtinterface} with it from the {\mentalrealm}.
	This still takes however long it would take you if you were physically touching the golem.
}

\feat{Thinking Engine}{stone-circle-golem}{15}{
	\skillref[2]{golemancy},
	\skillref[1]{projection},
	\featref{projection-golemancy},
	\featref{golem-intelligence},
	\featref{stone-golem}
}{
	Your experiments in combining \discref{projection} with \discref{golemancy} have borne fruit: an entirely new kind of golem, created from a \materialref{stone-circle}.
	Taking the immobility of a stone golem even further, this golem has no moving parts at all.
	It cannot wave an arm, waggle an eyebrow, or flap its lips.
	It has no senses either; it cannot see, hear, or even feel when someone touches it.
	In fact, its existence can only be detected from the {\mentalrealm}---though it cannot sense the {\mentalrealm} itself.
	
	The thinking engine's physical deficiencies are compensated by its mental prowess, however.
	It has no \attref{might} or \attref{grace} scores, but has a score of 5 in all other attributes.
	It also has \skillrefspeciality[3]{lore}{Astronomy}.
	These abilities make it very useful in a \featref{projection-merge}.
	
	You can only maintain one thinking engine at a time.
	However, it does not count against the number of golems you can maintain.
	
	The thinking engine's mind is tied to its physical form; it cannot be {\possessed}, and cannot be affected by \featref{projection-start-other}, \featref{projection-start-other-2}, or \featref{projection-start-other-3}.
	It is deanimated if the \materialref{stone-circle} that forms its body is destroyed.
	
	As normal, you can only animate a golem of this kind while erecting the \materialref{stone-circle}.
	As such, \featref{golem-reanimate} and \featref{golem-reanimate-2} might be very helpful.
	You may only use \featref{tree-standing-stone} to form a thinking engine if you also have \featref{wood-golem}.
}

\feat{Projection Engine}{stone-circle-golem-2}{15}{
	\skillref[2]{golemancy},
	\skillref[2]{projection},
	\featref{stone-circle-golem}
}{
	You can imbue your \featref{stone-circle-golem} with some of your knowledge of \discref{projection}.
	Firstly, this allows it to sense the {\mentalrealm}.
	At first, it can only sense minds \emph{within} the \materialref{stone-circle}.
	However, it benefits from \featref{projection-sense} and \featref{projection-sense-2}, if you have them.
	
	Secondly, the thinking engine can form {\interfaces} with minds that enter the \materialref{stone-circle}, either inside bodies or in the {\mentalrealm}.
	At first, it cannot do anything with these {\interfaces}.
	However, it can also use any feat you have that utilises an {\interface}.
	The exceptions are \featref{projection-start-other}, \featref{projection-start-other-2}, and \featref{projection-start-other-3}; the thinking engine cannot use these.
	
	Lastly, the thinking engine benefits from \featref{projection-interface-sense} and \featref{projection-interface-block}, if you have them.
	Even with \featref{projection-interface-block}, it always allows your {\interfaces} with it.
	
	Although the thinking engine gains your knowledge, it does not gain your experience.
	It gains no ranks in the \skillref{projection} as a result of this feat.
	
	The thinking engine's mind remains irrevocably tied to its body, and it cannot move within the {\mentalrealm}.
	Likewise, it cannot maintain an {\interface} with a mind that leaves the \materialref{stone-circle}.
}
