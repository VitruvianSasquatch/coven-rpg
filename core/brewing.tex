\chapter{Brewing}
\disclabel{brewing}{Brewer}{Brewers}

\discref{brewing} may well be the least magical of witchcraft's disciplines.
In fact, it is not even restricted to witches.
Many wise folk can cook up a remedy for the most common ailments.
And almost every village has someone to brew their hops into beer.

As such, the feats in this discipline are not necessary for common brewing; they represent only those brews that require some trick, skill, or guarded piece of knowledge.
Anyone, even those without ranks in the \skillref{brewing} skill, may attempt to create more common brews.
Some, such as alcohol, are relatively simple.
A Test may determine the quality of the result, but almost anyone can create \emph{something} appropriate.

Remedies for diseases and ailments are a little more complicated.
That's not to say that it takes much skill to boil up a few strips of willow bark, but half of the common knowledge of folk medicine is ungrounded superstition.
It requires a \testtype{ken}{brewing} Test to create an appropriate remedy for most ailments, with the {\tn} of the Test determined by the rarity and severity of the ailment.
This normally requires access to a wide array of \materialrefplural{herb} and a boiling cauldron; the {\tn} might be increased as normal by trying to make do with inadequate materials.

Particularly simple remedies, such as helping a person to sleep, or easing an ache, might not require the \materialrefplural{herb} to be brewed at all.
In this case, a \testtype{ken}{botany} Test might suffice.

\section{Creation and Application}

The \skillref{brewing} skill and \discref{brewing} discipline don't just cover potions brewed in a cauldron.
Potions, poultices, poisons, tinctures, salves, ointments, even beer, mead, wine and spirits.
Witches have many ways of turning \materialrefplural{herb}, and even other things, into more useful forms.

Each feat that allows a witch to prepare such a mixture lists the method of preparation and delivery.
The rules of such methods are presented below.

\subsection{Brewing and Chewing}

Different methods of preparation require different equipment, and take different periods of time.

\mixcreation{Cauldron}{cauldron}{
	Most potions are brewed in cauldrons, filled with water and brought to boil.
	This requires, obviously, a cauldron, as well as a fire to heat it.
	A smaller kettle might do in a pinch, but requires a Test.
	A full cauldron will typically yield several doses.
	Brewing in a cauldron requires around 15 minutes to bring the water to the boil, and another minute to mix the potion.
}

\mixcreation{Poultice}{poultice}{
	A poultice doesn't need to be brewed at all; the ingredients are simply chewed into a paste.
	Some of the more dangerous poultices should definitely be ground with a mortar and pestle, however, rather than allowed anywhere near the mouth.
	Creating a poultice requires less than a minute.
}

\mixcreation{Still}{still}{
	Some potions, or spirits, need to be distilled.
	This requires quite a lot of dedicated equipment, a carefully maintained heat source, and several hours.
}

\subsection{Method of Delivery}

Although most potions are drunk, there are many ways to get a mixture into a person's body.
Some faster, some slower, and some far easier to inflict on an unwilling victim.

\mixdelivery{Drink}{drink}{
	A \mixdeliveryref{drink} is about half a litre of liquid that must be drunk to take effect.
	It can be quaffed as an {\action}, and takes effect after 1 {\round}, unless specified otherwise.
}

\mixdelivery{Spike}{spike}{
	A \mixdeliveryref{spike} is a much smaller quantity of liquid than a \mixdeliveryref{drink}, little enough that it could be slipped into a glass of wine without being obvious.
	It can be drunk willingly, but typically isn't.
	It takes effect after 1 {\round}, unless specified otherwise.
	
	In a glass of wine or a cup of water, one dose will normally go unnoticed until it is too late.
	A second dose causes a noticeable change in taste, scent, or colour, which will typically be noticed unless the drinker isn't paying much attention.
	Larger and stronger drinks can conceal more doses, however.
}

\mixdelivery{Topical}{topical}{
	A \mixdeliveryref{topical} mixture is applied to the skin.
	It typically requires more than an {\action} to smear it on, or bind a wad in place.
	It generally only takes effect after a few minutes, but can kick in a little faster if applied to a wound or a mucous membrane.
	Some need to be applied to the correct part of the body.
	For example, if it is to treat a wound, it should be applied to the wound, and if it is to enhance the eyesight, it should be applied to the eyes.
}

\mixdelivery{Snuff}{snuff}{
	These mixtures are boiled or ground down to a powder, which must be inhaled into the nostrils.
	They can be taken as an {\action}, and take effect immediately.
	Giving them to someone unwilling requires forcing them to inhale in some fashion, but it can usually be achieved if you have the target at your mercy for a minute or more.
}

\mixdelivery{Injury}{injury}{
	These mixtures, typically harmful ones, must be delivered into the bloodstream via an injury.
	The most expedient way to do this is to smear it on an arrow or an edged weapon, requiring an {\action}.
	It's good for one cut, but otherwise remains on the weapon until rubbed off or washed away.
	Beware rain.
	It takes effect immediately, unless specified otherwise.
}

%TODO: Gaseous? Needs to be stored in an air-tight bottle?
%TODO: Incense? Needs to be burned, evaporated?

\subsection{Time to Effect}

Mixtures applied by different methods typically require different lengths of time to take effect, as specified above.
Some mixtures require more or less time to take effect, as specified in their descriptions.
If quaffed on a creature's {\turn}, a mixture that takes effect after 1 {\round} comes into play at the start of their {\turn} in the next {\round}.

The GM may allow a character to make \attref{might} Tests to stave off the effects of an unwanted mixture.
This might extend the time to effect by two or three times, but should not allow them to avoid the effect entirely, except in the most exceptional cases.

For ingested mixtures (a \mixdeliveryref{drink} or \mixdeliveryref{spike}), vomiting before it takes effect can massively ease this \attref{might} Test, and even allow avoiding the effect entirely.
Ingesting an emetic herb, such as \herb{veratrum or the toadstool ``emetic russula''}{2}, serves to induce vomiting.
However, these herbs do not act quickly enough to help with a mixture that takes effect in just 1 {\round}.
To prevent these, a witch might look into \featref{vomit-spike}.

\section{Side-Effects}

While some of the more noxious mixtures a witch can brew produce adverse effects by design, these are not the only ways a potion can hurt.
Many mixtures come with adverse side effects all by themselves, and these are compounded by the dangers of overdosing and combining brews.

\subsection{Overdosing}

Many potions carry harmful effects when taking too many doses.
These typically only occur if multiple doses would be in effect simultaneously; taking another dose after the first has worn off is safe unless specified.
Some of the effects of overdoses are given explicitly, but many are given as general guidelines.
The GM is left to adjudicate in the latter case.
Typically the worst effects of overdosing can be staved off with a \attref{might} Test, with the {\tn} affected by how many excess doses have been taken, and how close in succession they were taken.

\subsection{Mixing Mixtures}

Mixing multiple potions can have adverse and unexpected effects.
These kick in when a character is under the effect of two substances that both affect the same \seclink{Attribute}{attributes} or other statistic.
The effects are unpredictable.
The GM is free to apply any appropriate penalty, possibly calling for a \attref{might} Test to avoid or alleviate the effects.
The following table is provided for inspiration.
The GM may roll 2 six-sided dice and compare their sum against the table to randomly determine an effect, if desired.

\begin{simpletable}{rX}
	\toprule
	\dice{2} & Effect\\
	\midrule
	2 & Apply severe overdose effects of the first mixture.\\
	3 & Exhaustion, unconsciousness and/or oxygen deprivation.\\
	4 & Apply moderate overdose effects of the first mixture.\\
	5 & Ignore any positive effects of the first mixture.\\
	6 & Double any detrimental effects of the first mixture.\\
	7 & Re-roll twice on the table, taking both results.\\
	8 & Double any detrimental effects of the second mixture.\\
	9 & Ignore any positive effects of the second mixture.\\
	10 & Apply moderate overdose effects of the second mixture.\\
	11 & Twitching, seizure, overheating and/or organ failure.\\
	12 & Apply severe overdose effects of the second mixture.\\
	\bottomrule
\end{simpletable}

The effect of painkillers---to ignore {\damage}---does not count as a statistic for the purpose of mixing substances.
As such, a character may safely be under the effects of multiple painkillers as long as their other effects do not overlap.
Additionally, withdrawal effects (such as those of \featref{mental-stimulant}) do not count for mixing.
%TODO: Use a better example when I have a withdrawal effect that modifies statistics.

\section{Antidotes}
\seclabel{antidotes}

While vomiting can help to avoid the effects of a potion \emph{before} they kick in, ending the effects once they are ongoing requires an {\antidote}.
{\antidotes} exist not just for poisons, but for any potion or other mixture with an ongoing effect.
Each {\antidote} will only work on potions it is designed to counteract, however.
It will specify in its description what sorts of potion it is effective against.

{\antidotes} can be delivered through different methods, just like potions themselves, and take the usual lenght of time to kick in.
Once it kicks in, the {\antidote} ends all ongoing effects of any potions it is designed to counteract.
One dose of {\antidote} can end the effect of several different potions, as long as it is designed to counteract all of them, or several doses of the same potion.

When an {\antidote} takes effect, treat it as though the duration of the potion has expired.
An {\antidote} ends both beneficial and detrimental effects, but can only end \emph{ongoing} effects.
Any {\damage} that has been dealt stays dealt.
Overdose effects are ended, but organs that have failed are not repaired, and so on.
Withdrawal effects (such as \featref{mental-stimulant}), are not ended by an {\antidote}.
In fact, taking an {\antidote} causes the withdrawal effects to kick in, as it ends the effect of the potion.

The effect of an {\antidote} is immediate, not ongoing.
This means that you cannot have an {\antidote} to an {\antidote}.

\section{Feats}

\feat{Numbing Painkiller}{painkiller-grace}{15}{
	None
}{
	\mix{cauldron}{drink}{\Herb{willow bark}{2}}
	%Willow bark contains natural NSAIDs.
	
	The drinker may ignore 1 point of {\damage} for a few hours, but loses 1 \attref{grace} for the same duration.
	Two doses may be effective simultaneously.
	Further doses cause paralysis, and possibly organ failure.
}

\feat{Dimming Painkiller}{painkiller-mental}{10}{
	None
}{
	\mix{cauldron}{drink}{\Herb{poppy seed}{2}}
	%Poppies contain natural opioids.
	
	The drinker may ignore 1 point of {\damage} for a few hours, but loses 1 \attref{ken} and \attref{wit} for the same duration.
	Two doses may be effective simultaneously.
	Further doses cause unconsciousness, and possibly cessation of breathing.
}

\feat{Blurring Painkiller}{painkiller-heed}{15}{
	None
}{
	\mix{cauldron}{drink}{\Herb{barley}{2}}
	%Barley contains a lot of phenols, which are used in the synthesis of a lot of pharmaceuticals.
	%Malt barley is also used in producing alcohol, which can justify the blindness.
	
	The drinker may ignore 1 point of {\damage} for a few hours, but loses 1 \attref{heed} for the same duration.
	Two doses may be effective simultaneously.
	Further doses cause blindness, which can become permanent.
}

\feat{The Hard Stuff}{might-potion}{20}{
	None
}{
	You know how to make a drink that'll really put hairs on a man's chest.
	Or a woman's, at that.
	
	\mix{still}{drink}{Alcohol, \herb{apple}{2}}
	%Scumble is 'mostly apples'.
	
	The drinker gains 1 \attref{might} for a few hours, and loses 2 \attref{wit} and \attref{heed} for the same duration.
	A second dose will render the drinker unconscious.
	Further doses are dangerously poisonous, causing vomiting, seizures and oxygen deprivation.
}

\feat{The Pure}{might-potion-2}{15}{
	\skillref[1]{brewing},
	\featref{might-potion}
}{
	\mix{still}{drink}{Alcohol, \herb{fennel}{2}, \herb[aniseed]{green anise}{3}, \herb{wormwood}{3}}
	%Wormwood, sweet fennel and green anise are the three herbs in absinthe.
	
	You brew your drink clean and pure.
	This functions as \featref{might-potion}, except it decreases \attref{wit} and \attref{heed} by only 1 point.
}

\feat{The Green Fairy}{might-potion-3}{20}{
	\skillref[2]{brewing},
	\featref{might-potion-2}
}{
	\mix{still}{drink}{Alcohol, \herb{fennel}{2}, \herb[aniseed]{green anise}{3}, \herbcreature{grand-wormwood}{5}}
	
	Your drink gives men the strength of horses.
	You don't want to see what it does to horses.
	This functions as \featref{might-potion-2}, except it increases \attref{might} by 2 points.
}

\feat{Empathogen}{charm-potion}{20}{
	None
}{
	\mix{cauldron}{drink}{\Herb{violet}{2}}
	%Violet contains piperonal, a common precursor to the empathogen MDMA (ecstasy).
	%Violets were also emblematic flowers of Aphrodite and Priapus.
	
	This potion affords the drinker a greater sense of empathy and connection with those around them.
	The drinker gains 1 \attref{charm} for a few hours, and loses 2 \attref{will} for the same duration.
	A second dose causes agitation and paranoia, instead reducing \attref{charm} by 1.
	Further doses cause the drinker to overheat, suffering heat stroke, and may lead to internal bleeding and organ failure.
}

\feat{Alpha's Potion}{presence-potion}{20}{
	None
}{
	\mix{cauldron}{drink}{\Herb{onion}{2}}
	%Onions have an obvious and pervasive smell that makes people cry.
	
	This potion grants the drinker increased confidence, a slightly louder voice, and a certain indefinable \emph{obviousness}.
	There's a certain smell goes with it, a sort of threat pheromone, but it sits below the conscious level for all but the most attentive people.
	The increased confidence that the potion provokes tends to go a bit too far, however, veering into arrogance.
	If the drinker isn't careful, they come across as, frankly, a right prick.
	
	The drinker gains 1 \attref{presence} for a few hours, and loses 2 \attref{charm} for the same duration.
	A second dose causes a total loss of social graces, decreasing \attref{charm} by a further 2 points, without further increasing \attref{presence}.
	It also causes the victim to sweat profusely and smell strongly of onions.
	Further doses cause degeneration into raving, incoherent lunacy, and turns the sweat into a glistening mucus that coats the skin.
}

\feat{Stimulant}{mental-stimulant}{25}{
	None
}{
	\mix{cauldron}{drink}{Ants, vinegar}
	%Ants have a gland which can produce phenylacetic acid.
	%This can be converted to phenylacetone using acetic anhydride, which can be derived from acetic acid.
	%Acetic acid is the main component of vinegar (beside water). Vinegar is simply fermented from ethanol.
	%Phenylacetone can be converted to amphetamine using the Leuckart reaction and formic acid (also from ants).
	%Amphetamine is a widely-known nootropic (cognitive enhancer).
	
	The drinker gains 1 \attref{wit}, \attref{will} and \attref{heed} for about an hour.
	The potion also staves off tiredness for the duration. %TODO: A mechanical effect for this?
	After the potion wears off, you pay the price of your temporarily enhanced performance.
	You suffer a \negative{1} penalty to all rolls for the next 24 hours.
	Additional doses within this period are ineffective.
	
	Being under the effect of two doses simultaneously causes a headache that counteracts the increased attributes.
	Further doses can cause bleeding into the brain and death.
}

\feat{Stimulant Dragging}{mental-stimulant-extension}{15}{
	\skillref[1]{brewing},
	\featref{mental-stimulant}
}{
	A slight change to the formula of your \featref{mental-stimulant} allows its effect to be extended by additional doses.
	Drinking another dose as one begins wearing off extends the effect and staves off the withdrawal.
	However, the body can only sustain such enhanced performance for so long.
	Whenever you take a dose after the first, make a \attref{might} Test.
	The {\tn} is 9 for the second dose, and increases by 3 for every subsequent dose.
	On a failure, you pass out, gain no benefit from the additional dose, and the withdrawal effects kick in.
	You cannot be roused for several minutes.
}

\feat{Hysterical Strength}{physical-stimulant}{15}{
	None
}{
	A person's muscles are stronger than they normally get to use, strong enough to break their own bones.
	There's a good reason you don't get to use the full strength, you see.
	But you've figured out how to unlock that extra potential, and damn the consequences!
	
	\mix{cauldron}{drink}{\Herb{joint pine}{3}}
	%Joint pine is Ephedra, which contains ephedrine.
	%Ephedrine is similar in structure and function to epinephrine (adrenaline).
	%Adrenaline is oft-blamed for hysterical strength.
	
	The drinker gains 1 \attref{might} and 1 \attref{grace} for a few minutes.
	For the duration, any strenuous activity causes the drinker to suffer a \dice{2} {\damagetest}.
	Strenuous activity includes the \actionref{dash} and \actionref{attack} {\actions}, any Test using \attref{might} or \attref{grace} (excluding {\damagetests} as part of the \actionref{attack} {\action}), and other things at the GM's discretion.
	
	Being under the effect of two doses simultaneously does increase \attref{might} and \attref{grace} further, but causes {\damagetests} as a result of any movement at all; only lying still is safe.
	Further doses cause seizures, triggering the {\damagetests} themselves.
}

\feat{Oakenhide Brew}{resilience-potion}{15}{
	None
}{
	\mix{cauldron}{drink}{\Herb{parsnip}{2}, \herb{oak bark}{2}}
	%Parsnip stems and leaves can cause skin rash.
	
	This potion causes the drinkers skin to harden in patches, sprouting bark-like growths.
	It lends a remarkable toughness, at the price of an awful stiffness.
	
	It takes effect over the course several minutes, eventually increasing the drinker's \statref{res} by 1, but reducing their \attref{grace} by 2.
	A second dose further increases \statref{res}, but makes the drinker so stiff that they cannot move at all.
	Further doses risk making this immobility permanent.
	
	The growths flake off after a few hours, returning the drinker to normal.
	The energy expended in growing and shedding these patches takes a couple of extra meals to recoup---someone using this potion more than once a day for an extended period simply can't keep up, and will starve.
}

\feat{Stonehide Brew}{resilience-potion-2}{15}{
	\skillref[1]{brewing},
	\featref{resilience-potion}
}{
	You've found a rare succulent plant which disguises itself as a pebble.
	Boiling their tough skin, you can create a potion that lends the hardness of stone to those who drink it.
	
	\mix{cauldron}{drink}{\Herb{parsnip}{2}, \herb{pebble plant}{4}}
	%Pebble plants are the genus Lithops.
	
	This functions as \featref{resilience-potion}, except it increases \statref{res} by 2.
}

\feat{Ironhide Brew}{resilience-potion-3}{15}{
	\skillref[2]{brewing},
	\featref{resilience-potion-2}
}{
	Using the legendary strength of the \creatureref{ironwood} tree, you can grow a skin that will nearly turn knives.
	
	\mix{cauldron}{drink}{\Herb{parsnip}{2}, \herbcreature{ironwood}{5}}
	
	This functions as \featref{resilience-potion}, except it increases \statref{res} by 3.
}

\feat{Supple Hide}{resilience-potion-supple}{15}{
	\skillref[2]{brewing},
	\featref{resilience-potion}
}{
	The growths produced by your \featref{resilience-potion}, \featref{resilience-potion-2}, and \featref{resilience-potion-3} are more supple, less restrictive.
	They reduce \attref{grace} by only 1.
}

\feat{Rapid Hide}{resilience-potion-fast}{15}{
	\skillref[1]{brewing},
	\featref{resilience-potion}
}{
	Your \featref{resilience-potion}, \featref{resilience-potion-2}, and \featref{resilience-potion-3} are more potent, faster-acting.
	They take effect in only 1 round.
}

\feat{Eye Drops}{sight-potion}{15}{
	None
}{
	\mix{cauldron}{topical}{\Herb{eyebright}{2}}
	
	Dripped into the eyes, this concoction enhances the eyesight.
	Too much so, it could be argued, as bright lights are rendered blinding.
	
	For the next hour, the user rolls an extra die on \skillref{perception} Tests relying on sight.
	This replaces any existing bonus dice from an ability such as a \familiarrefpossessive{raptor}, but applies on top of the normal dice granted by the \skillref{perception} skill.
	However, in light as bright as sunlight, the user instead \emph{loses} 1 die from these rolls.
	A bright flash of light, or looking into the sky on a sunny day, blinds the user for several minutes.
}

\feat{Molenose Powder}{smell-potion}{10}{
	None
}{
	\mix{cauldron}{snuff}{\Herb{jasmine}{3}, an animal's nose}
	%Jasmine is noted for its fragrance.
	%The flowers look star-shaped, like a star-nosed mole's nose.
	
	Snorted into the nostrils, this concoction delivers an overwhelming fragrance of jasmine, which quickly fades to leave the user's sense of smell greatly enhanced.
	This enhanced sense lasts a few minutes, however, the concoction strikes the user completely blind for the same duration.
	
	Their sense of smell becomes fine enough to allow them detect people moving around a room, in real time.
	This should make walking around at a slow pace relatively easy, despite the blindness.
	With a difficult Test, they might even be able to run, without running into anything.
	Successfully attacking people, however, is all but out of the question, and certainly suffers a huge penalty.
	
	For the duration, the user rolls 2 extra dice on \skillref{perception} Tests relying on smell.
	This replaces any existing bonus dice from an ability such as a \familiarrefpossessive{rat}, but applies on top of the normal dice granted by the \skillref{perception} skill.
}

\feat{Mole's Eyes}{smell-potion-2}{10}{
	\skillref[1]{brewing},
	\featref{smell-potion}
}{
	By cutting \featref{smell-potion} with \herb{eyebright}{2}, you can ameliorate its effect on the eyesight.
	
	\mix{cauldron}{snuff}{\Herb{jasmine}{3}, \herb{eyebright}{2}, an animal's nose}
	
	This functions as \featref{smell-potion}, except it doesn't render the user completely blind.
	It still has a drastic effect on their vision, leaving even nearby objects as little more than blurry shapes.
	However, the user can safely walk around without trouble, and can even run without \emph{much} risk of running into something.
	Attacking people is even possible, albeit still at a penalty.
	They also halve the number of dice they roll on \skillref{perception} Tests relying on sight.
}

\feat{Recovering Mole}{smell-potion-3}{15}{
	\skillref[2]{brewing},
	\featref{smell-potion-2}
}{
	A subtle change to the formula of your \featref{smell-potion} makes it last much longer, without hindering the eyes any more.
	
	The duration of your \featref{smell-potion} (and \featref{smell-potion-2} concoction) increases to an hour.
	However, this increases the duration only of its effect on smell; vision still recovers to normal after only a few minutes.
}

\feat{Vomiting Drops}{vomit-spike}{15}{
	None
}{
	\mix{cauldron}{spike}{\Herb{mistletoe}{2}}
	
	These drops cause immediate vomiting.
	They take effect immediately, and act quickly enough to be effective against ingested mixtures that take effect in just 1 {\round}.
	The drinker loses at least one {\action}, and usually several, while they heave and retch.
}

\feat{Sleeping Solution}{sleep-spike}{10}{
	None
}{
	\mix{cauldron}{spike}{\Herb{camomile}{2}}
	
	This soporific kicks in quite slowly, but the drinker should be asleep with 10 minutes, and remain that way for a few hours.
	A few doses are quite safe, and will accelerate the effect a little, but too many can put the victim into a coma.
}

\feat{Garlic Solution}{garlic-spike}{15}{
	\skillref[1]{brewing}
}{
	\mix{still}{spike}{\Herb{garlic}{2}}
	
	A distillation of purest garlic essence, this solution goes straight to the sinuses and burns something awful.
	It takes effect immediately.
	The drinker must make a {\tn} 12 \attref{might} Test or pass out immediately, unable to be roused for several minutes.
	For the next hour or so, they have no useful sense of smell and a full breath to the face robs others of their own sense of smell.
	They can be easily tracked by scent as the stuff leaks from their pores.
	Remarkably, however, the solution itself has no obvious scent until ingested.
	
	Vampires suffer far worse, of course.
}

\feat{Kick From the Head}{projection-spike}{15}{
	\skillref[1]{brewing},
	\featref{garlic-spike},
	\featref{projection-start}
}{
	You can make a concoction with enough kick to knock the mind right out of the body.
	
	\mix{cauldron}{spike}{\Herb{garlic}{2}, \herb{morning glory}{3}}
	%Morning Glory is an Aztec entheogen.
	
	This concoction takes effect immediately, ejecting the drinker's mind from their body and into the {\mentalrealm}.
}

\feat{Medium's Potion}{medium-spike}{10}{
	\skillref[1]{brewing},
	\featref{projection-spike},
	\featref{medium}
}{
	By brewing the herb used to enter a \featref{medium} trance into the right concoction, you can skip the tedious meditating.
	Or feed it to someone else, if you like.
	
	\mix{cauldron}{spike}{\Herb{garlic}{2}, \herb{morning glory}{3}, \herb[stinking nightshade]{black henbane}{2}}
	
	This concoction takes effect immediately, ejecting the drinker's mind into the {\mentalrealm} and placing their body into a \featref{medium} trance.
	It grants finer control than use of the raw herb, and the \featref{medium} may end the trance at any point by re-entering her body.
	However, a soul ready and waiting to {\possess} the body still gets the first chance.
}

\feat{Nettle's Bite}{poison-pain}{15}{
	None
}{
	You can grind up nettles while preserving, and enhancing, their sting.
	The resulting mixture won't do much lasting harm, but it hurts like nobody's business.
	
	\mix{poultice}{injury}{\Herb{nettle}{1}}
	
	If this poison is delivered by an \actionref{attack}, the target's \statref{st} is treated as being 1 lower against the {\damagetest}.
	This does not increase the {\damage} dealt, but is more likely to send the target into {\shock}.
	
	Preparing this \mixcreationref{poultice} by chewing is incredibly painful, and leaves a swollen tongue that makes speaking difficult for hours---a mortar and pestle are recommended.
}

\feat{Festering Poison}{poison-infection}{10}{
	None
}{
	\mix{poultice}{injury}{\Herb{bloodwort}{2}}
	
	This wicked concoction invariably causes bleeding and infection when applied to a wound.
	Cleaning out the infection requires medical attention, and typically a Test.
	{\damage} caused by a weapon coated in this poison, or any wounds it is applied to, will not heal until the infection has been cleared.
	If the infection goes untreated for several days, it can become lethal.
	Even if the infection is treated and the wound heals, it will often leave a wicked scar.
	
	Chewing up one or two doses of this \mixcreationref{poultice} is relatively safe, as long as you have no wounds around your mouth.
	Chewing a larger batch can cause bleeding and subsequent infection in the mouth.
}

\feat{Embalming Fluid}{embalming-fluid}{10}{
	\skillref[1]{brewing}
}{
	\mix{still}{drink}{Alcohol, ants, iron shavings}
	%Basing this on formaldehyde.
	%Formaldehyde is produced industrially by oxidation of methanol.
	%One catalyst used for this is a mixture of metals including iron.
	%Ants are for the connection between formic acid and formaldehyde.
	
	This fluid is not intended for consumption, but rather for soaking corpses.
	When used in {\embalming}, the corpse can preserved for years, or even decades if it is well cared for.
	One dose suffices to {\embalm} a rat or bird, but two doses are required for a cat, or more than a dozen for a human.
	
	If, for some reason, the potion is imbibed, it proves quite toxic.
	It causes pain, nausea and convulsions, progressing, over the course of a few minutes to permanent blindness, and probably death.
}

\feat{Healing Salves}{brewing-healing}{10}{
	\skillref[1]{brewing}
}{
	You know a wide range of minor poultices, salves and remedies for cuts, bruises and other physical injuries.
	As long as you have access to a reasonable supply of various \herbtypeplural{2}, and time to chew up poultices, you may use your \skillref{brewing} skill in place of your \skillref{healing} skill on Tests to heal people and creatures of most physical injuries.
	Setting broken bones and performing surgery still requires \skillref{healing}.
	
	Similarly, you may use your \skillref{brewing} rank in place of your \skillref{healing} rank when determining the {\damage} healed by a patient during a day of rest.
}

\feat{Exposure}{poison-resistance}{15}{
	\skillref[1]{brewing}
}{
	You've tasted a few too many of your own concoctions, but you're still alive.
	In fact, you're beginning to build up a bit of resistance.
	
	You roll an extra die on Tests to resist poisons or the like.
}

\feat{Bottled Sobriety}{antidote-alcohol}{10}{
	\skillref[1]{brewing}
}{
	\mix{cauldron}{spike}{\Herb{tea leaves}{4}}
	%Tea contains caffeine, commonly touted as sobriety-inducing.
	
	This concoction acts as an {\antidote} to alcohol of any kind, as well as any other mixture which lists alcohol as an ingredient, such as \featref{might-potion}.
	Alas, it won't cure a hangover.
}
