\chapter{Character Creation Guide}
\chaplabel{character-creation-guide}

\section{Step Zero: Character Concept}

The most important part of a character is the concept.
Who is your character, what does she do?
You can try to flesh this all out now, or fill it in as you work through character creation.
Make sure your character is somebody you will enjoy role-playing.

Also make sure your witch fits into the coven; discuss this with the other players and your GM.
It can be quite painful for everyone involved playing an unscrupulous \practitioner{necromancy} in a coven of saccharine healers, or vice versa.
The GM should provide some idea of the tone intended for the game, to avoid this sort of trouble.
Diversity can also be good: make sure you know what you're letting yourselves in for if \emph{everybody} in the group wants to play a \practitioner{brewing}.

\section{Step One: Starting Experience}

Your GM will assign you an amount of experience (XP) to use during character creation.
By default, this is 100 XP, though the GM is free to adjust this to suit a different style of characters or campaign.

100 XP is suitable for witches who have just completed their apprenticeship and are taking over their own steading.
40 XP might be more suitable for witches who are still in an apprenticeship, and may have only just bound their familiar.
200 XP might be appropriate for witches with a few years of caring for a steading under their belt.
Even more experienced witches might require even more starting XP.
For fairness, the GM should probably give all characters the same starting XP, unless there is a good reason otherwise.

Make a note of how much XP you have, and keep track as you spend it later.
It can be well worth keeping a record of everything you have spent XP on over the course of character creation and any later development.

\section{Step Two: Attributes}

Attributes are a witch's broad, innate capabilities.
Is she skinny and lithe, or broad and well-muscled; quick-witted, or bullheaded; domineering, or silver-tongued?

At character creation, you have 20 points to spend on your character's attributes.
Set each of the eight attributes to between 0 and 4, inclusive, such that they sum to 20.

Attributes represent much more innate ability than skills.
While they can be developed through much hard work, they are not often passively improved over the course of a lifetime.
As such, it is not recommended to change the number of points available for attributes as readily as one might change the starting XP.

\section{Step Three: Skills}

A witch does not begin totally unskilled.
Select 1 \seclink{general skill}{general-skills}; you begin with 2 ranks in this skill.
Select an additional 3 general skills; you begin with 1 rank in each of these.
Lastly, select 1 \seclink{discipline skill}{discipline-skills} and one \seclink{speciality skill}{speciality-skills}; you begin with 1 rank in each of these skills.

The GM is also free to adjust the number of skills and ranks granted to starting characters.
General skills represent general life experience, speciality skills tend to result from vocational experience, and discipline skills represent experience with magic and witchcraft.
However, acquiring more than 1 rank in a discipline skill without learning magic from the associated discipline is very rare; such ranks ought to be acquired through XP rather than granted at character creation.

\section{Step Four: Familiar}

A witch's familiar is her essential and constant companion, and is usually bound early in her apprenticeship.
The available familiars are detailed in \chapref{familiars}.
Select one, spending XP (including the XP for any options) from your starting XP if necessary.

A witch can never have more than one familiar.
However, with the GM's approval, you may decline to select your familiar yet.
In this case, you begin play without a familiar and must acquire it during play, spending the necessary XP then.
This can be useful if you want a familiar with a high XP cost, but want to spend your starting XP on other things.

\section{Step Five: Spending Experience}

You should almost certainly have some XP left over after purchasing your witches familiar, and now you can spend it.
There are three main things to spend XP on: attributes, skills and feats.
The costs for \seclink{improving attributes}{improving-attributes} and \seclink{improving skills}{improving-skills} are given in \chapref{attributes-and-skills}.
Feats, along with their costs, can be found in the chapters of \partref{disciplines}.

Feats from some disciplines are cheaper if you are sufficiently advanced in the skill that governs that discipline.
A feat is 5 XP cheaper if you have a higher level of the discipline's governing skill than is required for the feat.
For example, a witch with \skillref[3]{brewing} who buys a \discref{brewing} feat that requires only \skillref[2]{brewing} gets a 5 XP discount.
A witch with \skillref[1]{brewing} gets a 5 XP discount when buying a \discref{brewing} feat that does not require the \skillref{brewing} skill at all.

The disciplines of \discref{headology} and \discref{druidcraft} do not have governing skills, and can never benefit from this discount, but all other disciplines do.

Don't worry about building the perfect witch and buying every feat you want at this point.
Many feats simply aren't accessible to a witch just starting out, and you will get more XP to spend as you play.
Similarly, don't worry if you have a little bit of XP left over; you'll be able to spend it once you get more.

\section{Derived Statistics}
\statlabel{Resilience}{res}
\statlabel{Shock Threshold}{st}
\statlabel{Dodge Rating}{dr}
\statlabel{Speed}{speed}

\begin{simpletable}{ll}
	\toprule
	Statistic & Derivation\\
	\midrule
	\statref{res} & $3$\\
	\statref{st} & $12 + \text{\attref{might}} + \text{\attref{will}}$\\
	\statref{dr} & $8 + \text{\attref{grace}} + \text{\attref{heed}}$\\
	\statref{speed} & $8 + \text{\attref{might}} + \text{\attref{grace}}$\\
	\bottomrule
\end{simpletable}

\section{Steading}

Most witches have a steading.
This is the area a witch watches over, a region she defends and protects the inhabitants of.
The duties a witch has to her steading are numerous and varied, but typically involve healing the inhabitants and protecting them from threats of a magical nature.
Some witches also perform midwifing, care for the land itself, or even take it upon themselves to deal with non-magical threats, such as invading armies.
A witch's responsibilities are not limited to her steading, and nothing stops her from responding to threats outside it.
But inside it, everything is certainly her responsibility.

Decide whether your witch has a steading.
How big is it?
One village, several, or an entire kingdom?
What duties does she perform within it?
Do the inhabitants appreciate what she does for them?

Also discuss this with your GM, and the other players.
Has the GM already described a village that could be your steading?
It is not unheard of for witches to share a steading, although this can obviously lead to disagreements.
Do you share a steading with your coven, or have you carved the local region into one steading each?
