\chapter{Attributes and Skills}
\chaplabel{attributes-and-skills}

\section{Attributes}
\seclabel{attributes}

Attributes are a character's broad, innate capabilities.
They represent physical capacity and natural talent.
That is not to say they can't be improved---one can grow muscle through exercise and the brain is no different---but such improvement represents a more significant investment than picking up a new skill.
A character has eight attributes: \attref{might}, \attref{grace}, \attref{ken}, \attref{wit}, \attref{will}, \attref{heed}, \attref{charm} and \attref{presence}.
For human characters, these range from 0 to 5, with 2 as the average for a human.
Non-human characters may have attributes outside this range.

A summary of these attributes is provided below, along with examples of using the attribute.
Note that many of the example Tests would be accompanied by an appropriate skill.

\subsection{Improving Attributes}
\seclabel{improving-attributes}

Attributes can be improved by spending XP, but they are more expensive than skills.
Improving an attribute represents substantial and continuous effort.

Improving an attribute by 1 point, to a maximum of 4, costs 25 XP.
Increasing an attribute to 5 represents the absolute peak of human ability and is even more expensive, costing 40 XP.

\attribute{Might}{might}

\attref{might} represents physical strength, endurance, and hardiness.
It's used to lift things, smash things, resist diseases and endure hard labour, to put the hurt on people and to resist having the hurt put back on you.
\attref{might} is the attribute you use when rolling damage with melee weapons, and also determines the amount of damage required to put you down.
It can also prove useful when a brewer or botanist feeds you something you shouldn't have eaten.
Lastly, powerful legs let you run faster.

\begin{simpletable}{rX}
	\toprule
	{\tn} & Example Task\\
	\midrule
	9 & Jumping across a \SI{3}{\metre} gap.\\
	12 & \\
	15 & \\
	18 & \\
	21 & \\
	\bottomrule
\end{simpletable}

\attribute{Grace}{grace}

\attref{grace} represents agility, dexterity, and reflexes.
It's used to dodge swords, manoeuvre broomsticks, do backflips, dance waltzes, and hastily scratch runic circles into the floor without smudging them and letting the demons in.
\attref{grace} determines how hard you are to hit with a weapon and also contributes toward your \statref{speed}.

%TODO: Table

\attribute{Ken}{ken}

\attref{ken} represents memory, knowledge, and education.
It's used to know and recall facts and details, from historical trivia to the intricacies of your own magical rites.
\attref{ken} is particularly important to \practitioners{brewing} and \practitioners{ritual-magic}, to remember their rituals and recipes.

%TODO: Table.

\attribute{Wit}{wit}

\attref{wit} represents reasoning, deduction, and intuition.
It's used to solve puzzles, unravel mysteries, lay intricate plans, follow philosophical arguments and perform research.
\attref{wit} is important to practitioners of more improvisational magic, such as \discref{sympathetic-magic}, \discref{projection}, and the cutting edges of many other disciplines.

%TODO: Table.

\attribute{Will}{will}

\attref{will} represents courage, dedication, and conviction.
It's used to stand your ground, resist the influence of others, remain unfazed in embarrassing situations, and push onwards in the face of adversity.
\attref{will} influences your pain threshold and is used to resist curses and mental influence, mundane or magical.
Some disciplines of magic also rely on pure force of \attref{will} to influence the world, such as \discref{willing} and the animation of golems and undead.

%TODO: Table.

\attribute{Heed}{heed}

\attref{heed} represents awareness, perception, and attention to detail.
It's used to avoid being caught unawares, to spot objects out of place or pick out details at a distance, to catch someone in a lie, or to pick up on the mood in a room.
As well as the five common senses, \attref{heed} covers the use of any other senses a witch might acquire in her magical dealings, making it particularly important to \practitioners{divination}.

%TODO: Table.

\attribute{Charm}{charm}

\attref{charm} represents eloquence, wile, and comeliness.
It's used to persuade, deceive or seduce people, to smarm your way into their good graces, and to imply things without outright saying them.
\attref{charm} is particularly important to \practitioners{headology}, as the discipline relies entirely upon convincing people of things.

%TODO: Table.

\attribute{Presence}{presence}

\attref{presence} represents force of personality, air of authority, and personal magnetism.
It's used to draw people's attention, boss them around, and make them wet themselves in terror.
Note that you don't have draw attention if you don't want to; a high \attref{presence} doesn't make it any harder to hide.
\attref{presence} is particularly important to \practitioners{headology}, as changing a person's mind so often relies upon commanding their attention.

\begin{simpletable}{rX}
	\toprule
	{\tn} & Example Task\\
	\midrule
	9 & \\
	12 & \\
	15 & \\
	18 & \\
	21 & Silencing a raucous town hall with a polite cough.\\
	\bottomrule
\end{simpletable}

\section{Skills}


%TODO: Intro fluff paragraph about the skill of a witch.

%TODO: Recap how skills affect dice rolled for Tests, and how not every Test has an applicable skill.

A witch's skills can be divided into three categories.
The first consists of {\generalskills}, pertaining to things any witch might find herself doing.
The second consists of {\disciplineskills}; a witch's skills in her particular disciplines of magic.
These skills are normally of little use to a witch who does not practice such a discipline, although they can often be used to identify, and sometimes to counteract, the effects from it.
The third consists of {\specialityskills}, pertaining to a particular craft or occupation.

The following table lists all the skills available to a witch, categorised by type.
The details of each skill, and examples of their use, are provided in the following sections.

\begin{simpletable}{XXX}
	\toprule
	{\generalskillbare} & {\disciplineskillbare} & {\specialityskillbare}\\
	\midrule
	\skillref{animals} & \skillref{brewing} & \skillref{crafting}\\
	\skillref{athletics} & \skillref{divination} & \skillref{lore}\\
	\skillref{botany} & \skillref{flying} & \skillref{performance}\\
	\skillref{deception} & \skillref{golemancy}\\
	\skillref{healing} & \skillref{necromancy}\\
	\skillref{insight} & \skillref{projection}\\
	\skillref{intimidation} & \skillref{ritual-magic}\\
	\skillref{perception} & \skillref{sympathetic-magic}\\
	\skillref{persuasion} & \skillref{willing}\\
	\skillref{socialising}\\
	\skillref{stealth}\\
	\skillref{weaponry}\\
	\bottomrule
\end{simpletable}

\subsection{Improving Skills}
\seclabel{improving-skills}

Ranks in skills may be purchased by spending XP.
The XP costs of increasing skills are provided in the following table.
A character must have the previous rank in a skill before purchasing the next rank.

\notedtable{lrr}{
	\toprule
	Type & Rank & XP Cost\\
	\midrule
	{\generalskillbare} & 1 & 15\\
	{\generalskillbare} & 2 & 25\\
	{\generalskillbare} & 3 & 35\\
	{\disciplineskillbare} & 1 & 40\tnote{*}\\
	{\disciplineskillbare} & 2 & 50\tnote{*}\\
	{\disciplineskillbare} & 3 & 60\tnote{*}\\
	{\specialityskillbare} & 1 & 10\\
	{\specialityskillbare} & 2 & 20\\
	{\specialityskillbare} & 3 & 30\\
	\bottomrule
}{
	\item[*] Reduced by feats from the governing discipline.
}

The XP cost for increasing a {\generalskill} or {\specialityskill} is fixed, while progression in {\disciplineskills} is closely tied to progression in the discipline itself.
While it is possible to learn a lot about a discipline of magic without ever practicing it, it is far easier to learn simply by setting out and using the magic.
As such, increasing a {\disciplineskill} costs less XP if a witch has already purchased feats from its governing discipline.

Many feats from a discipline require a certain rank of the skill that governs the discipline, while some do not require any ranks in the skill at all.
Increasing a {\disciplineskill} costs 5 XP less for each feat that requires exactly the rank below, to a minimum cost of 0 XP.
This applies only to feats from the governing discipline, even if feats from other disciplines require the same {\disciplineskill}.

For example, a witch buying \skillref[3]{brewing} gets a 5 XP discount for every feat she has from the \discref{brewing} discipline that lists exactly \skillref[2]{brewing} as a prerequisite.
A witch buying \skillref[1]{brewing} gets a 5 XP discount for every feat she has from the \discref{brewing} discipline, because all of the feats she has from that discipline require \skillref[0]{brewing} (i.e.\ don't require the \skillref{brewing} skill at all).

Similarly, a feat is 5 XP cheaper if you have a higher level of the discipline's governing skill than is required for the feat.
This is only ever 5 XP, regardless of how much higher your skill is than necessary.
For example, a witch with \skillref[3]{brewing} who buys a \discref{brewing} feat that requires only \skillref[2]{brewing} gets a 5 XP discount.
A witch with \skillref[1]{brewing} or higher gets a 5 XP discount when buying a \discref{brewing} feat that does not require the \skillref{brewing} skill at all.
This ensures that it does not matter whether a witch buys feats or skills first; her total XP expenditure should be the same either way.

\subsection{General Skills}
\seclabel{general-skills}

\skill{Animal Handling}{animals}

Used to understand animals and interact with them: to calm them, tame them, ride them, train them, command them, or predict how they might act.

\skill{Athletics}{athletics}

Used to run, jump, swim, climb, somersault, and generally get about the place more easily and impressively.

\skill{Botany}{botany}

Used to raise crops and herbs in a witch's garden, find them out in the forest, or identify a fishy-looking leaf.

\skill{Deception}{deception}

Used to mislead, lie, prevaricate, or filibuster, without anyone catching on that you're doing it.
Many witches make it a rule not to lie.
That doesn't mean they always need to tell the whole truth, so this can still be a useful skill for them.

\skill{Healing}{healing}

Used to bind wounds, set bones, diagnose diseases, and deliver children.
This covers first aid, extended care and even surgery.
It does not cover the use of herbs, poultices or potions; these fall under \skillref{botany} and \skillref{brewing}.
It can be used to diagnose a patient's sickness in the first place, however: an essential step in applying the correct potion.

\skill{Insight}{insight}

Used to read people, as individuals or crowds.
This can include judging people's attitude and confidence, telling when and why they're uncomfortable, picking up on tells that they're lying, or predicting whether an argument is likely to come to blows.
It can be particularly useful for guessing at people's levers and buttons when preparing to manipulate them.

\skill{Intimidation}{intimidation}

Used for making threats: anything from subtly suggesting that you know a secret somebody would rather wasn't public knowledge, to outright yelling that you'll break the bugger's knees if he doesn't sit down and shut up \emph{right now}!
This doesn't even have to involve speaking; turning half a mob into frogs can certainly discourage the rest from tangling with you.

\skill{Perception}{perception}

Used to see, hear, or smell things.
This includes noticing things that are out of place, such as hearing someone sneaking up behind you or spotting that your hat is missing from its peg.
It also covers active attempts to discern things, such as picking out details on someone at the other end of a street, eavesdropping, or identifying a faint smell.
Lastly, this is the skill used when trying to follow the trail of an animal or person.

\skill{Persuasion}{persuasion}

Used to influence or convince a person or crowd: to make them believe a particular thing or act in a particular way.
This can be through subtle suggestion and manipulation, or through reasoned, logical argument.
If you're attempting to persuade someone to act based on a falsehood, this might require both a \skillref{deception} Test to avoid being caught in the lie, and \skillref{persuasion} Test to motivate them to act.

\skill{Socialising}{socialising}

Used to befriend people, mingle with them, build rapport, and get into their good graces.
A good socialiser is everybody's best friend within a few minutes of meeting them, and might be trusted with secrets people would never otherwise give up.
Additionally, the GM may call for a \skillref{socialising} Test to determine how well you know a member of your steading or a nearby one.

\skill{Stealth}{stealth}

Used to do things without being noticed, such as sneaking up behind someone, peeking out through a bush, or lifting a guard's knife from his belt.
You can even try to blend in with a crowd (take the {\hat} off first), or rifle a man's purse while he watches your other hand.

\skill{Weaponry}{weaponry}

Used for everything from stabbing people with a concealed knife to clonking them over the head with a hefty staff, or even slugging them with a mean right hook.
Also used for throwing things, or shooting them with a bow.

\subsection{Discipline Skills}
\seclabel{discipline-skills}

\skill[brewing]{Brewing}{brewing}

Used to brew tinctures, tonics, elixirs, and other potions.
This doesn't always require a cauldron: it also covers mixing poultices and the like.
Even without any feats from the discipline, you can use this skill to brew remedies for diseases and other minor ailments.
Of course, you can also make booze.

\skill[divination]{Divination}{divination}

Used to see the past and future, and places many miles away.
It's not limited to seeing either; a diviner can eavesdrop on a conversation in the next village, or track a person better than any bloodhound.

\skill[flying]{Flying}{flying}

Used by a witch on a broomstick, whether she's settling in for a cross-country flight, showing off with a barrel roll, or pulling a stalled stick out of a deep dive.
This is also the skill used for feats of flying by a winged animal or familiar.

\skill[golemancy]{Golemancy}{golemancy}

Used to will life into inanimate creatures of clay, or other materials.
A skilled \practitioner{golemancy} can make more golems, make them smarter, and, of course, force life into increasingly substandard bodies.

\skill[necromancy]{Necromancy}{necromancy}

Used to pervert the natural order and bring the dead back to life, or at least commune with them from beyond the Veil.
Also used to send them on again, if hitting them over the head with a big stick won't suffice.

\begin{simpletable}{rX}
	\toprule
	{\tn} & Example Task\\
	\midrule
	9 & Discerning the power of the \discref{necromancy} animating a shambling corpse.\\
	12 & Identifying the purpose of a necromantic rite from the chalk circle left behind.\\
	15 & Filtering the true facts about vampires from the baseless rumours that surround them.\\
	18 & Discerning the power of the \discref{necromancy} that previously animated a no-longer-shambling corpse.\\
	21 & Performing a complex necromantic ritual using nothing but two small sticks and a fresh egg.\\
	\bottomrule
\end{simpletable}

\skill[projection]{Projection}{projection}

Used to shift your mind outside of your own head, and possibly to shunt it into other people's.
Vacating your body is a risky business, and one of the most important skills in \discref{projection} is finding your way back home.

\skill[ritual-magic]{Ritual Magic}{ritual-magic}

Used to perform rituals with magical circles.
\materialrefplural{ritual-circle} have many commonalities, even across the various disciplines, and this skill governs your familiarity with their intricacies and quirks.
Besides the discipline of \discref{ritual-magic}, you may also use this skill to quickly and accurately draw the \materialrefplural{ritual-circle} of other disciplines.

\skill[sympathetic-magic]{Sympathetic Magic}{sympathetic-magic}

Used to manipulate people or things using effigies, poppets, or other imitative talismans.
The idea of \discref{sympathetic-magic} is that one can't affect the imitation of a thing without affecting the thing itself.

\skill[willing]{Willing}{willing}

Used to force the universe to fall into line with what you know is true.
There is a real knack to convincing yourself of something well enough to make this work, and this skill governs how good you are at it.

\subsection{Speciality Skills}
\seclabel{speciality-skills}

Speciality skills behave a little differently to other skills.
A character does not gain ranks in the skills themselves, but in one of their specialities.
Ranks for each speciality are tracked independently.
For example, a witch might have \skillrefspeciality[1]{crafting}{Carpenter}, \skillrefspeciality[2]{crafting}{Cook} and \skillrefspeciality[2]{crafting}{Smith}.
Each speciality skill provides a list of recommended options, but the GM may approve others.

The GM may also approve specialities that don't quite fit into one of the categories provided by the skills, if a player can think of and requires one.

\skill{Crafting}{crafting}

Used to make things, quite broadly.
This covers the creation of most kinds of objects, although some kinds of crafting are still covered by other skills, such as \skillref{brewing}.
Available specialities include the following:

\begin{itemize}
	\item Carpenter
	\item Cook
	\item Embalmer
	\item Jeweller
	\item Mason
	\item Potter
	\item Seamstress
	\item Smith
	\item Woodcarver
	%TODO: Evaluate and maybe expand.
\end{itemize}

\skill{Lore}{lore}

Used to know and recall assorted knowledge from more academic fields, such as history, geography, and religious doctrine. %TODO: Double-check our stance on religion.
Many fields of knowledge, such as magic and \skillref{botany}, fall under their own skills; this covers those that don't.
Available specialities include the following:

\begin{itemize}
	\item Astronomy
	\item Geography
	\item History
	\item Nobility
	\item Philosophy
	\item Religion %TODO: Double-check our stance on religion.
	\item Weather
	%TODO: Evaluate and maybe expand.
\end{itemize}

\skill{Performance}{performance}

Used to entertain, amuse or impress people, or perhaps just to distract them.
Not everything that entertains people must use this skill; people can easily be entertained by a show of a magic or a sword fight, which might use another skill.
\skillref{performance} covers things done primarily for entertainment.
Available specialities include the following:

\begin{itemize}
	\item Dancer
	\item Drummer
	\item Harper
	\item Piper
	\item Puppeteer
	%TODO: Evaluate and maybe expand.
\end{itemize}
