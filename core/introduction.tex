\chapter{Introduction}
\chaplabel{introduction}

\titleemph{Coven} is a role-playing game designed upon a simple premise: the player characters are witches and the party is a coven.
Every character shares the common tools of witchcraft: a familiar, a broomstick and, most importantly, a pointed hat.
However, that is often where the similarities end.
There are many different disciplines to witchcraft, and many different approaches even within a discipline.
From meticulous ritualists to soaring broom-riders, from shy girls to terrifying matriachs, hunched over a cauldron or chatting with squirrels in the forest, a coven can be a diverse lot.

In light of this, they don't always get along.
Witches can be somewhat solitary creatures by nature, tending to their own villages, dealing with their own problems.
But they do tend to keep tabs on one another, and a good witch recognises when things are bit much to handle by herself.
When the great spirits of the land are threatened, when \emph{things} push through from other realms, or when one of their own begins to cackle: these are the times witches come together.
And these are the adventures the players have with them.

\section{The Craft}

The Craft, the Art, the Way.
Witchery, occultism, thaumaturgy.
There are many names for witchcraft.
Few things define it, however.
In truth, it is nothing but knowledge of the diverse disciplines of magic, and the skill to apply it.

Witchcraft is not like the enchantments of the faeries or the sorcery of warlocks.
It's not a power one is born with, nor one absorbed in a moment.
It is learned through years of training, grasped through decades of practice, and never truly mastered.
Anyone can pick it up, given enough patience and determination.
But few even have the inclination.

For the power of witchcraft comes with more responsbility than most.
The responsibility to care for one's neighbours, one's charges, one's village.
To see them through sickness and through strife, to see them into the world and back out of it.
The responsibility to take up arms and defend them from the horrors of the night, of other realms, even the ones they bring upon themselves.
To lay down one's own life in defence of others.
And finally, the responsibility to train a successor, that the Craft may continue to serve one's village after one dies.
Everything that goes on in a witch's realm is her responsibility, and that is too great a burdern for many to bear.

Which brings us to the topic of the Black Craft.
Witchcraft is simply knowledge, to be used how it will.
Even possession, voodoo and \discref{necromancy} are not evil acts in themselves, when turned to the purpose of good. %TODO: Link for voodoo
Evil begins when all the responsibility becomes too much for a witch.
When she wonders why she's doing so much for these people who never do anything for themselves.
When she believes that she is better than other people.
When she begins to cackle.
And so comes another responsibility of witches: to sit her down and give her a stern talking to.
Or, failing that, to show her the way out{\dots}

\section{The World}

%TODO: Assumptions of the setting.

\section{Tabletop Role Playing}

\section{Dice and Tests}

Like most tabletop roleplaying games, \titleemph{Coven} uses dice to determine the result of certain actions.
\titleemph{Coven} uses only six-sided dice (d6s), which you should be able to pilfer in abundance from a few board games.

Whenever the action of an outcome is in doubt, the GM may call for a Test by the acting character, specifying an attribute and optionally a skill with which to make the Test (attributes and skills are explained in \chapref{attributes-and-skills}).
A Test is made by rolling a number of six-sided dice determined by the character's skill and adding the highest 3 of these dice together
The relevant attribute is then added to the total, which is compared to a Target Number (TN) provided by the GM.
If the total meets or exceeds the TN, the Test has succeeded.
Otherwise, the Test has failed.
Sometimes, two characters will make directly opposed Tests.
Such a Test has no TN, and the character with the higher total succeeds.
In the event of equal totals, the situation remains as it was before the test, so far as possible.

The number of dice rolled for a Test is typically determined by a character's skill.
If a character is unskilled at a task, or there is no applicable skill for it, she rolls 3 dice (and hence keeps all of them).
If she has some relevant skill, she rolls 4.
If she is an expert in the applicable skill, she rolls 5.
If she is a veritable master of the skill, she rolls 6.
%TODO: Using numbers or names for skill ranks?

Sometimes, a character does more than simply succeed, she excels.
And sometimes, she fail catastrophically.
These are represented by critical successes and failures.
If every rolled die shows a 1 or a 2, the Test is a critical failure.
If the highest 3 dice all show 6, the Test is a critical success.
In addition to the Test automatically succeeding or failing, the GM is encouraged to apply an additional benefit or drawback to the result of the Test.
Critical failures on Tests involving dangerous magic can be especially catastrophic.

More details on Tests, including examples and prescribed tests for particular situations, can be found in \partref{rules}. %TODO: Ensure that's all there.
