\chapter{Sympathetic Magic}
\chaplabel{sympathetic-magic}

\section{Sympathetic Links \& Symbols}
\seclabel{sympathetic-links}

Central to the practice of \discref{sympathetic-magic} is the creation and manipulation of {\symbols}.
A {\symbol} is a representation of a creature or object, and by affecting the {\symbol} a witch may cause a mirroring effect upon the target.
Not every \materialref{poppet} or \materialref{effigy} is automatically a {\symbol}.
It must by magically bound to the target by a {\symlink}.

A novice witch can only maintain one {\symlink} at a time.
It's not that maintaining one is particularly arduous; once established, a {\symlink} remains in place indefinitely, even when the witch sleeps.
Rather, two {\symlinks} tend to tangle themselves up, like pieces of string left together in a drawer.
Soon enough, both are totally useless and it takes a pair of shears to separate them.
A witch can sever any {\symlink} she has established as an {\action}, or as part of establishing any new {\symlink}.

A {\symlink} by itself does nothing, but a practitioner of \discref{sympathetic-magic} soon learns to use it to transmit numerous things: sensations, physical effects and more.
A {\symlink} doesn't always transmit everything it is capable of transmitting: only what the witch who established it wants it to.
The witch can change what the link transmits at any point she chooses, regardless of proximity to the {\symbol} or the target.
However, she has no particular sense of what is being transmitted by the link, and must watch the {\symbol} or the target if she wants to know.
As such, leaving {\symbols} lying around is a slightly dangerous proposition.

\subsection{Establishing a Sympathetic Link}

Initially, establishing a {\symlink} is a rather involved process.
A novice witch can only establish {\symlinks} to creatures (people or animals), and requires an appropriate \materialref{poppet} or \materialref{effigy} as the {\symbol}.
The ritual that establishes the link requires tying the target and the {\symbol} together with a piece of thread.
It takes five minutes, and must occur in a \circleref{small}.

\section{Feats}

\feat{Taglock Tracing}{identify-taglock}{10}{
	None
}{
	You can touch a \materialref{taglock} and detect who it originates from.
	If you have met the target, you can identify them infallibly.
	
	If you have never met the target, you must make a \testtype{heed}{sympathetic-magic} Test, with higher results giving more information about the target.
	You can only get general information about the target this way, such as height, build, sex, appearance, and occupation.
	You can't get any information about their location, or even whether they are still alive.
}

\feat{Taglock Binding}{symlink-taglock}{20}{
	None
}{
	You can't always get someone co-operative (or unconscious) enough to sit still while you establish a {\symlink}.
	You've finally got the hang of using a \materialref{taglock} instead.
	
	You can establish a {\symlink} through the usual ritual by tying the {\symbol} to a \materialref{taglock} from the target, instead of the target themselves.
	However, you must use an \materialref{effigy} instead of a \materialref{poppet} as the {\symbol}, in order to make up the lost strength of the link.
}

\feat{Twin Links}{symlink-extra}{20}{
	\skillref[1]{sympathetic-magic}
}{
	You may maintain two {\symlinks} simultaneously.
}

\feat{Sympathetic Buoyancy}{sympathetic-weight}{10}{
	None
}{
	The mass of a {\symbol} affects the mass of its target: a stone or iron \materialref{poppet} will make a person heavier while a wood or cloth one will make them lighter.
	Not hugely so---no more than about \SI{20}{\percent}---but enough to make a person easily float or sink, and to aid or hinder jumping and climbing.
	%TODO: Mechanical effects on jumping, etc.
	
	This effect can be used on objects as well as creatures, making them easier or harder to lift and carry.
}

\feat{Sympathetic Sleep}{sympathetic-sleep}{10}{
	None
}{
	A {\symbol} can rest in place of its target, allowing the target to work through most of the night.
	The rest, the {\symbol} needs to be tucked into a small bed, with soft bedding, a pillow, and sheets.
	It needs to be in a quiet, dim location, and generally to be in conditions where a person could easily sleep.
	The {\symbol} cannot be used for any other \discref{sympathetic-magic} while it is resting.
	
	As long as the {\symbol} rests for at least 12 hours each day, the target can get by on only 1 hour of sleep each day without any ill effects.
	However, the target does not recover from {\damage} and {\exhaustion} as a result of this rest.
}

\feat{Sympathetic Insomnia}{sympathic-sleep-deprive}{20}{
	\skillref[1]{sympathetic-magic},
	\featref{sympathetic-sleep}
}{
	By keeping a {\symbol} awake, you can deprive its target of restful sleep.
	If the {\symbol} is subjected to loud noises, bright lights, stony bedding, or other significant discomforts while the target sleeps, the sleep will be fitful and restless.
	The sleep does not help them recover from {\damage} or {\exhaustion} (although they may still benefit from a day of rest).
	If this goes on for several nights, they may begin suffering {\exhaustion} due to sleep deprivation.
}

\feat{Sympathetic Warmth}{sympathetic-heat}{10}{
	None
}{
	The temperature of a {\symbol} affects the temperature of its target.
	Uncomfortable temperatures remain comfortable as long as the {\symbol} is at a comfortable temperature, and comfortable temperatures become uncomfortable if the {\symbol} is warmed or chilled.
	This effect cannot create dangerous temperatures---hot enough to cause heat stroke or cold enough to cause hypothermia---but can counteract them if the {\symbol} is inversely heated or cooled.
	Temperatures sufficiently extreme to cause {\damage}, such as fire or anything that would directly freeze the flesh, are outside the reach of this effect.
}

\feat{Sympathetic Malady}{sympathetic-attribute-reduce}{10}{
	None
}{
	You may afflict a target with various maladies by though a {\symlink}.
	You may reduce one of their attributes by 1 point by causing some appropriate affliction to the {\symbol}.
	For instance, you could reduce the target's \attref{grace} by binding their {\symbolpossessive} arms and legs, their \attref{heed} by blindfolding their {\symbol}, or their \attref{charm} by giving their {\symbol} some obvious disfigurement.
	A target may only be subject to one of these effects at a time, per witch who is affecting them.
}

\feat{Sympathetic Communication}{sympathetic-speak}{20}{
	\skillref[1]{sympathetic-magic}
}{
	You can send sounds along a {\symlink}, like a string telephone.
	A creature can hear sounds that originate near its {\symbol}, as long as it is conscious.
	The {\symbol} has a very short range of hearing; speaking through it essentially requires picking it up and holding it near the mouth.
}

\feat{Unbarred Sympathy}{sympathetic-ignore-barrier}{15}{
	\skillref[2]{sympathetic-magic}
}{
	Most barriers that interfere with magical effects don't break a {\symlink}, they just prevent it transmitting.
	But a finger on a string doesn't stop it from vibrating; it just restricts it.
	You can circumvent it if you know how.
	
	Barriers created by a \featref{circle-contain}, \featref{circle-exclude}, \featref{circle-contain-exclude}, or the like no longer impede transmission by your {\symlinks}.
	You still can't establish a {\symlink} that would be blocked by such a barrier, however.
}

\feat{Threading the Barrier}{sympathetic-ignore-barrier-2}{10}{
	\skillref[3]{sympathetic-magic},
	\featref{sympathetic-ignore-barrier},
	\featref{symlink-taglock}
}{
	If air can pass a magical barrier, why not a {\symlink}.
	It's just like threading a needle: it takes a bit of dexterity and your eyesight better be good, but it's hardly \emph{impossible}.
	
	You may establish a {\symlink} even through the barrier created by a \featref{circle-contain}, \featref{circle-exclude}, \featref{circle-contain-exclude}, or the like.
	You can't always do it first time, however, and the GM may require a Test if you are in a hurry.
}
