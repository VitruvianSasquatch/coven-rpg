\discipline{Sympathetic Magic}{sympathetic-magic}{Sympathist}{Sympathists}

\section{Sympathetic Links \& Symbols}
\seclabel{sympathetic-links}

Central to the practice of \discref{sympathetic-magic} is the creation and manipulation of {\symbols}.
A {\symbol} is a representation of a creature or object, and by affecting the {\symbol} a witch may cause a mirroring effect upon the target.
Not every \materialref{poppet} or \materialref{effigy} is automatically a {\symbol}.
It must by magically bound to the target by a {\symlink}.

A novice witch can only maintain one {\symlink} at a time.
It's not that maintaining one is particularly arduous; once established, a {\symlink} remains in place indefinitely, as long as the target is not resisting it.
Rather, two {\symlinks} tend to tangle themselves up, like pieces of string left together in a drawer.
Soon enough, both are totally useless and they have to be cut to separate them.

A {\symlink} by itself does nothing, but a \practitioner{sympathetic-magic} soon learns to use it to transmit numerous things: sensations, physical effects and more.
A {\symlink} doesn't always transmit everything it is capable of transmitting: only what the witch who established it wants it to.
The witch can change what the link transmits at any point she chooses, regardless of proximity to the {\symbol} or the target.
However, she has no particular sense of what is being transmitted by the link, and must watch the {\symbol} or the target if she wants to know.
As such, leaving {\symbols} lying around is a slightly dangerous proposition.

\subsection{Establishing a Sympathetic Link}

The simplest method for establishing a {\symlink} actually relies upon a trick of \discref{headology}.
The target must be \emph{expecting} the link, allowing the witch the opportunity to fasten it in place.
As such, at first, the witch can only establish {\symlinks} with people as the target, using a \materialref{poppet} or \materialref{effigy} as the {\symbol}.

Establishing the link requires an {\action}.
The target must see the {\symbol}, and the witch must declare to the target that she is binding them together.
Many witches adopt a standard incantation for this, often some piece of mumbo jumbo that suits the mystique they wish to cultivate.
The important thing is that the target understands the intent---that they are \emph{convinced} by it is not so important as in ``true'' \discref{headology}.

The target's expectation provides a hook that the witch may fasten the {\symlink} to.
If the target welcomes the {\symlink}, this is easy---it is established automatically and remains in place indefinitely.
Otherwise, establishing the link requires a \testtype{wit}{sympathetic-magic} Test {\opposed} by the target's \testtype{will}{sympathetic-magic}.

\subsection{Severing a Sympathetic Link}
\seclabel{break-sympathetic-link}

A witch can sever any {\symlink} she has established as an {\action}, or as part of establishing any new {\symlink}.
Additionally, a {\symlink} is severed if the {\symbol} or target are destroyed, or die.

Otherwise, a {\symlink} to an object, a willing creature, or a creature who is unaware they are the target of a {\symlink} at all, will persist indefinitely.
However, a {\symlink} to a creature that knows it is the target of a link, and does not wish to be, will be dislodged over time.
It automatically breaks after a minute, but can be broken sooner if it is {\stressed}.
This applies even if the creature previously accepted the link, but now wants rid of it.

Some uses of a {\symlink} will cause it considerable {\stress}, giving an unwilling creature another chance to dislodge the link.
In this case, repeat the Test used to establish a link---your \testtype{wit}{sympathetic-magic} {\opposed} by the target's \testtype{will}{sympathetic-magic}.
If the target wins the Test, the {\symlink} is broken.
Actions that {\stress} a link will say so in their relevant feats.

\section{Feats}

\feat{Stable Sympathy}{symlink-stable}{20}{
	\skillref[1]{sympathetic-magic}
}{
	An unwilling target will soon throw off a {\symlink}, but you've learned to stabilise your links against this, leaving them fastened strong in the face of adversity.
	However, this requires some preparation.
	
	By using an \materialref{effigy} in the likeness of the target as the {\symbol}, the {\symlink} does not expire over time, even when resisted.
	However, this does not allow it to resist {\stress}.
	This still requires the usual Test to establish the link in the first place.
}

\feat{Taglock Binding}{symlink-taglock}{20}{
	\skillref[1]{sympathetic-magic}
}{
	Normally, the hook to fasten a {\symlink} in place is provided by the target's \emph{expectation} of a link.
	This is the simplest and strongest way, but not the only one.
	
	You can establish {\symlinks} to creatures, using a \materialref{taglock}, and a \materialref{poppet} or \materialref{effigy} as the {\symbol}.
	Establishing the {\symlink} uses an {\action}, while touching the \materialref{taglock} and the {\symbol}.
	However, {\symlinks} fastened in this way are weaker.
	Anything that would {\stress} the link---or destroy it, as with \featref{sympathetic-damage}---simply snaps the {\symlink} without taking effect.
	Remember, however, that you can always choose not to try and transmit anything that would {\stress}, and hence break, the link.
}

\feat{Link Reversal}{symlink-reverse}{10}{
	None
}{
	You may use your own {\symlinks} in both directions.
	Anything that you could normally transmit from {\symbol} to target, you can choose to instead, or additionally, transmit from target to {\symbol}.
	
	Note that \emph{many} \discref{sympathetic-magic} effects will only affect creatures, and not objects.
	With the standard variety of {\symlink}, the reversed link is targeting an object, and many effects will be useless.
}

\feat{Twin Links}{symlink-extra}{20}{
	\skillref[1]{sympathetic-magic}
}{
	You may maintain two {\symlinks} simultaneously.
}

\feat{Sympathetic Jerk}{sympathetic-puppet}{15}{
	None
}{
	An expert \practitioner{sympathetic-magic} can make their target dance on the puppet strings of their {\symlink}.
	You aren't there yet, but you've taken the first step.
	
	You cannot control your target's movements, but you---or someone else holding the {\symbol}---can \emph{disrupt} them by jerking the {\symbolpossessive} limb the wrong way at the opportune time.
	If the target is just walking and talking normally, this doesn't do more than faintly disturb them.
	But if they are performing something highly physical or precise---running, jumping, aiming a weapon, or sewing, for example---it can severely disrupt them.
	Jerking the correct limb at the correct time requires knowing what the target is doing, or at least being able to take a very good guess.
	Normally, this means being able to see them.
	
	Typically, you can use this by taking the \actionref{ready} {\action} in order to disrupt the target's next {\action}, while holding their {\symbol}.
	Common disruptions include making them miss an \actionref{attack}, or making them trip and fall prone when jumping or taking the \actionref{dash} {\action}.
	The GM ultimately decides the result of any disruption.
	Disruptions like those listed above do not require a Test, but if the outcome is in doubt, the GM may call for an {\opposedtest}.
	This typically uses \testtype{wit}{sympathetic-magic} for the witch, and might use something like \testtype{grace}{athletics} or \testtype{grace}{weaponry} for the target.
}

\feat{Sympathetic Puppet}{sympathetic-puppet-2}{25}{
	\skillref[1]{sympathetic-magic},
	\featref{sympathetic-puppet}
}{
	You can control someone's actions through a {\symlink}.
	Only intermittently, and not precisely, but that doesn't make it much less terrifying.
	
	As an {\action}, someone can puppet a target by manipulating its linked {\symbolpossessive} limbs.
	The manipulator takes a physical {\action} on behalf of the target, which may be moving up to its \statref{speed} using the \actionref{dash} {\action}.
	This also deprives the target of their {\action} on their next {\turn}---unless that {\action} would be purely non-physical---although they may still make their usual movement.
	
	Using this {\stresses} the {\symlink}.
	
	Puppetry is quite difficult to do precisely.
	You can control limbs, and you can even open and close the hands and jaw, if the {\symbol} has the appropriate anatomy to manipulate.
	But speaking is impossible, and any work with the fingers requires you to manipulate the {\symbolpossessive} fingers with the same precision---a difficult proposition using your own bulky fingers.
	
	The manipulator suffers a \negative{6} penalty to any \emph{physical} Tests they must make on the target's behalf.
	These Tests typically use \attref{grace}, to finely manipulate the {\symbol}, and whichever skill would be used for performing the {\action} normally.
	However, \skillrefspeciality{performance}{Puppeteer} can be used in place of the normal skill.
}

\feat{Sympathetic Destruction}{sympathetic-damage}{20}{
	None
}{
	When a {\symbol} is destroyed, you can send its death throes lashing along the {\symlink}, tearing at its target.
	Roll a {\damagetest} against the target, using \skillref{sympathetic-magic} with no attribute.
	This works against objects, as well as creatures.
	
	Tearing a {\symbol} apart typically requires an {\action}, though you might find a faster way to destroy it.
	The destruction of the {\symbol} obviously terminates the {\symlink}.
}

\feat{Sympathetic Stabbing}{sympathetic-damage-2}{15}{
	\skillref[1]{sympathetic-magic},
	\featref{sympathetic-damage}
}{
	You no longer need to destroy a {\symbol} outright to wound the target.
	When a {\symbol} is significantly damaged in some way---sticking a pin in it is traditional---you may roll a {\damagetest} against the target, using \skillref{sympathetic-magic} with no attribute.
	This works against objects, as well as creatures.
	Using this effect {\stresses} the {\symlink}.
	
	Attacking a {\symbol} to activate this should typically require an {\action}, though you might find a faster way to damage it.
}

\feat{Sympathetic Surgery}{sympathetic-healing}{10}{
	\skillref[1]{sympathetic-magic},
	\featref{symlink-reverse},
	\featref{sympathetic-damage-2}
}{
	Using \featref{symlink-reverse}, you can cause a target's wounds to appear on their {\symbol}.
	Now, you can treat the {\symbolpossessive} wounds in order to heal the target.
	
	You can treat the injuries of a target by treating the injuries of their {\symbol}.
	This uses your choice of \skillref{healing}, or a \skillref{crafting} speciality appropriate to the construction of the {\symbol} you are using.
	Otherwise, it is as though you were treating the target directly.
	However, you can only treat wounds that the target has sustained since the {\symlink} was established, that were transmitted to the {\symbol}.
	
	You can only perform physical treatment this way; medicines cannot be transmitted.
	Any malpractice is treated as \featref{sympathetic-damage-2}, which {\stresses} the {\symlink}.
	
	The target does not need to be a creature to use this; you can also repair an object this way.
	This always uses \skillref{crafting}, not \skillref{healing}.
}

\feat{Sympathetic Re-Wounding}{sympathetic-healing-2}{10}{
	\skillref[1]{sympathetic-magic},
	\featref{sympathetic-healing}
}{
	You can cause a {\symbol} to develop wounds that its target sustained before the {\symlink} was established.
	This allows you to treat such wounds using \featref{sympathetic-healing}.
	
	You may only cause wounds to develop this way upon an object, not a creature.
}

\feat{Sympathetic Buoyancy}{sympathetic-weight}{10}{
	None
}{
	The mass of a {\symbol} affects the mass of its target: a stone or iron \materialref{poppet} will make a person heavier while a wood or cloth one will make them lighter.
	Not hugely so---no more than about \SI{25}{\percent}---but enough to make a person easily float or sink, and to aid or hinder jumping and climbing.
	%TODO: Mechanical effects on jumping, etc.
	
	This effect can be used on objects as well as creatures, making them easier or harder to lift and carry.
}

\feat{Sympathetic Sleep}{sympathetic-sleep}{10}{
	None
}{
	A {\symbol} can rest in place of its target, allowing the target to work through most of the night.
	The rest, the {\symbol} needs to be tucked into a small bed, with soft bedding, a pillow, and sheets.
	It needs to be in a quiet, dim location, and generally to be in conditions where a person could easily sleep.
	The {\symbol} cannot be used for any other \discref{sympathetic-magic} while it is resting.
	
	As long as the {\symbol} rests for at least 8 hours each day, the target can get by on only 1 hour of sleep each day without any ill effects.
	However, the target does not recover from {\damage} and {\exhaustion} as a result of this rest.
}

\feat{Sympathetic Insomnia}{sympathetic-sleep-deprive}{15}{
	\skillref[1]{sympathetic-magic},
	\featref{sympathetic-sleep},
	\featref{symlink-stable}
}{
	By keeping a {\symbol} awake, you can deprive its target of restful sleep.
	If the {\symbol} is subjected to loud noises, bright lights, stony bedding, or other significant discomforts while the target sleeps, the sleep will be fitful and restless.
	The sleep does not help them recover from {\damage} or {\exhaustion} (although they may still benefit from a day of rest).
	If this goes on for several nights, they may begin suffering {\exhaustion} due to sleep deprivation.
}

\feat{Sympathetic Narcolepsy}{sympathetic-sleep-cause}{15}{
	\skillref[1]{sympathetic-magic},
	\featref{sympathetic-sleep},
	\featref{symlink-stable}
}{
	\featref{sympathetic-sleep} lets a {\symbol} sleep instead of the target.
	You've reversed this, and may instead let the {\symbol} send the target to sleep.
	
	If you tuck a {\symbol} in, as you would for \featref{sympathetic-sleep}, then you may cause it to bring on tiredness in the target.
	This does not kick in for a minute, while the {\symbol} falls asleep.
	After this minute, make a \testtype{wit}{sympathetic-magic} {\opposed} by the target's \attref{will} Test.
	If you succeed, the target falls into a deep sleep.
	They cannot be roused for 8 hours (so long as the {\symbol} continues to sleep), but benefit as though they were sleeping naturally.
	
	Succeed or fail, this will not work on the same target again for another 24 hours.
	They've either slept off the tiredness, or fought through it.
}

\feat{Sympathetic Warmth}{sympathetic-heat}{10}{
	None
}{
	The temperature of a {\symbol} affects the temperature of its target.
	Uncomfortable temperatures remain comfortable as long as the {\symbol} is at a comfortable temperature, and comfortable temperatures become uncomfortable if the {\symbol} is warmed or chilled.
	This effect cannot create dangerous temperatures---hot enough to cause heat stroke or cold enough to cause hypothermia---but can counteract them if the {\symbol} is inversely heated or cooled.
	Temperatures sufficiently extreme to cause {\damage}, such as fire or anything that would directly freeze the flesh, are outside the reach of this effect.
}

\feat{Sympathetic Combustion}{sympathetic-fire}{15}{
	\skillref[1]{sympathetic-magic},
	\featref{sympathetic-damage},
	\featref{sympathetic-heat}
}{
	When you burn someone in effigy, they really burn.
	If a {\symbol} is destroyed by fire, and you use \featref{sympathetic-damage}, the target also catches fire.
	A person ignited this way begins at \dice{3} {\fire}.
}

\feat{Sympathetic Malady}{sympathetic-attribute-reduce}{10}{
	None
}{
	You may afflict a target with various maladies by though a {\symlink}.
	You may reduce one of their attributes by 1 point by causing some appropriate affliction to the {\symbol}.
	For instance, you could reduce the target's \attref{grace} by binding their {\symbolpossessive} arms and legs, their \attref{heed} by blindfolding their {\symbol}, or their \attref{charm} by giving their {\symbol} some obvious disfigurement.
	A target may only be subject to one of these effects at a time, per witch who is affecting them.
}

\feat{Sympathetic Communication}{sympathetic-speak}{20}{
	\skillref[1]{sympathetic-magic}
}{
	You can send sounds along a {\symlink}, like a string telephone.
	A creature can hear sounds that originate near its {\symbol}, as long as it is conscious and not deafened.
	It can avoid this by plugging its ears, although this obviously leaves it deaf to its own surroundings as well.
	The {\symbol} has a very short range of hearing; speaking through it essentially requires picking it up and holding it near the mouth.
}

\feat{Sympathetic Pestering}{sympathetic-speak-2}{15}{
	\skillref[1]{sympathetic-magic},
	\featref{sympathetic-speak},
	\featref{sympathetic-sleep-deprive}
}{
	When sending sounds along a {\symlink} using \featref{sympathetic-speak}, you may send them directly into the target's mind, bypassing its ears.
	The target hears them even if it is deaf, or has its ears plugged.
	You may even be able to wake the target up with loud enough sounds, if it is asleep.
}

\feat{Sympathetic Ventriloquism}{sympathetic-puppet-speak}{10}{
	\skillref[2]{sympathetic-magic},
	\featref{sympathetic-puppet-2},
	\featref{sympathetic-speak-2}
}{
	Puppeteering the vocal cords requires a lot more precision than swinging the limbs around.
	However, it doesn't take as much force---using this effect does not {\stress} the {\symlink}.
	
	While a {\symbolpossessive} jaw is flapped around, the target will speak anything said into the {\symbolpossessive} ear.
	This obviously requires that the {\symbol} possesses an appropriate jaw.
	The target speaks in its own voice, so an animal cannot be made to speak particularly well.
	
	This does not prevent the target from talking whenever this is not being actively used, so you have to force the target to talk constantly if you want to prevent it getting a word in edgeways.
}

\feat{Sympathetic Knot}{sympathetic-knot}{15}{
	\skillref[1]{sympathetic-magic},
	\featref{symlink-extra}
}{
	Normally when {\symlinks} get tangled, it renders both useless.
	However, if you knot them together intentionally, carefully, you can take advantage of it.
	
	You can knot together two or more {\symlinks} of the same kind---to creatures or to objects---as an {\action}.
	This requires that you are touching at least one end of each {\symlink} to be involved in the knot.
	For example, knotting together two {\symlinks} from \materialrefplural{poppet} to people requires you to be touching both \materialrefplural{poppet}, both people, or the \materialref{poppet} from one link and the person from the other.
	
	You can also undo a knot as an {\action}, but again you must be touching at least one end of every {\symlink} in the knot---you can only undo knots in their entirety, and not remove just one {\symlink}.
	Similarly, severing any {\symlink} in the knot severs all of them.
	You can only knot or unknot your own {\symlinks}.
	
	While two {\symlinks} are knotted, anything transmitted by any {\symbol} in the knot affects every target in the knot.
	You may still control what each {\symbol} transmits, but it always transmits to all targets.
}

\feat{Unbarred Sympathy}{sympathetic-ignore-barrier}{15}{
	\skillref[2]{sympathetic-magic}
}{
	Most barriers that interfere with magical effects don't break a {\symlink}, they just prevent it transmitting.
	But a finger on a string doesn't stop it from vibrating; it just restricts it.
	You can circumvent it if you know how.
	
	Barriers created by a \featref{circle-contain}, \featref{circle-exclude}, \featref{circle-contain-exclude}, or the like no longer impede transmission by your {\symlinks}.
	You still can't establish a {\symlink} that would be blocked by such a barrier, however.
}

\feat{Threading the Barrier}{sympathetic-ignore-barrier-2}{10}{
	\skillref[3]{sympathetic-magic},
	\featref{sympathetic-ignore-barrier}
}{
	If air can pass a magical barrier, why not a {\symlink}.
	It's just like threading a needle: it takes a bit of dexterity and your eyesight better be good, but it's hardly \emph{impossible}.
	
	You may establish a {\symlink} even through the barrier created by a \featref{circle-contain}, \featref{circle-exclude}, \featref{circle-contain-exclude}, or the like.
}
