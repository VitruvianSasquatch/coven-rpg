\discipline{Flying}{flying}{Pilot}{Pilots}

\dropcap{A} broom is primarily a witch's method of getting from A to B: from village to village, out into a distant forest, or all the way up the city.
It's not the easiest mode of transport, and it can be quite terrifying at first, but a witch can pick up the rudiments in a week or two's practice.
This is as far as most witches go.
But some, with enough practice, skill and flamboyance, can turn it into a real art.

\section{Laws of Aviation}

For an unpracticed witch, there are a lot limitations to broomstick flying.
After all, she is sitting on a thin stick floating hundreds of metres in the air.
First and foremost, it is easiest to balance on a broomstick if one sits side-saddle, and this is all an unpracticed witch is capable of.
This does, however, make it a lot harder to turn, and to fly at high speeds.
Barrel rolls are right out.

Additionally, the witch must keep at least one hand on the broom at all times, to prevent it spinning out of a control.
Manoeuvre is even easier with both hands, and the GM is encouraged to make \skillref{flying} Tests more difficult for a witch using only one hand.

\subsection{Taking Off}

Getting the broomstick off the ground in the first place is no easy task.
A broom needs a running start before the magic will catch, and even then it isn't consistent.
The witch must hold the broomstick level as she runs along the ground, then jump on it quickly when it starts.

Attempting to start a broom requires an {\action} and a 15 metre run-up.
A character must move this distance in a straight line on one {\turn}, and may \actionref{dash} as part of the broom-starting {\action} if necessary.
They must also succeed on a {\tn} 12 \testtype{grace}{flying} Test or the broom fails to start.
As normal, the Test is not required if there is no time pressure, as the witch may run up and down as many times as necessary until the broom starts.

The Test to start a broom may be more difficult in adverse conditions; the following table provides suggestions for the {\tn} of such Tests.
It is possible to achieve the necessary run-up through falling, although such a thing is \emph{very} difficult and the consequences for failure are obviously drastic.

\begin{simpletable}{rX}
	\toprule
	\capital{\tn} & Conditions\\
	\midrule
	12 & Nominal.\\
	15 & Blowing a gale.\\
	18 & In a bog.\\
	21 & While falling.\\
	\bottomrule
\end{simpletable}

\subsection{Climbing and Stalling}

There is a limit to the angle at which a broomstick can climb.
A novice witch can climb about 1 metre for every 5 metres of ground she covers, and can climb at this angle indefinitely.
Steeper climbs can be achieved for brief periods, but the broomstick loses speed, and eventually stalls.
As such, pulling off a steeper climb typically requires a Test.

A witch who tries and fails a steep climb soon finds her broomstick stalled.
She has until she hits the ground to point the broomstick downwards, restart it, and then pull out of the dive.
This requires a very difficult check, although more altitude will afford her more time, and make it slightly easier.

\subsection{Cruising and Turning}

A witch sitting sidesaddle doesn't have a particularly good grip on her broom, and this limits the speed she can go without the oncoming air ripping her clean off.
%TODO

A broomstick must also maintain a minimum speed in order to maintain lift: this speed is about \SI{10}{\kilo\metre\per\hour} or about 30 metres per turn.
Dropping below this speed for more than a moment causes the broomstick to stall.

Another consequence of sitting sidesaddle is a poor ability to steer the broom, leading to a turning circle several hundred metres in diameter.

\subsection{Landing}

There are two main techniques employed to land a broom.
In the first, the witch hits the ground running and performs a moving dismount.
This requires a flat stretch of ground to land on, but allows her to maintain momentum---perhaps important in a chase.
The second, slightly trickier technique is to bring the broom to a gentle stall just above the ground.
This allows an experienced witch to land with pinpoint accuracy, or an inexperienced witch to fall unceremoniously on her behind.

Unhurried, and with no care to accuracy, a witch can achieve either form of landing.
A witch trying to land with limited space available, or on rough terrain, may require a Test.
The GM is encouraged to adjust the {\tn} of the Test depending on how the landing is performed; a stall landing is generally trickier, but less dependent upon the terrain.

\subsection{Lift}

A broomstick can carry one witch, and about as much equipment as she could easily walk around with on the ground.
It can also carry a familiar, as long as it's of reasonable size.
A cat is fine, a beagle is borderline, a wolfhound is right out.

A little bit of extra weight, or something inconveniently large, makes the broomstick unwieldy.
Tests to take off or perform manoeuvres are more difficult, and the broom's maximum speed may be reduced.
A lot of extra weight, such as a passenger, makes proper flight impossible.
The broomstick cannot take off, cannot climb, and cannot even remain in level flight.
It might still be possible to bring it down and land safely, with an appropriate Test.



\section{Is This My Broom?}
\seclabel{broom-ownership}

Once trained, a broom can be ridden by any witch who lays a hand on it.
It has no concept of ownership.
Some feats, however, such as \featref{broom-summon}, allow a witch to influence a broom that she is not riding.
This calls into question which brooms she can influence---namely, which brooms are \emph{hers}?

Ownership is in the minds of witches, not brooms.
It isn't a hard and fast concept, but there are a few general guidelines.
The GM has the final say over whether a witch has sufficient claim to ownership in each instance.

In some cases, ownership if obvious.
If it's been your broom for years, and you ride it pretty often, it'll probably always be yours unless you give it away.
If another witch who owned it has given it to you, its yours now.
If you're the only person who's ever ridden it, even if you've only touched it once, it's probably still yours.

In other cases, it's definitely not yours.
Your enemy's broom isn't yours, even if you've just nicked it.
Nor is your best friend's, even if she's lent it to you a few times over the years.

Ownership can also change moment to moment.
Your best friend's broom might be yours if she's lent it to you \emph{right now}, up until she asks for it back.
You might even have a case if you've ``found'' a broom, and you've been joyriding on it for an hour or two.

If in doubt, the GM can call for a Test to affect a broom that might not be yours.
This relies on a mixture of skill in influencing brooms, and conviction that it \emph{is} yours, so it uses \testtype{will}{flying}.

Affecting a broom that you've never flown always requires a Test, but might be possible as long as it certainly isn't anyone else's broom.
This is more relevant with \featref{broom-untrained}, \featref{broom-improvise}, and \featref{broom-improvise-2}.

\section{Feats}

\feat{Ride Astride}{broom-astride}{15}{
	\noprereq
}{
	By sitting astride the broomstick, instead of side-saddle, you can go faster and turn more sharply without falling off.
	It's harder to balance, but you've got the hang of it now.
	
	%TODO: Speed
	
	Gripping the broom with your legs also allows you to turn much more sharply.
	Your turning radius is reduced to a dozen metres.
}

\feat{Bristlebrake Turn}{broom-turning}{20}{
	\skillref[1]{flying},
	\featref{broom-astride}
}{
	By flicking the back of the broom around, you can reduce your turning radius to just 1 metre.
	Turning more than \SI{90}{\degree} by this method require a difficult Test, and stalls the broom on a failure.
}

\feat{Angle of Attack}{broom-climb}{10}{
	\featref{broom-astride}
}{
	You can squeeze more thrust out of a broomstick, allowing you to climb at a steeper angle.
	You can climb one metre for every metre of ground you cover.
}

\feat{Hover}{broom-hover}{15}{
	\skillref[1]{flying},
	\featref{broom-turning},
	\featref{broom-climb}
}{
	By tipping the broomstick right back and balancing carefully, you can hover, staying roughly still in mid-air.
	Staying actually still is quite difficult, and you tend to drift around a fair bit.
	You can stay in roughly the same place, but staying still enough to, for example, reach out and touch a particular thing requires a Test.
}

\feat{Vertical Ascent}{broom-climb-2}{15}{
	\skillref[2]{flying},
	\featref{broom-hover}
}{
	You can ascend vertically on a broomstick, soaring straight upwards, indefinitely.
}

\feat{Chocks Away}{broom-start}{10}{
	\noprereq
}{
	There's a simple knack to starting a broom, and you've got it down pat now.
	You don't need a Test to start a broom under normal conditions (although you still need the run-up), and the {\tn} of any Test to start the broom under difficult conditions is reduced by 3.
}

\feat{Jumpstart}{broom-start-2}{20}{
	\skillref[1]{flying},
	\featref{broom-start}
}{
	You can perform a vertical take-off, by jumping.
	You do not need a run-up to start a broom, though it still requires an {\action}.
	You still don't need a Test to do this under normal conditions, but you might require a Test if anything makes it difficult for you to jump.
	
	Without \featref{broom-climb-2} you still cannot climb vertically, so you need to level off quickly to avoid stalling immediately.
}

\feat{Crash Starter}{broom-start-stall}{15}{
	\skillref[1]{flying},
	\featref{broom-astride},
	\featref{broom-start}
}{
	You've had rather a lot of practice restarting a broom in mid-air and, remarkably, you've survived to learn from it.
	Under nominal conditions, you can recover from a stall at an altitude of at least 500 metres without a Test.
	You still need a Test under adverse conditions or if stalling at a lower altitude, but your experience makes the Test considerably easier.
}

\feat{Passengers \& Cargo}{broom-weight}{15}{
	\noprereq
}{
	You can get enough lift out of a broomstick to carry a passenger.
	Unless they are a skilled \practitioner{flying} in their own right, they need to hold on to you while in flight.
	
	If you are not carrying a passenger, you can use the additional lift to carry cargo, such as saddlebags; no more than the weight of a person.
}

\feat{No Hands!}{broom-no-hands}{15}{
	\skillref[1]{flying},
	\featref{broom-astride}
}{
	You can fly a broomstick with just your legs, leaving both hands free.
	You ignore any penalties for flying with just one hand, but suffer those penalties when flying with no hands instead.
}

\feat{Autopilot}{broom-sleep}{15}{
	\skillref[1]{flying},
	\featref{broom-no-hands},
	\featref{projection-remain}
}{
	You could do this in your sleep.
	
	If you are flying a broom when you willingly enter the {\mentalrealm} or fall asleep, you can leave yourself sitting on the broom and flying forwards.
	The broom continues in a straight line, at a constant speed.
	However, this is not particularly stable.
	Strong winds or turbulent conditions may knock you off, and doing this at more than about half your top speed is incredibly dangerous.
	%TODO: Check this once top speeds are set.
	
	\capital{\featref{projection-remain-dodge}} allows you to travel at your full speed, and leaves you just as stable on the broom as when you are conscious.
	\capital{\featref{projection-remain-navigate}} allows you to steer along a predefined path, and even to land and continue on foot.
}

\feat{Broom Whisperer}{broom-untrained}{15}{
	\skillref[1]{flying}
}{
	You've got the knack of flying for yourself now, and don't need a broom to be trained to fly it.
	You can even train a broom this way, although without one of its own to learn from the process takes about 24 hours of flight time.
}

\feat{Tool Rider}{broom-improvise}{10}{
	\noprereq
}{
	Brooms are certainly very traditional, but really, any old tool will do.
	You can ride any long-handled, man-made, properly crafted tool, such as a rake, spade, scythe or wood-axe.
	It still needs to be trained as usual, unless you also have \featref{broom-untrained}.
}

\feat{Nearly a Broom}{broom-improvise-2}{15}{
	\skillref[2]{flying},
	\featref{broom-improvise},
	\featref{broom-untrained}
}{
	It's quite surprising quite what you can convince to be a broom, if you put your mind to it.
	You can ride just about any appropriately-sized piece of wood, as long as it's obvious which end is the back.
	Just a bit of a fork at one end of the stick will do, or you could just lash a couple of sticks on quickly.
}

\feat{Summon Broom}{broom-summon}{15}{
	\noprereq
}{
	Flying a broom that you're \emph{not riding} is quite difficult.
	But you can manage something rudimentary.
	Namely, you can summon it.
	
	As long as you can see your broom (see the section \secref{broom-ownership}), and it isn't being ridden by anybody, you can summon it as an {\action}.
	It flies into your outstretched hand.
	It flies in a straight line, so it can be obstructed by intervening obstacles.
	It cannot carry any significant weight---people or cargo---but you could tie messages to it.
	
	If you have \featref{broom-untrained}, \featref{broom-improvise}, and/or \featref{broom-improvise-2}, you may summon any ``broom'' that you could ride with those feats, as long as it is still \emph{yours}.
}

\feat{Moving Mount}{broom-summon-start}{15}{
	\skillref[1]{flying},
	\featref{broom-summon},
	\featref{broom-start-2}
}{
	You've perfected the art of leaping onto a moving broom.
	You may take off on a broom in the same {\action} which you use to summon it.
}

\feat{Swerving Summons}{broom-summon-2}{10}{
	\skillref[1]{flying},
	\featref{broom-summon}
}{
	When you summon a broom, you can steer it.
	This allows you to navigate it around a few obstacles on its way to you, choosing the route it takes.
	It must still take a fairly direct path to you; you can't send it out on a sweeping curve away from you.
}

\feat{Distant Summons}{broom-summon-3}{25}{
	\skillref[2]{flying},
	\featref{broom-summon-2}
}{
	You may summon your broom even if you cannot see it.
	You must know which broom you are summoning, and it must still be yours, but you don't need to know where it is.
	
	You can use \featref{broom-summon-2} once it comes within sight, but until then it tries to steer itself.
	It is not particularly smart, and can get stuck fairly easily, but will usually make it unless it has been trapped in some fashion.
	It can push open cupboard doors and the like, but obviously can't operate door handles.
	
	If the broom is outside the usual range of sight---a few hundred metres---it will not arrive on the same {\turn} you summon it.
	A broom many miles away might take minutes, or longer, to arrive.
	If you have \featref{broom-summon-start}, you may take off on the broom immediately, whenever it arrives.
}

\feat{Homeward Bound}{broom-send}{10}{
	\skillref[2]{flying},
	\featref{broom-summon-3}
}{
	Just as you can summon your broom to yourself, you can send it away.
	This requires an {\action}, while you are touching the broom.
	Just like \featref{broom-summon}, this must be \emph{your} broom, and it cannot be carrying any significant weight.
	
	The broom immediately sets off flying home.
	The broom's home is wherever it is most commonly kept---often in your {\cottage}, or a shed outside it.
	It steers itself when out of sight, following the same rules as \featref{broom-summon-3}.
	It has the precision to hang itself on a hook, or fly into a cupboard (if it can get it open).
}

\feat{Summon Cargo}{broom-summon-cargo}{10}{
	\featref{broom-summon},
	\featref{broom-weight}
}{
	When you summon your broom, it can carry passengers or cargo.
	When you use \featref{broom-summon} or \featref{broom-send}, your broom benefits from \featref{broom-weight}.
	Note that this only allows it to carry one person, not two---you and a passenger---like when you are riding it.
	
	Cargo must be securely fastened to the broom.
	Likewise, a broom with a passenger cannot exceed the speed at which that passenger could normally hold onto a broom, or they will lose their grip.
	%TODO: Note what this speed is for people without Ride Astride.
	You may limit your broom to this speed if you know, or expect, that it has a passenger.
	Or you may cause the broom to exceed that speed, if the passenger is unwanted.
	%TODO: Note that you cannot cause it to exceed this speed if the passenger has whatever Mach Speed feat ends up in the game.
}

\feat{Reclaim Broom}{broom-summon-ridden}{10}{
	\featref{broom-summon}
}{
	Firmly asserting your ownership, you may summon your broom even if someone is currently riding it.
	It still has to be \emph{your} broom, so this is normally only useful if another witch has stolen it.
}
