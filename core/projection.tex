\discipline{Projection}{projection}{Projector}{Projectors}

\dropcapdiscref{projection} is, broadly, a witch's ability to have her mind leave her own body.
Once freed from its earthly shackles, her mind can wander freely, unimpeded by walls, oceans, or even vast distances.
She can sense the minds of others, finding them, following them, listening to them, and reading them.
She can even displace them from their bodies, taking them over.

Some witches, always eager to put a mystical spin on everything, refer to the process as Astral Projection.
Most just stick to \discref{projection}.
The experts, however, often forgo even this phrase.
Many forget what it was ever like to be trapped within their own body, and treat \discref{projection} as a natural and integral part of their lives.
It isn't uncommon to hear a witch say ``I'm just popping out,'' a moment before her body slumps, catatonic, to the floor.

Expert \practitioners{projection} often treat their body like a favoured pair of shoes, to be worn and discarded as needed.
Many even prefer to go barefoot, and won't think twice about borrowing someone else's shoes in an emergency.
They make a complete separation between a person's mind and body, treating the natural coupling of the two as a mere convenience.
This lends such a witch a great flexibility, but can be terrifying to those less accustomed to the practice.

\capital{\discref{projection}} is a strictly mental discipline.
It features none of the circles and gestures of \discref{ritual-magic}, nor even the physical cues that often accompany \discref{willing}.
In fact, with her mind gone, an unskilled witch leaves her body totally catatonic.

But as much as it fails to engage the body, \discref{projection} engages almost every part of the mind.
A witch uses her \attref{heed} to sense the minds of others; her \attref{wit} to reach them, to probe them, to find the cracks and leverage them; and her \attref{will} to ram another mind, to shunt it from its body and replace it with her own.
Lastly, her \attref{wit} is once again important if she grows lost, and needs to navigate back to her own body.

\section{The Mental Realm}
\seclabel{the-mental-realm}

The {\mentalrealm} is not truly a different realm, so much as it is a different way of experiencing our own one.
\capital{\discref{projection}} is the ability to cast one's mind out into this realm, and to operate within it.

The {\mentalrealm} is not experienced with any of the normal senses, tied as these senses are to organs such as the eyes, ears and skin.
But if one had to describe it, it might go something like this:

Image a vast, grey plain.
It stretches away forever in all directions.
\emph{All} directions, for there is no ground, no sky.
You do not fall, just float.
There is no up, no down, no north, south, east or west.
Not unless you brought them with you.

And you had better hope you did, because there are no landmarks.
Everything is pervaded by a cold, grey fog.
It is thick.
So thick that you would be wading through it, if you still had legs.
You'd hardly see your own hand in front of your face, if you had a hand, or a face.

Only one thing shines through the fog.
It is dim, so dim, but it is close.
In fact, it seems to be inside of you.
It \emph{is} you---part of you.
The `you' that you left behind.
There isn't much there now.
Your body, catatonic, rigid.
The barest traces of thought, keeping you breathing, keeping your heart beating.

As you adjust to the gloom, you strain the eyes you do not have and peer deep into the fog.
The faintest hints of light shine through.
They move, dance, and play in the distance.
Some shine brighter, some softer.
They are the minds of others, still living in the realm you have left behind.

You could reach out and touch them, if you wanted.
Just one quick step over there, pushing your way into the fog.
But you can still feel the light beside, inside of you.
It is so dim.
If you stepped away, would you be able to find it again?
Or would you be lost in the fog forever?

\subsection{Using the Mental Realm}
\seclabel{using-the-mental-realm}

A witch may enter the {\mentalrealm} by taking a \featref{projection-start}.
Her body is left catatonic and helpless, although still alive and breathing, while her mind departs.
She enters the {\mentalrealm} with her mind safely beside her body, able to return to it at any point she chooses.

Even a novice entering the {\mentalrealm} can sense nearby minds.
The fog that fills the realm prevents this sense from reaching far.
Minds in the same room can be sensed automatically, while a \testtype{heed}{projection} might extend the sense to fill a house or more.
Interaction or communication with another mind, beyond simply observing it, is incredibly limited for a novice---see the section \secref{mental-interfaces}.

You can gather some information about a creature by observing its mind in the {\mentalrealm}.
Most obvious is the intelligence of the creature.
The mind of a human or familiar shines brightly, while animals are dimmer.
Particularly simple animals---worms, insects, and the like---are barely noticeable at all.
It is also fairly easy to tell what type of creature it is---the shapes of two creature's minds are normally as distinct as the shapes of their bodies.
If you know an individual, you can even recognise them in the {\mentalrealm} as easily as you would be seeing their face.
Additionally, it is easy to tell when a mind is sleeping, or otherwise unconscious.

By careful inspection of a mind, it is often possible to glean even more information.
This typically requires a \testtype{heed}{projection} Test.
Firstly, you can get some idea of the power of a mind, telling a canny human from a dullard.
A witch, in particular, tends to have a noticeably brighter mind than most folk.
Secondly, it is occasionally possible to tell when someone's mind does not match who they physically seem to be.
It is far harder to disguise the mind than the body, so this can be used to catch someone who is disguised as someone, or something, else.
Any instances of {\possession} are also fairly obvious from within the {\mentalrealm}.

Moving within the {\mentalrealm} is easy.
Navigating is not.
Time and space work strangely, allowing a mind to shift itself miles in a mere moment.
However, the same effect allows an untrained or careless mind to shift itself a long way out of place, just as quickly.
Furthermore, there is no sense of location or direction, and no landmarks besides the minds themselves.
A mind can shift itself to any other mind that it can sense, and follow it around as it moves, but anything else is pure guesswork.
Such guesswork uses \testtype{wit}{projection} Tests.

While in the {\mentalrealm}, a mind is freed from all the trappings of a body.
It has no \attref{might} or \attref{grace} scores, though no Tests using \attref{might} or \attref{grace} are ever necessary in the {\mentalrealm}.
It is also unaffected by any diseases, poisons, or potions that affected its body.
Even non-physical attributes are returned to their normal scores, unaffected by any potions.
It is still affected by {\exhaustion} of non-physical attributes, however.

Furthermore, without access to the physical world, a witch cannot use magic from other disciplines while in the {\mentalrealm}.
Feats from outside the \discref{projection} discipline cannot be used in the {\mentalrealm}, unless specified.
In most cases, the requirement for physical components makes this obvious, but even feats such as \featref{hindsight} and \featref{foresight}, from \discref{divination}, cannot naturally be used there.

\subsection{Denizens of the Mental Realm}

The most obvious denizens of the mental realm are other humans: their busy, intelligent minds shine brightly there.
Familiars, too, are intelligent and obvious there.
Other animals, however, have smaller and duller minds.
Most are not hidden, by any means, but are not so bright and obvious.
The minds of earthworms, fleas and such as so small and simple as to be almost imperceptible.

Most undead also have simpler, duller minds, much like animals.
Mindless creatures, such as golems, have no presence in the {\mentalrealm} at all.
The strangest creatures, however, are those with a single mind composed of many smaller pieces, such as a hive of bees.
It is always obvious that there is \emph{something} there, but it is so diffuse that a novice might not recognise it as a mind.
Interacting with such a mind is certainly impossible for all but the most experienced witches.

Occasionally, spirits are also detectable in the {\mentalrealm}.
In fact, several kinds of spirit exist solely there.
But beware: not all are benevolent.

Lastly, the minds of other witches performing \discref{projection} are present and easily noticeable.
As usual, they cannot communicate there, but they can still identify and follow one another.
It is common practice for an apprentice to follow her mentor into the {\mentalrealm}, to receive a tour and be led safely back to her body afterwards.

\subsection{Lifelines}
\seclabel{mental-lifeline}

A mind is always connected to its own body by a {\lifeline}.
This {\lifeline} serves two purposes.

Firstly, the mind, departed as it may be, still depends on the brain for survival.
If a witch's body is killed, her {\lifeline} is severed, and her mind dies too.
This occurs regardless of where her mind is, even if she is {\possessing} someone else.
As such, she must return to her body periodically to eat and drink, just as usual.
Furthermore, \discref{projection} is not a restful exercise.
A witch still suffers sleep deprivation if she does not return to her body and enjoy natural sleep at the usual rate.

The second use of a {\lifeline} is equally essential to a witch who wishes to make use of \discref{projection}.
While {\lifelines} are imperceptible to the untrained, a skilled witch comes to detect her {\lifeline} and follow it back to her body.
The distance at which she can do this is limited---the {\lifeline} stretches thinner and grows harder to detect as she moves from her body---but inside that distance she is in no danger of getting lost, and may return to her own body in an instant (not even requiring a {\action}).

A witch who has not yet learned to sense her {\lifeline} ought to remain where she can still sense her own mind.
As a novice, this range is only a handful of metres, severely limiting activities.

A witch can only sense {\lifelines}---her own, or any others she has learned to sense---while she is within the {\mentalrealm}.

\section{Possession}
\seclabel{possession}

Besides getting lost, another major risk of \discref{projection} is {\possession}.
The body a witch leaves behind when she enters the {\mentalrealm} is not just physically helpless, but vulnerable to mental intrusion.
Thankfully, most spirits don't have the skill or the power to invade even an undefended mind.
But every so often, a witch might return to her body to find it occupied by something else.

However, {\possession} is also a power that a witch skilled in \discref{projection} can use to her own advantage.
Such a witch learns to take up residence in an unoccupied body, or even to invade and displace an existing occupant.

A displaced occupant always finds themselves in the {\mentalrealm}, even if they haven't the skill to enter it willingly.
This comes as a frightful shock to the uninitiated, but most have the basic instinct to remain close to the nearest light---the mind now possessing their own body---until such a time as they might return to it.

\subsection{Possessed Creatures}

A {\possessed} creature uses the body's \attref{might} and \attref{grace}, but the {\possessing} mind's other attributes, skills, and feats.
The {\possessing} mind has full control of the body, and may retain it indefinitely.

However, the body's owner always has the strongest claim to it, and may attempt to displace a {\possessing} mind.
To do so, it must be adjacent in the {\mentalrealm}, and trying to force its way back in.
Whenever the {\possessing} creature is shocked in some fashion, such as by {\damage}, a slap to the face, a dunk in icy water, or the revelation of a terrible secret, the body's original owner gets an attempt to reassert control.
She makes a \testtype{will}{projection} Test, {\opposed} by the {\possessing} mind's \testtype{will}{projection} Test.
If she succeeds, she displaces the {\possessing} mind back into the {\mentalrealm}, and regains control of her body.
If the {\possessing} mind leaves at any point, the original owner can immediately regain control.

One risk for a {\possessing} creature to bear in mind is that of unconsciousness, and death.
If a body would be knocked unconscious, the possessing mind blacks out with them.
And if the body dies, the shock always kills the occupying mind.

If {\possession} becomes particularly prevalent in a game, it may become possible for a body to contain multiple minds over the course of a {\round}.
Even if this occurs, the body is limited to one {\action} and its usual quantity of movement per {\round}.
Similarly, a mind that occupies multiple bodies is limited to one {\action} and one allotment of movement on its {\turn}.

Note that a {\possessing} mind which wants to leave the body must do so by the usual means.
For example, if it only has the feat \featref{projection-start}, it still requires a minute to leave the body.
It may willingly cede the body to its original owner as an {\action}, but only if the original owner is trying to force its way back in.

\section{Mental Interfaces}
\seclabel{mental-interfaces}

From the {\mentalrealm}, there are many ways to interact with other minds, besides shunting them aside and {\possessing} their bodies.
You can read minds, see through other people's eyes, and talk straight into their heads.
But first, you must establish an {\interface}.

To establish an {\interface} with another mind, you must be in the {\mentalrealm}, and touching the other mind.
That mind can be inside a body or loose in the {\mentalrealm} beside you, and it can be a human mind, or the mind of some other creature.
Establishing an {\interface} does not require an {\action}, but it must be done on your {\turn}.

\capital{\interfaces} come in two kinds: {\overtinterfaces} and {\subtleinterfaces}.
The {\interfaces} are not mutually exclusive---you may establish both at the same time.

\capital{\subtleinterfaces} are simpler, but weaker.
You may establish a {\subtleinterface} with any mind you touch, with no restrictions, and the affected mind remains unaware of it.

\capital{\overtinterfaces} are stronger, but require the active participation of both recipients.
You can only try to establish an {\overtinterface} with a mind that is awake.
The intended recipient feels the attempt, and it feels your presence.
It gets as much information as it would get from observing your mind within the {\mentalrealm}---enough to identify you if it knows you.
To establish the {\interface}, it must actively accept it.
Doing so is easy---anyone can manage it.

An {\interface} only lasts while your minds remain touching in the {\mentalrealm}, though one mind can easily follow another around.
You can break it off at any time.
In the case of an {\overtinterface}, the target can also break it off at any time, and it automatically breaks if the target falls unconscious.
Furthermore, the target can allow only certain aspects of an {\overtinterface}, allowing you to use the abilities of one feat, but not another, on it.

An {\interface} does nothing for a novice witch---she needs feats to be able to make use of it.
Note, however, that the attempt to establish an {\overtinterface} can be used to inform someone that she is there, in the {\mentalrealm}, beside it.

\section{Feats}

\feat{Step Outside}{projection-start}{10}{
	\noprereq
}{
	You may enter the {\mentalrealm} by closing your eyes and meditating for a minute.
}

\feat{Mindlift}{projection-start-other}{15}{
	\featref{projection-start}
}{
	You may lift others out of their own bodies and into the {\mentalrealm}, if they come willingly.
	You may, as an {\action}, attempt to mindlift a creature with which you have an {\overtinterface}.
	If the creature co-operates, its mind enters the {\mentalrealm} beside yours, leaving its body open to {\possession}.
}

\feat{Mindjack}{projection-start-other-2}{20}{
	\skillref[1]{projection},
	\featref{projection-start-other}
}{
	A \featref{projection-start-other} requires that the target \emph{co-operates}, but with a little skill you can find success as long as they \emph{are unable to resist}.
	If you have a {\subtleinterface} with an \emph{unconscious} creature's mind, you may attempt to mindjack it as an {\action}.
	You and the creature make {\opposed} \testtype{will}{projection} Tests.
	On a success, its mind enters the {\mentalrealm} beside yours, leaving its body open to {\possession}.
	
	Whether you succeed or fail, the creature is awakened, unless it is unconscious due to poisoning or the like, and hence unable to be roused in any fashion.
	It is aware of the attempt to invade its mind.
	To save repeated rolling, the GM might allow automatic success against a target that would not be roused by an unsuccessful attempt.
	In this case, however, a creature that subsequently {\possesses} the body might find itself affected by whatever poison rendered the creature unconscious in the first place.
}

\feat{Mindram}{projection-start-other-3}{25}{
	\skillref[2]{projection},
	\featref{projection-start-other}
}{
	With the application of sufficient force, you can dislodge even a resilient mind from its roost.
	If you have a {\subtleinterface} with a creature's mind, you may attempt to ram it out of its body as an {\action}.
	You and the creature make {\opposed} \testtype{will}{projection} Tests.
	On a success, its mind enters the {\mentalrealm} beside yours, leaving its body open to {\possession}.
	
	It is very much aware of the attempt to displace its mind.
	Furthermore, ramming a fortified position with your bare mind is a good recipe for hurting yourself.
	If you fail the {\opposedtest}, you suffer one level of {\exhaustion} affecting \attref{wit} and \attref{will}.
}

\feat{Possess}{projection-possession}{10}{
	\featref{projection-start}
}{
	An unoccupied body is an enticing prospect for a roaming mind, and it's not particularly hard to slide one's own mind into it.
	When you find an unoccupied body in the {\mentalrealm}---a tiny sliver of mind left behind makes it detectable---you may slip into the body and {\possess} it.
	This does not even require an {\action}, and hence may be done in the same {\turn} as you perform a \featref{projection-start-other}, \featref{projection-start-other-2}, or \featref{projection-start-other-3}.
	However, an adjacent and prepared mind might just have time to sneak into the unoccupied body before you do.
}

\feat{Mental Bootstrapping}{projection-start-2}{25}{
	\skillref[1]{projection},
	\featref{projection-start-other}
}{
	You've perfected the neat trick of performing a \featref{projection-start-other} upon yourself from inside your own body.
	Pulling yourself up by your bootstraps, as it were.
	You may enter the {\mentalrealm} as an {\action}.
}

\feat{Sense Interface}{projection-interface-sense}{10}{
	\noprereq
}{
	A careful awareness of your own mind makes you aware of when someone establishes a {\subtleinterface} to your mind.
	Of course, there isn't necessarily anything you can \emph{do} about it.
	As well as just sensing when it has happened, you get all the information you would get if it was an {\overtinterface}---the other mind's identity, which powers they are using through the interface, and so on.
	
	This sense also wakes you up if the {\interface} is established while you are sleeping, just as if someone had touched your face.
	You may, however, turn the sense off if you are sick of people waking you up with it.
	You are still aware if there is an {\interface} in place when you awake.
}

\feat{Block Interface}{projection-interface-block}{15}{
	\skillref[1]{projection},
	\featref{projection-interface-sense}
}{
	Intense mental rigour allows you to block someone establishing a {\subtleinterface} to your mind; it now follows the same rules as establishing an {\overtinterface}.
	The person establishing the {\interface} gets no sense of this, however, and is not aware that you \emph{could} have blocked them if you decide to let them through.
	
	When you sleep, you may choose to block all {\interfaces} without being awoken, allow them through without being awoken, or be awoken by each attempt in order to choose.
}

\feat{One Eye Outside}{projection-mental-realm-sense}{15}{
	\skillref[2]{projection},
	\featref{projection-interface-sense},
	\featref{projection-start-2}
}{
	You always keep a tiny sliver of your mind in the {\mentalrealm}; just enough to see it.
	Even while inhabiting a body, you may sense the {\mentalrealm} as though you were there.
	This includes surrounding minds, as well as {\lifelines} you can detect in the {\mentalrealm}.
	
	This only senses the area around the body you are currently inhabiting, at the present time, unaffected by \discref{divination}.
	If you wish to use this in concert with \discref{divination}, see \featref{scrying-projection} and \featref{vision-projection}.
}

\feat{Mental Leap}{projection-start-3}{20}{
	\skillref[2]{projection},
	\featref{projection-start-2}
}{
	You are nearly more at home in the {\mentalrealm} than in your own body now.
	You may enter the {\mentalrealm} at any point on your {\turn}, without using an action.
}

\feat{Flit Away}{projection-start-4}{15}{
	\skillref[3]{projection},
	\featref{projection-mental-realm-sense}
}{
	You can't even be said to inhabit your own body at this point; you simply puppet it from the {\mentalrealm}.
	You may leave your body into the {\mentalrealm} at any point, even outside your own {\turn}.
	Even in the brief instant between a sword hitting your skull and it cleaving the brain inside, should it come to that.
	Similarly, you may always avoid unconsciousness by cutting loose into the {\mentalrealm}, if you wish.
}

\feat{Stand in Absentia}{projection-remain}{10}{
	\featref{projection-start}
}{
	When you willingly enter the {\mentalrealm} (or, incidentally, when you fall asleep), you may leave your body standing or sitting upright, and your eyes open.
	It won't fool anyone who tries to interact with you---you're completely unresponsive, and you'll fall over if someone shoves you---but they might not notice if nobody tries to talk to you.
	
	When you use this feat, or any of its derivative ones, you cannot remember anything that happened to your body while you were away.
	Take \featref{projection-remain-memory} to remedy this.
}

\feat{Stroll in Absentia}{projection-remain-move}{10}{
	\skillref[1]{projection},
	\featref{projection-remain},
	\featref{projection-start-2}
}{
	When you willingly enter the {\mentalrealm} (or fall asleep), you may leave your body walking.
	It walks up to your \statref{speed} each {\turn}, in a fixed direction.
	It walks blindly, banging into or tripping over any major obstacles.
}

\feat{Survive in Absentia}{projection-remain-dodge}{25}{
	\skillref[1]{projection},
	\featref{projection-remain},
	\featref{projection-start-2}
}{
	When you willingly enter the {\mentalrealm} (or fall asleep), you may leave your body aware, and evading danger.
	Your body is not considered helpless, and maintains your full \statref{dodge-rating}.
	It still remains standing in roughly one spot, unless you also have \featref{projection-remain-move}.
	
	If you do use \featref{projection-remain-move}, however, your body can avoid minor obstacles.
	It is unlikely to trip, and can navigate around narrow obstructions such as trees, in order to keep going in the same direction.
	If it comes to an obstruction it can't navigate around, it stops before crashing into it.
}

\feat{Seek in Absentia}{projection-remain-navigate}{20}{
	\skillref[2]{projection},
	\featref{projection-remain-move},
	\featref{projection-remain-dodge}
}{
	When you willingly enter the {\mentalrealm} (or fall asleep), you may leave your body walking to someplace else.
	It navigates along some route you know, to some place you know.
	For example, you could make it walk up to the castle, or walk home, as long as you know the way.
	
	Your body can turn corners, open doors (and close them behind it), cross bridges, and the like, although it cannot do anything that would require a Test.
	It will stop if it gets lost, or the way is blocked, or the like.
}

\feat{Sprint in Absentia}{projection-remain-move-2}{10}{
	\skillref[1]{projection},
	\featref{projection-remain-move},
	\featref{projection-remain-dodge}
}{
	When you use \featref{projection-remain-move} or \featref{projection-remain-navigate}, you can make your body sprint.
	It takes the \actionref{dash} {\action} each {\turn}, moving twice your \statref{speed} each {\turn}.
	
	This is no less tiring than running normally, and your body suffers {\exhaustion} of \attref{might} and \attref{grace} if it keeps it up for very long.
	However, it won't stop running until the {\exhaustion} causes it to pass out.
}

\feat{Speak in Absentia}{projection-remain-speak}{20}{
	\skillref[2]{projection},
	\featref{projection-remain-dodge}
}{
	When you willingly enter the {\mentalrealm} (or fall asleep), you can leave yourself speaking, singing, making small talk, or otherwise on a sort of social autopilot.
	You can even keep up one-on-one conversation well enough to fool someone, as long as the topic remains entirely small talk.
	Fooling someone always requires a \testtype{charm}{socialising} or \testtype{charm}{deception} Test, however.
	Note that you won't remember the conversation unless you also have \featref{projection-remain-memory}.
}

\feat{Absent Memory}{projection-remain-memory}{15}{
	\featref{projection-remain}
}{
	When you use \featref{projection-remain}, or any of its derivative feats, your body still stores the memories of everything it sees, hears, senses, or does.
	You can recall these memories when your mind returns to your body.
}

\feat{Sleep in Absentia}{projection-sleep}{10}{
	\featref{projection-start}
}{
	You have trained your body to rest in your mind's absence.
	While your mind is in the {\mentalrealm} (or {\possessing} a creature), and your body is unoccupied and resting, it may sleep.
	Such sleep can heal {\damage} as usual, and remove {\exhaustion} that affects \attref{might} or \attref{grace}.
	However, while your mind does not rest, you cannot recover from {\exhaustion} affecting your other attributes, and you still suffer sleep deprivation as usual.
}

\feat{Hide Mind}{projection-hide}{10}{
	\noprereq
}{
	With a strict form of meditation, you can dull your own mind, hiding it from prying \practitioners{projection}.
	You can do so while within your body or in the {\mentalrealm}, but you cannot move, take {\actions}, use {\interfaces}, or the like while you do so.
	Your mind becomes harder to detect from the {\mentalrealm}.
	Your senses still work fine, whether those are your bodily senses, or your senses within the {\mentalrealm}.
	
	Hiding like this is not a guaranteed success, much like hiding in the real world.
	Any Tests you make to hide use \attref{wit} and your choice of \skillref{stealth} or \skillref{projection}.
	Tests made to detect you use \attref{heed} and either \skillref{perception} or \skillref{projection}.
	However, if your mind is still in your body, and someone can find your body from \emph{outside} the {\mentalrealm}, it subsequently becomes very easy to find your mind \emph{inside} the {\mentalrealm}.
}

\feat{Mindless}{projection-hide-2}{20}{
	\skillref[1]{projection},
	\featref{projection-hide}
}{
	You have trained yourself to keep all your thoughts subtle, making hiding your mind even easier.
	You can move, take {\actions}, use {\interfaces}, and the like while using \featref{projection-hide}.
	You must still be conscious, but you can use it all through your waking hours, if you wish.
}

\feat{Hide and Sleep}{projection-hide-3}{10}{
	\skillref[1]{projection},
	\featref{projection-hide-2}
}{
	You have become so adept at hiding your mind that you can do it unconsciously.
	You can use \featref{projection-hide} even while asleep.
}

\feat{Hide in Absentia}{projection-remain-hide}{10}{
	\skillref[1]{projection},
	\featref{projection-hide-3},
	\featref{projection-sleep}
}{
	\capital{\possession} of the body left behind is a risk all \practitioners{projection} take.
	But a malicious mind cannot {\possess} your body if it can't \emph{find} it.
	
	You can use \featref{projection-hide} to hide the tiny sliver of mind you leave behind in your body, making it harder to detect from the {\mentalrealm}.
	Remember, however, that if someone can find your body in the physical realm, it becomes far easier to find it in the {\mentalrealm}.
}

\feat{Piercing the Fog}{projection-sense}{15}{
	\skillref[1]{projection},
	\featref{projection-start}
}{
	You are becoming accustomed to the fog of the {\mentalrealm}, and can sense things deeper within it.
	You can detect minds out to about 100 metres.
}

\feat{Banishing the Fog}{projection-sense-2}{20}{
	\skillref[2]{projection},
	\featref{projection-sense}
}{
	The fog of the {\mentalrealm} does not truly exist; it is a defence the mind automatically casts against the stark, unending void suddenly thrown before it.
	Few have realised this, but with the realisation you have succeeded in banishing it entirely.
	
	Inside the {\mentalrealm}, your senses are unimpeded.
	You can sense a human mind from as far away as you could see a human, about 3 kilometres.
	Smaller minds might not be detectable from quite such a distance.
	But with no horizon to impede the view, you can sense a larger mind, or a crowd, from much further off.
}

\feat{Sense Lifeline}{projection-lifeline}{10}{
	\featref{projection-start}
}{
	You may detect your {\lifeline} when within the {\mentalrealm}, and use it return to your own body, as long as you remain within about 100 metres.
}

\feat{Long Lifeline}{projection-lifeline-2}{20}{
	\skillref[1]{projection},
	\featref{projection-lifeline}
}{
	You may detect your {\lifeline} at any range.
	This lets you return to your body at any time, no matter how far you stray from it.
}

\feat{Borrow Sight}{projection-read-sight}{10}{
	\featref{projection-start}
}{
	They say that the eyes are the windows to the soul.
	That you can read a person's thoughts be staring into their eyes.
	What most people forget is that windows work both ways.
	By peering into someone's mind, you can see out of their eyes.
	
	While you have an {\overtinterface} with a creature's mind, you can see what it sees.
	You don't have any control over where it points it eyes, of course.
	As such, anything it hasn't noticed or isn't paying attention to is somewhat indistinct.
	Conversely, this does make it rather easy to tell what the creature is paying attention to.
}

\feat{Borrow Hearing}{projection-read-hearing}{10}{
	\featref{projection-start}
}{
	The sounds a mind hears echo within it, and a witch listening closely can hear them.
	While you have an {\overtinterface} with a creature's mind, you can hear what it hears.
}

\feat{Borrow Senses}{projection-read-senses}{5}{
	\featref{projection-read-sight},
	\featref{projection-read-hearing}
}{
	You've seen the sights, you've heard the sounds, and now you can take it one step further: borrowing a creature's whole sensorium.
	While you have an {\overtinterface} with a creature's mind, you gain input from all its senses: sight, hearing, smell, taste and touch, as well as the senses of temperature, balance, pain, and so on.
	You can also feel senses foreign to humans, such a pigeon's sense of north, and a python's heat vision.
	However, you don't gain supernatural senses, such as that granted by \featref{death-detection}.
}

\feat{Subtle Borrowing}{projection-read-senses-unwilling}{15}{
	\featref{projection-read-sight} or \featref{projection-read-hearing}
}{
	You can borrow a creature's senses without its permission, and without it even noticing.
	You can use \featref{projection-read-sight}, \featref{projection-read-hearing}, or \featref{projection-read-senses}, if you have them, through a {\subtleinterface}.
}

\feat{Think Tank}{projection-speak}{10}{
	\featref{projection-start}
}{
	You can speak directly into people's heads from within the {\mentalrealm}, and they can speak back to you.
	For now, this requires their co-operation.
	
	While you have an {\overtinterface} with another mind, you and it can communicate.
	This communication happens using words, at about the same speed as speech; perhaps slightly faster.
	This typically limits it to humans and other intelligent creatures, though you can still say ``stop'' or ``sit'' to a dog.
	Familiars understand language---it is normally only anatomy that prevents them speaking, and they can communicate just fine this way.
}

\feat{Speak to the Mind}{projection-speak-unwilling}{15}{
	\skillref[1]{projection},
	\featref{projection-speak}
}{
	You can use \featref{projection-speak} over a {\subtleinterface}---the words arrive in the target's head unbidden.
	This can be quite disconcerting for common folk, and can also be used to wake sleeping people by shouting into their head.
	
	Because this is not an {\overtinterface}, the target does not know your identity.
	However, because of the one-way nature of the {\interface}, you also cannot hear any reply it gives.
	You may want \featref{projection-read-mind-2} for that; or \featref{projection-read-hearing} and \featref{projection-read-senses-unwilling}.
}

\feat{Mental Images}{projection-speak-2}{10}{
	\skillref[1]{projection},
	\featref{projection-speak}
}{
	When you use \featref{projection-speak}, or \featref{projection-speak-unwilling}, you may establish a slightly deeper connection.
	Instead of communicating using words, you can send raw thoughts---mental images, and even more abstract ideas.
	
	Firstly, this lets you communicate with creatures that don't understand a language, such as animals.
	Secondly, it can often prove faster than speech.
	Instead of describing a person's face, you can visualise it directly for someone.
}

\feat{Dream Shaping}{projection-dreams}{10}{
	\skillref[1]{projection},
	\featref{projection-speak-2},
	\featref{projection-speak-unwilling}
}{
	You can already send images flashing, unbidden, into people's heads.
	But this always comes across as obviously supernatural.
	If you're a little subtler about it, and do it while they're asleep, you can make them seem like natural dreams.
	
	While you have a {\subtleinterface} with a sleeping creature's mind, you can shape its dreams.
	You choose what it sees, hears, smells, and so on, and even whether it dreams in the first place.
	You can even take control of how it acts within its dreams, or allow it of act of its own volition.
	In the latter case, however, you cannot sense how it chooses to act without an effect like \featref{projection-read-mind-2}.
	
	The creature is not normally aware that it is dreaming, even in the most outlandish situations.
	The GM may, however, allow the creature to make a Test---a difficult one---in order to notice.
	
	You may force the creature to remember the dream, if you wish.
	Otherwise, it must make a \attref{ken} Test to remember it.
	If it does remember the dream, it normally realises that it was a dream after it awakes.
	
	If you spend most of the night causing the creature to have nightmares, its sleep is fitful and restless.
	The sleep does not help it recover from {\damage} or {\exhaustion}, although they may still recover 1 point of {\damage} and 1 level of {\exhaustion} from a {\dayofrest}.
	If this goes on for several nights, it may begin suffering {\exhaustion} due to sleep deprivation.
	Of course, if you spend all night doing this, you are not sleeping yourself.
	
	This only works on a \emph{sleeping} creature---sleeping naturally, or due to something like \featref{sleep-potion}.
	It does not work on a creature knocked unconscious by {\shock}, or the like.
}

\feat{Shape of a Mind}{projection-read-mind}{10}{
	\featref{projection-start}
}{
	Emotions are a powerful force, gripping a creature's mind, and altering its entire nature.
	Though subtle, this change is noticeable in the {\mentalrealm}.
	
	While you have a {\subtleinterface} with another mind, you can feel its emotions.
	This is not simply a vague impression, but a very particular sense which you can feel varying moment-to-moment.
	However, the GM might call for a Test to detect better-hidden emotions, such as a twinge of guilt overwhelmed by a great sense of triumph.
	
	It is true that emotions can often just be read on a person's face, without risking a trip to the {\mentalrealm}, however it is far harder for a liar to control their mind than their face.
	Animal's faces, too, can prove harder to interpret than their minds.
	This technique can be used very effectively in concert with an accomplice to direct questions at the target.
}

\feat{Mental Eavesdropping}{projection-read-mind-2}{20}{
	\featref{projection-read-mind}
}{
	While the shape of a creature's mind betrays its emotions, listening to its \emph{thoughts} requires a slightly deeper probing.
	While you have a {\subtleinterface} with another mind, you can listen to its thoughts.
	This isn't an active mind reading so much as a passive mind eavesdropping.
	You can only hear its active train of thought, the words running through its head whenever you happen to be listening.
	If it is speaking, these are normally the words it is saying.
	
	You can only understand these thoughts if the creature is thinking in a language you understand, which tends to limit the ability to humans, familiars, and other intelligent creatures.
	%TODO: Reference a druidcraft feat that lets you understand animal speech/thought.
	Furthermore, a rigidly controlled and disciplined mind can prevent certain thoughts from slipping into this internally vocalised stream, if it has reason to suspect it is being eavesdropped upon.
	There are even rumours of witches who have fooled mental eavesdroppers by intentionally lying in their own thoughts.
	%TODO: Add feats for this sort of fooling? Feats that don't actually require the ability to Project at all?
}

\feat{Mind Meld}{projection-merge}{15}{
	\skillref[1]{projection},
	\featref{projection-speak-2},
	\featref{projection-read-senses},
	\featref{projection-possession}
}{
	When you establish an {\overtinterface}, you can strengthen it, merging both minds together.
	You can think as one, and even move as one if the target is currently inhabiting a body.
	
	While your minds are merged, you may make any Tests using the attributes of skills from either participant in the mind meld.
	You may even use the attribute score from one participant and the skill rank from the other.
	Note that a mind in the {\mentalrealm} does not have a \attref{might} or \attref{grace} score, so Tests using these scores must always use the body's score.
	
	Ultimately, the mind you have merged with remains in control.
	It always retains the final say in any movements and actions.
	Furthermore, you cannot work magic through a body using a mind meld.
	You cannot use any feats that you would not normally be able to use while in the {\mentalrealm}.
}

\feat{Magic Meld}{projection-merge-2}{15}{
	\skillref[1]{projection},
	\featref{projection-merge},
	\featref{projection-start-other}
}{
	When you use \featref{projection-merge}, you may work your magic through someone else.
	The melded mind can use the feats of both participants.
	
	For feats which require sustained magic---those that would end if you died---maintenance is tracked separately for each mind in the meld.
	This primarily affects the number of {\symlinks}, scrying sensors, golems, or controlled undead that can be maintained.
	When a new one is established or animated, it must be assigned to one mind or the other.
	Note that a witch with no relevant feats cannot maintain even one of these things.
	For example, if only one of you has any \discref{sympathetic-magic} feats, the melded mind can only maintain one {\symlink}.
	But, if you both have a \discref{sympathetic-magic} feat, you can maintain one or more {\symlinks} each.
}

\feat{Shaping a Mind}{projection-change-mind}{15}{
	\featref{projection-read-mind}
}{
	By feeling the shape of a mind, you can read its emotions.
	And by shaping the mind, you can affect those emotions.
	
	This requires a {\subtleinterface} with the target's mind.
	For now, you can only affect the foremost emotion in the creature's mind, as you sense it with \featref{projection-read-mind}.
	Amplifying or suppressing the emotion is easy, though particularly extreme cases might require a Test.
	
	With a Test, you might even be able to redirect the emotion.
	For example, you could make someone angry at themselves, instead of at their child.
	You must still redirect it to a reasonable target; you can't make them angry at themselves unless they carry at least a portion of the blame for what they're angry about.
	
	Tests are {\opposed}, using \testtype{will}{projection} from both you and the target.
	The GM should apply modifiers as appropriate.
}

\feat{Shape Senses}{projection-change-senses}{20}{
	\skillref[2]{projection},
	\featref{projection-dreams},
	\featref{projection-read-senses-unwilling},
	\featref{projection-read-sight} or \featref{projection-read-hearing}
}{
	It is one thing to change what a person senses in their dreams, when everything they see is a fiction of their mind.
	But to change what they senses while they are awake---to fool their eyes and ears, as well as their mind---requires another trick entirely.
	Thankfully, it's a trick you know.
	
	While you have a {\subtleinterface} with a mind that currently resides within a body, you can alter what it senses.
	The senses you can affect are the same senses you can borrow, as determined by which of the feats \featref{projection-read-sight}, \featref{projection-read-hearing}, and \featref{projection-read-senses} you have.
	
	For now, your influence is subtle; you cannot fabricate \emph{everything} they sense, like you can while they are asleep.
	If you affect vision, you can make them see flashes of movement, change the colour of something, hide a small object from their view, alter someone enough such that they don't recognise them, or the like.
	You cannot form an entirely new object, unless it is very simple, and you cannot make someone look like someone else entirely.
	If you affect hearing, you can make noises, even loud ones, coming from a particular direction, but not with enough precision to synthesise speech.
	Or you could silence a noise that they would otherwise hear.
	Affecting other senses, you could make a single, possibly pungent smell, remove the taste from something, or give them a tap on the shoulder.
}

\feat{Minding the Hive}{projection-hive}{15}{
	\skillref[1]{projection},
	\featref{projection-start}
}{
	Creatures such as bees and ants don't have any real individual presence in the {\mentalrealm}, for they have no individual minds.
	Rather, they are ruled by a hive mind.
	They appear as a diffuse glow within the {\mentalrealm}, which some people might not even register as a mind at all.
	A novice cannot interact with such a mind, but you can.
	
	You may form {\interfaces} to hive minds, and use them in the normal ways.
	However, a hive mind is far harder to separate from its physical form---you cannot \featref{projection-start-other}, \featref{projection-start-other-2}, \featref{projection-start-other-3}, or {\possess} it.
}

\feat{Mob Rule}{projection-hive-2}{25}{
	\skillref[3]{projection},
	\featref{projection-hive}
}{
	It is said that the intelligence of a crowd is inversely proportional to its number of members.
	In truth, it's not quite that bad, but that's beside the point.
	The point is that a crowd has its own mind, separate to, but formed from, the minds of its members.
	
	You may treat a group of minds that would normally be considered distinct minds as a hive mind, and interface with it using \featref{projection-hive}.
	However, it is not enough that they are all gathered in one place; there must be some sort of mob, or herd mentality going on.
}

\feat{Familiar Lifeline}{projection-lifeline-familiar}{15}{
	\featref{projection-lifeline}
}{
	Your familiar shares a shard of your own {\soul}, leaving you just as tethered to it as you are to your own body.
	You gain a {\lifeline} leading to your familiar, at the same range you can sense you own.
	Likewise, you can follow it to appear alongside your familiar's mind in an instant.
	However, this {\lifeline} does nothing to keep you alive, and you will still die if your body does.
}

\feat{Living Vicariously}{projection-lifeline-familiar-2}{10}{
	\skillref[1]{projection},
	\featref{projection-lifeline-familiar}
}{
	By strengthening your {\lifeline} to your familiar, you can keep your {\soul} shackled to this realm in the event of your death.
	If you are outside your own body when your body dies, but your familiar still lives, your mind survives.
	Of course, if your familiar dies at any point after this, so do you.
	
	You should discuss matters with your GM before taking this feat.
	You should have a plan for keeping your character in play in the event of their body's death, and the GM is free to preclude you from taking the feat if they disagree with these plans.
	On the other hand, the GM may also help you to hatch a plan---perhaps a quest---for resurrecting your character's body, or finding them a new one.
}

\feat{Lifeline Hunting}{projection-lifeline-other}{10}{
	\featref{projection-lifeline}
}{
	While you have a {\subtleinterface} with another mind, you can sense its {\lifeline}, at the same range you can sense your own.
	You may follow this {\lifeline} straight to its body in one leap.
	If the mind is loose in the {\mentalrealm}, its body is likely unoccupied and open to {\possession}.
	
	This only allows you to feel the primary {\lifeline}, to its body.
	This may not exist for some non-humans, or for humans whose body has died, that are using an alternative {\lifeline}.
}

\feat{Another Lifeline}{projection-lifeline-other-2}{10}{
	\skillref[1]{projection},
	\featref{projection-lifeline-other},
	\featref{projection-lifeline-familiar}
}{
	While you have a {\subtleinterface} with another mind, you can sense any {\lifelines} other than its primary, at the same range you can sense your own.
	First and foremost, you can sense a {\lifeline} from a witch to her familiar, or from a familiar to its witch.
	You gain this sense regardless of whether the witch herself can sense her {\lifeline} to her familiar.
	
	Some creatures may have other non-primary {\lifelines}, at the GM's discretion.
	You can also sense these, whether or not the creature itself can.
}

\feat{Phylactery Projection}{projection-phylactery}{20}{
	\skillref[1]{projection},
	\skillref[1]{necromancy},
	\featref{phylactery-self}
}{
	Being trapped in a {\phylactery} is normally rather crippling, much akin to being dead.
	With discipline, however, you may separate the death of your mind from the death of your body, and remain active in the {\mentalrealm} even while your {\soul} resides in a {\phylactery}.
	
	While your {\soul} is in a {\phylactery}, you may enter the {\mentalrealm} from your {\phylactery} as through it were your body.
	You may also use feats such as \featref{projection-mental-realm-sense}.
	
	While in this state, your primary {\lifeline} connects to your {\phylactery}.
	This is the one you can sense and follow with \featref{projection-lifeline}, and the ones others sense with \featref{projection-lifeline-other}.
}

\feat{Shards of My Mind}{projection-golemancy}{10}{
	\featref{projection-start},
	\anyfeat{golemancy}
}{
	Although a golem is a mindless creature, you have imbued it with a shard of your own mind and will.
	Just enough to sense, if you know what you're looking for.
	
	You can sense your own golems within the {\mentalrealm}, just as would sense any other mind.
	You still can't sense other people's golems.
	You can also establish {\interfaces} with your golems, and use all the associated feats.
	Your golems always accept your {\overtinterfaces}.
	
	Note that if you \featref{projection-merge} with a golem, you still cannot control its actions directly.
	You can help it with any Tests, but its instructions still determine everything that it does.
}

\feat{Shards of Another Mind}{projection-golemancy-2}{10}{
	\skillref[1]{projection},
	\featref{projection-golemancy}
}{
	It's harder to find parts of another mind than it is to find parts of your own, because you're never quite as familiar with them.
	But all \practitioners{golemancy} have a few techniques in common, and you've learned to search for these signatures.
	
	You gain all the benefits of \featref{projection-golemancy} with other people's golems.
	Other people's golems will never accept your {\overtinterfaces}, unless they have been specifically instructed to.
}

\feat{Golem Lifelines}{projection-golemancy-location}{10}{
	\skillref[1]{projection},
	\skillref[1]{divination},
	\featref{projection-lifeline-familiar},
	\featref{projection-golemancy},
	\featref{divination-golems}
}{
	Your golems have their own {\lifelines}, leading from them back to you.
	You have learned to sense these lifelines, finding your golems within the {\mentalrealm}.
	
	You can sense your golems' {\lifelines} at the same range you can sense your own, and can follow it to appear alongside a golem in the {\mentalrealm} in an instant.
	However, these {\lifelines} do nothing to keep you alive, and you will still die if your body does.
}

\feat{Remote Access}{golem-change-instructions-projection}{15}{
	\skillref[1]{golemancy},
	\featref{projection-golemancy},
	\featref{golem-change-instructions}
}{
	You can do more than just find your golems in the mental realm: you can alter them.
	You can reprogram your own golems while you have an {\overtinterface} with them from the {\mentalrealm}.
	This still takes however long it would take you if you were physically touching the golem.
}

\feat{Vessel of Clay}{projection-golemancy-possession}{25}{
	\skillref[2]{projection},
	\skillref[2]{golemancy},
	\featref{projection-possession},
	\featref{golem-change-instructions-projection}
}{
	Golems have no minds.
	This made them difficult to even \emph{find} within the {\mentalrealm}, so most are quick to dismiss them as worthless in the art of \discref{projection}.
	You, however, have come to the realisation that a vessel without a mind is a great boon.
	
	You may {\possess} a golem that you can sense within the {\mentalrealm}, following the normal rules for possessing an unoccupied body.
}

\feat{Mental Circle Bypass}{projection-barrier-bypass}{10}{
	\skillref[1]{projection},
	\skillref[1]{ritual-magic},
	\featref{projection-start},
	\featref{circle-barrier-bypass}
}{
	Using your \featref{circle-barrier-bypass}, your mind, while in the {\mentalrealm}, can pass through your \emph{own} \featref{circle-contain}, \featref{circle-exclude}, or \featref{circle-contain-exclude} in either direction.
	Passing through in a direction that would normally be blocked requires an {\action}, but, unlike in the physical realm, you don't need any equipment to do it.
}

\feat{Mental Circle Intrusion}{projection-barrier-bypass-2}{15}{
	\skillref[2]{projection},
	\skillref[2]{ritual-magic},
	\featref{projection-barrier-bypass},
	\featref{circle-barrier-break-2}
}{
	Having slipped your mind through your own barrier circles, you've figured out how to push through other people's.
	This works like \featref{projection-barrier-bypass}, requiring an {\action}.
	Furthermore, unlike \featref{circle-barrier-break-2}, it does not require a Test.
}
