\chapter{Projection}
\chaplabel{projection}

\discref{projection} is a strictly mental discipline.
It features none of the circles and gestures of \discref{ritual-magic}, nor even the physical cues that often accompany \discref{willing}.
In fact, with her mind gone, an unskilled witch leaves her body totally catatonic.

But as much as it fails to engage the body, \discref{projection} engages every part of the mind.
A witch uses her \attref{heed} to sense the minds of others; her \attref{wit} to reach them, to probe them, to find the cracks and leverage them; and her \attref{will} to ram another mind, to shunt it from its body and replace it with her own.
Lastly, her \attref{wit} is important in returning to her own body.

\section{The Mental Realm}
\seclabel{the-mental-realm}

The mental realm is not truly a different realm, so much as it is a different way of experiencing our own one.
\discref{projection} is the ability to cast one's mind out into this realm, and to operate within it.

The mental realm is not experienced with any of the normal senses, tied as these senses are to organs such as the eyes, ears and skin.
But if one had to describe it, it might go something like this:

Image a vast, grey plain.
It stretches away forever in all directions.
\emph{All} directions, for there is no ground, no sky.
You do not fall, just float.
There is no up, no down, no north, south, east or west.
Not unless you brought them with you.

And you had better hope you did, because there are no landmarks.
Everything is pervaded by a cold, grey fog.
It is thick.
So thick that you would be wading through it, if you still had legs.
You'd hardly see your own hand in front of face, even if you still had one.

Only one thing shines through the fog.
It is dim, so dim, but it is close.
In fact, it seems to be inside of you, so far as there is a `you' to be inside.
It is you.
There is not much there now.
Your body, catatonic, rigid.
The barest traces of mind you have left behind, keeping you breathing, your heart beating.

As you adjust to the gloom, you strain the eyes you do not have and peer deep into the fog.
The faintest hints of light shine through.
They move, dance, and play in the distance.
Some shine brighter, some softer.
They are the minds of others, still living in the realm you have left behind.

You could reach out and touch them, if you wanted.
Just one quick step over there, pushing your way into the fog.
But you can still feel the light beside, inside of you.
It is so dim.
If you stepped away, would you be able to find it again?
Or would you be lost in the fog forever?

\section{Possession}

\subsection{The Displaced}
