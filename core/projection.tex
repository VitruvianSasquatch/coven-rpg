\discipline{Projection}{projection}{Projector}{Projectors}

\dropcapdiscref{projection} is, broadly, a witch's ability to have her mind leave her own body.
Once freed from its earthly shackles, her mind can wander freely, unimpeded by walls, oceans, or even vast distances.
She can sense the minds of others, finding them, following them, listening to them, and reading them.
She can even displace them from their bodies, taking them over.

Some witches, always eager to put a mystical spin on everything, refer to the process as Astral Projection.
Most just stick to \discref{projection}.
The experts, however, often forgo even this phrase.
Many forget what it was ever like to be trapped within their own body, and treat \discref{projection} as a natural and integral part of their lives.
It isn't uncommon to hear a witch say ``I'm just popping out,'' a moment before her body slumps, catatonic, to the floor.

Expert \practitioners{projection} often treat their body like a favoured pair of shoes, to be worn and discarded as needed.
Many even prefer to go barefoot, and won't think twice about borrowing someone else's shoes in an emergency.
They make a complete separation between a person's mind and body, treating the natural coupling of the two as a mere convenience.
This lends such a witch a great flexibility, but can be terrifying to those less accustomed to the practice.

\discref{projection} is a strictly mental discipline.
It features none of the circles and gestures of \discref{ritual-magic}, nor even the physical cues that often accompany \discref{willing}.
In fact, with her mind gone, an unskilled witch leaves her body totally catatonic.

But as much as it fails to engage the body, \discref{projection} engages almost every part of the mind.
A witch uses her \attref{heed} to sense the minds of others; her \attref{wit} to reach them, to probe them, to find the cracks and leverage them; and her \attref{will} to ram another mind, to shunt it from its body and replace it with her own.
Lastly, her \attref{wit} is once again important if she grows lost, and needs to navigate back to her own body.

\section{The Mental Realm}
\seclabel{the-mental-realm}

The {\mentalrealm} is not truly a different realm, so much as it is a different way of experiencing our own one.
\discref{projection} is the ability to cast one's mind out into this realm, and to operate within it.

The {\mentalrealm} is not experienced with any of the normal senses, tied as these senses are to organs such as the eyes, ears and skin.
But if one had to describe it, it might go something like this:

Image a vast, grey plain.
It stretches away forever in all directions.
\emph{All} directions, for there is no ground, no sky.
You do not fall, just float.
There is no up, no down, no north, south, east or west.
Not unless you brought them with you.

And you had better hope you did, because there are no landmarks.
Everything is pervaded by a cold, grey fog.
It is thick.
So thick that you would be wading through it, if you still had legs.
You'd hardly see your own hand in front of your face, if you had a hand, or a face.

Only one thing shines through the fog.
It is dim, so dim, but it is close.
In fact, it seems to be inside of you.
It \emph{is} you---part of you.
The `you' that you left behind.
There isn't much there now.
Your body, catatonic, rigid.
The barest traces of thought, keeping you breathing, keeping your heart beating.

As you adjust to the gloom, you strain the eyes you do not have and peer deep into the fog.
The faintest hints of light shine through.
They move, dance, and play in the distance.
Some shine brighter, some softer.
They are the minds of others, still living in the realm you have left behind.

You could reach out and touch them, if you wanted.
Just one quick step over there, pushing your way into the fog.
But you can still feel the light beside, inside of you.
It is so dim.
If you stepped away, would you be able to find it again?
Or would you be lost in the fog forever?

\subsection{Using the Mental Realm}
\seclabel{using-the-mental-realm}

A witch may enter the {\mentalrealm} by taking a \featref{projection-start}.
Her body is left catatonic and helpless, although still alive and breathing, while her mind departs.
She enters the {\mentalrealm} with her mind safely beside her body, able to return to it at any point she chooses.

Even a novice entering the {\mentalrealm} can sense nearby minds.
The fog that fills the realm prevents this sense from reaching far.
Minds in the same room can be sensed automatically, while a \testtype{heed}{projection} might extend the sense to fill a house or more.
Vague attributes of the minds---whether they belong to an animal, a human, or a familiar---are obvious.
Gleaning the actual identity of a mind requires a Test.
Interaction or communication with another mind, beyond simply observing it, is impossible for a novice.

Moving within the {\mentalrealm} is easy.
Navigating is not.
Time and space work strangely, allowing a mind to shift itself miles in a mere moment.
However, the same effect allows an untrained or careless mind to shift itself a long way out of place, just as quickly.
Furthermore, there is no sense of location or direction, and no landmarks besides the minds themselves.
A mind can shift itself to any other mind that it can sense, and follow it around as it moves, but anything else is pure guesswork.
Such guesswork uses \testtype{wit}{projection} Tests.

While in the {\mentalrealm}, a mind is freed from all the trappings of a body.
It has no \attref{might} or \attref{grace} scores, though no Tests using \attref{might} or \attref{grace} are ever necessary in the {\mentalrealm}.
It is also unaffected by any diseases, poisons, or potions that affected its body.
Even non-physical attributes are returned to their normal scores, unaffected by any potions.
It is still affected by {\exhaustion} of non-physical attributes, however.

\subsection{Denizens of the Mental Realm}

The most obvious denizens of the mental realm are other humans: their busy, intelligent minds shine brightly there.
Familiars, too, are intelligent and obvious there.
Other animals, however, have smaller and duller minds.
Most are not hidden, by any means, but are not so bright and obvious.
The minds of earthworms, fleas and such as so small and simple as to be almost imperceptible.

Most undead also have simpler, duller minds, much like animals.
Mindless creatures, such as golems, have no presence in the {\mentalrealm} at all.
The strangest creatures, however, are those with a single mind composed of many smaller pieces, such as a hive of bees.
It is always obvious that there is \emph{something} there, but it is so diffuse that a novice might not recognise it as a mind.
Interacting with such a mind is certainly impossible for all but the most experienced witches.

Occasionally, spirits are also detectable in the {\mentalrealm}.
In fact, several kinds of spirit exist solely there.
But beware: not all are benevolent.

Lastly, the minds of other witches performing \discref{projection} are present and easily noticeable.
As usual, they cannot communicate there, but they can still identify and follow one another.
It is common practice for an apprentice to follow her mentor into the {\mentalrealm}, to receive a tour and be led safely back to her body afterwards.

\subsection{Lifelines}
\seclabel{mental-lifeline}

A mind is always connected to its own body by a {\lifeline}.
This {\lifeline} serves two purposes.

Firstly, the mind, departed as it may be, still depends on the brain for survival.
If a witch's body is killed, her {\lifeline} is severed, and her mind dies too.
This occurs regardless of where her mind is, even if she is {\possessing} someone else.
As such, she must return to her body periodically to eat and drink, just as usual.
Furthermore, \discref{projection} is not a restful exercise.
A witch still suffers sleep deprivation if she does not return to her body and enjoy natural sleep at the usual rate.

The second use of a {\lifeline} is equally essential to a witch who wishes to make use of \discref{projection}.
While {\lifelines} are imperceptible to the untrained, a skilled witch comes to detect her {\lifeline} and follow it back to her body.
The distance at which she can do this is limited---the {\lifeline} stretches thinner and grows harder to detect as she moves from her body---but inside that distance she is in no danger of getting lost, and may return to her own body in an instant (not even requiring a {\action}).

A witch who has not yet learned to sense her {\lifeline} ought to remain where she can still sense her own mind.
As a novice, this range is only a handful of metres, severely limiting activities.

\section{Possession}
\seclabel{possession}

Besides getting lost, another major risk of \discref{projection} is {\possession}.
The body a witch leaves behind when she enters the {\mentalrealm} is not just physically helpless, but vulnerable to mental intrusion.
Thankfully, most spirits don't have the skill or the power to invade even an undefended mind.
But every so often, a witch might return to her body to find it occupied by something else.

However, {\possession} is also a power that a witch skilled in \discref{projection} can use to her own advantage.
Such a witch learns to take up residence in an unoccupied body, or even to invade and displace an existing occupant.

A displaced occupant always finds themselves in the {\mentalrealm}, even if they haven't the skill to enter it willingly.
This comes as a frightful shock to the uninitiated, but most have the basic instinct to remain close to the nearest light---the mind now possessing their own body---until such a time as they might return to it.

\subsection{Possessed Creatures}

A {\possessed} creature uses the body's \attref{might} and \attref{grace}, but the {\possessing} mind's other attributes, skills and feats.
The {\possessing} mind has full control of the body, and may retain it indefinitely.

However, the body's owner always has the strongest claim to it, and may attempt to displace a {\possessing} mind.
To do so, it must be adjacent in the {\mentalrealm}, and trying to force its way back in.
Whenever the {\possessing} creature is shocked in some fashion, such as by {\damage}, a slap to the face, a dunk in icy water or the revelation of a terrible secret, the body's original owner gets an attempt to reassert control.
She makes a \testtype{will}{projection} Test, {\opposed} by the {\possessing} mind's \testtype{will}{projection} Test.
If she succeeds, she displaces the {\possessing} mind back into the {\mentalrealm}, and regains control of her body.
If the {\possessing} mind leaves at any point, the original owner can immediately regain control.

One risk for a {\possessing} creature to bear in mind is that of unconsciousness, and death.
If a body would be knocked unconscious, the possessing mind blacks out with them.
And if the body dies, the shock always kills the occupying mind.

If {\possession} becomes particularly prevalent in a game, it may become possible for body to contain multiple minds over the course of a {\round}.
Even if this occurs, the body is limited to one {\action} and its usual quantity of movement per {\round}.
Similarly, a mind that occupies multiple bodies is limited to one {\action} and one allotment of movement on its {\turn}.

\section{Feats}

\feat{Step Outside}{projection-start}{10}{
	\noprereq
}{
	You may enter the {\mentalrealm} by closing your eyes and meditating for a minute.
}

\feat{Mindlift}{projection-start-other}{15}{
	\featref{projection-start}
}{
	You may lift others out of their own bodies and into the {\mentalrealm}, if they come willingly.
	When you touch a creature's mind in the {\mentalrealm} you may, as an {\action}, attempt to mindlift it.
	The creature feels the attempt to lift its mind, which is only successful if it co-operates.
	If the creature does co-operate, its mind enters the {\mentalrealm} beside yours, leaving its body open to {\possession}.
}

\feat{Mindjack}{projection-start-other-2}{20}{
	\skillref[1]{projection},
	\featref{projection-start-other}
}{
	A \featref{projection-start-other} requires that the target \emph{co-operates}, but with a little skill you can find success as long as they \emph{are unable to resist}.
	When you touch a \emph{unconscious} creature's mind in the {\mentalrealm} you may attempt to mindjack it as an {\action}.
	You and the creature make {\opposed} \testtype{will}{projection} Tests.
	On a success, its mind enters the {\mentalrealm} beside yours, leaving its body open to {\possession}.
	
	Whether you succeed or fail, the creature is awakened, unless it is unconscious due to poisoning or the like, and hence unable to be roused in any fashion.
	It is aware of the attempt to invade its mind.
	To save repeated rolling, the GM might allow automatic success against a target that would not be roused by an unsuccessful attempt.
	In this case, however, a creature that subsequently {\possesses} the body might find itself affected by whatever poison rendered the creature unconscious in the first place.
}

\feat{Mindram}{projection-start-other-3}{25}{
	\skillref[2]{projection},
	\featref{projection-start-other}
}{
	With the application of sufficient force, you can dislodge even a resilient mind from its roost.
	When you touch a conscious creature's mind in the {\mentalrealm} you may attempt to ram it out of its body as an {\action}.
	You and the creature make {\opposed} \testtype{will}{projection} Tests.
	On a success, its mind enters the {\mentalrealm} beside yours, leaving its body open to {\possession}.
	
	It is very much aware of the attempt to displace its mind.
	Furthermore, ramming a fortified position with your bare mind is a good recipe for hurting yourself.
	If you fail the {\opposedtest}, you suffer one level of exhaustion affecting \attref{wit} and \attref{will}.
}

\feat{Possess}{projection-possession}{10}{
	\featref{projection-start}
}{
	An unoccupied body is an enticing prospect for a roaming mind, and it's not particularly hard to slide one's own mind into it.
	When you are in the {\mentalrealm} and touch the sliver of mind left in an unoccupied body, you may slip into the body and {\possess} it.
	This does not even require an {\action}, and hence may be done in the same {\turn} as you perform a \featref{projection-start-other}, \featref{projection-start-other-2}, or \featref{projection-start-other-3}.
	However, an adjacent and prepared mind might just have time to sneak into the unoccupied body before you do.
}

\feat{Mental Bootstrapping}{projection-start-2}{25}{
	\skillref[1]{projection},
	\featref{projection-start-other}
}{
	You've perfected the neat trick of performing a \featref{projection-start-other} upon yourself from inside your own body.
	Pulling yourself up by your bootstraps, as it were.
	You may enter the {\mentalrealm} as an {\action}.
}

\feat{Mental Leap}{projection-start-3}{20}{
	\skillref[2]{projection},
	\featref{projection-start-2}
}{
	You are nearly more at home in the {\mentalrealm} than in your own body now.
	You may enter the {\mentalrealm} at any point on your {\turn}, without using an action.
}

\feat{One Eye Outside}{projection-mental-realm-sense}{15}{
	\skillref[2]{projection},
	\featref{projection-start-3}
}{
	You can enter the {\mentalrealm} with such ease that you have begun to keep a part of your mind there constantly.
	Even while inhabiting a body, you may sense the {\mentalrealm} as though you were there.
}

\feat{Flit Away}{projection-start-4}{15}{
	\skillref[3]{projection},
	\featref{projection-mental-realm-sense}
}{
	You can't even be said to inhabit your own body at this point; you simply puppet it from the {\mentalrealm}.
	You may leave your body into the {\mentalrealm} at any point, even outside your own {\turn}.
	Even in the brief instant between a sword hitting your skull and it cleaving the brain inside, should it come to that.
	Similarly, you may always avoid unconsciousness by cutting loose into the {\mentalrealm}, if you wish.
}

\feat{Stand in Absentia}{projection-remain}{10}{
	\skillref[1]{projection},
	\featref{projection-start}
}{
	When you willingly enter the {\mentalrealm} (or, incidentally, when you fall asleep), you may leave your body standing or sitting upright, and your eyes open.
	It won't do much to fool people---your eyes glaze over or stare into the middle distance, and you'll fall over if someone knocks you---but they might not notice if they're not paying attention either, or if you're not facing them.
}

\feat{Dreaming}{projection-sleep}{10}{
	\featref{projection-start}
}{
	You have trained your body to rest in your mind's absence.
	While your mind is in the {\mentalrealm} (or {\possessing} a creature), and your body is unoccupied and resting, it may sleep.
	Such sleep can heal {\damage} as usual, and remove {\exhaustion} that affects \attref{might} or \attref{grace}.
	However, while your mind does not rest, you cannot recover from {\exhaustion} affecting your other attributes, and you still suffer sleep deprivation as usual.
}

\feat{Piercing the Fog}{projection-sense}{15}{
	\skillref[1]{projection},
	\featref{projection-start}
}{
	You are becoming accustomed to the fog of the {\mentalrealm}, and can sense things deeper within it.
	You can detect minds out to about 100 metres.
}

\feat{Banishing the Fog}{projection-sense-2}{20}{
	\skillref[2]{projection},
	\featref{projection-sense}
}{
	The fog of the {\mentalrealm} does not truly exist; it is a defence the mind automatically casts against the stark, unending void suddenly thrown before it.
	Few have realised this, but with the realisation you have succeeded in banishing it entirely.
	
	Inside the {\mentalrealm}, your senses are unimpeded.
	You can sense a human mind from as far away as you could see a human, about 3 kilometres.
	Smaller minds might not be detectable from quite such a distance.
	But with no horizon to impede the view, you can sense a larger mind, or a crowd, from much further off.
}

\feat{Sense Lifeline}{projection-lifeline}{10}{
	\featref{projection-start}
}{
	You may detect your {\lifeline} when within the {\mentalrealm}, and use it return to your own body, as long as you remain within about 100 metres.
}

\feat{Shape of a Mind}{projection-read-mind}{10}{
	\featref{projection-start}
}{
	The simple shape of a creature's mind in the {\mentalrealm} can betray a surprising amount.
	When in the {\mentalrealm}, you can glean this information by merely brushing your own mind against its.
	Firstly, you know the creature's identity.
	This is not their \emph{name}, or any such concrete information.
	But it is enough to recognise them if you know them, and normally to realise if they are not who they say they are.
	You can certainly tell when {\possession} is in play.
	
	Secondly, you can feel the creature's emotions.
	This is not simply a vague impression, but a very particular sense which you can feel varying moment-to-moment.
	However, the GM might call for a Test to detect better-hidden emotions, such as a twinge of guilt overwhelmed by a great sense of triumph.
	
	It is true that emotions often just be read on a person's face, without risking a trip to the {\mentalrealm}, however it is far harder for a liar to control their mind than their face.
	Animal's faces, too, can prove harder to interpret than their minds.
	This technique can be used very effectively in concert with an accomplice to direct questions at the target.
}

\feat{Borrow Sight}{projection-read-sight}{15}{
	\featref{projection-read-mind}
}{
	They say that the eyes are the windows to the soul.
	That you can read a person's thoughts be staring into their eyes.
	What most people forget is that windows work both ways.
	By peering into someone's mind, you can see out of their eyes.
	
	While touching a creature's mind in the {\mentalrealm}, you can see what it sees.
	You don't have any control over where it points it eyes, of course.
	As such, anything it hasn't noticed or isn't paying attention to is somewhat indistinct.
	Conversely, this does make it rather easy to tell what the creature is paying attention to.
}

\feat{Borrow Hearing}{projection-read-hearing}{15}{
	\featref{projection-read-mind}
}{
	Much as an attuned witch can hear the words and thoughts issuing from a mind, it is possible to hear the words echoed \emph{through} a mind.
	While touching a creature's mind in the {\mentalrealm}, you can hear what it hears.
}

\feat{Borrow Senses}{projection-read-senses}{10}{
	\skillref[1]{projection},
	\featref{projection-read-sight},
	\featref{projection-read-hearing}
}{
	You've seen the sights, you've heard the sounds, and now you can take it one step further: borrowing a creature's whole sensorium.
	While touching a creature's mind in the {\mentalrealm}, you gain input from all its senses: sight, hearing, smell, taste and touch, as well as the senses of temperature, balance, pain, and so on.
	You can also feel senses foreign to humans, such a pigeon's sense of north and a python's heat vision.
	However, you don't gain supernatural senses, such as that granted by \featref{death-detection}.
}

\feat{Mental Eavesdropping}{projection-read-mind-2}{20}{
	\featref{projection-read-mind}
}{
	While the shape of a creature's mind betrays its emotions, listening to its \emph{thoughts} requires a slightly deeper probing.
	While touching a creature's mind in the {\mentalrealm}, you can listen to its thoughts.
	This isn't an active mind reading so much as a passive mind eavesdropping.
	You can only hear its active train of thought, the words running through its head whenever you happen to be listening.
	If it is speaking, these are normally the words it is saying.
	
	You can only understand these thoughts if the creature is thinking in a language you understand, which tends to limit the ability to humans and other intelligent creatures.
	%TODO: Reference a druidcraft feat that lets you understand animal speech/thought.
	Furthermore, a rigidly controlled and disciplined mind can prevent certain thoughts from slipping into this internally vocalised stream, if it has reason to suspect it is being eavesdropped upon.
	There are even rumours of witches who have fooled mental eavesdroppers by intentionally lying in their own thoughts.
	%TODO: Add feats for this sort of fooling? Feats that don't actually require the ability to Project at all?
}

\feat{Familiar Lifeline}{projection-lifeline-familiar}{15}{
	\featref{projection-lifeline}
}{
	Your familiar shares a shard of your own soul, leaving you just as tethered to it as you are to your own body.
	You gain an additional {\lifeline} leading to your familiar, which you can sense at the same range you can sense you own.
	Likewise, you can follow it to appear alongside your familiar's mind in an instant.
	However, this {\lifeline} does nothing to keep you alive, and you will still die if your body does.
}

\feat{Living Vicariously}{projection-lifeline-familiar-2}{10}{
	\skillref[1]{projection},
	\featref{projection-lifeline-familiar}
}{
	By strengthening your {\lifeline} to your familiar, you can keep your soul shackled to this realm in the event of your death.
	If you are outside your own body when your body dies, but your familiar still lives, your mind survives.
	Of course, if your familiar dies at any point after this, so do you.
	
	You should discuss matters with your GM before taking this feat.
	You should have a plan for keeping your character in play in the event of their body's death, and the GM is free to preclude you from taking the feat if they disagree with these plans.
	On the other hand, the GM may also help you to hatch a plan---perhaps a quest---for resurrecting your character's body, or finding them a new one.
}

\feat{Phylactery Lifeline}{projection-lifeline-phylactery}{20}{
	\skillref[1]{projection},
	\featref{projection-lifeline-familiar-2},
	\featref{phylactery}
}{
	Much as you share a {\lifeline} to the shard of soul in your familiar, you may connect one to another shard tucked elsewhere, in a {\phylactery}.
	You gain a third {\lifeline}, leading to your {\phylactery}, if you have one.
	
	You may sense this {\lifeline} at the same range you can sense your original, and it also serves to keep your mind alive, regardless of distance.
	As such, your mind may continue to survive as long as any one of your own body, your familiar, or your {\phylactery} is alive or intact.
	Combined with the ability to {\possess} people, this may even allow you to reanimate your own body from you {\phylactery}.
}

\feat{Shards of My Mind}{projection-golemancy}{10}{
	\featref{projection-start},
	\featref{gingerbread-golem}
}{
	Although a golem is a mindless creature, you have imbued it with a shard of your own mind and will.
	Just enough to sense, if you know what you're looking for.
	
	You can sense your own golems within the {\mentalrealm}, although not other people's.
	You can use \featref{projection-read-sight}, \featref{projection-read-hearing} and \featref{projection-read-senses} on them, if you have those feats.
}

\feat{Shards of Another Mind}{projection-golemancy-2}{10}{
	\skillref[1]{projection},
	\featref{projection-golemancy}
}{
	It's harder to find parts of another mind than it is to find parts of your own, because you're never quite as familiar with them.
	But all \practitioners{golemancy} have a few techniques in common, and you've learned to search for these signatures.
	You gain all the benefits of \featref{projection-golemancy} with other people's golems.
}

\feat{Vessel of Clay}{projection-golemancy-possession}{25}{
	\skillref[2]{projection},
	\skillref[2]{golemancy},
	\featref{projection-possession},
	\featref{change-golem-instructions-projection}
}{
	Golems have no minds.
	This made them difficult to even \emph{find} within the {\mentalrealm}, so most are quick to dismiss them as worthless in the art of \discref{projection}.
	You, however, have come to the realisation that a vessel without a mind is a great boon.
	
	You may {\possess} a golem that you can sense within the {\mentalrealm}, following the normal rules for possessing an unoccupied body.
}
