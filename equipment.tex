\chapter{Tools of the Craft}
\chaplabel{equipment}

\section{The Hat}

A witch's pointed hat is the most important of her tools, in many regards.
There are no particular rules about the hat; its effects are left up to the GM.
But it always has an effect on people.
It may make them angry, reverent, reassured or afraid, but most importantly it makes sure they know that they are in the presence of a witch.

A witch's hat says a lot about her, particularly to other witches.
When you create your character, you can answer the following questions about your hat.

\begin{itemize}
	\item Did you make it yourself?
	\item How tall is it?
	\item Is it the traditional black, or some other colour?
	\item How long have you had it?
		Is it visibly worn?
		Well cared for?
	\item Is it plain, tastefully decorated, or covered in stars and sequins?
	\item Does it have any useful accessories?
		Pockets?
\end{itemize}

Many witches accompany their hats by a black cloak or other such attire.
Opinions on occult jewellery are mixed: some witches wear masses, others frown on it heavily.



\section{Broomsticks}

Sometimes, walking from one village to another just takes too long.
A lot of witches---to maintain their mystique or simply because the townsfolk wouldn't be happy otherwise---even choose to live quite a way from the nearest village.
Such circumstances make a broomstick an essential accessory for any witch.

Broomstick flight is no mean feat and while every witch picks up the rudiments, most can use it for nothing more than getting from A to B.
The broom needs a running start, has to be ridden sidesaddle, and has a turning circle several hundred metres across.
Detailed rules for flying a broomstick can be found in \chapref{broomcraft}.

Before it can be used, a broomstick needs to be trained to to fly.
This requires someone to fly it around on another broomstick so that it can learn its craft from one of its fellows.
It must be held parallel to the broom being ridden, to ensure it learns to fly in the correct direction.
The process takes about eight hours.
These hours need not be consecutive, but should all be done within a couple of weeks.
Once trained, a broom retains its flight skill for a long time.
Taking it out for a few hours each year is enough to keep its hand in.

At character creation, every witch is assumed to own a trained broomstick one way or another.
It was probably trained using the broom of whoever taught her witchcraft, at least if she's still using their first broom.
It might feel like an old friend at this point, the witch familiar with every knot and notch in its handle.
A more careless witch might have gone through a few brooms during her career.



\section{Common Magical Components}

The various rites and magics of the various disciplines of witchcraft require too many different materials to enumerate here.
However, there a few components that make a regular appearance.
Some details of their acquisition, construction and use are given here.

\subsection{Ritual Circles}

A ritual circle describes any large arrangement of symbols or shapes required by a rite.
They are traditionally drawn on the floor in chalk, but other methods are far from uncommon; the visibility and accuracy are the only important aspects for most rites.
Some witches use paint for permanence, or even chisel their circles into stone.
Many a witch in a hurry has scratched their circles into the dirt with the toe of their boot.
Some witches even embroider their circles upon sheets of fabric that can be rolled up and laid down where needed.
However, a roll bearing even the smallest of circles is most of the height of a man.

Each rite requires a ritual circle of a particular design, different for every rite, but the same each time the rite is performed.
This means that scribing a circle just once and using it for many performances of the rite is a common practice.
Ritual circles are not even universally circular, although it is the most common shape and almost all have some sort of symmetry.
Squares, triangles and hexagons are not uncommon, and pentagrams are particularly common in certain disciplines.

Ritual circles are classified primarily by their size.
\begin{itemize}
	\item A small ritual circle can be scribed entirely in arm's reach while standing in one spot.
		It can comfortably be drawn in a couple of minutes.
	\item A medium sized circle is a few paces across.
		Most houses should have a room large enough to draw one in, if the furniture is moved.
		It can comfortably be drawn in a quarter of an hour.
	\item A large ritual circle is at least two dozen paces across.
		A ballroom or village hall is probably the only place one could be drawn indoors, so most are drawn outside.
		At least a couple of hours are required to draw such a circle without haste.
\end{itemize}

\subsection{Megalithic Circles}

Some rites require a circle of standing stones, called a megalithic circle.
Such a circle must be at least the size of a large ritual circle, with at least a dozen stones each taller than a man.
The arrangement and shape of the stones is unimportant, as long as it is recognisable as a ring of standing stones, and so the same circle can be used for all rites that require one.
Constructing a megalithic circle is no easy task, typically requiring weeks of work by much of a village, even if the site is quite close to a stone quarry.

\subsection{Taglocks}

A taglock is any part of a person's body, such as a piece of flesh, a strand of hair, a nail clipping, a drop of blood, or a gob of saliva.
It is often used to bind a spell to a particular target.
It can always be picked off a person---although taking a hair without being noticed might be difficult---but people often leave taglocks behind them, especially in places they frequent.
In place you suspect someone might have left a taglock, such as their house or a bed they've slept in, finding a taglock typically uses Perception.



\section{Improvised Tools}



\section{Weapons}

Weapons are divided into several broad categories.
Players are free to describe their character's weapons how they wish, within the bounds of reason, placing them in one of the categories.
Anything a character might find at hand and hit people with can also be placed into a category.

A weapon's accuracy is added a flat bonus to rolls to hit, in place of an attribute.
A weapon's damage determines the number of dice rolled upon hitting.
The highest 3 dice are kept, as always, but the number of dice rolled are determined by the weapon instead of the wielder's skill.
The wielder's Might is added to the damage roll for melee or thrown weapons, but not for bows.

\begin{simpletable}{X[2.4]XXX[1.3]}
	\toprule
	Weapon & Accuracy & Damage & Range (metres)\\
	\midrule
	Fist & +2 & 3 & Melee\\
	Club & +4 & 4 & Melee\\
	Knife & +2 & 5 & Melee\\
	Hand Weapon & +4 & 5 & Melee\\
	Thrown Rock & +0 & 3 & $5\times\text{Might}$\\
	Thrown Weapon & +0 & 5 & $5\times\text{Might}$\\
	Bow & +2 & 5 & 100\\
	\bottomrule
\end{simpletable}

\subsubsection{Fist}
A punch, a kick, or a headbutt.
Covers any attack you make without any weapon at all.

\subsubsection{Club}
A club, a walking stick, a chair, or a cauldron.
A club is just about anything you pick up and hit someone with.

\subsubsection{Knife}
A knife or dagger.
Easily concealed, and a staple of blood witches.
The short blade costs the wielder reach, but can do as much damage as a sword if you get the enemy in the tender parts.

\subsubsection{Hand Weapon}
A sword, an axe, a mace, a spear, a pike.
This category covers most things actually designed as a weapon and larger than a knife.

\subsubsection{Thrown Rock}
A genuine rock, but also a teapot, a boot or a frog.
Anything you might pick up and throw.
This includes weapons that aren't designed to be thrown.

\subsubsection{Thrown Weapon}
A spear, a knife, a hatchet.
Any weapon you can throw that was actually designed for the purpose.
Rocks from slingshots fall in this category too.

\subsection{Bow}
A bow and arrow.
Also covers crossbows, if the setting includes them.
