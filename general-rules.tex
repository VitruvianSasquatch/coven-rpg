\chapter{The Broad of It}
\chaplabel{general-rules}

This chapter covers rules essential to day-to-day play.
Players and GMs alike should be familiar with at least the major points in here in order to play.
More specific rules, pertaining to various disciplines of magic, can be found in the appropriate chapters of \partref{disciplines}.

It is important to remember that this book cannot cover every situation that may arise during play.
The role of the GM includes adjudicating such scenarios, and the following section should contain guidelines to assist in that.
Furthermore, it is often helpful to do the same when the players simply cannot remember a rule, to avoid slowing down play while someone looks it up.
And lastly, remember that all the rules contained in this book are guidelines and suggestions.
Feel free to change them all that you want!
The most important thing is that everyone is having fun.

\section{Tests}

Tests are the dice rolls used to determine the outcome of an action when there is element of chance and risk involved.
Several of the rules in this chapter and others will specify the appropriate Test to make with a particular action, but the GM should be calling for other kinds of Tests whenever appropriate as well.

A Test is typically made with a skill and an attribute, although having no applicable skill is not uncommon.
Often, the rule that required the Test specifies these.
Otherwise, the GM chooses as appropriate.
The character's skill determines how many dice she rolls for a Test.
If there is no skill applicable to the test, or if the character has no ranks in the applicable skill, she rolls 3 dice.
Each rank in the skill gives an additional rolled die, to a maximum of 6 with all three ranks.
Total together the highest 3 of the rolled dice and add the character's relevant attribute to this total.
The final total is compared against a Target Number (TN) set by the GM: if it meets or exceeds the TN the Test succeeds; otherwise it fails.

A Test where every die shows a 1 or 2 is a critical failure, and a Test where all 3 kept dice show a 6 is a critical success.
In addition to the Test automatically succeeding or failing, the GM is encouraged to apply an additional drawback or benefit to the result of the Test.
Critical failures on Tests involving dangerous magic can be especially catastrophic.

\subsection{An Example Test}

As an example, suppose Mistress Talbot is attempting to identify which manner of undead dog has just shambled into her garden.
The GM declares this to be a Necromancy + Wit Test, as she is attempted to recall information about the undead.
Mistress Talbot dabbled in Necromancy as a youth, and has one rank in the skill, so she rolls 4 dice.
However, she is not the sharpest knife in the drawer, and her memory has begun to fade with age, so she has only 1 Wit.
The four dice show a 4, a 6, a 2 and 3.
Her player totals the three highest dice, the 6, 4 and 3, for 13.
Then she adds Mistress Talbot's Wit, 1, for a grand total of 14.
Her player announces the total to the table.

The GM knows that the dog is a simple zombie, the most common of undead, but it was only killed and raised yesterday so the characteristic rot hasn't properly set in yet.
In light of this, she assigns a Target Number of 12: not too easy, but not particuarly difficult either.
Hearing Mistress Talbot's total of 14, the GM knows that this has exceeded the TN of 12: the Test has succeeded.
She tells Mistress Talbot that the midnight intruder is merely a zombie, nothing too threatening.
Mistress Talbot is reassured and heads outside to see what the beast wants, though not without grabbing the poker from beside the fireplace, just in case.

\subsection{Target Numbers}

\subsection{Using Tests}

Be careful not to call for a Test when it's not necessary.
If the action is a simple one that the character should be able to routinely perform, such as walking through a door or ransacking a room for something that isn't hidden, it doesn't require a Test.
(Although what is routine for one character might not be for another; a closed door can present a serious obstacle to a feline familiar.)
If the action is impossible, such as jumping over the moon or convincing the King to give up his crown without solid leverage, the character shouldn't make a Test.
If the charcater wouldn't succeed even with a critical success, a Test should never be rolled.
Lastly, if there is no penalty for failure, there is no need for a Test.
If the character will keep on trying until she succeeds, there's no need to make the player keep rolling Tests.

%TODO: Graded success, success at a price.

\section{Combat}

\section{Injury}

\section{Movement}

\section{Magic}
