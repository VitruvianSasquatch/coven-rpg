\chapter{The Broad of It}
\chaplabel{general-rules}

This chapter covers rules essential to day-to-day play.
Players and GMs alike should be familiar with at least the major points in here in order to play.
More specific rules, pertaining to various disciplines of magic, can be found in the appropriate chapters of \partref{disciplines}.

It is important to remember that this book cannot cover every situation that may arise during play.
The role of the GM includes adjudicating such scenarios, and the following section should contain guidelines to assist in that.
Furthermore, it is often helpful to do the same when the players simply cannot remember a rule, to avoid slowing down play while someone looks it up.
And lastly, remember that all the rules contained in this book are guidelines and suggestions.
Feel free to change them all that you want!
The most important thing is that everyone is having fun.

\section{Rounding Fractions}

In general, round down whenever you get a fraction, even if the fractional part is one half or greater.

\section{Tests}

Tests are the dice rolls used to determine the outcome of an action when there is element of chance and risk involved.
Several of the rules in this chapter and others will specify the appropriate Test to make with a particular action, but the GM should be calling for other kinds of Tests whenever appropriate as well.

A Test is typically made with a skill and an attribute, although having no applicable skill is not uncommon.
Often, the rule that required the Test specifies these.
Otherwise, the GM chooses as appropriate.
The character's skill determines how many dice she rolls for a Test.
If there is no skill applicable to the test, or if the character has no ranks in the applicable skill, she rolls 3 dice.
Each rank in the skill gives an additional rolled die, to a maximum of 6 with all three ranks.
Total together the highest 3 of the rolled dice and add the character's relevant attribute to this total.
The final total is compared against a Target Number (TN) set by the GM: if it meets or exceeds the TN the Test succeeds; otherwise it fails.

A Test where every die shows a 1 or 2 is a critical failure, and a Test where all 3 kept dice show a 6 is a critical success.
In addition to the Test automatically succeeding or failing, the GM is encouraged to apply an additional drawback or benefit to the result of the Test.
Critical failures on Tests involving dangerous magic can be especially catastrophic.

\subsection{Dice Notation}

A variant of standard RPG dice notation is used for Tests.
The size of the dice and the fact that only three are kept is omitted, as these are constants.
For example, \dice{4} indicates a 4 die Test with no bonus, and \dice[2]{3} indicates a 3 die Test with an attribute bonus of 2.

\subsection{An Example Test}

As an example, suppose Mistress Talbot is peering out of her window and attempting to identify which manner of undead dog has just shambled into her garden.
The GM declares this to be a \testtype{wit}{necromancy} Test, as she is attempting to recall information about the undead.
Mistress Talbot dabbled in \discref{necromancy} as a youth, and has one rank in the skill, so she rolls 4 dice.
However, she is not the sharpest knife in the drawer, and her memory has begun to fade with age, so she has only 1 \attref{wit}.
The four dice show 4, 6, 2 and 3.
Her player totals the three highest dice, the 6, 4 and 3, for 13.
Then she adds Mistress Talbot's \attref{wit}, 1, for a grand total of 14.
Her player announces the total to the table.

The GM knows that the dog is a simple zombie, the most common variety of undead, but it was killed and raised only yesterday so the characteristic rot hasn't properly set in yet.
In light of this, she assigns a Target Number of 12: not too easy, but not particularly difficult either.
Hearing Mistress Talbot's total of 14, the GM knows that she has met the TN of 12: the Test has succeeded.
She announces that Mistress Talbot, by the creature's glassy eyes and stumbling gait, realises the midnight intruder is merely a zombie.
Reassured---she'd been fearing a ghoul or a hellhound---Mistress Talbot heads outside to see what the beast wants.
Though not without grabbing the poker from beside the fireplace, just in case.

\subsection{Target Numbers}

A Target Number (TN) represents the difficulty of the action that requires a Test.
The more difficult the action, the higher the target number, and the less likely the Test is to succeed.
In some situations, the same rule that requires a Test will specify its TN.
In other situations, the GM should select a TN she feels is appropriate.

Typical TNs range from approximately 9 to 21.
A Test with a TN lower than 9 is not normally worth it: a character with no skill and an average score in the relevant attribute will succeed more than \SI{95}{\percent} of the time.
Similarly, a Test with a TN higher than 21 is not normally worth it: a character needs a 5 in the relevant attribute to succeed without a critical success.
The following table shows a brief summary of the sorts of task particular TNs are suited to.

\begin{simpletable}{rX}
	\toprule
	TN & Task Difficulty\\
	\midrule
	9 & Easy: An average, unskilled person would normally manage this.\\
	12 & Moderate: An average, unskilled person would manage this about half the time.\\
	15 & Challenging: It takes skill to pull this off consistently.\\
	18 & Difficult: Even a skilled person is unlikely to achieve this consistently.\\
	21 & Legendary: This takes great skill, ability and good luck to perform.\\
	\bottomrule
\end{simpletable}

Instead of assigning a simple pass-or-fail TN, the GM may also employ graded success.
This is when a higher roll gives a higher level of success.
For instance, a higher roll on a Test to recall knowledge might mean that the character recalls more knowledge about the situation, while a higher roll on a check to influence a crowd might influence a greater proportion of the crowd.
This can also be used to apply success at a cost, where an intermediate roll, neither particularly high nor particualarly low, means that the character succeeds at their task but incurs some drawback in doing so.
For example, a coven might try to intimidate a guard to allow them into the castle.
Failure could indicate the guard calls for backup and resists, while a very high result on the Test would mean he is cowed and allows them to pass.
an intermediate result might mean that he allows the coven to pass, but sneaks off to find reinforcements and confront them later, while they are inside the castle.

\subsection{Using Tests}

Be careful not to call for a Test when it's not necessary.
If the action is a simple one that the character should be able to routinely perform, such as walking through a door or ransacking a room for something that isn't hidden, it doesn't require a Test.
(However, what is routine for one character might not be for another; a closed door can present a serious obstacle to many familiars.)
If the action is impossible, such as jumping over the moon or convincing the King to give up his crown without solid leverage, the player shouldn't make a Test.
If the character wouldn't succeed even with a critical success, a Test should never be rolled.
Lastly, if there is no penalty for failure, there is no need for a Test.
If the character will keep on trying until she succeeds, there's no need to make the player keep rolling Tests.

\subsection{Rolling Fewer Than Three Dice}

Some effects will modify the number of dice a character rolls for a Test, and this can bring the number of rolled dice below three.
In this case, all the rolled dice are added to the total as normal, but the maximum total that can be reached is obviously reduced.
Additionally, critical success is no longer possible, as this require three dice showing 6.
Critical failure, however, becomes far more likely, as it only requires that all dice show 1 or 2.

If the number of dice rolled for a Test would be reduced to zero, the Test cannot be performed.
If it is unavoidable, it is automatically treated as a critical failure.

\section{The Flow of Time}

\subsection{Narrative Time}

During normal play, the exact timing and duration of characters' action are unimportant, and not carefully tracked.
It is enough to know whether something took a matter of seconds or minutes, an hour or two, or a couple of days.
This is Narrative Time, and the GM is free to be as accurate or as loose as necessary with time periods.

The one element of Narrative Time with an impact upon the rules is that of Scenes, which are often used to measure the duration of effects.
It may be helpful to think of scenes like in a play.
The Scene typically changes when the action changes location (everyone walks from the church down to the village green), when there is a timeskip (everyone waits an hour for the sun to set) or there is a change in the cast of characters (the preparations for the party finish and the guests begin to arrive).
Changes in Scene, and the duration of effects that rely on them, are ultimately left up to the GM, but should often be obvious.

\subsection{Structured Time}

In tense situations with two opposing parties, exact timing and duration becomes for important to track.
For this purpose, and to aid tactical thinking in such scenarios, the GM can move the game into Structured Time.
Direct combat is perhaps the most common application of this, but chase scenes may also use them.
With the correct magic, some of the participants might even be many miles apart.

Structured Time is divided into Rounds and Turns.
Every character participating in the Scene gets one Turn each Round.
Although the Turns are resolved in some order, all characters are assumed to be acting simultaneously and continuously.
If it becomes particularly relevant for some reason, assume each Round takes approximately 10 seconds.

On each Turn, a character may move a number of metres equal to their Speed and take one Action.
An Action is something that requires most of the character's effort during their turn, such as attacking someone, performing a brief bit of magic, knocking a hole in a wall or quaffing a potion.
They may also take a reasonable number of minor actions that shouldn't require their full concentration, such as opening or slamming a door, drawing a sword, pointing at something or speaking a short sentence.
Not everything can be accomplished in one Action.
For example, winching a drawbridge closed may take several Actions, as might even one of the faster magical rites.
Some of the Actions available to a character are given in \secref{combat-actions}, but the GM is free adjudicate anything the characters try as one or more Actions.

\subsection{Initiative}

%TODO

\section{Distances and Ranges}

Each Turn in Structured Time, a character can move a number of metres equal to their Speed, as well as taking an Action.

%TODO: Long-range travel times.

%TODO: Define close, medium, long range, etc.

\section{Injury}

Witchcraft is a dangerous business.
Between mad spirits, evil demons, foul undead, and disgruntled mobs of villagers, injury is inevitable.
And it's not only her own injuries that a witch has to deal with.
One of a witch's duties is to tend to the injuries of her neighbours, nursing them back to health after an accident or disease has laid them low.
Or, when they are beyond her help, easing their moments.

A character's resistance to injury is determined by two statistics: Resilience and Shock Threshold.
A character's Resilience is equal to their \attref{might} plus 5, all halved.
For example, a character with $0$ \attref{might} has $(0 + 5)/2 = 2$ Resilience, and a character with $3$ \attref{might} has $(3+5)/2 = 4$ Resilience.
A character's Shock Threshold is equal to 12 plus their \attref{will}.

\subsection{Damage Tests}
\seclabel{damage-tests}

A Damage Test is a special type of Test used to determine how much an effect hurts a character.
It is made like a normal Test, by rolling some number of dice and adding the highest 3 together, with a flat bonus.
In the case of an attack by one character upon another, the number of dice are determined by the weapon used and the flat bonus by the wielder's strength.
In other cases, the GM or the rules of the damaging effect assign the number of dice and the bonus.
For small effects, this can often be fewer than 3 dice.

The following table provides examples of the number of dice and the bonus for Damage Tests.

\begin{simpletable}{Xl}
	\toprule
	Effect & Damage\\
	\midrule
	Touching a hot cauldron & \dice{1}\\
	Crawling through brambles & \dice{2}\\
	Wave-tossed against a boulder & \dice{3}\\
	Hit by a falling brick & \dice{4}\\
	Falling on a sword & \dice{5}\\ %TODO: Evaluate and expand.
	Hit by a falling tree & \dice[4]{5}\\
	\bottomrule
\end{simpletable}

Additionally, a Damage Test is not made against a particular TN like most Tests.
Instead, it applies two effects to the target, Shock and Damage.
Shock is always tested for before Damage is applied.

Critical failure on a Damage Test means no effect is applied at all; the blow was glancing and won't do more than bruise slightly.
Critical success on a Damage Test may immediately kill the target or leave them with a lasting injury, at the GM's option, and always applies Shock.

\subsection{Shock}

If a Damage Test meets or exceeds the target's Shock Threshold, or critically succeeds, the target goes into Shock.
A character in Shock falls unconscious and cannot be roused while they remain in Shock.
If a character in Shock would go into Shock again due to another Damage Test, they die.

Additionally, at the start of each of the Shocked character's Turns, roll a special Test against them.
This Test applies no flat bonus, and uses the same number of dice as the Damage Test that sent the character into Shock: a character is more likely to bleed out from a sword wound than a punch.
If it meets or exceeds the Shocked character's Shock Threshold, they die.
This Test is not considered to be a Test made by any character.

If the Test made every Turn ever totals 9 or less, unless it also meets or exceed their Shock Threshold, the character is no longer in Shock.
However, the character remains unconscious and cannot be naturally roused before the end of the Scene.
A character can also be brought out of Shock by another character tending to them.
This requires an Action and a successful \testtype{wit}{healing} Test.
The TN for this Test is 3 times the number of dice that would be rolled against the Shocked character each round.

\subsection{Damage}

After Shock has been tested for, whether or not it occurs, the Damage Test causes damage.
To calculate damage, divide the result of the Damage Test by the target's Resilience.
For example, if the result of the Damage Test is 13 and the damaged creature has 4 Resilience, they suffer 3 damage.
Damage accumulates: a character who has previously suffered 3 damage and suffers an additional 2 is now suffering from 5 damage.

Damage has several effects.
Firstly, a character subtracts their current damage from their Shock Threshold.

Secondly, increasing damage can cause pain, having an adverse effect upon a character's actions.
For every 3 points by which a character's damage exceeds their \attref{will}, they suffer a $-1$ penalty to all Tests except Damage Tests.
For example, if a character with 1 \attref{will} is suffering 7 damage, their damage exceeds their \attref{will} by 6 and they suffer a $-2$ penalty.
However, if the character had 5 \attref{will}, their damage would exceed their \attref{will} by only 2 and they would not suffer any penalty.

Lastly, if a character's Shock Threshold ever reaches zero, they die immediately.
This is very unlikely to happen through repeated damage, as an earlier blow would send them into Shock, but can occur if a lot of painkillers wear off all at once.
%Lastly, if a character's current damage ever equals or exceeds their original Shock Threshold (unmodified by damage), they die immediately.
%This applies even if they are ignoring the effect of some of their damage, such as through painkillers.
%As effects that allow a character to ignore damage are so common, it can be helpful to track a character's actual damage and the damage they are considered to be suffering from separately.
%The former represents injury: scrapes, bruises and cuts.
%The latter represents the pain suffered as a result of these.

\section{Combat}
\seclabel{combat}

\subsection{Actions in Combat}
\seclabel{combat-actions}

\subsubsection{Attack}
\seclabel{attack-action}

You attack a creature or object, with a \seclink{weapon}{weapons} or unarmed.
You must be adjacent to the target to attack with a melee weapon, or within the listed range of a ranged weapon.
Make a Test using a number of dice determined, as normal, by your \skillref{weaponry} skill, and a flat bonus determined by your weapon's accuracy.
The Test is made against a TN equal to the target's Dodge Rating: 8 plus twice their \attref{grace}.

If you succeed in your Test, you hit.
Make a \seclink{Damage Test}{damage-tests} against the Target, rolling dice as determined by your weapon and adding you \attref{might}.

\subsubsection{Dash}
\seclabel{dash-action}

You may move an additional number of metres equal to your Speed this turn.

\subsubsection{Ready}
\seclabel{ready-action}

You don't act immediately, but prepare to take an Action later.
Decide what Action you will take, and which circumstances trigger it.
When those circumstances come around, you may choose to take the readied Action or not.
If your next turn comes around without you taking the readied Action, you lose the benefits of readying.
You must take the Ready action again if you want to continue to wait.

\section{Movement}

\section{Magic}

Magic consists of too many diverse disciplines and effects to be effectively summarised in this section; indeed this is the entire topic of \partref{disciplines}.
However, a few general guidelines apply.

It is generally assumed that any witch who knows a spell, rite or technique has the knowledge and practice to pull it off consistently; doing so does not require a Test unless specified otherwise.
However, this practice only applies under normal conditions, with adequate time and materials.
A witch may attempt to rush her magic, perform it using whatever she has to hand, or to perform it in difficult conditions, and each of these requires a Test.
Such Tests typically use \attref{wit} and the relevant skill for the discipline of magic, but not always.
For example, drawing a chalk circle hurriedly might use \attref{grace}, and grinding a poultice while on a broomstick might use \skillref{flying}.

TNs for rushing or improvising magic are ultimately left up the GM, but some guidelines are provided below.

\subsection{Rushing Magic}

Generally, magic that would normally take at least an Action in combat cannot be performed in less than that time.
Exceptions may be made where the magic is used as part of the Action already being taken, to aid it or improve its effect, but the GM should still be careful allowing such things.
Otherwise, common sense may apply a limit to the minimum time magic can be performed in.
For example, if a potion requires boiling water, a witch needs some way to bring water to the boil in the time they want to brew their potion.

Where magic can be rushed, guideline TNs for doing so are given in the following table.

\begin{simpletable}{rX}
	\toprule
	TN & Example Task\\
	\midrule
	9 & Performing a simple rite in half the normal time.\\
	12 & Performing a complex rite in half the normal time.\\
	15 & Performing a simple rite in a tenth the normal time.\\
	18 & Performing a complex rite in a tenth the normal time.\\
	21 & Performing a simple 5 minute rite in one Action.\\
	\bottomrule
\end{simpletable}

\subsection{Improvising Materials}

This applies to both the tools used to conduct magic and the ingredients consumed by it, and works equally well in brewing and rites.
The most important part is that the witch can justify any substitution to herself.
From a gameplay perspective, this also means that the player should justify such improvisations to the GM.
This can be as simple as using a pool of water in place of a mirror, because both are reflective, or more extreme, such as using a fresh egg in place of blood, as both are the fluids of life.

\begin{simpletable}{rX}
	\toprule
	TN & Example Task\\
	\midrule
	9 & An unusual component that still meets the specifications, e.g.\ a ritual circle scratched in the dirt instead of drawn in chalk.\\
	12 & A component that retains the fundamental property, e.g.\ scrying through a pool of still water instead of a mirror.\\
	15 & A component that is close, but violates a specification, e.g.\ pig blood instead of human blood.\\
	18 & A componment with a strong justification for relatedness, e.g.\ a fresh egg in place of blood.\\
	21 & A component with a weak justifcation for relatedness, e.g.\ apple juice in place of blood.\\
	\bottomrule
\end{simpletable}

\subsection{Consequences}

Magic is dangerous, especially when rushed or improvised.
The GM should feel free to reflect this in the consequences of failure on a magic Test, even when it is not a critical failure.
Failure on a magic Test need not indicate that nothing occurred, but might indicate that something unwanted or something rather tangential has occurred, or that the magic has succeeded, but with side effects.

For example, suppose a witch is attempting to brew a potion for hair regrowth, but has substituted several of the ingredients for similar ones they hoped would work.
A failure on the Test might mean that the potion successfully causes hair regrowth, but that the hair is the wrong colour or grows in more places that desired.

Other magics can have even more dangerous consequences.
A witch trying to scry through a puddle instead of a mirror might, on a narrow failure, only get an unclear image as the puddle is disturbed by wind.
But a more dire failure could mean that the target instead sees the witch herself through any nearby reflective surfaces, or that the imperfect scrying draws the attention of \emph{things} from other dimensions that look, reach or even climb out of the puddle.
Rituals to summon demons and the like can obviously have some of the most dangerous consequences of all, should they go wrong.
