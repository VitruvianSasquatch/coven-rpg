\chapter{Golemancy}
\chaplabel{golemancy}

\section{Animating a Golem}

A golem must be animated as part of creation, and the witch doing the animation must be involved in its creation, even if she is not the primary craftswoman.
To animate a golem, a witch simply touches it and wills it life; many consider Golemancy to be a particularly specialised application of \discref{willing}.
A novice golemancer has only enough animating force to maintain one golem at a time.
If she animates a new golem, the previous golem immediately becomes inanimate.
A witch may also withdraw the animating force from a golem she has animated at any time, though if this is to be done urgently (perhaps the golem has gone rogue), the GM may require a Test.

The crafting and animation, although strongly interlinked, are separate processes.
Tests related to the craftsmanship use an appropriate Craft skill and an appropriate attribute.
Tests related to the golem's animation, such as giving it instructions, use \testtype{wit}{golemancy}.
A witch can only animate a particular material into a golem if she has taken the appropriate feat.

\section{A Golem's Instructions}

A witch just beginning to dabble in \discref{golemancy} only has the skill to make very simple-minded, single-purpose creatures, although she will learn more nuance later.
These golems are imbued with a single instruction at the moment of their creation.
They will follow this until its completion, whereupon they will simply stand still and await destruction.
The instruction must be very simple, and the golem has minimal ability to improvise around it.
It should not have any conditional aspects, and the golem is unable to respond to any form of communication.
Example instruction are given below.

\begin{itemize}
	\item Deliver this note to the castle.
	\item Fetch my broom.
	\item Kill that man.
	\item Sweep the floor every evening.
	\item Extinguish any fires you see.
\end{itemize}

Additional information necessary to the completion of the task, such as the location of the castle or the identity of an intended victim, may be imparted with the instruction.
The golem will trust this information and cannot adapt if it is wrong, for example if the victim has been disguised.

Giving instructions with nuance, or instructions with multiple linked parts (such as ``go to the castle and kill the King'') requires a Test, with a TN set by the GM based on the complexity of the instruction.
A failure either prevents the golem from animating or, at the GM's option, corrupts the instructions.

\section{A Golem's Statistics}

A golem's physical statistics are determined by the material and method of its construction, and are specified in the appropriate feat.
These include its \attref{might}, \attref{grace}, and response to damage.
The GM is also advised to apply common sense to other consequences of a golem's construction: for example, a clay golem will sink in water, a wooden golem will float, and a gingerbread golem will go soggy and fall apart.
A golem lacks the other four attributes (\attref{wit}, \attref{will}, \attref{charm} and \attref{presence}) entirely; it automatically fails any Test that would require them.
It has no ranks in any skills.

A golem knows no languages; it cannot read, write, or comprehend speech.
It cannot speak, and furthermore cannot vocalise in any fashion.
The sounds it can make are limited to such things as clapping its hands and stamping its feet.

A golem has senses as good as a human, although only if its craftsmanship gives it the appropriate anatomy.
For example, a gingerbread golem with two currants for eyes can see, but if baked without the currants it will be blind.
A clay golem can only smell if a nose is sculpted upon its face.

\section{Feats}

\feat{Gingerbread Man}{gingerbread-golem}{4}{
	None
}{
	The simplest golems are not baked of clay, but of dough.
	When you bake a humanoid figure from gingerbread, approximately the size of a human hand, you may animate it as a gingerbread golem.
	
	A gingerbread golem has \negative{2} \attref{might} and 2 \attref{grace}.
	It is destroyed if it suffers any damage.
	
	Additionally, a gingerbread golem has a limited lifespan.
	After about a week, it grows stale and can no longer move.
	Moisture or water, even a couple of minutes in rain, will destroy it sooner.
}
