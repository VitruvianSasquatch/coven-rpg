\chapter{Golemancy}
\chaplabel{golemancy}

\section{Animating a Golem}

A golem must be animated as part of creation, and the witch doing the animation must be involved in its creation, even if she is not the primary craftswoman.
To animate a golem, a witch simply touches it and wills it life; many consider Golemancy to be a particularly specialised application of \discref{willing}.
A novice golemancer has only enough animating force to maintain one golem at a time.
If she animates a new golem, the previous golem immediately becomes inanimate.
A witch may also withdraw the animating force from a golem she has animated at any time, though if this is to be done urgently (perhaps the golem has gone rogue), the GM may require a Test.

The crafting and animation, although strongly interlinked, are separate processes.
Tests related to the craftsmanship use \skillref{crafting} and an appropriate attribute.
Tests related to the golem's animation, such as giving it instructions, use \testtype{will}{golemancy}.
A witch can only animate a particular material into a golem if she has taken the appropriate feat.

\section{A Golem's Instructions}

A witch just beginning to dabble in \discref{golemancy} only has the skill to make very simple-minded, single-purpose creatures, although she will learn more nuance later.
These golems are imbued with a single instruction at the moment of their creation.
They will follow this until its completion, whereupon they will simply stand still and await destruction.
The instruction must be very simple, and the golem has minimal ability to improvise around it.
It should not have any conditional aspects, and the golem is unable to respond to any form of communication.
Example instruction are given below.

\begin{itemize}
	\item Deliver this note to the castle.
	\item Fetch my broom.
	\item Kill that man.
	\item Sweep the floor every evening.
	\item Extinguish any fires you see.
\end{itemize}

Additional information necessary to the completion of the task, such as the location of (and directions to) the castle or the identity of an intended victim, may be imparted with the instruction.
The golem will trust this information and cannot adapt if it is wrong, for example if the victim has been disguised.
Furthermore, such information must be quite explicit.
For example, a golem instructed to ``attack intruders'' has no mechanism for distinguishing between invited guests and intruders.

Giving instructions with nuance, or instructions with multiple linked parts (such as ``go to the castle and kill the King'') requires a Test, with a TN set by the GM based on the complexity of the instruction.
A failure either prevents the golem from animating or, at the GM's option, corrupts the instructions.

\section{A Golem's Statistics}

A golem's physical statistics are determined by the material and method of its construction, and are specified in the appropriate feat.
These include its \attref{might}, \attref{grace}, Speed, Resilience and Shock Threshold.
A golem whose Shock Threshold is met or exceeded by a Damage Test is immediately destroyed, instead of knocked out.
The GM is also advised to apply common sense to other consequences of a golem's construction: for example, a clay golem will sink in water, a wood or wax golem will float, and a gingerbread golem will go soggy and fall apart.
All golems are immune to poisons and diseases, and unaffected by potions, poultices and the like.

As for its other attributes, a golem lacks \attref{ken}, \attref{wit}, \attref{will}, \attref{charm} and \attref{presence} entirely; it automatically fails Tests that would require them.
It has 0 \attref{heed}.
However, a golem is unrelenting and untiring, and it has no mind to affect.
As such, it may be considered to automatically succeed at many Tests that would require \attref{will}.
Lastly, a golem has no ranks in any skills.

Initially, a witch can only animate small golems: about a handspan in height.
She doesn't have enough animating force to manage anything bigger, and anything smaller can't hold the magic required.

A golem knows no languages; it cannot read, write, or comprehend speech.
It cannot speak, and furthermore cannot vocalise in any fashion.
The sounds it can make are limited to such things as clapping its hands and stamping its feet.

A golem has senses as good as a human, although only if its craftsmanship gives it the appropriate anatomy.
For example, a gingerbread golem with two currants for eyes can see, but if baked without the currants it will be blind.
A clay golem can only smell if a nose is sculpted upon its face.

\section{Feats}

\feat{Gingerbread Man}{gingerbread-golem}{20}{
	None
}{
	The simplest golems are not baked of clay, but of dough.
	When you bake a humanoid figure from gingerbread, you may animate it as a gingerbread golem.
	The entire preparation and baking process takes approximately half an hour, although an entire batch of golems can be crafted at once if the size of the oven allows.
	
	\materials{Flour, sugar, eggs, butter, \herb{ginger}{3}}
	
	A gingerbread golem has \negative{2} \attref{might}, 1 \attref{grace} and 15 Speed.
	It has an effective Shock Threshold of 1; it is destroyed if it takes any damage.
	
	Additionally, a gingerbread golem has a limited lifespan.
	After about a week, it grows stale and can no longer move.
	Moisture or water, even a couple of minutes in rain, will destroy it sooner.
}

\feat{Wood Golem}{wood-golem}{15}{
	\featref{gingerbread-golem}
}{
	Wood offers a far more robust and permanent golem than gingerbread.
	When you whittle or assemble a humanoid figure from wood, you may animate it as a golem.
	The time required to do this depends on the size of the golem and the piece of wood you are crafting from.
	Whittling a small golem from an approximately man-shaped piece of wood may take as little as ten minutes, but carving one from a solid chunk of log might take an hour or more.
	Carving a much larger one could take days, and it would likely be faster to assemble it from multiple pieces of wood.
	
	A wooden golem has \negative{1} \attref{might}, 1 \attref{grace} and 10 Speed.
	It has 4 Resilience and a Shock Threshold of 14.
	Damage to a wooden golem cannot be repaired.
}

\feat{Wax Golem}{wax-golem}{15}{
	\featref{gingerbread-golem}
}{
	Wax isn't quite so robust as wood, but it can be very quick to mould and repair.
	When you mould or cast a humanoid figure from tallow or beeswax, you may animate it as a golem.
	The wax or tallow needs to be warmed to be moulded.
	Leaving it in the sun on a warm summer's day provides about the temperature required, as does sitting near a fire.
	Once warmed, the golem can be moulded by hand in a couple of minutes.
	
	A wax golem has \negative{1} \attref{might}, 1 \attref{grace} and 10 Speed.
	It has 2 Resilience and a Shock Threshold of 12.
	Damage to the golem can be easily repaired by the application of a little more warm wax.
	
	Wax golems are susceptible to heat.
	A hot summer's day won't hurt, just make them a little softer, but coming too close to a fire or forge will leave them in a dribbly pool on the ground.
}

\feat{Clay Golem}{clay-golem}{15}{
	\skillref[1]{golemancy},
	\featref{wood-golem} or \featref{wax-golem}
}{
	Wood, wax, tallow and gingerbread contain at least traces of life, making them easier to animate.
	Clay has never known life at all, but you've finally figure out how to teach it.
	When you mould a humanoid figure from clay and fire it in a kiln, you may animate it as a golem.
	The firing process takes at least ten hours.
	
	A clay golem has 0 \attref{might}, 1 \attref{grace} and 10 Speed.
	It has 14 Resilience and a Shock Threshold of 18.
	Damage to a clay golem can be repaired by filling the cracks with clay and refiring the golem.
	
	Clay golems are all but immune to heat.
	After all, they were fired to over \SI{1000}{\celsius} at their creation.
	Only rapid quenching from red-hot poses any threat at all.
}

\feat{Golem Programming}{change-golem-instructions}{20}{
	\featref{gingerbread-golem}
}{
	You can change the instruction imbued into one of your golems, allowing you to reuse the same golem for multiple tasks.
	Reprogramming a golem requires you to be touching it, and takes a minute of concentration.
	The normal restrictions apply to the new instruction, and the golem can only have one instruction at a time; adding a new instruction removes the previous one.
	You may only reprogram golems powered by your own animating force.
}

\feat{Delegated Programming}{change-golem-instructions-familiar}{10}{
	\featref{change-golem-instructions}
}{
	You've taught your familiar a few tricks of golemancy, and it may use the bond it shares with you to tap into your own animating force.
	Your familiar may reprogram golems, using the same rules as \featref{change-golem-instructions}.
	It reprograms your golems, however, instead of its own.
	If imbuing the new insruction requires a Test, your familiar uses its own \testtype{will}{golemancy}, not yours.
}

\feat{Advanced Instructions}{advanced-golem-instructions}{20}{
	\skillref[1]{golemancy},
	\featref{gingerbread-golem}
}{
	You can imbue your golems with more advanced instructions.
	The instructions can contain several steps, and conditional elements.
	On the whole, the golem can contain instructions that would take no more than a minute to convey by reasonably-paced speech.
	
	The golem still has next to no ability to improvise around the instructions, and will unreasoningly attempt to carry them out until it completes them or is destroyed.
	Information can still be imparted alongside the instructions, but it must still be explicitly.
	For example, an instruction to ``attack intruders'' will still fail, however the golem can now be instructed to ``attack anyone except me who enters this house'' or ``attack anyone who enters this house unless the door is unlocked with the key.''
	
	The golem can respond to some degree of communication, such as pointing and nodding, if explicitly instructed to.
	However, it still cannot \emph{comprehend} the communication.
	For example, it can follow an instruction to ``go where this man points,'' but not to ``follow this man's directions''.
}

\feat{Golem Language}{golem-understand-language}{20}{
	\skillref[1]{golemancy},
	\featref{advanced-golem-instructions},
	\featref{change-golem-instructions}
}{
	You imbue golems with your own understanding of language, allowing them to understand speech, as well as to read and write.
	This also includes some understanding of body language, though sarcasm, metaphor and the like continue to elude the golem.
	
	The golem still exclusively follows the instructions it has been imbued with, but you may now give it instructions such as ``follow my orders,'' and give further orders verbally.
	Verbal instructions must be just as explicit as imbued ones, however.
}

\feat{Golem Familiar Interpretation}{golem-understand-familiar}{10}{
	\skillref[1]{golemancy},
	\featref{golem-understand-language},
	\featref{change-golem-instructions-familiar}
}{
	Your golems gain the same ability to innately understand your familiar that you have, as effectively as though your familiar was speaking.
}

\feat{Golem Intelligence}{golem-intelligence}{20}{
	\skillref[1]{golemancy},
	\featref{advanced-golem-instructions},
	\featref{change-golem-instructions}
}{
	You may imbue your golems with some degree of intelligence.
	Although, to be honest, they're still a little dull.
	They gain \attref{ken}, \attref{wit}, \attref{charm} and \attref{presence} scores of 0, and may make Tests requiring these attributes.
	It can also perform a little improvisation around the best way to carry out its instructions.
	For example, if instructed to kill someone, it might pick up a club or sword instead of using its fists.
	It may finally be given instructions such as ``attack intruders,'' and will use its best judgement to determine whether someone is an intruder.
}

\feat{Golem Speech}{golem-speak}{15}{
	\skillref[2]{golemancy},
	\featref{golem-understand-language},
	\featref{golem-intelligence}
}{
	If you create your golem's body with a mouth and a tongue, you may imbue it with the ability to speak.
	%And sing, though not necessarily well.
}

\feat{Twin Golems}{more-golems}{20}{
	\skillref[1]{golemancy},
	\featref{gingerbread-golem}
}{
	You can muster enough animating force to maintain a second golem at the same time.
	If you try to animate a third golem, you may choose which existing golem becomes inanimate.
}

\feat{Dwarf Golem}{medium-golem}{20}{
	\featref{wood-golem} or \featref{wax-golem}
}{
	You can muster enough animating force to create larger golems, about mid-thigh height.
	Such golems have their \attref{might} increased by 1, and their \attref{grace} decreased by 1.
	You can maintain only one golem of this size, regardless of how many golems you can animate in total.
	
	Obviously, crafting the body of a larger golem takes more material and normally more time.
	Acquiring an oven, kiln or forge large enough can also present an obstacle for some kinds of golem.
}

\feat{Man-Sized Golem}{large-golem}{25}{
	\skillref[1]{golemancy},
	\featref{medium-golem}
}{
	It takes a lot of clay to make a golem the size of a man, but this pales in comparison to the force required to bring such a golem to life.
	You should know, because you can finally muster that much force.
	Such golems have their \attref{might} increased by 2, and their \attref{grace} decreased by 1.
	You can maintain only one golem of this size, regardless of how many golems you can animate in total.
}
