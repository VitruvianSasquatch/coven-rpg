\chapter{Broomcraft}
\chaplabel{broomcraft}

A broom is primarily a witch's method of getting from A to B: from village to village, out into a distant forest, or all the way up the city.
It's not the easiest mode of transport, and it can be quite terrifying at first, but a witch can pick up the rudiments in a week or two's practice.
This is as far as most witches go.
But some, with enough practice, skill and flamboyance, can turn it into a real art.

\section{Laws of Aviation}

For an unpracticed witch, there are a lot limitations to broomstick flying.
After all, she is sitting on a thin stick floating hundreds of metres in the air.
First and foremost, it is easiest to balance on a broomstick if one sits side-saddle, and this is all an unpracticed witch is capable of.
This does, however, make it a lot harder to turn, and to fly at high speeds.
Barrel rolls are right out.

Additionally, the witch must keep at least one hand on the broom at all times, to prevent it spinning out of a control.
Manoeuvre is even easier with both hands, and the GM is encouraged to make \skillref{flying} Tests more difficult for a witch using only one hand.

\subsection{Taking Off}

Getting the broomstick off the ground in the first place is no easy task.
A broom needs a running start before the magic will catch, and even then it isn't consistent.
The witch must hold the broomstick level as she runs along the ground, then jump on it quickly when it starts.

Attempting to start a broom requires an Action and a 15 metre run-up.
A character must move this distance in a straight line on one Turn, and may Dash as part of the broom-starting action if necessary. %TODO: Ensure Dashing is in the rules.
They must also succeed on a TN 12 \testtype{grace}{flying} Test or the broom fails to start.
As normal, the Test is not required if there is no time pressure, as the witch may run up and down as many times as necessary until the broom starts.

The Test to start a broom may be more difficult in adverse conditions; the following table provides suggestions for the TN of such Tests.
It is possible to achieve the necessary run-up through falling, although such a thing is \emph{very} difficult and the consequences for failure are obviously drastic.

\begin{simpletable}{rX}
	\toprule
	TN & Conditions\\
	\midrule
	12 & Nominal.\\
	15 & Blowing a gale.\\
	18 & In a bog.\\
	21 & While falling.\\
	\bottomrule
\end{simpletable}

\subsection{Climbing and Stalling}

A novice witch can climb indefinitely at a \SI{30}{\degree} angle, although steeper climbs can be achieved by experienced witches or for short periods.
Not even the very best witches can perform an indefinite vertical climb, although with enough care and a lot of momentum on entering the climb, it is possible to maintain one for 50 metres or more.

A witch who tries and fails too steep a climb soon finds her broomstick stalling.
She has until she hits the ground to point the broomstick downwards, restart it, and then pull out of the dive.
This requires a very difficult check, although more altitude will afford her more time, and make it slightly easier.

\subsection{Cruising and Turning}

A witch sitting sidesaddle doesn't have a particularly good grip on her broom, and this limits the speed she can go without the oncoming air ripping her clean off.
%TODO

A broomstick must also maintain a minimum speed in order to maintain lift: this speed is about \SI{10}{\kilo\metre\per\hour} or about 30 metres per turn.
Dropping below this speed for more than a moment causes the broomstick to stall.

Another consequence of sitting sidesadle is a poor ability to steer the broom, leading to a turning circle several hundred metres in diameter.

\subsection{Landing}

There are two main techniques employed to land a broom.
In the first, the witch hits the ground running and performs a moving dismount.
This requires a flat stretch of ground to land on, but allows her to maintain momentum, perhaps important in a chase.
The second, slightly trickier technique is to bring the broom to a gentle stall just above the ground.
This allows an experienced witch to land with pinpoint accuracy, or an unexperienced witch to fall unceremoniously on her behind.

Unhurried, and with no care to accuracy, a witch can achieve either form of landing.
A witch trying to land with limited space available space, or on rough terrain, may require a Test.
The GM is encouraged to adjust the TN of the Test depending on how the landing is performed; a stall landing is generally trickier, but less dependent upon the terrain.

\subsection{Lift}

A broomstick can carry one witch, and about as much equipment as she could easily walk around with on the ground.
It can also carry a familiar, as long as it's of reasonable size.
A cat is fine, a beagle is borderline, a wolfhound is right out.

A little bit of extra weight, or something inconveniently large, makes the broomstick unwieldy.
Tests to take off or perform manoeuvres are more difficult, and the broom's maximum speed may be reduced.
A lot of extra weight, such as a passenger, makes proper flight impossible.
The broomstick cannot take off, and cannot remain in level flight.
It might still be possible to bring it down and land safely, with an appropriate Test.



\section{Feats}

\feat{Ride Astride}{broom-astride}{15}{
	None
}{
	By sitting astride your broom, instead of side-saddle, you can go faster and turn more sharply without falling off.
	It's harder to balance, but you've got the hang of it now.
	
	%TODO: Speed
	
	Gripping the broom with your legs also allows you to turn much more sharply.
	Your turning radius is reduced to about 15 metres.
}

\feat{Chocks Away}{broom-take-off}{15}{
	None
}{
	There's a simple knack to starting a broom, and you've got it down pat now.
	You don't need a Test to start a broom under normal conditions (although you still need the run-up), and the TN of any Test to start the broom under difficult conditions is reduced by 3.
}

\feat{Passengers \& Cargo}{broom-passenger}{15}{
	None
}{
	You can get enough lift out of your broom to carry a passenger.
	Unless they are a skilled flier in their own right, they need to hold on to you while in flight.
	
	If you are not carrying a passenger, you can use the additional lift to carry saddlebags, containing no more than the weight of a person.
}

\feat{Tool Rider}{broom-improvise}{10}{
	None
}{
	Brooms are certainly very traditional, but really, any old tool will do.
	You can ride any long-handled, man-made, properly crafted tool, such as a rake, spade, scythe or wood-axe.
	It still needs to be trained as usual, unless you also have \featref{broom-whisperer}.
}

\feat{Bristlebrake Turn}{broom-turning}{20}{
	\skillref[1]{flying},
	\featref{broom-astride}
}{
	By flicking the back of the broom around, you can reduce your turning radius to just 1 metre.
	Turning more than \SI{90}{\degree} by this method require a difficult Test, and stalls the broom on a failure.
}

\feat{Broom Whisperer}{broom-untrained}{15}{
	\skillref[1]{flying}
}{
	You've got the knack of flying for yourself now, and don't need a broom to be trained to fly it.
	You can even train a broom this way, although without one of its own to learn from the process takes about 24 hours of flight time.
}

\feat{No Hands!}{broom-no-hands}{15}{
	\skillref[2]{flying}
}{
	You can fly a broomstick with just your legs, leaving both hands free.
	You ignore any penalties for flying with just one hand, but suffer those penalties when flying with no hands instead.
}

\feat{Hover}{broom-hover}{15}{
	\skillref[2]{flying},
	\featref{broom-astride}
}{
	By tipping your broomstick right back and balancing carefully, you can hover, staying roughly still in mid-air.
	Staying actually still is quite difficult, and you tend to drift around a fair bit.
	You can stay in roughly the same place, but staying still enough to, for example, reach out and touch a particular thing requires a Test.
}

\feat{Nearly a Broom}{broom-improvise-2}{15}{
	\skillref[2]{flying},
	\featref{broom-improvise},
	\featref{broom-untrained}
}{
	It's quite surprising quite what you can convince to be a broom, if you put your mind to it.
	You can ride just about any appropriately-sized length of wood, as long as it's obvious which end is the back.
	Even just a bit of a fork at one end of the stick will do, or you could just lash a few sticks on quickly.
}
