\chapter{Broomcraft}
\chaplabel{broomcraft}

A broom is primarily a witch's method of getting from A to B: from village to village, out into a distant forest, or all the way up the city.
It's not the easiest mode of transport, and it can be quite terrifying at first, but a witch can pick up the rudiments in a week or two's practice.
This is as far as most witches go.
But some, with enough practice, skill and flamboyance, can turn it into a real art.

\section{Laws of Aviation}

For an unpracticed witch, there are a lot limitations to broomstick flying.
After all, she is sitting on a thin stick floating hundreds of metres in the air.
First and foremost, it is easiest to balance on a broomstick if one sits side-saddle, and this is all an unpracticed witch is capable of.
This does, however, make it a lot harder to turn, and to fly at high speeds.
Barrel rolls are right out.

\subsection{Taking Off}

Getting the broomstick off the ground in the first place is no easy task.
A broom needs a running start before the magic will catch, and even then it isn't consistent.
The witch must hold the broomstick level as she runs along the ground, then jump on it quickly when it starts.

Attempting to start a broom requires an Action and a 15 metre run-up.
A character must move this distance in a straight line on one Turn, and may Dash as part of the broom-starting action if necessary. %TODO: Ensure Dashing is in the rules.
They must also succeed on a TN 12 Grace + Flying Test or the broom fails to start.
As normal, the Test is not required if there is no time pressure, as the witch may run up and down as many times as necessary until the broom starts.

The Test to start a broom may be more difficult in adverse conditions; the following table provides suggestions for the TN of such Tests.
It is possible to achieve the necessary run-up through falling, although such a thing is \emph{very} difficult and the consequences for failure are obviously drastic.

\begin{simpletable}{rX}
	\toprule
	TN & Conditions\\
	\midrule
	12 & Nominal.\\
	15 & Blowing a gale.\\
	18 & In a bog.\\
	21 & While falling.\\
	\bottomrule
\end{simpletable}

\subsection{Lift}

A broomstick can carry one witch, and about as much equipment as she could easily walk around with on the ground.
It can also carry a familiar, as long as it's of reasonable size.
A cat is fine, a beagle is borderline, a wolfhound is right out.

A little bit of extra weight, or something inconveniently large, makes the broomstick unwieldy.
Tests to take off or perform manoeuvres are more difficult, and the broom's maximum speed may be reduced.
A lot of extra weight, such as a passenger, makes proper flight impossible.
The broomstick cannot take off, and cannot remain in level flight.
It might still be possible to bring it down and land safely, with an appropriate Test.



\section{Feats}

\feat{Ride Astride}{astride}{
	None
}{
	By sitting astride your broom, instead of side-saddle, you can go faster and turn more sharply without falling off.
	It's harder to balance, but you've got the hang of it now.
	
	%TODO
}

\feat{Chocks Away}{take-off}{
	None
}{
	There's a simple knack to starting a broom, and you've got it down pat now.
	You don't need a Test to start a broom under normal conditions (although you still need the run-up), and the TN of any Test to start the broom under difficult conditions is reduced by 3.
}

\feat{Passengers \& Cargo}{broom-passenger}{
	None
}{
	You can get enough lift out of your broom to carry a passenger.
	Unless they are a skilled flier in their own right, they need to hold on to you while in flight.
	
	If you are not carrying a passenger, you can use the additional lift to carry saddlebags, containing no more than the weight of a person.
}

\feat{Broom Whisperer}{untrained-broom}{
	\skillref[1]{flying}
}{
	You've got the knack of flying for yourself now, and don't need a broom to be trained to fly it.
	You can even train a broom this way, although without one of its own to learn from the process takes about 24 hours of flight time.
}
