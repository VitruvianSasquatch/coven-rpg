\chapter{Contracts}
\chaplabel{contracts}

\section{Contract Law}
\seclabel{contracts}

\capital{\contracts} are among the most complicated of all magics.
In fact, much of the skill in using {\contracts} lies in making them complicated enough to hide loopholes, or catches, to dupe the other party and come out on top.
The other side of the coin is understanding them well enough to avoid being caught out by the same tricks.

Some players, and some GMs, will revel in this complication, loving the battle of wits as they open and close loopholes in the {\contracts} they write.
Others, however, will think it too much like work, preferring not to bog down play with such details.
As such, there are two ways to use {\contracts} in your game.
The players and GM should agree on which method is being used before anyone takes feats from this chapter, to avoid confusion and disappointment.

The first way is to write them yourself---word for word.
You can even put your character's signature upon them---perhaps in red pen---and use the piece of paper as a prop around the game table.
The GM might offer you the chance to make a Test to spot a loophole, if they see one that you haven't, but ultimately, the words written upon the paper are the words that form the {\contract} in-game.

The second method is for you just describe the intent of the contract to the GM.
The GM can then call for a Test---possibly {\opposed} by the other party in the {\contract}---for your character to draft the {\contract}.
On a failure, the GM can invent a loophole to catch you out on, while on a success you write a clean {\contract}, or could even catch the other party out.

Tests related to {\contracts} use the \skillrefspeciality{lore}{Contracts} skill.
Unlike most \skillref{lore} skills, which are typically paired with \attref{ken}, most \skillrefspeciality{lore}{Contracts} Tests will use \attref{wit}.
Writing and spotting loopholes and other tricks is less about rote learning, and more about outwitting the other parties to the {\contract}.

\subsection{Creating Contracts}

In-game, a {\contract} is a magical agreement brought into existence using a written document, signed in \materialref{blood}.
A {\contract} consists of three parts: {\stipulations}, {\penalties}, and {\signatories}.
All three must be present on the written document.
The words upon the document form the binding rules of the {\contract}---any spoken agreements do not matter, only what is written.

There are two steps to creating a {\contract}.
The first is writing the document, specifying the {\stipulations}, the {\penalties}, and who the {\signatories} are to be.
The second is getting each {\signatory} to {\sign} the document.
The document must not be modified during the second step.
If it is altered at any point after the first {\signatory} {\signs}, but before the last {\signatory} {\signs} it becomes {\void}.

Once the last {\signatory} signs it, the {\contract} takes effect.
Every {\signatory} is immediately, magically aware of this, regardless of whether they are currently present.
From this point onwards, the document used to create the {\contract} is no longer relevant.
It can be modified, or destroyed, without affecting the terms of the {\contract} itself.

Not just anyone can write a magically-binding {\contract}, although anyone can {\sign} one.
At least one of the people writing the {\contract} must have the feat \featref{contracts}, in order to make it magically binding.
Furthermore, everyone who helps to write the {\contract} must be a {\signatory} upon it.

Several feats allow extra clauses to be specified upon a {\contract} that a person helps to write.
Firstly, note that a person must be a {\signatory} upon the {\contract} in order to use these.
Secondly, while a witch must learn the feat \featref{contracts} before learning any of these feats, some other creatures may have these feats without being able to write a {\contract} independently. %TODO: Link to demons.
In this case, they can add these clauses to a {\contract} they help to write, and {\sign}, but they need someone with the \featref{contracts} to create the {\contract} in the first place.

\subsection{Signatories}
\seclabel{contract-signatories}

\subsection{Stipulations}
\seclabel{contract-stipulations}

\subsection{Penalties}
\seclabel{contract-penalties}

\subsection{Voiding a Contract}
\seclabel{contract-void}

\section{Feats}

\feat[headology]{Signed in Blood}{contracts}{20}{
	\noprereq
}{
	You may create {\contracts}, as long as you are one of the {\signatories}.
	Your {\contracts} are limited to only two {\signatories}, unless you are helped by someone who can create {\contracts} with more.
}
