\chapter{Character Creation Guide}
\chaplabel{character-creation-guide}

\section{Step Zero: Character Concept}

The most important part of a character is the concept.
Who is your character, what does she do?
You can try to flesh this all out now, or fill it in as you work through character creation.
Make sure your character is somebody you will enjoy roleplaying.

Also make sure your witch fits into the coven; discuss this with the other players and your GM.
It can be quite painful for everyone involved playing an unscrupulous necromancer in a coven of saccharine healers, or vice versa.
The GM should provide some idea of the tone intended for the game, to avoid this sort of trouble.
Diversity can also be good: make sure you know what you're letting yourselves in for if everybody in the group wants to play a potion-brewer.

\section{Step One: Attributes}

Attributes are a witch's broad, innate capabilities.
Is she skinny and lithe or broad and well-muscled, quick-witted or bullheaded, domineering or silver-tongued?

At character creation, you have 15 points to spend on your character's attributes.
Spend these points on each of the six attributes, setting each attribute to between 0 and 4, inclusive.

\section{Derived Statistics}

\begin{simpletable}{ll}
	\toprule
	Statistic & Derivation\\
	\midrule
	Resilience & $(5 + \text{\attref{might}}) \div 2$\\
	Shock Threshold & $12 + \text{\attref{will}}$\\
	Speed & $8 + \text{\attref{might}} + \text{\attref{grace}}$\\
	\bottomrule
\end{simpletable}

\section{Steading}

Most witches have a steading.
This is the area a witch watches over, a region she defends and protects the inhabitants of.
The duties a witch has to her steading are numerous and varied, but typically involve healing the inhabitants and protecting them from threats of a magical nature.
Some witches also perform midwifing, care for the land itself, or even take it upon themselves to deal with non-magical threats, such as invading armies.
A witch's responsibilities are not limited to her steading, and nothing stops her from responding to threats outside it.
But inside it, everything is certainly her responsibility.

Decide whether your witch has a steading.
How big is it?
One village, several, or an entire kingdom?
What duties does she perform within it?
Do the inhabitants appreciate what she does for them?

Also discuss this with your GM, and the other players.
Has the GM already described a village that could be your steading?
It is not unheard of for witches to share a steading, although this can obviously lead to disagreements.
Do you share a steading with your coven, or have you carved the local region into one steading each?
